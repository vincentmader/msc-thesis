\section{Continuous Formulation}

    \todo{Is the below correct?}
    \begin{align}
        L_\text{coag}(m)
            &= \int\limits_0^\infty R_\text{coll}(m, m') \cdot n(m) \cdot n(m') \ \dd m \ \dd m' \\
        G_\text{coag}(m)
            &= \int\limits_0^\infty \int\limits_0^\infty R_\text{coll}(m', m'') \cdot
                n(m') \cdot n(m'') \cdot \delta_D(m-m'-m'') \ \dd m' \ \dd m'' \\
            &= \int\limits_0^\infty R_\text{coll}(m', m-m') \cdot n(m') \cdot n(m-m') \ \dd m'
    \end{align}

    \todo{The question is: How does the number of particles with mass $m$ change?} \\

    \todo{For the loss:}
    \begin{itemize}
        \item Particle with mass $m$ could collide with any other mass, labeled $m'$.
        \item This would then (here: pure coag.) lead to the merging of the particles.
        \item (A new particle with mass $m+m'$ is created and has to be added to the distribution.)
        \item A particle has to be removed from bin $m$ and $m'$.
        \item We sum (integrate) over all these masses $m'$.
    \end{itemize}

    \todo{For the gain}:
    \begin{itemize}
        \item How can a particle with mass $m$ be created via coagulation?
        \item For this to happen, two particles with masses $m'$ and $m''$ have to collide \& merge.
        \item Condition: Mass conservation $m=m'+m''$ ($\to\ \delta_D(m-m'-m'')$).
        \item We can reduce the 2D integral to a 1D integral, making use of the relationship
            \begin{equation}
                f(x_0)
                    = \int\limits_{-\infty}^{\infty} f(x) \cdot \delta_D(x-x_0) \ \dd x 
            \end{equation}
    \end{itemize}

\section{Discretized Formulation}

    Even in a highly simplified scenario, where only hit-and-stick coagulation is included, the 
    definition of the kernel $K_{kij}$ is not at all trivial. To assure both the consistency and 
    accuracy of the algorithm, one has to take care of two separate problems, namely:
    \begin{enumerate}
        \item The conservation of mass \textit{has} to be assured, otherwise the numerical solution can 
            not be assumed to remain stable for long. In the case of stick-and-hit coagulation, this 
            means that for every pair of colliding particles, a single new particle has to be created. 
            At the same time, the two initial particles have to be removed from the distribution. 
            During this process, the total mass should remain unaffected down to machine precision.
        \item When using a logarithmically spaced grid for the discretized mass axis, it can not be 
            assumed that after a collision of two dust particles with masses $m_i$ and $m_j$ the 
            resulting particle will carry a mass $m_k=m_i+m_j$ whose value can be mapped trivially onto 
            the grid. In general, the corresponding index will not be an integer, and instead lie 
            between somewhere between the two neighboring grid points with indices $k$ and $k+1$.
            Therefore, the result of the merging of $m_i$ and $m_j$ has to be divided in some sensible 
            way between these two neighboring bins.
    \end{enumerate}
    
    % The Kovetz-Olund Algorithm {{{ 
    \subsection{The Kovetz-Olund Algorithm}
        
        An elegant way for solving the two problems listed above is given in the 1969 paper
        by Kovetz \& Olund \cite{kovetz_olund_1969}, where they used the following procedure:
        \begin{enumerate}
            \item The stick-and-hit coagulation kernel is split into two parts. The first is
                the \textit{gain} of particles in bin $k$ due to the collision of particles from the
                bins $i$ and $j$. The second is the \textit{loss} of particles from bin $k$ due to
                collisions of particles in bin $k$ with particles from any other bin $j$.
            
                Using this separation into gain \& loss, the dust particle mass distribution's
                temporal derivative can be expressed in the following form:
                \begin{equation}
                    \pderiv{n_k}{t}
                        =\sum_{i=0}^{\mathcal N_m}\sum_{j=0}^{\mathcal N_m}
                            K_{kij}^\text{gain}\cdot n_i\cdot n_j
                        -\sum_{j=0}^{\mathcal N_m}K_{kj}^\text{loss}\cdot n_k\cdot n_j
                \end{equation}
                In other words, the total kernel [from above, cite eq.] can be written as
                \begin{equation}
                    K_{kij}
                        =K_{kij}^\text{gain}
                        -K_{ij}^\text{loss}\delta_{ki}
                \end{equation}
                Splitting the kernel like this into a gain \& a loss term is a quite general
                approach, and can be used in more complex scenarios as well (including e.g.
                particle fragmentation processes).
            \item For the scenario of pure hit-and-stick coagulation, a unique discretization
                of the kernel can be defined such that both the number of particles and the
                conservation of total mass are handled correctly. To do this, consider a
                pair of colliding particles with indices $i$ and $j$. Then, let the index
                $\bar k$ be chosen in such a way that the condition
                \begin{equation}
                    m_{\bar k}\leq m_i+m_j<m_{\bar k+1}
                \end{equation}
                is satisfied.
            \item As stated before, in hit-and-stick coagulation, a single new particle emerges
                for each pair of colliding particles. Using the definitions from above, this
                condition can be expressed as follows:
                \begin{equation}
                    K_{\bar k,ij}^\text{gain}
                    +K_{\bar k+1,ij}^\text{gain}
                    \overset{!}{=}K_{ij}^\text{loss}
                \end{equation}
            \item The second condition is that of mass conservation, which can be written as:
                \begin{equation}
                    m_{\bar k}K_{\bar k,ij}^\text{gain}
                    +m_{\bar k+1}K_{\bar k+1,ij}^\text{gain}
                    \overset{!}{=}(m_i+m_j)K_{ij}^\text{loss}
                \end{equation}
            \item Now, to map the resulting particle's mass onto two neighboring bins, let us define a 
                parameter $\varepsilon$ such that
                \begin{align}
                    K_{\bar k,ij}^\text{gain}
                        &=K_{ij}^\text{loss}\cdot(1-\varepsilon),\ \text{and}\\
                    K_{\bar k+1,ij}^\text{gain}
                        &=K_{ij}^\text{loss}\cdot\varepsilon
                \end{align}
                This assures that [equation from pt 3] is satisfied. If we now plug this
                definition into [equation from pt 4] and solve for $\varepsilon$, we
                arrive at
                \begin{equation}
                    \varepsilon
                        :=\frac{m_i+m_j-m_{\bar k}}{m_{\bar k+1}-m_{\bar k}}
                \end{equation}
        \end{enumerate}
        This is the Kovetz-Olund algorithm \cite{kovetz_olund_1969}, and it was also used in subsequent 
        coagulation studies by e.g. \cite{brauer_dullemond_henning_2007} and 
        \cite{birnstiel_dullemond_brauer_2010}.
    
    % }}}
    % Near-Zero Cancellation Handling {{{ 
    \newpage\subsection{Near-Zero-Cancellation Handling}
    
        When using floating-point numbers following the representation defined
        by the IEEE-754 standard, it can occur that
        \begin{align*}
          a+b=a
          \ \ \ \ \ \text{for} \ \ \ \ \
          b\neq0
        \end{align*}
        Typically, this happens when
        \begin{align*}
            |b|<\varepsilon_m\cdot|a|
        \end{align*}
        Here, $\varepsilon_m$ labels the \textit{machine precision}.\\
        \todo{Add: How big is $\varepsilon_m$ for an f32, how big for an f64?}\\
        
        Let $i$ and $j$ once again be the indices used to label two colliding particles. Additionally, 
        assume now that particle $i$ is \textit{much smaller} than particle $j$.\\
        
        The detailed balance approach from above requires the removal of both the big and the small 
        particle from the mass distribution, followed by the re-insertion of a new particle carrying the 
        initial pair's combined mass. This new particle would then have a mass which is nearly identical 
        to that of the bigger one of the original two particles, it would be only a tiny bit heavier.\\
        
        In the approach defined above this would mean that $\bar k=j$, i.e. the resulting particle will 
        reside in the same bin as the larger original one. Also, it would follow that 
        $\varepsilon\ll1$.\\
        
        Let us now take a look at the particle mass distribution in the bin $\bar k$ and, more 
        specifically, by how much it changes from one timestep to the next. For this particular pair of 
        $i$ and $\bar k=j$, we can write:
        \begin{align*}
            \pderiv{n_{\bar k}}{t}
                =K_{\bar k,i\bar k}^{\text{gain}}\cdot n_i\cdot n_{\bar k}
                -K_{\bar ki}^{\text{loss}}\cdot n_i\cdot n_{\bar k}
        \end{align*}
        Plugging in [equation from above] leads to
        \begin{align*}
            \pderiv{n_{\bar k}}{t}
                =(1-\varepsilon)K_{\bar ki}^{\text{loss}}\cdot n_i\cdot n_{\bar k}
                -K_{\bar ki}^{\text{loss}}\cdot n_i\cdot n_{\bar k}
        \end{align*}
        Here, the two terms almost cancel each other out. What remains is a contribution which is 
        proportional to $\varepsilon$.\\
        
        If $\varepsilon$ is small enough, the double-precision accuracy of the floating point 
        representation will lead to breakdown of the method.\\
        \todo{rewrite this sentence, copied almost exactly from Kees}\\
        
        It is relatively easy to identify the particle pairs $(i,j)$ for which the scenario detailed 
        above will occur. Let $i$ (without loss of generality) be the index of the larger one of the two 
        colliding masses. Cancellation may then occur when the resulting $k$ is equal to $j$.\\
        
        In that case, we carry out the subtraction in [previous equation] analytically, and write:
        \begin{align*}
            \pderiv{n_{\bar k}}{t}
                =-\varepsilon K_{\bar ki}^{\text{loss}}\cdot n_i\cdot n_{\bar k}
        \end{align*}
        \todo{Elaborate on this, see "Dust Evolution with Binning Methods"}
    
        \newpage

% \begin{figure}[h!]
%     \centering
%     \begin{minipage}{.5\linewidth}
%         \centering
%       	\subfloat[]{
%             \label{:a}
%       	  	\includegraphics[width=\linewidth]{34/canc.pdf}
%       	}
%     \end{minipage}%
%     \begin{minipage}{.5\linewidth}
%         \centering
%       	\subfloat[]{
%             \label{:b}
%       	  	\includegraphics[width=\linewidth]{34/nocanc.pdf}
%       	}
%     \end{minipage}
%     \caption{}
% \end{figure}

% \newpage


\begin{figure}[h!]
    \centering
    \begin{minipage}{.5\linewidth}
        \centering
      	\subfloat[Coagulation Kernel without Cancellation Handling]{
            \label{:a}
      	  	% \includegraphics[width=\linewidth]{34/K_coag_canc.pdf}
      	  	\includegraphics[width=\linewidth]{34/error_K_coag_canc.pdf}
      	}
    \end{minipage}%
    \begin{minipage}{.5\linewidth}
        \centering
      	\subfloat[Coagulation Kernel with Cancellation Handling]{
            \label{:b}
      	  	% \includegraphics[width=\linewidth]{34/K_coag_nocanc.pdf}
      	  	\includegraphics[width=\linewidth]{34/error_K_coag_nocanc.pdf}
      	}
    \end{minipage}
    \caption{
        Influence of Near-Zero Cancellation on the Mass Error of the
        Coagulation Kernel
    }
    \label{}
\end{figure}


% \begin{figure}[h!]
%     \centering
%     \end{minipage}
%     \begin{minipage}{.5\linewidth}
%         \centering
%       	\subfloat[]{
%             \label{:a}
%       	}
%     \end{minipage}%
%     \begin{minipage}{.5\linewidth}
%         \centering
%       	\subfloat[]{
%             \label{:b}
%       	}
%     \end{minipage}
%     \caption{
%         Kernel Mass Error:
%         (a) Mass Error of Coagulation Kernel without Cancellation Handling
%         (b) Mass Error of Coagulation Kernel with Cancellation Handling
%         (c) Mass Error of Fragmentation Kernel
%         (d) Mass Error of Total Kernel
%     }
% \end{figure}

% \newpage


    % }}}
