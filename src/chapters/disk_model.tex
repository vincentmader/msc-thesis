In the following we will concern ourselves with the definition of a simple model
for the proto-planetary disk, as well as for its gas and dust contents. 
It needs to be said here that due to the complexity of the involved processes and the (literally)
astronomically large number of particles present in the disk, our model can only ever be 
a highly simplified approximation to reality. \\

Nonetheless, it is our goal to formulate a model which captures a "typical" PPD's most relevant
properties, in order to allow the study conduction regarding dust particle coagulation 
and/or fragmentation in the subsequent chapters. \\

% Preliminary Definitions and Assumptions {{{
\section{Preliminary Definitions and Assumptions}
\label{sec:first_assumptions_about_the_disk}

% \begin{enumerate}
%     \item
        % \assume{} 
        We will base this model on the aforementioned
        nebular hypothesis: The origin of the modeled proto-planetary disk is assumed to be
        found in the collapse of a giant cloud of gas and dust in the interstellar medium.
        Having inherited its contents, the disk's elementary and material composition mirror
        that of the interstellar medium. As such, the disk contains mainly molecular hydrogen
        mono-atomic $\text{H}_2$, to a lesser extent helium $\text{He}$, and small trace amounts 
        of metals. 
        % The disk's elementary and material composition equal that 
        % of the interstellar medium, from which it inherited its material contents.
        % \todo{Elements?}
        % As such, the elements most prevalent in the disk are hydrogen and, to a lesser extent, helium,
        % with trace amounts of metals.
        % \assume{Gas mean particle mass (weight?).}
        % TODO Rewrite
        The mean molecular gas particle mass is given as
        \begin{equation}
            m_g := \mu\cdot m_p
        \end{equation}
        where $m_p$ labels the mass of a single proton, and $\mu$ denotes the mean molecular 
        weight of the gas particles, which we will set to
        \begin{equation}
            \mu := 2.3
        \end{equation}

    % \item
        % \assume{} 
        The collapse of the cloud is assumed to have happened sufficiently long ago such
        that a star has already formed at the center of the disk. 
        For the studies that will be carried out in the context of this thesis, this star's 
        most relevant properties will be its mass $M_*$ as well as its luminosity $L_*$. 
        % \assume{}
        Both of these properties will be assumed to equal those of our own solar system's Sun. 
        Thus, we define
        \begin{align}
            M_*&:=M_\odot=\SI{1.989e30}{\kg},\ \text{and} \\
            L_*&:=L_\odot\ =\SI{3.828e26}{\joule\second^{-1}},
        \end{align}
        where $M_\odot$ and $L_\odot$ label the solar mass and luminosity, respectively.
        % \todo{What type of star is it? Where in the main sequence is it? How old is it?} 
        % \begin{itemize}
        %     \item G-type main sequence star
        %     \item often imprecisely called "yellow dwarf" 
        %           (imprecisely, because its light is actually white)
        % \end{itemize}
        % \todo{The Sun was not as it is today when it formed $\approx$ 4.6 billion years ago.} \\
        \\

    % \item
        Like in our own solar system, where the Sun is responsible for more than $99.9\%$ of the 
        total mass in the Solar System, we will assume most of the original gas cloud's mass to be
        concentrated inside the star in our model as well, while the disk surrounding it holds only 
        a much smaller fraction of the total mass. 
        Let $M_*$ and $M_\text{disk}$ label the masses of the star and the disk, respectively.
        The ratio between the two shall be defined as
        \begin{equation}
            q_\text{disk-to-star}
                =\frac{M_\text{disk}}{M_*}  
                :=0.01
        \end{equation}

    % \clearpage
    % \item
        % \assume{
        Furthermore, the disk's mass is assumed to be dominated by the contribution 
        of circumstellar gas, with only a comparatively tiny contribution of dust particles.
        Let $M_\text{dust}$ and $M_\text{gas}$ label the total mass present in the form of dust 
        and gas, respectively. Since in the context of our model the disk consists entirely out 
        of gas and dust, they must satisfy the relationship
        \begin{equation}
            M_\text{disk} = M_\text{gas} + M_\text{disk}
        \end{equation}
        The relative abundance between the two particle classes can be encoded into the 
        dust-to-gas ratio, which we define as
        \begin{equation}
            \label{eq:dust-to-gas_ratio}
            q_\text{dust-to-gas}=\frac{M_\text{dust}}{M_\text{gas}}:=0.01
        \end{equation}

        For the gas we will adopt an isothermal model of an ideal gas in hydrostatic equilibrium.
        The isothermal assumption allows us to relate gas pressure $P_g$, 
        gas volume density $\rho_g$, and sound speed $c_s$ via the expression
        \begin{equation}
            \label{eq:isothermal_condition}
            P_g = \rho_g \cdot c_s^2
        \end{equation}
        The hydrostatic assumption in turn allows us to express the relationship between 
        the gas pressure $P_g(r,z)$ and the gas volume density $\rho_g(r,z)$ 
        % at a position $\vec r=(r,\varphi,z)^T$ (\todo{symmetry, cyl. coords.})
        as
        \begin{equation}
            \label{eq:hydrostatic_equilibrium}
            \deriv{P_g}{z} = - \rho_g \cdot g_z,  % TODO Why `g_z` and not `g` ?
        \end{equation}
        where $g_z$ labels the acceleration due to gravity along the $z$-axis. \\

        % In this simplified model, the gas is assumed to be distributed
        % For the purpose of this thesis, mass $M_\text{gas}$ is assumed to be distributed rad
        % Chemically, the gas distribution is made up 

    % \item \assume{ideal gas}
   
    % \item 
        % Furthermore, we assume the validity of the ideal gas law.
        % To arrive at a simplified model of the complex processes involved in 
        % gas particle interaction, we and make the following three assumptions:
        % \begin{enumerate}
        %     \item The gas can be described by the gas law.
        %     \item The gas behaves isothermal. 
        %     \item The gas is in hydrostatic equilibrium.
        % \end{enumerate}

% \end{enumerate}

    In the context of this simplified model, the modeled disk is assumed to obey 
    radial symmetry on macroscopic scales. \\
    
    Let us now adopt a radially resolved one-dimensional disk model 
    with an inner and outer disk radius $r_\text{min}$ and $r_\text{max}$, respectively.
    We will define them as
    % We will model the disk radially, 
    % \todo{Consider a disk 
    \begin{align}
        r_\text{min} &= \SI{0.01}{\astronomicalunit}
        \ \ \ \ \text{and} 
        \\
        r_\text{max} &= \SI{100}{\astronomicalunit}
    \end{align}
    Along that axis, we will now define the radial dependency of a number of the disk's mos
    t relevant properties, like gas surface \& volume density, midplane temperature, ...

   \todo{Adopt cylindrical coordinates:}
   \begin{equation}
       \vec r = (r, \phi, z)^T
   \end{equation}

    \todo{The flaring angle is defined as}
    \begin{equation}
        f := 0.05
    \end{equation}

    Disk is symmetric. \\

    To keep things simple, we will not attempt to resolve the disk in all three spatial 
    dimensions $r$, $\phi$ and $z$. Instead, we will focus mostly on the dependence of the 
    disk's features on the distance $r$ to the disk center.
    For numerical treatment, this $r$-axis will be discretized into $\mathcal N_r$ bins,
    with a lower and upper bound of
    \begin{align}
        r_\text{min} = 0.01\ \text{AU} 
        \ \ \ \ \text{and} \\
        r_\text{max} = 100\ \text{AU}
    \end{align}
    respectively. \\

    After having defined the distribution of gas and dust in the disk, we will focus on one 
    specific location in the disk for the rest of this thesis. This location shall be location at 
    a distance of $r=10\ \text{AU}$ from the disk center in the midplane, i.e. $z=0$. Due to 
    the assumed radial disk symmetry we can neglect the definition of the azimuthal angle $\phi$.

    % \clearpage
    % \begin{enumerate}
    
        % \item The next assumption we will make concerns itself with disk symmetry. For this, let us 
        %     adopt cylindrical coordinates $(r, \varphi, z)$. In reality, each of the disk's 
        %     position-dependent properties could best be described by a function $f(r, \varphi, z)$ 
        %     that depends on all three of these coordinates. \\
    
            % On large scales, we will assume the disk's features to not depend significantly on 
            % $\varphi$ and $z$. \\
            % \todo{Assume perfect radial symmetry?}
    % \end{enumerate}
    
    \todo{Plot top-down view of the disk.}
    
% }}}
% Midplane Temperature {{{ 
\clearpage\subsection{Midplane Temperature}

    % The stellar luminosity $L_*$ is related to the radiation flux 
    % (i.e. energy per unit surface area per unit time) at a distance $r$ from the star.
    % \begin{align}
    %             L_* 
    %         % &=  \int B_*\ \text dA \\
    %         &=  \int B_* \cdot 4\pi r\ \text dr \\
    %         &=  B_*(r) \cdot 4\pi r^2 \\
    %             B_*(r) 
    %         &=  \frac{L_*}{4\pi r^2} 
    %     % L_* = \int\limits B(r) 4\pi r \text dr
    % \end{align}
    % here: assumption of isotropy (?)

    % to derive an expression for the temperature at the disk midplane,
    % we have to think about how much of that flux arrives there.

    % almost no radiation arrives along the midplane
    % high extinction coefficient, disk more wide than thick/tall

    % indidence angle (flaring index)
    % \begin{equation}
    %     f = 0.05
    % \end{equation}

    The stellar luminosity $L_*$ labels the total amount of energy emitted per time by the star at 
    the center of the disk. It is related to the stellar radiation flux $B_*$ via the following 
    surface integral:
    \begin{equation}
        L_* = \int_A B_*\ \text d A
        % \ \ \ \ \ \Rightarrow\ \ \ \ \
        % B_*(r) = \frac{L_*}{4\pi r^2}
    \end{equation}
    Assuming the star's radiative power output is directed isotropically all directions,
    we can simplify the above and then solve for the radiation flux $B_*(r)$ at 
    a distance $r$ from the star.
    \begin{equation}
        B_*(r) = \frac{L_*}{4\pi r^2}
    \end{equation}
    % Due to the high extinction coefficients in the inner midplane regions, most of the 
    % light emitted directly into the midplane will not reach the outer disk regions. 
    % Instead, these outer regions are heated by indirect radiation reaching them from 
    % the upper and lower regions of the disk. There, light from the star is scatterd at the 
    % "edge" of the disk.
    % the disk where light from the star is scattered into or out of the disk.
    % "boundary" of the disk, where it is scattered. 
    Let $f$ denote the angle of incidence (flaring angle) of the light onto the disk, and assume
    that one half of the light is scattered into, the other out of the disk.
    Then, the radiation flux $B_\text{mid}(r)$ reaching the midplane can be written as
    \begin{equation}
        B_\text{mid}(r) = \frac{f}{2} \cdot B_*(r)
        \ \ \ \text{with}\ \ \
        f := 0.05
    \end{equation}
    % light that reaches them indirectly, via 
    % scattering 
    % resulting from the 
    % disk flaring
    % At a distance of $r_0 = 10\ \text{au}$
    % Only a small part of the light actually reaches the midplane at a distance $r$. 
    % Along the 
    % line of sight of 
    % A large part of 
    % the light is lost due to being emitted 
    We make the assumption that the gas and dust particles in the disk can be modeled as perfect 
    black-body objects. 
    Then, the relationship between the radiation flux $B_\text{mid}$
    % reaching the disk midplane from the star, 
    and the resulting temperature $T_\text{mid}$ 
    % at the midplane of the irradiated disk 
    is given by the Stefan-Boltzmann law, which reads
    % coming from the star  
    % relationship between the
    %  and the midplane temperature 
    \begin{equation}
        \label{eq:stefan_boltzmann_law}
        B_\text{mid}(T) = \sigma_\text{SB} \cdot T_\text{mid}^4
    \end{equation}
    % Here, $\sigma_\text{SB}$ labels the Stefan-Boltzmann constant, which is given by 
    % \begin{equation}
    %     \sigma_\text{SB}
    %         = \frac{2\pi k_B^4}{15h^3 c^2}
    %         \approx \SI{5.670e-8}{\ \watt\ \kelvin^{-1}\ \meter^{-2}}
    % \end{equation}
    Plugging this all in and rearranging for the midplane temperature yields
    \begin{equation}
        \boxed{T_\text{mid}(r) = \bigg( \frac{f}{2} \cdot \frac{L_*}{4\pi r^2 \cdot
        \sigma_\text{SB}} \bigg)^{1/4}}
    \end{equation}

    % The relationship between the stellar luminosity $L_*$, (the total emitted energy 
    % by the star per time), and the radiation flux $B_\text{mid}$, (the energy reaching the 
    % considered disk region per unit surface area and per unit time), is given by the surface 
    % integral
    % % The luminosity, (i.e.  total energy emitted per time, can be determined by evaluating the
    % % integral of the radiation flux over the entire surface of the considered absorver/emitter.
    % % Thus, the relationship between radiation flux $B(T)$ and luminosity $L(T)$ is given by
    % \begin{equation}
    %     L_* = \frac{f}{2} \cdot \int_A B_* \ \text dA
    % \end{equation}
    % where $f$ labels the flaring angle of the disk.

    % \vfill

\begin{figure}[h!]
    \centering
    \begin{tikzpicture}
        \draw (0,0) -- (12,0) -- (12, 3) -- cycle;
        % TODO
    \end{tikzpicture}
    \caption{}
    \label{}
\end{figure}

\clearpage
  % TODO

    % \begin{equation}
    %     L_* = B_*(r=R_*) \cdot 4\pi R_*^2
    % \end{equation}
    % In our case we can simplify this, since we are assuming spherical symmetry (isotropy) of the 
    % star's radiative power output. For a fixed stellar luminosity $L_*$, the radiative flux $B(r)$ 
    % at a distance $r$ from the star can be written as
    % \begin{equation}
    %     B_*(r) = \frac{L_*}{4\pi r^2}
    % \end{equation}
    
    % \assume{
    % Since we are assuming spherical symmetry (isotropy) of the star's radiative power output, we 
    % can simplify this expression. For a fixed stellar luminosity $L_*$, the radiative flux $B(r)$ 
    % at a distance $r$ from the star can be written as}
    % \begin{equation}
    %     L_* = \frac{f}{2} \cdot B(r) \cdot 4\pi r^2
    % \end{equation}
    % \todo{} \\

    % Plugging in \cref{eq:stefan_boltzmann_law} and rearranging for the temperature $T$ leads to
    % \begin{equation}
    %     T(r) = \left(
    %         \frac{L_*}{4\pi r^2 \cdot \sigma_\text{SB}}
    %     \right)^{1/4}
    % \end{equation}

    % \todo{Inclusion of $f/2$:}
    % \begin{equation}
    % \boxed{
    %     T(r) = \left(
    %         \frac{f}{2}\cdot
    %         \frac{L_*}{4\pi r^2 \cdot \sigma_\text{SB}}
    %     \right)^{1/4}
    % }
    % \end{equation}

    \vfill

\begin{figure}[h!]
    \makebox[\textwidth]{
        \includegraphics[width=\paperwidth]{12/midplane_temperature.pdf}
    }
    \caption{Radial profile of the midplane gas temperature $T_\text{mid}(r)$.}
    \label{fig:radial_profile_of_midplane_temperature}
\end{figure}


    % \begin{itemize}
    %     \item midplane temperature $T_\text{mid}(r)$ (b.c. isothermal) \\
    %         \todo{Plot $T_\text{mid}$ vs. $r$ (lin-log?)}
    %     \item thermal velocity $u_\text{th}$ \\
    %         \todo{Plot $u_\text{th}$ vs. $r$ (lin-log?)}
    %     \item sound speed (here?)
    % \end{itemize}

% }}}
    % Sound Speed {{{ 
    \clearpage\subsection{Sound Speed}

        The speed of sound is defined as the speed at which pressure disturbances travel in a 
        given medium. It is given by 
        \begin{equation}
            \label{eq:general_definition_of_sound_speed}
            c_s := \sqrt{\frac{\partial P}{\partial \rho}}
        \end{equation}
        where $P$ and $\rho$ label the pressure and mass density, respectively. \\
        
        \assume{As noted before, we assume the gas to behave in an isothermal manner (i.e. the
        temperature stays constant across time.)} As such, the ideal gas equation of state reads 
        % A gas can be approximated to be isothermal if the sound wave period is much higher than the (radiative) cooling time of the gas, as any increase in temperature due to compression by the wave will be immediately followed by radiative cooling to the original equilibrium temperature well before the next compression occurs. Many astrophysical situations in the ISM are close to being isothermal, thus the isothermal sound speed is often used. For example, in conditions where temperature and density are independent such as H II regions (where the gas temperature is set by the ionizing star’s spectrum), the gas is very close to isothermal.
        \begin{equation}
            \label{eq:isothermal_ideal_gas_equation_of_state}
            P = \frac{\rho k_B T}{m_g}
        \end{equation}

        Plugging \cref{eq:isothermal_ideal_gas_equation_of_state} into
        \cref{eq:general_definition_of_sound_speed}, we arrive at the isothermal sound speed
        \begin{equation}
            c_s = \sqrt{\frac{k_BT}{m_g}}
        \end{equation}

        % TODO Decide on usage of $P$ vs. $P_g$ and $\rho$ vs. $\rho_g$

        \vfill

\begin{figure}[h!]
    \makebox[\textwidth]{
        \includegraphics[width=\paperwidth]{12/sound_speed.pdf}
    }
    \caption{Radial Profile of the Sound Speed $c_s(r)$}
\end{figure}

    % }}}
% Radial Profile of the Gas Density {{{ 
\clearpage\section{Gas and Dust Density}

    Our next task will be to formulate a simple model for the spatial distribution of the disk's 
    gaseous contents. In this context, we will adopt cylindrical coordinates 
    $\vec r=(r, \varphi, z)^T$ and define both the surface and volume density profiles along the 
    radial axis $r$.

    % Radial Profile of the Gas Surface Density {{{ 
    \subsection{Radial Profile of the Gas Surface Density}

        The gas surface density $\Sigma_g(r,\varphi)$ is directly related to the gas volume density
        $\rho_g(r,\varphi,z)$, and can be derived from it by integrating the volume density from 
        $-\infty$ to $+\infty$ along the $z$-axis, i.e. along the axis perpendicular to the disk:
        \begin{equation}
            \label{eq:relationship_between_gas_surface_density_and_gas_volume_density}
            \Sigma_g(r,\varphi)
                = \int\limits_{-\infty}^{\infty} \rho_g(r,\varphi,z)\ \dd z
        \end{equation}

        As noted before in \cref{sec:first_assumptions_about_the_disk}, % TODO assert citation
        we will assume the large-scale structure of the disk to be radially symmetric.
        This is of course quite a substantial simplification 
        that neglects the evolution of the disk structure on macroscopic scales,
        but will have to suffice as a first approximation in the context of this thesis.
        For the density distributions, assuming radial symmetry means that
        \begin{align}
            \Sigma_g(r, \varphi) 
                = \Sigma_g(r) 
            \ \ \ \ \text{and}\ \ \ \
            % \\
            \rho_g(r, \varphi) 
                = \rho_g(r)
        \end{align}
        Thus, we can simplify 
        \cref{eq:relationship_between_gas_surface_density_and_gas_volume_density} and write:
        \begin{equation}
            \Sigma_g(r)
                = \int\limits_{-\infty}^{\infty} \rho_g(r,z)\ \dd z
        \end{equation}
        
        Note that we do not yet have an expression for the gas volume density $\rho_g$ so far.
        Therefore, we cannot derive the surface density $\Sigma_g$ by simply evaluating the above
        integral. Instead, we will go the other way round and make an ansatz for the gas surface
        density. Then we will be able to derive the volume density from that. 
        Here, we will 
        follow the work of Brauer, Dullemond, \& Henning \cite{brauer_dullemond_henning_2007}
        % Following the works of \cite{brauer_dullemond_henning_2007}
        % , \cite{andrews_2007}, and \cite{kitamura_2002}, 
        and make the assumption that the radial profile of the gas surface density can be 
        approximated by using an inverse power law of the form
        \begin{equation}
            \label{eq:gas_surface_density_profile_inverse_power_law_ansatz}
            \Sigma_g(r) = \Sigma_0\cdot\frac{1}{r^\sigma}
            \ \ \ \text{with}\ \ \
            \sigma := 0.8
        \end{equation}  % TODO Why 0.8 ?
        Before we can use this though, we still need to define $\Sigma_0$, which we will do now: \\

        The total gas mass present in the disk is given by $M_\text{gas}$, the numerical value of 
        which can easily be derived from what we defined in
        \cref{sec:first_assumptions_about_the_disk}. Since the integration of the surface density 
        over the entire disk surface must yield this value, we can formulate the following 
        condition:
        \begin{align}
            M_\text{gas}
                % &= \int_{A_\text{disk}} \Sigma_g(r) \ \dd A \\
                &= \int\limits_{r_\text{min}}^{r_\text{max}} \Sigma_g(r) \cdot 2\pi r \ \dd r
        \end{align}
        % TODO Define r_min and r_max before.

        Since we discretized the radial axis into $\mathcal N_r$ "bins", we can rewrite this as
        \begin{equation}
            M_\text{gas} = \sum_{i=1}^{\mathcal N_r} \Sigma_g(r_i) \cdot 2\pi r_i \cdot \Delta r_i
        \end{equation}
        % TODO Define N_r before.

        Now plug in \cref{eq:gas_surface_density_profile_inverse_power_law_ansatz} and 
        rearrange for $\Sigma_0$ to arrive at
        \begin{equation}
            \Sigma_0
                = M_\text{gas} \cdot \left[
                    \sum_{i=1}^{\mathcal N_m} 2\pi r_i^{1-\sigma} \cdot \Delta r_i
                \right]^{-1},
        \end{equation}
        which is all we need to formulate the radial profile of the gas surface density.
        % TODO Do this with dA instead of 4pi r dr ?

    % }}}
    % Radial Profile of the Gas Volume Density {{{ 
    \newpage\subsection{Radial Profile of the Gas Volume Density}
        
        Consider again a point $\vec r=(r,\phi,z)^T$ in the disk. Let $r = r_0$, which we defined 
        as $10 \ \text{au}$ in \cref{sec:first_assumptions_about_the_disk}. Also, this point shall 
        be situated sufficiently close to the midplane as to allow us to write
        \begin{equation}
            \label{eq:r_much_bigger_than_z}
            r\gg z 
        \end{equation}

        The acceleration due to Newtonian gravity exerted by the star at the disk center 
        on a massive object at position $\vec r$ is given by
        \begin{equation}
            \vec g = -\frac{GM_*}{r^2+z^2} \cdot \frac{\vec r}{|\vec r|}
        \end{equation}

        Now consider the schematic drawing in
        \cref{fig:trigonometry_schematic_for_gas_volume_density_plot}.
        Making use of basic trigonometry, we can write
        \begin{equation}
            g_z = |\vec g| \cdot \sin\theta
        \end{equation}
        and
        \begin{equation}
            \sin\theta=\frac{z}{\sqrt{r^2+z^2}}
        \end{equation}

        As already mentioned in \cref{sec:first_assumptions_about_the_disk}, the gas in the 
        proto-planetary disk is assumed to be in perfect hydrostatic equilibrium, which lets us 
        make use of \cref{eq:hydrostatic_equilibrium}. If we then plug in the trigonometric 
        relations from above, we can write
        \begin{align}
            \deriv{P_g}{z}
                &= -\rho_g \cdot g_z \\
                &= -\rho_g \cdot |\vec g| \cdot \sin\theta \\
                &= -\rho_g \cdot \frac{GM_*}{r^2+z^2} \cdot\frac{z}{\sqrt{r^2+z^2}} \\
                &= -\rho_g \cdot z \cdot\frac{GM_*}{ (r^2+z^2)^{\frac{3}{2}} }
                \label{eq:hydrostatic_equilibrium_with_plugged_in_trigonmetry}
        \end{align}

        Let us now make use of \cref{eq:r_much_bigger_than_z}, which allows us to make 
        the approximation
        \begin{equation}
            \frac{GM_*}{(r^2+z^2)^{3/2}} \approx \frac{GM_*}{r^3} = \Omega_K^2
        \end{equation}
        and thus, together with \cref{eq:hydrostatic_equilibrium_with_plugged_in_trigonmetry},
        lets us write
        \begin{equation}
            \deriv{P_g}{z} \approx -\rho_g \cdot z \cdot \Omega_K^2
        \end{equation}

        \vfill

\begin{figure}[h!]
    \centering
    \begin{tikzpicture}
        % Draw triangle.
        \draw (0,0) -- (12,0) -- (12, 3) -- cycle;
        % Draw angle.
        \draw (0, 0) ++ (0:5) arc (0:14:5);
        % Draw star and massive object.
        \node at (-0.2,  0) {$\star$};
        \node at (12.1, 3)    {$\cdot$};
        % Draw text.
        \node at (12.5,  1.5) {$z$};
        \node at (6,    -0.5) {$r$};
        \node at (6,     2.1) {$|\vec r|$};
        \node at (3,     0.4) {$\theta$};
        \node at (12.5,  3.1)   {$\vec r$};
    \end{tikzpicture}
    \caption{Illustration of Trigonometric relations used for expressing the $z$-component of the
        acceleration due to the gravitational influence of the star at the center of the disk.
    }
    \label{fig:trigonometry_schematic_for_gas_volume_density_plot}
\end{figure}

\clearpage


        The second assumption we made about the gas in \cref{sec:first_assumptions_about_the_disk}
        was that its behavior can be modeled as perfectly isothermal. If we differentiate the 
        isothermal condition from 
        \cref{eq:isothermal_condition} with respect to $z$, we get
        \begin{equation}
            \deriv{P_g}{z} = \deriv{\rho_g}{z}\cdot c_s^2
        \end{equation}

        Thus, we now have two expressions for $\text dP/\text dz$. Equating both of these and then 
        solving for $\text d\rho_g/\text dz$ leads to the following differential equation:
        % TODO Is it an ODE or a PDE ? Should I write `dy/dx` here, or `del y / del x` ?
        \begin{equation}
            \boxed{
                \deriv{\rho_g}{z}(r,z) 
                = -\left(\frac{\Omega_K}{c_s}\right)^2 \cdot \rho_g(r,z) \cdot z
            }
        \end{equation}

        This differential equation can be solved by making the ansatz of a Gaussian distribution
        \begin{equation}
            \boxed{
                \rho_g(r,z) = \rho_g^\text{mid}\cdot\exp\left(-\frac{z^2}{2\cdot H_p^2}\right)
            }
        \end{equation}
        where
        \begin{equation}
            H_p(r) :=\frac{c_s}{\Omega_K}  % TODO Define Kepler frequency.
        \end{equation}
        is the so-called \textit{pressure scale height} of the disk
        % TODO socalled vs so-called
        and
        \begin{equation}
            \rho_g^\text{mid}(r) := \rho_g(r,z=0)
        \end{equation}
        is the gas volume density at the midplane of the disk.
        Now that we know the behavior of the gas volume density, we can easily derive it from 
        the gas surface density like this: % (\todo{Why can we do this?})
        \begin{equation}
            \label{eq:gas_volume_density_vs_distance_from_star}
            \boxed{\rho_g^\text{mid}(r) =\frac{\Sigma_g}{\sqrt{2\pi}\cdot H_p}}
        \end{equation}

    % }}}
    % Radial Profile of the Gas Number Density {{{ 
    % \subsection{Radial Profile of the Gas Number Density}

        We can use this to express the number density, i.e. 
        the number of gas particles per unit volume, as
        % is given by the number density
        \begin{equation}
            N_g
                =\frac{\rho_g}{m_g}
        \end{equation}

        % \todo{Is this definition really even needed? (Don't I only need dust number density?)} \\
        % \todo{Plot gas number density $N_g(r_i)$ vs $r_i$.} \\  % `N_g -> N`

    % }}}
    % Radial Profile of Dust Volume Density {{{
    \subsection{Radial Profile of the Dust Volume Density}

        After having derived an expression for 
        the radial profile of 
        the gas volume density at 
        the disk midplane % TODO MIDPLANE!!!?
        in \cref{eq:gas_volume_density_vs_distance_from_star}, 
        we can use the definition of 
        the dust-to-gas ratio 
        in \cref{eq:dust-to-gas_ratio}
        to obtain 
        the \textit{dust volume density}, 
        i.e. the amount of dust mass per unit volume:
        \begin{equation}
            \rho_d(\vec r, t) = q_\text{dust-to-gas} \cdot \rho_g(\vec r, t)
        \end{equation}

    % }}}
    % Plots of Density Profiles {{{

        \clearpage
        % \begin{figure}[h!]
    \makebox[\textwidth]{
        \includegraphics[width=\paperwidth]{12/gas_surface_density.pdf}
    }
    \caption{Radial Profile of the Gas Surface Density $\Sigma_g$}
    \label{}
\end{figure}

\begin{figure}[h!]
    \makebox[\textwidth]{
        \includegraphics[width=\paperwidth]{12/gas_volume_density.pdf}
    }
    \caption{Radial Profile of the Gas Volume Density $\rho_g$}
    \label{}
\end{figure}


        \begin{figure}[h!]
            \makebox[\textwidth]{
                \includegraphics[width=\paperwidth]{108/2x2 gas density.pdf}
            }
            \caption{}
        \end{figure} 

    % }}}

% }}}
% Radial Profile of the Disk Scale Height {{{ 
% \clearpage\section{Radial Profile of the Disk Scale Height}

    % \todo{!!}
    % \vfill

\begin{figure}[h!]
    \centering
    \begin{tikzpicture}

        \def\R{0.25}  % This is the radius of the star.
        \def\D{0.75}  % This is the x-padding between star & disk/axis.
        \def\Y{0.5}   % This is the y-intersect.
        \def\MAX{7}   % This is the maximum value along the abscissa.
        \def\M{1/20}  % This factor controls the slope of the power law plots.
        \def\T{\D/3}  % This factor controls the size & y-padding of the abscissa ticks.

        % Draw the star.
        \draw (0, 0) circle (\R);

        % Draw the 4 power-law components for the "disk".
        \draw[domain= \D  : \MAX,smooth,variable=\x,black] plot ({\x},{ \x^2*\M+\Y});  % top-right
        \draw[domain=-\MAX:-\D,  smooth,variable=\x,black] plot ({\x},{-\x^2*\M+\Y});  % top-left
        \draw[domain= \MAX: \D,  smooth,variable=\x,black] plot ({\x},{-\x^2*\M-\Y});  % bot-right
        \draw[domain=-\D  :-\MAX,smooth,variable=\x,black] plot ({\x},{ \x^2*\M-\Y});  % bot-left

        % Draw the abscissa, i.e. the radial axis.
        \draw[->] (\D,   0)     -- (1.1*\MAX, 0) node[below] {$r$};
        % Draw the ticks & labels.
        \draw     (\D,   -\T/3) -- (\D,       \T/3);  % left tick
        \node at  (\D,   -\T  ) {$r_\text{min}$};     % left label
        \draw     (\MAX, -\T/3) -- (\MAX,     \T/3);  % right tick
        \node at  (\MAX, -\T  ) {$r_\text{max}$};     % right label

    \end{tikzpicture}
    \caption{Schematic View of the Disk Structure's Dependence on Radial Distance from the Star}
    \label{}
\end{figure} \ \\ 

% \begin{figure}[h!]
%     \centering
% \begin{center}
%     \begin{tikzpicture}
%         \draw (0, 0) circle (0.5cm);

%         % \foreach \x in {0.5, 1, 1.5} {
%         %     \draw (\x, 0) -- (\x, {1/\x^2});
%         %     \draw (-\x, 0) -- (-\x, {1/\x^2});
%         % }


%          % Draw the axes
%         % \draw[->] (-3, 0) -- (3, 0) node[right] {$x$};
%         % \draw[->] (0, -1) -- (0, 3) node[above] {$y$};
    
%         % Draw the quadratic sloped line
%     \end{tikzpicture}
%     \caption{}
% \end{center}
    % \label{}
% \end{figure} \ \\ 


% }}}
% Radial Profile of the Gas Pressure {{{ 
% \clearpage\section{Radial Profile of the Gas Pressure}

    % The gas pressure is given by: [\todo{Cite}]
    % \begin{equation}
    %     P_g = \rho_g \cdot c_s^2
    % \end{equation}

% }}}
% Gas Particle Kinematics {{{ 
\clearpage\section{Gas Particle Kinematics}

    % \todo{There are both stochastic and systematic contributions to the gas kinematics.}

    % Gas Viscosity {{{ 
    \subsection{Gas Viscosity}

        The kinematic viscosity gas viscosity can be expressed as [\todo{cite}]
        \begin{equation}
            \nu_g = \frac{1}{2} \cdot u_\text{th} \cdot \lambda_\text{mfp}
        \end{equation}

        We define the absolute value of the thermal velocity to be the mean of the magnitude
        of the molecular velocity. It is given by [\todo{cite}]
        \begin{equation}
            u_\text{th} = \sqrt{\frac{8}{\pi}} \cdot c_s
        \end{equation}

        A particle's mean free path in a given medium is given by the inverse product between the 
        number of particles per unit volume and the cross section for a collision:
        \begin{equation}
            \lambda_\text{mfp} = \frac{1}{n \cdot \sigma_{\text{H}_2}}
        \end{equation}
        Here, we follow \cite{birnstiel_dullemond_brauer_2010} and use 
        $\sigma_{\text{H}_2} = \SI{2e-19}{\meter^2}$ for the collision cross section. \\

        % \vfill

\todo{Why so small?}

\begin{figure}[h!]
    \makebox[\textwidth]{
        \includegraphics[width=\paperwidth]{12/thermal_gas_velocity.pdf}
    }
    \caption{Thermal Gas Velocity $u_th(r)$}
    \label{}
\end{figure}


    % }}}
    % Radial Gas Velocity {{{ 
    \subsection{Radial Gas Velocity}

        % \todo{Plot radial profile of "speed" (thermal + ..?)} \\

        The viscous evolution of the gas disk can be described by the continuity equation 
        \cite{birnstiel_dullemond_brauer_2010}:
        \begin{equation}
            \pderiv{\Sigma_g}{t}-\frac{1}{r}\pderiv{}{r}\bigg(\Sigma_g \cdot r \cdot u_g\bigg) = S_g
        \end{equation}
        Here, $u_g$ labels the \textit{radial gas velocity}.
        \\
        % TODO: What is $S_g$
        % TODO: Does $u_g$ label radial velocity or speed?
        % TODO: Choose either $uor and $v$ as variable names for velocities.
        %       Be consistent across the entire thesis.

        A solution to this equation is given by \cite{lynden-bell_pringle_1974}, where the 
        radial velocity of the gas is given by
        \begin{equation}
            \label{eq:radial_gas_velocity}
            u_{g}
            =-\frac{3}{\Sigma_g\cdot\sqrt{r}}
                \cdot\pderiv{}{r}\bigg(\Sigma_g \cdot \nu_g \cdot \sqrt{r}\bigg)
        \end{equation}
        % TODO: Have I already defined the viscosity somewhere? Else define here.

        % \vfill

\begin{figure}[h!]
    \makebox[\textwidth]{
        \includegraphics[width=\paperwidth]{12/radial_gas_velocity.pdf}
    }
    \caption{Radial Gas Velocity $u_g(r)$}
    \label{}
\end{figure}


    % }}}
% }}}

% \section{Symmetry Properties of the Disk}  
%     \todo{Rename section? $\to$ "Disk Structure" ?} \\
%     \todo{Plot disk structure (flaring disk, flaring index)} \\
%     \todo{What exactly is on the ordinate? ($r$ on the abscissa)} \\
%     \assume{Disk is radially symmetric: $f(r,\varphi,z)=f(r,z)$ (cylindrical coordinates)} \\
