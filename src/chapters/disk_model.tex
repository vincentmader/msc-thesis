\section{The Star at the Center of the Disk}

    In our model, the star's most relevant properties will be its mass $M_*$, as well as its
    luminosity $L_*$. Both of these properties will be assumed to equal those of our own solar
    system's Sun. Thus, we define
    \begin{align}
        M_*&:=M_\odot=\SI{1.989e30}{\kg},\ \text{and} \\
        L_*&:=L_\odot\ =\SI{3.828e26}{\joule\second^{-1}},
    \end{align}
    where $M_\odot$ and $L_\odot$ label the solar mass and luminosity, respectively. \\

    \todo{What type of star is it?}
    \todo{Where in the main sequence is it?}
    \todo{How old is the star?}

\section{Distribution of Mass in the Disk}

   Most of the original gas cloud's mass is assumed to be concentrated inside the central star,
   while the disk surrounding it holds only a much smaller fraction of the total mass. \\ 

   We label the stellar mass $M_*$, and the disk mass $M_\text{disk}$. 
   The ratio between the two shall be defined as
   \begin{equation}
       q_\text{disk-to-star}
           =\frac{M_\text{disk}}{M_*}  
           :=0.01
   \end{equation}

\section{Material Composition of the Disk}

    Furthermore, the disk's mass is assumed to be dominated by the contribution of circumstellar 
    gas, with only a comparatively tiny contribution of dust particles. We define the
    dust-to-gas ratio as 
    \begin{equation}
        q_\text{dust-to-gas}=\frac{M_\text{dust}}{M_\text{gas}}:=0.01
    \end{equation}

\section{Radial Midplane Temperature Profile}
