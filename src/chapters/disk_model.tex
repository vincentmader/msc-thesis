Our first task will be the definition of a model for the proto-planetary disk, 
as well as its contents. \\

Here, it has to be said that due to the complexity of the involved processes and the (literally)
astronomically large number of particles present in the disk, our model can only ever be a highly
simplified approximation to reality. \\

Nonetheless, the goal of the following chapters will be the formulation of a model that captures
a typical disk's most relevant properties, and allows us to conduct studies of the processes
involved in dust particle coagulation and/or fragmentation. \\

\assume{Start with gas cloud: Composition like interstellar medium} \\
\assume{Then: Disk collapses, star forms at center (how long ago?)}

\section{The Star at the Center of the Disk}

    In our model, the star's most relevant properties will be its mass $M_*$, as well as its
    luminosity $L_*$. \\

    \assume{Star is sun-like: Both of these properties will be assumed to equal those of our own
    solar system's Sun.} Thus, we define
    \begin{align}
        M_*&:=M_\odot=\SI{1.989e30}{\kg},\ \text{and} \\
        L_*&:=L_\odot\ =\SI{3.828e26}{\joule\second^{-1}},
    \end{align}
    where $M_\odot$ and $L_\odot$ label the solar mass and luminosity, respectively. \\

    \todo{What type of star is it? Where in the main sequence is it? How old is it?} 
    \begin{itemize}
        \item G-type main sequence star
        \item often imprecisely called "yellow dwarf" 
              (imprecisely, because its light is actually white)
    \end{itemize}
    \todo{The Sun was not as it is today when it formed $\approx$ 4.6 billion years ago.} \\

    \todo{As "mentioned" above, we assume the star to have formed as a result of the gravitational
    collapse of a large molecular cloud.} \\

\section{Mass Distribution and Material Composition}

   Most of the original gas cloud's mass is assumed to be concentrated inside the central star,
   while the disk surrounding it holds only a much smaller fraction of the total mass. \\ 

   We label the stellar mass $M_*$, and the disk mass $M_\text{disk}$. 
   The ratio between the two shall be defined as
   \begin{equation}
       q_\text{disk-to-star}
           =\frac{M_\text{disk}}{M_*}  
           :=0.01
   \end{equation}

    \assume{Furthermore, the disk's mass is assumed to be dominated by the contribution of
    circumstellar gas, with only a comparatively tiny contribution of dust particles.}
    Let $M_\text{dust}$ and $M_\text{gas}$ label the total mass present in the form of dust and 
    gas, respectively. Then, we define the dust-to-gas ratio as 
    \begin{equation}
        q_\text{dust-to-gas}=\frac{M_\text{dust}}{M_\text{gas}}:=0.01
    \end{equation}

\section{Symmetry Properties of the Disk}  

    \todo{Rename section? $\to$ "Disk Structure" ?} \\

    \todo{Consider a disk with an inner and outer disk radius
        $r_\text{min}=\SI{0.01}{\astronomicalunit}$ 
    and 
        $r_\text{max}=\SI{100}{\astronomicalunit}$,
    respectively.} \\

    \todo{Plot disk structure (flaring disk, flaring index)} \\
    \todo{What exactly is on the ordinate? ($r$ on the abscissa)} \\

    \assume{Disk is radially symmetric: $f(r,\varphi,z)=f(r,z)$ (cylindrical coordinates)} \\

    \ \\ % TODO Add padding to diagram.
    \vfill

\begin{figure}[h!]
    \centering
    \begin{tikzpicture}

        \def\R{0.25}  % This is the radius of the star.
        \def\D{0.75}  % This is the x-padding between star & disk/axis.
        \def\Y{0.5}   % This is the y-intersect.
        \def\MAX{7}   % This is the maximum value along the abscissa.
        \def\M{1/20}  % This factor controls the slope of the power law plots.
        \def\T{\D/3}  % This factor controls the size & y-padding of the abscissa ticks.

        % Draw the star.
        \draw (0, 0) circle (\R);

        % Draw the 4 power-law components for the "disk".
        \draw[domain= \D  : \MAX,smooth,variable=\x,black] plot ({\x},{ \x^2*\M+\Y});  % top-right
        \draw[domain=-\MAX:-\D,  smooth,variable=\x,black] plot ({\x},{-\x^2*\M+\Y});  % top-left
        \draw[domain= \MAX: \D,  smooth,variable=\x,black] plot ({\x},{-\x^2*\M-\Y});  % bot-right
        \draw[domain=-\D  :-\MAX,smooth,variable=\x,black] plot ({\x},{ \x^2*\M-\Y});  % bot-left

        % Draw the abscissa, i.e. the radial axis.
        \draw[->] (\D,   0)     -- (1.1*\MAX, 0) node[below] {$r$};
        % Draw the ticks & labels.
        \draw     (\D,   -\T/3) -- (\D,       \T/3);  % left tick
        \node at  (\D,   -\T  ) {$r_\text{min}$};     % left label
        \draw     (\MAX, -\T/3) -- (\MAX,     \T/3);  % right tick
        \node at  (\MAX, -\T  ) {$r_\text{max}$};     % right label

    \end{tikzpicture}
    \caption{Schematic View of the Disk Structure's Dependence on Radial Distance from the Star}
    \label{}
\end{figure} \ \\ 

% \begin{figure}[h!]
%     \centering
% \begin{center}
%     \begin{tikzpicture}
%         \draw (0, 0) circle (0.5cm);

%         % \foreach \x in {0.5, 1, 1.5} {
%         %     \draw (\x, 0) -- (\x, {1/\x^2});
%         %     \draw (-\x, 0) -- (-\x, {1/\x^2});
%         % }


%          % Draw the axes
%         % \draw[->] (-3, 0) -- (3, 0) node[right] {$x$};
%         % \draw[->] (0, -1) -- (0, 3) node[above] {$y$};
    
%         % Draw the quadratic sloped line
%     \end{tikzpicture}
%     \caption{}
% \end{center}
    % \label{}
% \end{figure} \ \\ 


\section{Radial Midplane Temperature Profile}

    To keep things simple, let us make the assumption that the gas and dust particles 
    in the disk behave like perfect black-body objects. As such, the relationship between the
    thermal radiation flux $B$ coming from the star (i.e. the total energy per unit surface area 
    per unit time) and the temperature $T$ is given by the Stefan-Boltzmann law, which states that
    \begin{equation}
        \label{eq:stefan_boltzmann_law}
        B(T) = \sigma_\text{SB}\cdot T^4   % TODO Is this true?
    \end{equation}

    Here, $\sigma_\text{SB}$ labels the Stefan-Boltzmann constant, which is given by 
    \begin{equation}
        \sigma_\text{SB}
            = \frac{2\pi k_B^4}{15h^3 c^2}
            \approx \SI{5.670e-8}{\ \watt\ \kelvin^{-1}\ \meter^{-2}}
    \end{equation}

    In general, the relationship between the luminosity $L$, which labels the total emitted energy 
    per time, and the radiation flux $B$ is given by the surface integral
    % The luminosity, i.e. the total energy emitted per time, can be determined by evaluating the
    % integral of the radiation flux over the entire surface of the considered absorver/emitter.
    % Thus, the relationship between radiation flux $B(T)$ and luminosity $L(T)$ is given by
    \begin{equation}
        L(T) = \int_A B(T) \ \text dA
    \end{equation}
    
    \assume{
    Since we are assuming spherical symmetry (isotropy) of the star's radiative power output, we 
    can simplify this expression. For a fixed stellar luminosity $L_*$, the radiative flux $B(r)$ 
    at a distance $r$ from the star can be written as}
    \begin{equation}
        L_* = B(r) \cdot 4\pi r^2
    \end{equation}

    Plugging in \cref{eq:stefan_boltzmann_law} and rearranging for the temperature $T$ leads to
    \begin{equation}
        T(r) = \left(
            \frac{L_*}{4\pi r^2 \cdot \sigma_\text{SB}}
        \right)^{1/4}
    \end{equation}

    \todo{Inclusion of $f/2$:}
    \begin{equation}
    \boxed{
        T(r) = \left(
            \frac{f}{2}\cdot
            \frac{L_*}{4\pi r^2 \cdot \sigma_\text{SB}}
        \right)^{1/4}
    }
    \end{equation}

    \vfill

\begin{figure}[h!]
    \makebox[\textwidth]{
        \includegraphics[width=\paperwidth]{12/midplane_temperature.pdf}
    }
    \caption{Radial profile of the midplane gas temperature $T_\text{mid}(r)$.}
    \label{fig:radial_profile_of_midplane_temperature}
\end{figure}


    \begin{itemize}
        \item midplane temperature $T_\text{mid}(r)$ (b.c. isothermal) \\
            \todo{Plot $T_\text{mid}$ vs. $r$ (lin-log?)}
        \item thermal velocity $u_\text{th}$ \\
            \todo{Plot $u_\text{th}$ vs. $r$ (lin-log?)}
        \item sound speed (here?)
    \end{itemize}
