Our first task will be the definition of a model for the proto-planetary disk, as well as for
the properties of its gas and dust contents. It needs to be said here that due to the complexity of
the involved processes and the (literally) astronomically large number of particles present in the 
disk, our model can only ever be a highly simplified approximation to reality. \\

Nonetheless, the goal of the following chapter(s) (?) will be to formulate a model that captures a 
typical PPD's most relevant properties, which would then allow us to conduct studies of the 
processes involved in dust particle coagulation and/or fragmentation in the subsequent chapters. 
To be able to construct such a framework, it will be necessary to make a couple of assumptions:

% \todo{Do what? (Star, Disk, Gas, Dust, Coagulation/Fragmentation)} \\

% First Assumptions {{{
\section{First Assumptions}
\label{sec:first_assumptions_about_the_disk}

    \begin{enumerate}
        \item \assume{} First of all, we will base our model on the aforementioned nebular 
            hypothesis: The origin of the modeled proto-planetary disk is assumed to be found in 
            the collapse of a giant interstellar cloud. % consisting of mostly gas and dust. 
            As such, the disk's elementary and material composition is assumed to equal that of the 
            interstellar medium as well. \\
            \todo{Elements?}
    
        \item \assume{} Also, we will assume that the cloud collapse happened sufficiently long ago 
            that a star could already form at the center of the disk (while not long enough for 
            planet formation to occur).
            % (\todo{How long ago?}) [cite subsection]
            For the studies that will be carried out in the context of this thesis, the star's most
            relevant properties will be its mass $M_*$ as well as its luminosity $L_*$. 
            \assume{}
            Both of these properties will be assumed to equal those of our own solar system's Sun. 
            Thus, we define
            \begin{align}
                M_*&:=M_\odot=\SI{1.989e30}{\kg},\ \text{and} \\
                L_*&:=L_\odot\ =\SI{3.828e26}{\joule\second^{-1}},
            \end{align}
            where $M_\odot$ and $L_\odot$ label the solar mass and luminosity, respectively.
            % \todo{What type of star is it? Where in the main sequence is it? How old is it?} 
            % \begin{itemize}
            %     \item G-type main sequence star
            %     \item often imprecisely called "yellow dwarf" 
            %           (imprecisely, because its light is actually white)
            % \end{itemize}
            % \todo{The Sun was not as it is today when it formed $\approx$ 4.6 billion years ago.} \\
    
        \item
            We will assume most of the original gas cloud's mass to be concentrated inside this 
            star, while the disk surrounding it holds only a much smaller fraction of the total 
            mass. We label the stellar mass $M_*$, and the disk mass $M_\text{disk}$. 
            The ratio between the two shall be defined as
            \begin{equation}
                q_\text{disk-to-star}
                    =\frac{M_\text{disk}}{M_*}  
                    :=0.01
            \end{equation}
    
        \item \assume{Furthermore, the disk's mass is assumed to be dominated by the contribution 
            of circumstellar gas, with only a comparatively tiny contribution of dust particles.}
            Let $M_\text{dust}$ and $M_\text{gas}$ label the total mass present in the form of dust 
            and gas, respectively. 
            With
            \begin{equation}
                M_\text{disk} = M_\text{gas} + M_\text{disk}
            \end{equation}
            we define the dust-to-gas ratio as
            \begin{equation}
                q_\text{dust-to-gas}=\frac{M_\text{dust}}{M_\text{gas}}:=0.01
            \end{equation}
    
        \item The next assumption we will make concerns itself with disk symmetry. For this, let us 
            adopt cylindrical coordinates $(r, \varphi, z)$. In reality, each of the disk's 
            position-dependent properties could best be described by a function $f(r, \varphi, z)$ 
            that depends on all three of these coordinates. \\
            To keep things simple though, we will not attempt to resolve the disk in all three 
            spatial dimensions. Instead, we will focus mostly on the radial dependence of the 
            disk's features.
    
            On large scales, we will assume the disk's features to not depend significantly on 
            $\varphi$ and $z$.
    
            \todo{Consider a disk with an inner and outer disk radius
                $r_\text{min}=\SI{0.01}{\astronomicalunit}$ 
            and 
                $r_\text{max}=\SI{100}{\astronomicalunit}$,
            respectively.} \\
            \todo{For numerical treatment, discretize radial axis into $\mathcal N_r$ bins.}

            \todo{Assume perfect radial symmetry?}
    
        \item \assume{Gas mean particle mass (weight?).}
            \begin{equation}
                m_g := \mu\cdot m_p
                \ \ \ \text{with}\ \ \
                \mu := 2.3
            \end{equation}
            Here, $m_p$ labels the mass of a single proton, and $\mu$ the mean molecular weight of 
            the gas particles.
            % TODO Rewrite

        \item \assume{ideal gas}
    
        \item \assume{Gas in the disk is in hydrostatic equilibrium.} \\
            Let $P_g(r,z)$ label the gas pressure and $\rho_g(r,z)$ the gas volume density 
            at a position $\vec r=(r,\varphi,z)^T$. (\todo{symmetry, cyl. coords.})
            \begin{equation}
                \label{eq:hydrostatic_equilibrium}
                \deriv{P_g}{z} = - \rho_g \cdot g_z  % TODO Why `g_z` and not `g` ?
            \end{equation}
    
        \item \assume{Gas behaves isothermically.}
            \begin{equation}
                \label{eq:isothermal_condition}
                P_g = \rho_g \cdot c_s^2
            \end{equation}

        \item \assume{Position in the disk: $r=?$}
    
    \end{enumerate}
    
    \todo{Plot top-down view of the disk.}
    
    \vfill

\begin{figure}[h!]
    \centering
    \begin{tikzpicture}

        \def\R{0.25}  % This is the radius of the star.
        \def\D{0.75}  % This is the x-padding between star & disk/axis.
        \def\Y{0.5}   % This is the y-intersect.
        \def\MAX{7}   % This is the maximum value along the abscissa.
        \def\M{1/20}  % This factor controls the slope of the power law plots.
        \def\T{\D/3}  % This factor controls the size & y-padding of the abscissa ticks.

        % Draw the star.
        \draw (0, 0) circle (\R);

        % Draw the 4 power-law components for the "disk".
        \draw[domain= \D  : \MAX,smooth,variable=\x,black] plot ({\x},{ \x^2*\M+\Y});  % top-right
        \draw[domain=-\MAX:-\D,  smooth,variable=\x,black] plot ({\x},{-\x^2*\M+\Y});  % top-left
        \draw[domain= \MAX: \D,  smooth,variable=\x,black] plot ({\x},{-\x^2*\M-\Y});  % bot-right
        \draw[domain=-\D  :-\MAX,smooth,variable=\x,black] plot ({\x},{ \x^2*\M-\Y});  % bot-left

        % Draw the abscissa, i.e. the radial axis.
        \draw[->] (\D,   0)     -- (1.1*\MAX, 0) node[below] {$r$};
        % Draw the ticks & labels.
        \draw     (\D,   -\T/3) -- (\D,       \T/3);  % left tick
        \node at  (\D,   -\T  ) {$r_\text{min}$};     % left label
        \draw     (\MAX, -\T/3) -- (\MAX,     \T/3);  % right tick
        \node at  (\MAX, -\T  ) {$r_\text{max}$};     % right label

    \end{tikzpicture}
    \caption{Schematic View of the Disk Structure's Dependence on Radial Distance from the Star}
    \label{}
\end{figure} \ \\ 

% \begin{figure}[h!]
%     \centering
% \begin{center}
%     \begin{tikzpicture}
%         \draw (0, 0) circle (0.5cm);

%         % \foreach \x in {0.5, 1, 1.5} {
%         %     \draw (\x, 0) -- (\x, {1/\x^2});
%         %     \draw (-\x, 0) -- (-\x, {1/\x^2});
%         % }


%          % Draw the axes
%         % \draw[->] (-3, 0) -- (3, 0) node[right] {$x$};
%         % \draw[->] (0, -1) -- (0, 3) node[above] {$y$};
    
%         % Draw the quadratic sloped line
%     \end{tikzpicture}
%     \caption{}
% \end{center}
    % \label{}
% \end{figure} \ \\ 


% }}}
% Radial Profile of the Midplane Temperature {{{ 
\clearpage\section{Radial Profile of the Midplane Temperature}

    \assume{To keep things simple, let us make the assumption that the gas and dust particles 
    in the disk behave like perfect black-body objects.} As such, the relationship between the
    thermal radiation flux $B$ coming from the star (i.e. the total energy per unit surface area 
    per unit time) and the temperature $T$ is given by the Stefan-Boltzmann law, which states that
    \begin{equation}
        \label{eq:stefan_boltzmann_law}
        B(T) = \sigma_\text{SB}\cdot T^4   % TODO Is this true?
    \end{equation}

    Here, $\sigma_\text{SB}$ labels the Stefan-Boltzmann constant, which is given by 
    \begin{equation}
        \sigma_\text{SB}
            = \frac{2\pi k_B^4}{15h^3 c^2}
            \approx \SI{5.670e-8}{\ \watt\ \kelvin^{-1}\ \meter^{-2}}
    \end{equation}

    In general, the relationship between the luminosity $L$, which labels the total emitted energy 
    per time, and the radiation flux $B$ is given by the surface integral
    % The luminosity, i.e. the total energy emitted per time, can be determined by evaluating the
    % integral of the radiation flux over the entire surface of the considered absorver/emitter.
    % Thus, the relationship between radiation flux $B(T)$ and luminosity $L(T)$ is given by
    \begin{equation}
        L(T) = \int_A B(T) \ \text dA
    \end{equation}

    % \vfill

\begin{figure}[h!]
    \centering
    \begin{tikzpicture}
        \draw (0,0) -- (12,0) -- (12, 3) -- cycle;
        % TODO
    \end{tikzpicture}
    \caption{}
    \label{}
\end{figure}

\clearpage
  % TODO

    % \begin{equation}
    %     L_* = B_*(r=R_*) \cdot 4\pi R_*^2
    % \end{equation}
    
    \assume{
    Since we are assuming spherical symmetry (isotropy) of the star's radiative power output, we 
    can simplify this expression. For a fixed stellar luminosity $L_*$, the radiative flux $B(r)$ 
    at a distance $r$ from the star can be written as}
    \begin{equation}
        L_* = B(r) \cdot 4\pi r^2
    \end{equation}

    Plugging in \cref{eq:stefan_boltzmann_law} and rearranging for the temperature $T$ leads to
    \begin{equation}
        T(r) = \left(
            \frac{L_*}{4\pi r^2 \cdot \sigma_\text{SB}}
        \right)^{1/4}
    \end{equation}

    \todo{Inclusion of $f/2$:}
    \begin{equation}
    \boxed{
        T(r) = \left(
            \frac{f}{2}\cdot
            \frac{L_*}{4\pi r^2 \cdot \sigma_\text{SB}}
        \right)^{1/4}
    }
    \end{equation}

    \vfill

\begin{figure}[h!]
    \makebox[\textwidth]{
        \includegraphics[width=\paperwidth]{12/midplane_temperature.pdf}
    }
    \caption{Radial profile of the midplane gas temperature $T_\text{mid}(r)$.}
    \label{fig:radial_profile_of_midplane_temperature}
\end{figure}


    % \begin{itemize}
    %     \item midplane temperature $T_\text{mid}(r)$ (b.c. isothermal) \\
    %         \todo{Plot $T_\text{mid}$ vs. $r$ (lin-log?)}
    %     \item thermal velocity $u_\text{th}$ \\
    %         \todo{Plot $u_\text{th}$ vs. $r$ (lin-log?)}
    %     \item sound speed (here?)
    % \end{itemize}

% }}}
% Radial Profile of the Gas Density {{{ 
\clearpage\section{Radial Profile of the Gas Density}

    Our next task will be to formulate a simple model for the spatial distribution of the disk's 
    gaseous contents. In this context, we will adopt cylindrical coordinates 
    $\vec r=(r, \varphi, z)^T$ and define both the surface and volume density profiles along the 
    radial axis $r$.

    % Radial Profile of the Gas Surface Density {{{ 
    \subsection{Radial Profile of the Gas Surface Density}

        The gas surface density $\Sigma_g(r,\varphi)$ is directly related to the gas volume density
        $\rho_g(r,\varphi,z)$, and can be derived from it by integrating the volume density from 
        $-\infty$ to $+\infty$ along the $z$-axis, i.e. along the axis perpendicular to the disk:
        \begin{equation}
            \label{eq:relationship_between_gas_surface_density_and_gas_volume_density}
            \Sigma_g(r,\varphi)
                = \int\limits_{-\infty}^{\infty} \rho_g(r,\varphi,z)\ \dd z
        \end{equation}

        \assume{As noted before in (cite), we will assume the disk to be perfectly radially 
        symmetric.} (\todo{At least macroscopically!})
        For the density distributions, this means that
        \begin{align}
            \Sigma_g(r, \varphi) 
                &= \Sigma_g(r) 
            \ \ \ \text{and}\ \ \
            \\
            \rho_g(r, \varphi) 
                &= \rho_g(r)
        \end{align}
        Thus, we can simplify 
        \cref{eq:relationship_between_gas_surface_density_and_gas_volume_density} and write:
        \begin{equation}
            \Sigma_g(r)
                = \int\limits_{-\infty}^{\infty} \rho_g(r,z)\ \dd z
        \end{equation}
        
        But: We do not yet have an expression for the gas volume density $\rho_g$ so far.
        Therefore, we cannot derive the surface density $\Sigma_g$ by simply evaluating the above
        integral. Instead, we will go the other way round and make an ansatz for the gas surface
        density. Then we will be able to derive the volume density from that. \\

        In order to do this, we will 
        follow the work of Brauer, Dullemond, \& Henning \cite{brauer_dullemond_henning_2007}
        % Following the works of \cite{brauer_dullemond_henning_2007}
        % , \cite{andrews_2007}, and \cite{kitamura_2002}, 
        and make the assumption that the radial profile of the gas surface density can be 
        approximated by using an inverse power law of the form
        \begin{equation}
            \label{eq:gas_surface_density_profile_inverse_power_law_ansatz}
            \Sigma_g(r) = \Sigma_0\cdot\frac{1}{r^\sigma}
            \ \ \ \text{with}\ \ \
            \sigma := 0.8
        \end{equation}  % TODO Why 0.8 ?
        Before we can use this though, we still need to define $\Sigma_0$, which we will do now: \\

        The total gas mass present in the disk is given by $M_\text{gas}$, the numerical value of 
        which can easily be derived from what we defined in
        \cref{sec:first_assumptions_about_the_disk}. Since the integration of the surface density 
        over the entire disk surface must yield this value, we can formulate the following 
        condition:
        \begin{align}
            M_\text{gas}
                % &= \int_{A_\text{disk}} \Sigma_g(r) \ \dd A \\
                &= \int\limits_{r_\text{min}}^{r_\text{max}} \Sigma_g(r) \cdot 2\pi r \ \dd r
        \end{align}
        % TODO Define r_min and r_max before.

        Since we discretized the radial axis into $\mathcal N_r$ "bins", we can rewrite this as
        \begin{equation}
            M_\text{gas} = \sum_{i=1}^{\mathcal N_r} \Sigma_g(r_i) \cdot 2\pi r_i \cdot \Delta r_i
        \end{equation}
        % TODO Define N_r before.

        Now plug in \cref{eq:gas_surface_density_profile_inverse_power_law_ansatz} and 
        rearrange for $\Sigma_0$ to arrive at
        \begin{equation}
            \Sigma_0
                = M_\text{gas} \cdot \left[
                    \sum_{i=1}^{\mathcal N_m} 2\pi r_i^{1-\sigma} \cdot \Delta r_i
                \right]^{-1},
        \end{equation}
        which is all we need to formulate the radial profile of the gas surface density.
        % TODO Do this with dA instead of 4pi r dr ?

    % }}}
    % Radial Profile of the Gas Volume Density {{{ 
    \newpage\subsection{Radial Profile of the Gas Volume Density}
        
        Consider again a point $\vec r=(r,\phi,z)^T$ in the disk. (\todo{Rewrite:)} Since the disk
        is much more wide than tall (it is a disk, after all), we can make the assumption that
        % TODO Be more specific: We focus on a point where that makes sense.
        \begin{equation}
            \label{eq:r_much_bigger_than_z}
            r\gg z 
        \end{equation}

        The acceleration due to Newtonian gravity exerted by the star at the disk center 
        on a massive object at position $\vec r$ is given by
        \begin{equation}
            \vec g = -\frac{GM_*}{r^2+z^2} \cdot \frac{\vec r}{|\vec r|}
        \end{equation}

        Now consider the schematic drawing in
        \cref{fig:trigonometry_schematic_for_gas_volume_density_plot}.
        Making use of basic trigonometry, we can write
        \begin{equation}
            g_z = |\vec g| \cdot \sin\theta
        \end{equation}
        and
        \begin{equation}
            \sin\theta=\frac{z}{\sqrt{r^2+z^2}}
        \end{equation}

        As already mentioned in \cref{sec:first_assumptions_about_the_disk}, the gas in the 
        proto-planetary disk is assumed to be in perfect hydrostatic equilibrium, which lets us 
        make use of \cref{eq:hydrostatic_equilibrium}. If we then plug in the trigonometric 
        relations from above, we can write
        \begin{align}
            \deriv{P_g}{z}
                &= -\rho_g \cdot g_z \\
                &= -\rho_g \cdot |\vec g| \cdot \sin\theta \\
                &= -\rho_g \cdot \frac{GM_*}{r^2+z^2} \cdot\frac{z}{\sqrt{r^2+z^2}} \\
                &= -\rho_g \cdot z \cdot\frac{GM_*}{ (r^2+z^2)^{\frac{3}{2}} }
                \label{eq:hydrostatic_equilibrium_with_plugged_in_trigonmetry}
        \end{align}

        Let us now make use of \cref{eq:r_much_bigger_than_z}, which allows us to make 
        the approximation
        \begin{equation}
            \frac{GM_*}{(r^2+z^2)^{3/2}} \approx \frac{GM_*}{r^3} = \Omega_K^2
        \end{equation}
        and thus, together with \cref{eq:hydrostatic_equilibrium_with_plugged_in_trigonmetry},
        lets us write
        \begin{equation}
            \deriv{P_g}{z} \approx -\rho_g \cdot z \cdot \Omega_K^2
        \end{equation}

        \vfill

\begin{figure}[h!]
    \centering
    \begin{tikzpicture}
        % Draw triangle.
        \draw (0,0) -- (12,0) -- (12, 3) -- cycle;
        % Draw angle.
        \draw (0, 0) ++ (0:5) arc (0:14:5);
        % Draw star and massive object.
        \node at (-0.2,  0) {$\star$};
        \node at (12.1, 3)    {$\cdot$};
        % Draw text.
        \node at (12.5,  1.5) {$z$};
        \node at (6,    -0.5) {$r$};
        \node at (6,     2.1) {$|\vec r|$};
        \node at (3,     0.4) {$\theta$};
        \node at (12.5,  3.1)   {$\vec r$};
    \end{tikzpicture}
    \caption{Illustration of Trigonometric relations used for expressing the $z$-component of the
        acceleration due to the gravitational influence of the star at the center of the disk.
    }
    \label{fig:trigonometry_schematic_for_gas_volume_density_plot}
\end{figure}

\clearpage


        The second assumption we made about the gas in \cref{sec:first_assumptions_about_the_disk}
        was that its behavior can be modeled as perfectly isothermal. If we differentiate the 
        isothermal condition from 
        \cref{eq:isothermal_condition} with respect to $z$, we get
        \begin{equation}
            \deriv{P_g}{z} = \deriv{\rho_g}{z}\cdot c_s^2
        \end{equation}

        Thus, we now have two expressions for $\text dP/\text dz$. Equating both of these and then 
        solving for $\text d\rho_g/\text dz$ leads to the following differential equation:
        % TODO Is it an ODE or a PDE ? Should I write `dy/dx` here, or `del y / del x` ?
        \begin{equation}
            \boxed{
                \deriv{\rho_g}{z}(r,z) 
                = -\left(\frac{\Omega_K}{c_s}\right)^2 \cdot \rho_g(r,z) \cdot z
            }
        \end{equation}

        This differential equation can be solved by making the ansatz of a Gaussian distribution
        \begin{equation}
            \boxed{
                \rho_g(r,z) = \rho_g^\text{mid}\cdot\exp\left(-\frac{z^2}{2\cdot H_p^2}\right)
            }
        \end{equation}
        where
        \begin{equation}
            H_p(r) :=\frac{c_s}{\Omega_K}  % TODO Define Kepler frequency.
        \end{equation}
        is the so-called \textit{pressure scale height} of the disk
        % TODO socalled vs so-called
        and
        \begin{equation}
            \rho_g^\text{mid}(r) := \rho_g(r,z=0)
        \end{equation}
        is the gas volume density at the midplane of the disk. \\

        Now that we know the behavior of the gas volume density, we can easily derive it from 
        the gas surface density like this: (\todo{Why can we do this?})
        \begin{equation}
            \boxed{\rho_g^\text{mid}(r) =\frac{\Sigma_g}{\sqrt{2\pi}\cdot H_p}}
        \end{equation}

    % }}}
    % Radial Profile of the Gas Number Density {{{ 
    \subsection{Radial Profile of the Gas Number Density}

        The number $N_g$ of gas particles per unit volume is given by the number density
        \begin{equation}
            N_g
                =\frac{\rho_g}{m_g}
        \end{equation}

        \todo{Is this definition really even needed? (Don't I only need dust number density?)} \\
        \todo{Plot gas number density $N_g(r_i)$ vs $r_i$.} \\  % `N_g -> N`

    % }}}
    % Plots of Density Profiles {{{

        \clearpage
        \begin{figure}[h!]
    \makebox[\textwidth]{
        \includegraphics[width=\paperwidth]{12/gas_surface_density.pdf}
    }
    \caption{Radial Profile of the Gas Surface Density $\Sigma_g$}
    \label{}
\end{figure}

\begin{figure}[h!]
    \makebox[\textwidth]{
        \includegraphics[width=\paperwidth]{12/gas_volume_density.pdf}
    }
    \caption{Radial Profile of the Gas Volume Density $\rho_g$}
    \label{}
\end{figure}

        \todo{Plot $\rho_g(z)$.}

    % }}}

% }}}
% Radial Profile of the Gas Pressure {{{ 
% \clearpage\section{Radial Profile of the Gas Pressure}

    % The gas pressure is given by: [\todo{Cite}]
    % \begin{equation}
    %     P_g = \rho_g \cdot c_s^2
    % \end{equation}

% }}}
% Gas Particle Kinematics {{{ 
\newpage\section{Gas Particle Kinematics}

    \todo{There are both stochastic and systematic contributions to the gas kinematics.}

    \subsection{Sound Speed (\todo{Move to \cref{sec:first_assumptions_about_the_disk}})}

        The speed of sound is defined as the speed at which pressure disturbances travel in a 
        given medium. It is given by 
        \begin{equation}
            \label{eq:general_definition_of_sound_speed}
            c_s := \sqrt{\frac{\partial P}{\partial \rho}}
        \end{equation}
        where $P$ and $\rho$ label the pressure and mass density, respectively. \\
        
        \assume{As noted before, we assume the gas to behave in an isothermal manner (i.e. the
        temperature stays constant across time.)} As such, the ideal gas equation of state reads 
        % A gas can be approximated to be isothermal if the sound wave period is much higher than the (radiative) cooling time of the gas, as any increase in temperature due to compression by the wave will be immediately followed by radiative cooling to the original equilibrium temperature well before the next compression occurs. Many astrophysical situations in the ISM are close to being isothermal, thus the isothermal sound speed is often used. For example, in conditions where temperature and density are independent such as H II regions (where the gas temperature is set by the ionizing star’s spectrum), the gas is very close to isothermal.
        \begin{equation}
            \label{eq:isothermal_ideal_gas_equation_of_state}
            P = \frac{\rho k_B T}{m_g}
        \end{equation}

        Plugging \cref{eq:isothermal_ideal_gas_equation_of_state} into
        \cref{eq:general_definition_of_sound_speed}, we arrive at the isothermal sound speed
        \begin{equation}
            c_s = \sqrt{\frac{k_BT}{m_g}}
        \end{equation}

        % TODO Decide on usage of $P$ vs. $P_g$ and $\rho$ vs. $\rho_g$

        \vfill

\begin{figure}[h!]
    \makebox[\textwidth]{
        \includegraphics[width=\paperwidth]{12/sound_speed.pdf}
    }
    \caption{Radial Profile of the Sound Speed $c_s(r)$}
\end{figure}


    % Thermal Gas Velocity {{{ 
    \clearpage\subsection{Thermal Gas Velocity (\todo{Rename: Gas Viscosity?})}

        We define the absolute value of the thermal velocity to be the mean of the magnitude
        of the molecular velocity. It is given by (\todo{Derive this})
        \begin{equation}
            u_\text{th} = \sqrt{\frac{8}{\pi}} \cdot c_s
        \end{equation}

        A particle's mean free path in a given medium is given by the inverse product between the 
        number of particles per unit volume and the cross section for a collision:
        \begin{equation}
            \lambda_\text{mfp} = \frac{1}{n \cdot \sigma_{\text{H}_2}}
        \end{equation}
        Here, we follow \cite{birnstiel_dullemond_brauer_2010} and use 
        $\sigma_{\text{H}_2} = \SI{2e-19}{\meter^2}$ for the collision cross section. \\

        \todo{Viscosity: (kinematic viscosity)}
        \begin{equation}
            \nu_g = \frac{1}{2} \cdot u_\text{th} \cdot \lambda_\text{mfp}
        \end{equation}

        \vfill

\todo{Why so small?}

\begin{figure}[h!]
    \makebox[\textwidth]{
        \includegraphics[width=\paperwidth]{12/thermal_gas_velocity.pdf}
    }
    \caption{Thermal Gas Velocity $u_th(r)$}
    \label{}
\end{figure}


    % }}}
    % Radial Gas Velocity {{{ 
    \newpage\subsection{Radial Gas Velocity}

        \todo{Plot radial profile of "speed" (thermal + ..?)} \\

        The viscous evolution of the gas disk can be described by the continuity equation 
        \cite{birnstiel_dullemond_brauer_2010}:
        \begin{equation}
            \pderiv{\Sigma_g}{t}-\frac{1}{r}\pderiv{}{r}\bigg(\Sigma_g \cdot r \cdot u_g\bigg) = S_g
        \end{equation}
        Here, $u_g$ labels the \textit{radial gas velocity}.
        \\
        % TODO: What is $S_g$
        % TODO: Does $u_g$ label radial velocity or speed?
        % TODO: Choose either $uor and $v$ as variable names for velocities.
        %       Be consistent across the entire thesis.

        A solution to this equation is given by \cite{lynden-bell_pringle_1974}, where the 
        radial velocity of the gas is given by
        \begin{equation}
            u_{g}
            =-\frac{3}{\Sigma_g\cdot\sqrt{r}}
                \cdot\pderiv{}{r}\bigg(\Sigma_g \cdot \nu_g \cdot \sqrt{r}\bigg)
        \end{equation}
        % TODO: Have I already defined the viscosity somewhere? Else define here.

        \vfill

\begin{figure}[h!]
    \makebox[\textwidth]{
        \includegraphics[width=\paperwidth]{12/radial_gas_velocity.pdf}
    }
    \caption{Radial Gas Velocity $u_g(r)$}
    \label{}
\end{figure}


    % }}}
% }}}

% \section{Symmetry Properties of the Disk}  
%     \todo{Rename section? $\to$ "Disk Structure" ?} \\
%     \todo{Plot disk structure (flaring disk, flaring index)} \\
%     \todo{What exactly is on the ordinate? ($r$ on the abscissa)} \\
%     \assume{Disk is radially symmetric: $f(r,\varphi,z)=f(r,z)$ (cylindrical coordinates)} \\
