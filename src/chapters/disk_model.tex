Our first task will be the definition of a model for the proto-planetary disk, as well as for
the properties of its gas and dust contents. It needs to be said here that due to the complexity of
the involved processes and the (literally) astronomically large number of particles present in the 
disk, our model can only ever be a highly simplified approximation to reality. \\

Nonetheless, the goal of the following chapter(s) (?) will be to formulate a model that captures a 
typical PPD's most relevant properties, which would then allow us to conduct studies of the 
processes involved in dust particle coagulation and/or fragmentation in the subsequent chapters. 
To be able to construct such a framework, it will be necessary to make a couple of assumptions:

% \todo{Do what? (Star, Disk, Gas, Dust, Coagulation/Fragmentation)} \\

\section{First Assumptions}

    \begin{enumerate}
        \item \assume{} First of all, we will base our model on the aforementioned nebular 
            hypothesis: The origin of the modeled proto-planetary disk is assumed to be found in 
            the collapse of a giant interstellar cloud. % consisting of mostly gas and dust. 
            As such, the disk's elementary and material composition is assumed to equal that of the 
            interstellar medium as well. \\
            \todo{Elements?}
    
        \item \assume{} Also, we will assume that the cloud collapse happened sufficiently long ago 
            that a star could already form at the center of the disk (while not long enough for 
            planet formation to occur).
            % (\todo{How long ago?}) [cite subsection]
            For the studies that will be carried out in the context of this thesis, the star's most
            relevant properties will be its mass $M_*$ as well as its luminosity $L_*$. 
            \assume{}
            Both of these properties will be assumed to equal those of our own solar system's Sun. 
            Thus, we define
            \begin{align}
                M_*&:=M_\odot=\SI{1.989e30}{\kg},\ \text{and} \\
                L_*&:=L_\odot\ =\SI{3.828e26}{\joule\second^{-1}},
            \end{align}
            where $M_\odot$ and $L_\odot$ label the solar mass and luminosity, respectively.
            % \todo{What type of star is it? Where in the main sequence is it? How old is it?} 
            % \begin{itemize}
            %     \item G-type main sequence star
            %     \item often imprecisely called "yellow dwarf" 
            %           (imprecisely, because its light is actually white)
            % \end{itemize}
            % \todo{The Sun was not as it is today when it formed $\approx$ 4.6 billion years ago.} \\
    
        \item
            We will assume most of the original gas cloud's mass to be concentrated inside this 
            star, while the disk surrounding it holds only a much smaller fraction of the total 
            mass. We label the stellar mass $M_*$, and the disk mass $M_\text{disk}$. 
            The ratio between the two shall be defined as
            \begin{equation}
                q_\text{disk-to-star}
                    =\frac{M_\text{disk}}{M_*}  
                    :=0.01
            \end{equation}
    
        \item \assume{Furthermore, the disk's mass is assumed to be dominated by the contribution 
            of circumstellar gas, with only a comparatively tiny contribution of dust particles.}
            Let $M_\text{dust}$ and $M_\text{gas}$ label the total mass present in the form of dust 
            and gas, respectively. 
            With
            \begin{equation}
                M_\text{disk} = M_\text{gas} + M_\text{disk},
            \end{equation}
            we define the dust-to-gas ratio as
            \begin{equation}
                q_\text{dust-to-gas}=\frac{M_\text{dust}}{M_\text{gas}}:=0.01
            \end{equation}
    
        \item The next assumption we will make concerns itself with disk symmetry. For this, let us 
            adopt cylindrical coordinates $(r, \varphi, z)$. In reality, each of the disk's 
            position-dependent properties could best be described by a function $f(r, \varphi, z)$ 
            that depends on all three of these coordinates. \\
            To keep things simple though, we will not attempt to resolve the disk in all three 
            spatial dimensions. Instead, we will focus mostly on the radial dependence of the 
            disk's features.
    
            On large scales, we will assume the disk's features to not depend significantly on 
            $\varphi$ and $z$.
    
            \todo{Consider a disk with an inner and outer disk radius
                $r_\text{min}=\SI{0.01}{\astronomicalunit}$ 
            and 
                $r_\text{max}=\SI{100}{\astronomicalunit}$,
            respectively.} \\
    
        \item \assume{Gas mean particle mass.}
            \begin{equation}
                m_g
                    := 2.3 \cdot m_p
            \end{equation}
            Here, $m_p$ labels the mass of a single proton. \\
    
        \item \assume{Gas is in hydrostatic equilibrium.}
            \begin{equation}
                \deriv{P}{z} = - \rho_g \cdot g_z  % TODO Why `g_z` and not `g` ?
            \end{equation}
    
        \item \assume{Gas behaves isothermically.}
            \begin{equation}
                P = \rho_g \cdot c_s^2
            \end{equation}
    
    \end{enumerate}
    
    \todo{Plot top-down view of the disk.}
    
    \vfill

\begin{figure}[h!]
    \centering
    \begin{tikzpicture}

        \def\R{0.25}  % This is the radius of the star.
        \def\D{0.75}  % This is the x-padding between star & disk/axis.
        \def\Y{0.5}   % This is the y-intersect.
        \def\MAX{7}   % This is the maximum value along the abscissa.
        \def\M{1/20}  % This factor controls the slope of the power law plots.
        \def\T{\D/3}  % This factor controls the size & y-padding of the abscissa ticks.

        % Draw the star.
        \draw (0, 0) circle (\R);

        % Draw the 4 power-law components for the "disk".
        \draw[domain= \D  : \MAX,smooth,variable=\x,black] plot ({\x},{ \x^2*\M+\Y});  % top-right
        \draw[domain=-\MAX:-\D,  smooth,variable=\x,black] plot ({\x},{-\x^2*\M+\Y});  % top-left
        \draw[domain= \MAX: \D,  smooth,variable=\x,black] plot ({\x},{-\x^2*\M-\Y});  % bot-right
        \draw[domain=-\D  :-\MAX,smooth,variable=\x,black] plot ({\x},{ \x^2*\M-\Y});  % bot-left

        % Draw the abscissa, i.e. the radial axis.
        \draw[->] (\D,   0)     -- (1.1*\MAX, 0) node[below] {$r$};
        % Draw the ticks & labels.
        \draw     (\D,   -\T/3) -- (\D,       \T/3);  % left tick
        \node at  (\D,   -\T  ) {$r_\text{min}$};     % left label
        \draw     (\MAX, -\T/3) -- (\MAX,     \T/3);  % right tick
        \node at  (\MAX, -\T  ) {$r_\text{max}$};     % right label

    \end{tikzpicture}
    \caption{Schematic View of the Disk Structure's Dependence on Radial Distance from the Star}
    \label{}
\end{figure} \ \\ 

% \begin{figure}[h!]
%     \centering
% \begin{center}
%     \begin{tikzpicture}
%         \draw (0, 0) circle (0.5cm);

%         % \foreach \x in {0.5, 1, 1.5} {
%         %     \draw (\x, 0) -- (\x, {1/\x^2});
%         %     \draw (-\x, 0) -- (-\x, {1/\x^2});
%         % }


%          % Draw the axes
%         % \draw[->] (-3, 0) -- (3, 0) node[right] {$x$};
%         % \draw[->] (0, -1) -- (0, 3) node[above] {$y$};
    
%         % Draw the quadratic sloped line
%     \end{tikzpicture}
%     \caption{}
% \end{center}
    % \label{}
% \end{figure} \ \\ 


% \section{Symmetry Properties of the Disk}  

%     \todo{Rename section? $\to$ "Disk Structure" ?} \\
%     \todo{Plot disk structure (flaring disk, flaring index)} \\
%     \todo{What exactly is on the ordinate? ($r$ on the abscissa)} \\

%     \assume{Disk is radially symmetric: $f(r,\varphi,z)=f(r,z)$ (cylindrical coordinates)} \\

% \clearpage\section{Mass Distribution and Material Composition}

\clearpage\section{Radial Midplane Temperature Profile}

    \assume{To keep things simple, let us make the assumption that the gas and dust particles 
    in the disk behave like perfect black-body objects.} As such, the relationship between the
    thermal radiation flux $B$ coming from the star (i.e. the total energy per unit surface area 
    per unit time) and the temperature $T$ is given by the Stefan-Boltzmann law, which states that
    \begin{equation}
        \label{eq:stefan_boltzmann_law}
        B(T) = \sigma_\text{SB}\cdot T^4   % TODO Is this true?
    \end{equation}

    Here, $\sigma_\text{SB}$ labels the Stefan-Boltzmann constant, which is given by 
    \begin{equation}
        \sigma_\text{SB}
            = \frac{2\pi k_B^4}{15h^3 c^2}
            \approx \SI{5.670e-8}{\ \watt\ \kelvin^{-1}\ \meter^{-2}}
    \end{equation}

    In general, the relationship between the luminosity $L$, which labels the total emitted energy 
    per time, and the radiation flux $B$ is given by the surface integral
    % The luminosity, i.e. the total energy emitted per time, can be determined by evaluating the
    % integral of the radiation flux over the entire surface of the considered absorver/emitter.
    % Thus, the relationship between radiation flux $B(T)$ and luminosity $L(T)$ is given by
    \begin{equation}
        L(T) = \int_A B(T) \ \text dA
    \end{equation}
    
    \assume{
    Since we are assuming spherical symmetry (isotropy) of the star's radiative power output, we 
    can simplify this expression. For a fixed stellar luminosity $L_*$, the radiative flux $B(r)$ 
    at a distance $r$ from the star can be written as}
    \begin{equation}
        L_* = B(r) \cdot 4\pi r^2
    \end{equation}

    Plugging in \cref{eq:stefan_boltzmann_law} and rearranging for the temperature $T$ leads to
    \begin{equation}
        T(r) = \left(
            \frac{L_*}{4\pi r^2 \cdot \sigma_\text{SB}}
        \right)^{1/4}
    \end{equation}

    \todo{Inclusion of $f/2$:}
    \begin{equation}
    \boxed{
        T(r) = \left(
            \frac{f}{2}\cdot
            \frac{L_*}{4\pi r^2 \cdot \sigma_\text{SB}}
        \right)^{1/4}
    }
    \end{equation}

    \vfill

\begin{figure}[h!]
    \makebox[\textwidth]{
        \includegraphics[width=\paperwidth]{12/midplane_temperature.pdf}
    }
    \caption{Radial profile of the midplane gas temperature $T_\text{mid}(r)$.}
    \label{fig:radial_profile_of_midplane_temperature}
\end{figure}


    % \begin{itemize}
    %     \item midplane temperature $T_\text{mid}(r)$ (b.c. isothermal) \\
    %         \todo{Plot $T_\text{mid}$ vs. $r$ (lin-log?)}
    %     \item thermal velocity $u_\text{th}$ \\
    %         \todo{Plot $u_\text{th}$ vs. $r$ (lin-log?)}
    %     \item sound speed (here?)
    % \end{itemize}
