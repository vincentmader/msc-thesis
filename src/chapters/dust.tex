% A Single Dust Particle {{{ 

    \assume{Dust particles are perfectly spherical.}
    \begin{align}
        m
            =\rho_d\cdot\frac{4\pi}{3}a^3
        \ \ \ \ \ \text{and}\ \ \ \ \
        a
            =\bigg(\frac{3}{4\pi}\cdot\frac{m}{\rho_d}\bigg)^{1/3}
    \end{align}

    \assume{Dust particles have density $\rho_d$.} \\

    Building upon prior work done by Birnstiel, Dullemond, \& Brauer % TODO Use Oxford comma?
    \cite{birnstiel_dullemond_brauer_2010}, we will assume that the volume density of the dust 
    particles carries a value of
    $$\rho_d=\SI{1600}{\kilogram\ \meter^{-3}}$$
    % TODO Increase spacing in `\SI` renders, here & elsewhere.

% }}}
% The Dust Particle Mass Distribution Function {{{ 
\section{The Dust Particle Mass Distribution Function}

    % Continuous Formulation {{{ 
    \subsection{Continuous Formulation}

        This function is defined in such a way that, at a given time $t$ and position $\vec r$ 
        in the disk, the expression
        \begin{equation}
            N(\vec r, t)
                =\int\limits_0^\infty n(m,\vec r, t)\ \dd m
        \end{equation}
        denotes the total number of dust particles $N$ per unit volume, i.e. the dust particle
        number density, at the considered region in the proto-planetary disk. \\

        Furthermore, with this we can define 
        \begin{equation}
            \rho_d(\vec r, t)
                =\int\limits_0^\infty n(m,\vec r, t)\cdot m\ \dd m,
        \end{equation}
        which gives us the total dust mass per volume, i.e. the dust mass density, 
        at that specific point in time and space.

    % }}}
    % Discretized Formulation {{{ 
    \clearpage\subsection{Discretized Formulation}

        Let us introduce the notation
        \begin{equation}
            n_i:=n(m_i)
        \end{equation}

        \todo{The total number of dust particles per unit unit volume sitting in a given bin $i$ is 
        then calculated by integrating the particle mass distribution function from the lower to 
        the upper bin boundary:}
        \begin{align}
          N_i=\int_{m_{i-1/2}}^{m_{1+1/2}}n(m)\ \dd m
        \end{align}
        
        \todo{The mass density $\rho_i$ in the bin $i$ is given by}
        \begin{align}
          \rho_i=\int_{m_{i-1/2}}^{m_{1+1/2}}m\cdot n(m)\ \dd m
        \end{align}
        
        \todo{To arrive at the total mass density independent of particle mass, one has to sum over 
        all bins:}
        \begin{align}
          \rho=\sum_{i=1}^{\mathcal N_m}\rho_i
        \end{align}
        % TODO Make approximation here? Need to switch from integral to sum at some point.

    % }}}

% }}}
% Dust Particle Kinematics {{{ 
\clearpage\section{Dust Particle Kinematics}
    
    \todo{This is relevant to us mainly in the context of relative velocity. 
    (for collision rate)} \\

    \todo{Consider a particle suspended in a fluid flow. The ratio between the particle's 
    characteristic time scale and the characteristic time scale of the fluid flow
    is often referred to as \textit{Stokes' number}, named after George Gabriel Stokes.} \\
    % TODO I already talked about this above.
    % TODO: Write "time scale" or "time-scale" or "timescale"?
    % TODO Write "Stokes" or "Stokes'"?

    In our model, it is given by
    \begin{equation}
        \text{St}_i=\frac{\rho_d\cdot a_i}{\Sigma_g\cdot\frac{\pi}{2}}
    \end{equation}

    \todo{maximum drift velocity of a particle}
    \begin{equation}
        u_{n}
        =-\pderiv{P}{r}\cdot\frac{E_d}{2\rho_g\Omega_K}
    \end{equation}

    % Radial Drift {{{ 
    \subsection{Radial Drift}

        For the dust, the radial velocity can be modeled as
        \begin{equation}
            u_{r,i}
            =\frac{u_{g}}{1+\text{St}_i^2}-\frac{2u_{n}}{\text{St}_i+\text{St}_i^{-1}}
        \end{equation}

        \todo{Talk about Stokes' numbers.}
        % TODO "Stokes number" vs. "Stokes' number"

        \todo{first term: drag term}
            comes from the radial movement of the gas 
            which moves with a radial velocity of $u_g$
            Since the dust is coupled to the gas to a certain extend, 
            the radially moving gas is able to partially drag the dust along. \\

        \todo{second term: radial drift velocity with respect to the gas}
             gas in a Keplerian disk does in fact move sub-keplerian, 
             since it feels the force of its own pressure gradient which is usually pointing inwards
             Larger dust grains do not feel this pressure and move on a keplerian orbit
             Therefore, from a point of view of a (larger) dust particle, 
             there exists a constant head wind, 
             which causes the particle to loose angular momentum and to drift inwards. \\

        \todo{particles of different size are coupled differently to the gas}
        \begin{equation}
            \Delta v_{ij}^\text{RD}
            =|u_{r,i}-u_{r,j}|
        \end{equation}
    
        See fig.
        \hyperref[fig:relative_dust_particle_velocities]{\ref*{fig:relative_dust_particle_velocities}a}.

    % }}}
    % Azimuthal Motion {{{ 
    \subsection{Azimuthal Motion}
 
        The azimuthal relative velocity $\Delta v^\text{AZ}_{ij}$ between a pair 
        of particles $(i, j)$ can be written as
        \begin{equation}
            \Delta v^\text{AZ}_{ij}=\bigg|
                u_n\cdot\bigg(
                    \frac{1}{1+\text{St}_i^2}+
                    \frac{1}{1+\text{St}_j^2}
                \bigg)
            \bigg|
        \end{equation}
    
        For a visualization, see fig. 
        \hyperref[fig:relative_dust_particle_velocities]{\ref*{fig:relative_dust_particle_velocities}b}.

    % }}}
    % Differential Settling {{{ 
    \subsection{Differential Settling}
    % }}}
    % Brownian Motion {{{ 
    \subsection{Brownian Motion}

      The Maxwell-Boltzmann distribution function is given by 
        \begin{equation}
            f(v)
            =\left(\frac{m}{2\pi\cdot k_BT}\right)^{3/2}
            \cdot4\pi v^2\cdot\exp\left(-\frac{mv^2}{2k_BT}\right)
        \end{equation}

        The most probable dust particle speed can easily be determined by 
        setting the first derivative of $f$ with respect to $v$ equal to 0:
        \begin{equation}
            0\overset{!}{=}\deriv{f}{v}
        \end{equation}
        Rearanging for the velocity leads to the expression
        \begin{equation}
            \hat{v}=\sqrt{\frac{2k_BT}{m}}
        \end{equation}

        The RMS speed can be calculated via
        \begin{align}
            \sqrt{\overline{v^2}}
            &=\sqrt{\int\limits_0^\infty v^2\cdot f(v)\ \text{d}v}\\
            &=\sqrt{\frac{3k_BT}{m}}
        \end{align}

        We're not actually interested in the most probable particle speed.
        Instead, what we want to look at is the arithmetic mean particle speed.
        It can be calculated from the integral 
        \begin{equation}
            \bar{v}
            =\int\limits_0^\infty v\cdot f(v)\ \text{d}v
        \end{equation}
        and then be expressed as
        \begin{equation}
            \bar{v}
            =\sqrt{\frac{8k_BT}{\pi\cdot m}}
        \end{equation}
        \todo{Now with reduced particle mass:}
        \begin{equation}
            \Delta v_{ij}^\text{BR}
            =\sqrt{\frac{8k_BT}{\pi}\cdot\frac{m_i+m_j}{m_i\cdot m_j}}
        \end{equation}

        See fig. 
        \hyperref[fig:relative_dust_particle_velocities]{\ref*{fig:relative_dust_particle_velocities}c}.

    % }}}
    % Turbulent Motion {{{ 
    \subsection{Turbulent Motion}

        \todo{See \cite{ormel_cuzzi_2007}.} \\
    
        See fig.
        \hyperref[fig:relative_dust_particle_velocities]{\ref*{fig:relative_dust_particle_velocities}d}.

    % }}}
    % Total Relative Velocity {{{ 
    \newpage\subsection{Total Relative Velocity}

        The total relative dust particle velocity can be expressed as the root mean square (RMS) of 
        the individual relative velocity contributions:
        \begin{equation}
            \Delta v_{ij}
                = \sqrt{
                    \big(\Delta v^\text{RD}_{ij}\big)^2
                    + \big(\Delta v^\text{AZ}_{ij}\big)^2
                    + \big(\Delta v^\text{BR}_{ij}\big)^2
                    + \big(\Delta v^\text{TU}_{ij}\big)^2
                }
        \end{equation}
        A visualization of the total relative velocity's dependence on the involved masses of the
        two colliding particles can be seen in \cref{fig:total_relative_dust_particle_velocity}.

        \clearpage

\ \\
\vfill

\begin{figure}[h!]
    \centering
    \begin{minipage}{.5\linewidth}
      \centering
      \subfloat[Relative Velocity $\Delta v_{ij}^\text{RD}$ due to Radial Drift]{
        \label{:a}
        \includegraphics[width=\linewidth]{21/dv_RD.pdf}
      }
    \end{minipage}%
    \begin{minipage}{.5\linewidth}
      \centering
      \subfloat[Relative Velocity $\Delta v_{ij}^\text{AZ}$ due to Azimuthal Drift]{
        \label{:b}
        \includegraphics[width=\linewidth]{21/dv_AZ.pdf}
      }
    \end{minipage}
    \begin{minipage}{.5\linewidth}
      \centering
      \subfloat[Relative Velocity $\Delta v_{ij}^\text{TU}$ due to Turbulent Motion]{
        \label{:c}
        \includegraphics[width=\linewidth]{21/dv_TU.pdf}
      }
    \end{minipage}%
    \begin{minipage}{.5\linewidth}
      \centering
      \subfloat[Relative Velocity $\Delta v_{ij}^\text{BR}$ due to Brownian Motion]{
        \label{:d}
        \includegraphics[width=\linewidth]{21/dv_BR.pdf}
      }
    \end{minipage}
    \caption{Relative dust particle velocities due to 
        (a) Radial drift $\Delta v_{ij}^\text{RD}$, 
        (b) Azimuthal drift $\Delta v_{ij}^\text{AZ}$, 
        (c) Turbulent motion $\Delta v_{ij}^\text{TU}$, and
        (d) Brownian motion $\Delta v_{ij}^\text{BR}$.
    }
    \label{fig:relative_dust_particle_velocities}
    % TODO Assure consistent capitalization in captions.
\end{figure} 

% \begin{figure}[h!]
%     \makebox[\textwidth]{
%         \includegraphics[width=\paperwidth]{21/dv_all_2x2.pdf}
%     }
%     \caption{Total Relative Dust Particle Velocity}
%     \label{fig:total_relative_dust_particle_velocity}
% \end{figure}


    % }}}

% }}}
% Dust Particle Collisions {{{ 
\clearpage\section{Dust Particle Collisions}

    % Collision Cross Section {{{ 
    \subsection{Collision Cross Section}
        
        As already mentioned in (\todo{cite}), in the context of this thesis we will assume the
        dust particles to possess a perfectly spherical shape. Under this assumption, the cross
        section $\sigma_{ij}$ for a collision of two dust particles with radii $a_i$ and $a_j$ can
        be approximated by the area of a circle the radius of which equals the sum of the two
        particles' radii. \\

        Therefore, the collision cross section can be written as:
        \begin{equation}
            \sigma_{ij} = \pi \cdot (a_i+a_j)^2
        \end{equation}

    % }}}
    % Collision Rate {{{ 
    \subsection{Collision Rate}

        The rate $R^\text{coll}_{ij}$ of collisions between pairs of particles from bins $i$ 
        and $j$ is given by the product of the corresponding collision cross section, the
        particles' relative velocity, and the number density of particles at the considered 
        location in the disk. 
        \begin{equation}                                            % ^ TODO: "considered?"
            R^\text{coll}_{ij}
                \approx \sigma_{ij} \cdot \Delta v_{ij} \cdot \textcolor{red}{n}
        \end{equation}

        \todo{Watch out: Different units! (not 1/second)}

        \vfill

\begin{figure}[h!]
    \centering
    \begin{minipage}{.5\linewidth}
      \centering
      \subfloat[]{
        \label{:a}
        \includegraphics[width=\linewidth]{22/collision_cross_section.pdf}
      }
    \end{minipage}%
    \begin{minipage}{.5\linewidth}
      \centering
      \subfloat[]{
        \label{:b}
        \includegraphics[width=\linewidth]{23/collision_rate.pdf}
      }
    \end{minipage}
    \caption{
        Dust Particle Collision Cross Section \& Collision Rate
    }
    \label{}
\end{figure} \ \\

        % TODO Define `minipage` plot with both coll. cross section & coll. rate.

    % }}}
    % Collision Outcomes {{{ 
    \subsection{Collision Outcomes}

        Since the kinematics of circumstellar dust particles are influenced by both systematic and 
        stochastic contributions (e.g. Keplerian and Brownian motion, respectively), occasional
        particle collisions are to be expected. At a given time $t$ and position $\vec r$ in the
        disk, let us write $$R^\text{coll}(m,m')$$ to denote the rate of collisions between two
        particles carrying the masses $m$ and $m'$, respectively. \\

        \todo{This rate will be defined further in... [ref. chapter]} \\
        
        The outcome of a collision between such a particle pair depends (\todo{among other things})
        on the relative velocity between the two particles, and can be classified into 3 distinct 
        categories, namely (pure hit-and-stick) coagulation, fragmentation, and bouncing.\\
        
        Let $P^\text{coag}$, $P^\text{frag}$, and $P^\text{bounce}$
        label the probabilities that a collision will lead to the respective outcome. \\

        \todo{They will be defined in... [ref. chapter]} \\
        
        The rates at which such events occur are then given by
        \begin{equation}
            R^\text{coag}(m,m')
                =R^\text{coll}(m,m')\cdot P^\text{coag}(m,m'),
        \end{equation}
        \begin{equation}
            R^\text{frag}(m,m')
                =R^\text{coll}(m,m')\cdot P^\text{frag}(m,m'),
        \end{equation}
        \begin{center}and\end{center}
        \begin{equation}
            R^\text{bounce}(m,m')
                =R^\text{coll}(m,m')\cdot P^\text{bounce}(m,m').
        \end{equation}
        
        % to denote the probability that such a collision will lead to a 
        % merging of the two collision partners into a single new particle. Analogously, 
        % labels that rate of collisions leading to fragmentation of the two 
        % particles into a range of newly created, smaller particles.
        % TODO "newly created" or "newly-created" ?
        % $R^\text{coag}(m,m',\vec r,t)$ label the rate of 
        % collisions leading to the merging of two particles carrying the masses $m$ and $m'$, respectively. 
        % Analogously, let $R^\text{frag}(m,m',\vec r,t)$ be the rate of collisions leading to particle 
        % fragmentation.\\
        
        \todo{Threshold value for relative velocity:}
        
        \begin{equation}
            v_\text{threshold}:=\SI{1}{\meter\second^{-1}}
        \end{equation}
        
        \begin{equation}
            P^\text{frag}_{ij}
            =
            \begin{cases}
                1 & \text{if}\ \Delta v_{ij} >= v_\text{threshold}\\
                0 & \text{else}
            \end{cases}
        \end{equation}
        
        Correction using Maxwell-Boltzmann distribution:
        \begin{equation}
            x:=\frac{3}{2}\cdot\bigg(
                \frac{v_\text{frag}}{\Delta v_{ij}}
            \bigg)^2
        \end{equation}
        
        \begin{equation}
            P^\text{frag}_{ij}=(1+x)\cdot e^{-x}
        \end{equation}
        
        \todo{Line plot: outcome probability vs. particle radius}
        
        % \newpage
\vfill

\begin{figure}[h!]
    \centering
    % \begin{minipage}{.5\linewidth}
    %   \centering
    %   \subfloat[]{
    %     \label{:a}
    %     \includegraphics[width=\linewidth]{24/coagulation_probability_from_cutoff_velocity.pdf}
    %   }
    % \end{minipage}%
    \begin{minipage}{.5\linewidth}
      \centering
      \subfloat[]{
        \label{:b}
        \includegraphics[width=\linewidth]{24/fragmentation_probability_from_cutoff_velocity.pdf}
      }
    \end{minipage}%
    % \begin{minipage}{.5\linewidth}
    %   \centering
    %   \subfloat[]{
    %     \label{:a}
    %     \includegraphics[width=\linewidth]{24/coagulation_probability_from_Maxwell-Boltzmann.pdf}
    %   }
    % \end{minipage}%
    \begin{minipage}{.5\linewidth}
      \centering
      \subfloat[]{
        \label{:b}
        \includegraphics[width=\linewidth]{24/fragmentation_probability_from_Maxwell-Boltzmann.pdf}
      }
    \end{minipage}
    \caption{
        Dust Particle Collision Outcome Probabilities.\\
        \todo{Why not a perfect square in 1st case?}
    }
    \label{}
\end{figure}

\newpage


    % }}}

% }}}
