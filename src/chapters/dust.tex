% TODO: Write "time scale" or "time-scale" or "timescale"?
% TODO Write "Stokes" or "Stokes'"?


% Dust Particle Kinematics {{{ 
\clearpage\section{Dust Particle Kinematics}

    The construction of the kernel matrix requires the definition of the rate at which collision 
    occur between pairs of differently sized particles.
    Therefore, it is essential to model the kinematics of the dust particles in an appropriate
    manner and arrive at an expression for the relative velocities involved in a given particle
    collision.
    These relative velocities will influence both the collision rate as well as their outcomes, 
    since the fragmentation probability is influenced by the relative velocity as well. \\

    We include four processes in the calculation of the relative dust particle velocities.
    There are the two systematic contributions of radial and azimuthal drift, 
    and the two stochastic contributions of Brownian and turbulent motion. \\

    We will start with the definition of the systematic contributions, then move on to define 
    the stochastic contributions, and finally combine the four different effects into the 
    total relative velocity value 
    \begin{equation}
        \Delta v_{ij} := \Delta v( m_i, m_j)
    \end{equation}
    for a given pair of particles carrying masses $m_i$ and $m_j$, respectively. 
    % \\

    % For the definition of the radial and azimuthal velocities, as well as for Brownian motion, 
    % we closely follow the work done in earlier studies by 
    % Weidenschilling \cite{weidenschilling_1977}, 
    % Nakagawa \cite{nakagawa_1986}, 
    % Birnstiel, Dullemond, and Brauer \cite{birnstiel_dullemond_brauer_2010}, 
    % as well as 
    % Dullemond and Dominik
    % \cite{dullemond_dominik_2004}.
    % The definition of the relative velocity due to turbulence is based on the work done by 
    % Ormel and Cuzzi \cite{ormel_cuzzi_2007}.

    % Radial Drift {{{ 
    \subsection{Radial Drift}

        Consider a particle suspended in a fluid flow. The ratio between the particle's 
        characteristic time scale and the characteristic time scale of the fluid flow
        is referred to as \textit{Stokes' number}, named after George Gabriel Stokes. 
        In our model, it is given by
        \begin{equation}
            \text{St}_i=\frac{\rho_s\cdot a_i}{\Sigma_g\cdot\frac{\pi}{2}}
        \end{equation}
        % \todo{Talk about Stokes' numbers.}
        % \todo{Epstein regime} \\
        % TODO "Stokes number" vs. "Stokes' number"
        % Small particles have small Stokes numbers.

        Building upon the 2010 paper written by Birnstiel, Dullemond, and Brauer 
        \cite{birnstiel_dullemond_brauer_2010}, we will model the
        radial velocity $u_{r,i}$ of particles in the dust population corresponding to 
        a given mass bin $i$ as
        \begin{equation}
            \label{eq:radial_dust_velocity}
            u_{r,i}
            =\frac{u_{g}}{1+\text{St}_i^2}-\frac{2u_{n}}{\text{St}_i+\text{St}_i^{-1}}
        \end{equation}

        Here, the first term in the above equation is the \textit{drag term}. 
        The gas moves along 
        the radial axis with a radial velocity $u_g$
        which we defined in \cref{eq:radial_gas_velocity}.
        The value of the Stokes number affects 
        the extent to which the dust particles are coupled to the gas. In the case of a low 
        Stokes number, the dust is dragged along with the gas, whereas large Stokes numbers 
        describe particles whose kinematics is less influenced by the motion of the gas. \\

        The second term is related to the fact that due to the disk's own gas pressure 
        gradient, the gas in the disk orbits the disk center on a sub-keplerian 
        trajectory. Heavier particles do not feel this pressure as much and 
        thus move keplerian. As a result, there is a difference in the radial velocity
        of the dust particle and the gas through which it moves. Slowed down by the gas, 
        the particle loses momentum and starts drifting inwards. 
        The term $u_n$ labels the maximum drift velocity of a particle, which was derived 
        by Weidenschilling \cite{weidenschilling_1977}, and is given by
        \begin{equation}
            u_{n}
            =-\pderiv{P_g}{r}\cdot\frac{E_d}{2\rho_g\Omega_K}
        \end{equation}
        In it, the $E_d$ labels the drift efficiency parameter that was introduced by 
        \cite{birnstiel_dullemond_brauer_2010}. Here, it will be set to 1 for simplicity. \\
        % \todo{particles of different size are coupled differently to the gas}

        The relative velocity due to radial drift of dust particles is then given by
        \begin{equation}
            \Delta v_{ij}^\text{RD}
            =|u_{r,i}-u_{r,j}|
        \end{equation}
    
        For a visualization of the results, see 
        \hyperref[fig:relative_dust_particle_velocities]
        {\cref*{fig:relative_dust_particle_velocities}a}.

    % }}}
    % Azimuthal Motion {{{ 
    \subsection{Azimuthal Motion}

        The azimuthal relative velocities of differently-sized dust particles are induced in a 
        similar way to the radial drift. The extent to which dust particles are coupled to the gas 
        depends directly on their Stokes number, and thus on the particle's size and mass. \\

        Building upon the results of earlier studies done by 
        Weidenschilling \cite{weidenschilling_1977} and Nakagawa \cite{nakagawa_1986}, 
        we will express the relative azimuthal velocity $\Delta v^\text{AZ}_{ij}$ for 
        gas-dominated drag between a pair of particles with mass bin indices $(i, j)$ as
        \begin{equation}
            \Delta v^\text{AZ}_{ij}=\bigg|
                u_n\cdot\bigg(
                    \frac{1}{1+\text{St}_i^2}+
                    \frac{1}{1+\text{St}_j^2}
                \bigg)
            \bigg|
        \end{equation}

        For a visualization of the resulting distribution of relative velocities, see 
        \hyperref[fig:relative_dust_particle_velocities]
        {\cref*{fig:relative_dust_particle_velocities}b}.

    % }}}
    % Differential Settling {{{ 
    % \subsection{Differential Settling}
    % }}}
    % Turbulent Motion {{{ 
    \subsection{Turbulent Motion}

        For the calculation of the relative velocity contributions due to turbulent motion, we 
        adopt the model outlined in the 2007 work by Ormel and Cuzzi \cite{ormel_cuzzi_2007}. \\

        There, the relative velocity $\Delta v^\text{TU}_{ij}$ due to turbulence is defined as
        \begin{equation}
            \Delta v^\text{TU}_{ij} =
            \begin{cases}
                \Delta v_{ij}^\text{A} & \text{if } t_{sm} \leq t_s \\
                \Delta v_{ij}^\text{B} & \text{if } t_s < t_{sm} \leq t_n \\
                \Delta v_{ij}^\text{C} & \text{if } t_{sm} > t_n \\
            \end{cases}
        \end{equation}

        For this definition, three different cases have to be handled separately.
        \begin{enumerate}
            \item Limit of tightly coupled particles (below the Kolmogorov scale) \\
                % \begin{equation}
                %     \Delta v_{ij}^1 
                %     = u_g \cdot \sqrt{\frac{t_1 - t_2}{t_1 + t_2} \cdot (\Delta u_1 - \Delta u_2)}
                % \end{equation}
                % with the two "velocities" $u_1$ and $u_2$ given by
                % \begin{equation}
                %     \Delta u_{1,2} 
                %     = \left(\frac{t_{1,2}}{t_n}\right)^2 \cdot 
                %         \left(\frac{t_{1,2}}{t_n} + \frac{1}{\sqrt{\text{Re}} }\right)^{-1} \\
                % \end{equation}
                \begin{equation}
                    \label{eq:turbulence_ormel_cuzzi_eq_1}
                    \Delta v_{ij}^\text{A}
                    = u_g \cdot \sqrt{
                        \frac{\text{St}_i - \text{St}_j}{\text{St}_i + \text{St}_j} \cdot 
                        \left(
                            \frac{\text{St}_i^2}{\text{St}_i + \text{Re}^{-1/2}} -
                            \frac{\text{St}_j^2}{\text{St}_j + \text{Re}^{-1/2}}
                        \right)
                    }
                \end{equation} 
                For a detailed explanation, see section 3.4.1 of \cite{ormel_cuzzi_2007}, 
                and their equation 26.

            \item At least one of the particles is not in the tightly coupled regime.
                Both particles have $\text{St} < 1$. \\
                \begin{equation}
                    \Delta v_{ij}^\text{B} = u_g \cdot \sqrt{
                        \text{St}_i \cdot \left[
                            2y_a - (1+\varepsilon) + \frac{2}{1+\varepsilon} \cdot \left(
                                \frac{1}{1+y_a} + \frac{\varepsilon^3}{y_a+\varepsilon}
                            \right)
                        \right] 
                    }
                \end{equation}
                where $\varepsilon < 1$ is the ratio of the stopping times 
                and $y_a := 1.6$. 
                See section 3.4.2 of \cite{ormel_cuzzi_2007}, and their equation 28.
                % \begin{equation}
                %     \varepsilon = \frac{\min(t_1, t_2)}{\max(t_1, t_2)} < 1
                % \end{equation}
                % Ratio of St_smallest/St_largest
                % St1 = St_largest
                % \begin{equation}
                %     \Delta v_{ij}^\text{B} = u_g \cdot \sqrt{\Delta u_2 \cdot \text{St}_i}
                % \end{equation}
                % with 
                % \begin{align}
                %     \Delta u_1 
                %         &= \frac{1}{1 + 1.6} + \frac{\varepsilon^3}{1.6 + \varepsilon} \\
                %     \Delta u_2 
                %         &= 3.2 - (1 + \varepsilon) + \frac{2}{1 + \varepsilon} \cdot \Delta u_1
                % \end{align}
                % with 
                % and
                % \begin{equation}
                %     \text{St}_1 = \frac{\max(t_1, t_2)}{t_n}
                % \end{equation}

            \item At least one of the particles is not in the tightly coupled regime. 
                Also, here at least one of the particles has $\text{St} \geq 1$. \\
                \begin{equation}
                    \Delta v_{ij}^\text{C} = u_g \cdot \sqrt{
                        \frac{1}{1+\text{St}_i} + \frac{1}{1+\text{St}_j}
                    }
                \end{equation}
                For a detailed explanation, see section 3.4.3 of \cite{ormel_cuzzi_2007}, 
                their equation 29.
                % \begin{equation}
                %         \frac{t_n}{t_n + t_1} + \frac{t_n}{t_n + t_2}
                % \end{equation}

        \end{enumerate}

        In \cref{eq:turbulence_ormel_cuzzi_eq_1}, the Reynolds number $\text{Re}$ is defined as
        \begin{equation}
            \text{Re} = \frac{\alpha c_s H_p}{\nu_g}
        \end{equation}
        with the viscosity parameter set to $\alpha := 10^{-3}$. 
        For a visualization of the results, see \hyperref[fig:relative_dust_particle_velocities]
        {\cref*{fig:relative_dust_particle_velocities}d}.

    % }}}
    % Brownian Motion {{{ 
    \clearpage\subsection{Brownian Motion}

        Here, we follow the work published in the 2004 paper written by Dullemond and Dominik 
        \cite{dullemond_dominik_2004}, where the contribution to relative velocity due to Brownian 
        motion is given by
        \begin{equation}
            \Delta v_{ij}^\text{BR}
            =\sqrt{\frac{8k_BT}{\pi}\cdot\frac{m_i+m_j}{m_i\cdot m_j}}
        \end{equation}

        Note that this term will be the largest when both collision partners carry a very low mass.
        Brownian motion thus favors collisions in which at least one collision partner is a 
        low-mass particles. This can be seen reflected in 
        \hyperref[fig:relative_dust_particle_velocities]{
        \cref*{fig:relative_dust_particle_velocities}c}.

    % }}}
    % Total Relative Velocity {{{ 
    \subsection{Total Relative Velocity}

        The total relative dust particle velocity can be arrived at by calculating the 
        root mean square (RMS) of the individual relative velocity contributions:
        \begin{equation}
            \Delta v_{ij}
                = \sqrt{
                    \big(\Delta v^\text{RD}_{ij}\big)^2
                    + \big(\Delta v^\text{AZ}_{ij}\big)^2
                    + \big(\Delta v^\text{BR}_{ij}\big)^2
                    + \big(\Delta v^\text{TU}_{ij}\big)^2
                }
        \end{equation}
        Having arrived at this expression now allows us to use it in the definition of the 
        collision rates between differently-sized particles, which we will do in 
        \cref{sec:dust_particle_collisions}. 
        A visualization of the total relative velocity's dependence on the sizes of the
        two colliding particles can be seen in \cref{fig:total_relative_dust_particle_velocity}. 
        Note that while the relative dust particle velocities are visualized up to a maximum 
        particle radius of $>10^2\ \text{m}$, when integrating numerically an equilibrium state 
        will be reached before particles can grow to that size, such that there we have an 
        effective maximum particle radius of $10^{-4}\ \text{m}$.

        \vfill

\begin{figure}[h!]
    \begin{center}
        \includegraphics[width=.5\linewidth]{21/dv_tot.pdf}
    \end{center}
    \caption{Total Relative Dust Particle Velocity}
    \label{fig:total_relative_dust_particle_velocity}
\end{figure}

        \clearpage

\ \\
\vfill

\begin{figure}[h!]
    \centering
    \begin{minipage}{.5\linewidth}
      \centering
      \subfloat[Relative Velocity $\Delta v_{ij}^\text{RD}$ due to Radial Drift]{
        \label{:a}
        \includegraphics[width=\linewidth]{21/dv_RD.pdf}
      }
    \end{minipage}%
    \begin{minipage}{.5\linewidth}
      \centering
      \subfloat[Relative Velocity $\Delta v_{ij}^\text{AZ}$ due to Azimuthal Drift]{
        \label{:b}
        \includegraphics[width=\linewidth]{21/dv_AZ.pdf}
      }
    \end{minipage}
    \begin{minipage}{.5\linewidth}
      \centering
      \subfloat[Relative Velocity $\Delta v_{ij}^\text{TU}$ due to Turbulent Motion]{
        \label{:c}
        \includegraphics[width=\linewidth]{21/dv_TU.pdf}
      }
    \end{minipage}%
    \begin{minipage}{.5\linewidth}
      \centering
      \subfloat[Relative Velocity $\Delta v_{ij}^\text{BR}$ due to Brownian Motion]{
        \label{:d}
        \includegraphics[width=\linewidth]{21/dv_BR.pdf}
      }
    \end{minipage}
    \caption{Relative dust particle velocities due to 
        (a) Radial drift $\Delta v_{ij}^\text{RD}$, 
        (b) Azimuthal drift $\Delta v_{ij}^\text{AZ}$, 
        (c) Turbulent motion $\Delta v_{ij}^\text{TU}$, and
        (d) Brownian motion $\Delta v_{ij}^\text{BR}$.
    }
    \label{fig:relative_dust_particle_velocities}
    % TODO Assure consistent capitalization in captions.
\end{figure} 

% \begin{figure}[h!]
%     \makebox[\textwidth]{
%         \includegraphics[width=\paperwidth]{21/dv_all_2x2.pdf}
%     }
%     \caption{Total Relative Dust Particle Velocity}
%     \label{fig:total_relative_dust_particle_velocity}
% \end{figure}


    % }}}

% }}}
% Dust Particle Collisions {{{ 
\clearpage\section{Dust Particle Collisions}
\label{sec:dust_particle_collisions}

    As we saw in the last section, the relative velocities between differently-sized dust particls 
    are influenced by partly random, and partly systematic contributions. 
    Due to this, occasional collision events between dust particles are to be expected. \\ 

    In the context of this thesis, these collisions are assumed to always be two-body 
    interactions. The probability of three or more dust particles colliding at the exact same time
    is so small, that the occurance rate of three-body interactions can be neglected entirely.
    Thus, we can safely assume the particle collisions to be two-particle interactions.

    % \todo{alwats 2-body collisions}
    % Such collisions are always two-body collisions. The occurrance rate of three-body interactions (where by chance three dust particles collide with each other at the same time) is so tiny that it is absolutely negligible. If two particles with masses m1

    % Collision Cross Section {{{ 
    \subsection{Collision Cross Section}
        
        As was mentioned in \cref{sec:dust_particle_mass_distribution}, in the context of 
        this thesis we will assume the dust particles to possess a perfectly spherical shape. \\

        Under this assumption, the cross section $\sigma^\text{coll}_{ij}$ for a collision of two
        dust particles with radii $a_i$ and $a_j$ can be approximated by the area of a circle the 
        radius of which equals the sum of the two particles' radii. \\

        Therefore, the collision cross section 
        $\sigma^\text{coll}_{ij} := \sigma^\text{coll}(m_i, m_j)$ 
        can be written as:
        \begin{equation}
            \sigma^\text{coll}_{ij} = \pi \cdot (a_i+a_j)^2
        \end{equation}

        It can be seen visualized in 
        \hyperref[fig:collision_cross_section_and_rate_coefficient]
        {\cref*{fig:collision_cross_section_and_rate_coefficient}a}.

        \vfill

\begin{figure}[h!]
    \centering
    \begin{minipage}{.5\linewidth}
      \centering
      % \subfloat[Dust Particle Collision Cross Section]{
      \subfloat[Dust Particle Collision Cross Section $\sigma^\text{coll}_{ij}$]{
        \label{:a}
        \includegraphics[width=\linewidth]{22/collision_cross_section.pdf}
      }
    \end{minipage}%
    \begin{minipage}{.5\linewidth}
      \centering
      % \subfloat[Dust Particle Collision Rate]{
      \subfloat[Dust Particle Collision Rate Coefficient $R_{ij}^\text{coll}$]{
        \label{:b}
        \includegraphics[width=\linewidth]{23/collision_rate_coefficient.pdf}
      }
    \end{minipage}
    \caption{
        Visualization of the dependence of the
        (a) Dust particle collision cross section $\sigma^\text{coll}_{ij}$ and 
        (b) Dust particle collision rate coefficient $R_{ij}^\text{coll}$
        on the mass values of two colliding particles.
    }
    \label{fig:collision_cross_section_and_rate_coefficient}
\end{figure}


    % }}}
    % Collision Rate {{{ 
    \clearpage\subsection{Collision Rate}

        The rate of collisions between pairs of particles from bins $i$ and $j$ can be 
        described using the definition of the \textit{collision rate coefficient} 
        $R^\text{coll}_{ij}$, which is given as the product of the corresponding collision 
        cross section and the two colliding particles' relative velocity: 
        \begin{equation}
            R^\text{coll}_{ij} \approx \sigma^\text{coll}_{ij} \cdot \Delta v_{ij}
        \end{equation}

        Note: The units of this collision rate coefficent are \textit{not} $\text{s}^{-1}$! 
        Instead, this coefficient is to be interpreted as a collision rate ``per particle density'',
        with units of m$^3$ s$^{-1}$. \\

        The actual rate of collisions per time can be calculated by multiplying the 
        collision coefficient $R_{ij}^\text{coll}$ with the number density of particles $N_i$ 
        per unit volume at a given position of interest in the disk. 

    % }}}
    % Collision Outcomes {{{ 
    \subsection{Collision Outcomes}
        \label{sec:definition_of_reaction_rate_coefficient}
        % Depending on the relative velocity of the two particles involved in a collision,
        % we will differentiate between three different collision outcome scenarios:

        Now that we have a description for \textit{how often} collisions between dust particles 
        will occur in our model, we have to think about how to model \textit{what happens} as a 
        result of a collision. \\

        In the context of this thesis, we will classify the different outcome scenarios into 
        three main categories, which are briefly outlined in the following. 
        \begin{enumerate}
            \item \textit{Coagulation:} Here, the two colliding particle merge into a single 
                  new particle. Mass is conserved, such that the sum of the two initial particles'
                  masses exactly equals that of newly created one.
                % If the relative velocity is low, the two particles will merge into a single one.
                % \\ (\todo{pure "hit-and-stick" coagulation})
            \item \textit{Bouncing:} The two particles collide and bounce off of each other.
                  No particles are created or destroyed, and no mass is transferred between the 
                  particles. Since this scenario does not lead to a change in the dust particle 
                  mass distribution, it will be neglected throughout this work.
            \item \textit{Fragmentation:} One or both of the particles involved in the collision 
                  breaks apart into a possibly large number of smaller and less massive particles.
                  Figuring out how one might model the distribution of masses among the newly 
                  created particles in such a scenario is a problem that we will address in 
                  \cref{sec:modeling_the_mass_distribution_resulting_from_fragmentation} and
                  \cref{sec:the_mrn_distribution}.

                % If the relative velocity is large enough, the collision will lead to a shattering 
                % (fragmenting) of the two particles.
        \end{enumerate}

        A schematic representation of the collision types is displayed in
        \cref{fig:collision_outcomes}.\footnote{
            For studies and visualizations regarding measurements of actual dust particle 
            collisions (here: BPCA) in a laboratory see e.g. \cite{wada_2009}.
        } \\

        Which of these outcomes actually occurs as a result of a given collision of course heavily 
        depends on the details of said collision. 
        In our highly simplified model, the most notable of these details is the 
        relative dust particle velocity. \\

        To get an intuition for how the relative velocity influences the probability of a 
        certain outcome to occur, one might imagine that collisions at low relative velocities have 
        a high likelyhood of leading to a merge event. Higher velocity values may lead to 
        bouncing, and still higher velocities would eventually have the effect 
        of breaking apart the particles involved in the collision, i.e. fragmentation \\

        For each of these three possible outcomes, we will define an associated probability 
        distribution, each of which depends on the mass values of the two particles participating 
        in a given collision. \\

        Let $P^\text{coag}$, $P^\text{frag}$, and $P^\text{bounce}$
        label the probabilities that a collision will lead to the respective outcome. \\

        \clearpage

        The rates at which such reactions occur can then described by the 
        \textit{reaction rate coefficients}, which we shall label 
        $R_{ij}^\text{coag}$, $R_{ij}^\text{frag}$, and $R_{ij}^\text{bounce}$. 

        They are defined as 
        \begin{align}
            R^\text{coag}_{ij}   &= R^\text{coll}_{ij} \cdot P^\text{coag}_{ij} \\
            R^\text{frag}_{ij}   &= R^\text{coll}_{ij} \cdot P^\text{frag}_{ij} \\
            R^\text{bounce}_{ij} &= R^\text{coll}_{ij} \cdot P^\text{bounce}_{ij}
        \end{align}

        Collisions in the scenario of bouncing do not have an effect on the dust mass distribution
        among particles. But since a high value of $P^\text{bounce}_{ij}$ will lead to lower 
        reaction rates for the other two collision outcomes, the inclusion of bouncing effectively 
        ``slows down'' the simulation. \\

        In the context of this thesis, we will not take the effect of bouncing into account, and 
        only model the influence of coagulation and fragmentation processes. As such, we will adopt 
        a bouncing probability 
        \begin{equation}
            P_{ij}^\text{bounce}=0 
        \end{equation}

        To help with the decision whether a given collision ends up resulting in coagulation or
        fragmentation, the definition of a \textit{threshold value for the relative velocity} makes 
        sense. Adopting the value used by Dullemond and Dominik in their 2003 paper 
        \cite{dullemond_dominik_2023}, we will define the value of $v_\text{frag}$ as
        \begin{equation}
            v_\text{frag} := \SI{1}{\meter~\second^{-1}}
        \end{equation}

        The idea is that collisions with slow relative velocities lead to a sticking of the two 
        particles, while high velocities lead to them breaking apart. \\

        \vfill

\newcommand{\particle}[3]{%
    \ifcase#1
        % Zero
    \or
        \draw[fill={black!0!white}] (#2          , #3         ) circle (\R);
    \or
        \draw[fill={black!0!white}] (#2 - 0.65*\R, #3 + 0.30*\R) circle (\R);
        \draw[fill={black!0!white}] (#2 + 0.65*\R, #3 - 0.30*\R) circle (\R);
    \or
        \draw[fill={black!0!white}] (#2 - 0.70*\R, #3 - 0.50*\R) circle (\R);
        \draw[fill={black!0!white}] (#2 + 0.00*\R, #3 + 0.50*\R) circle (\R);
        \draw[fill={black!0!white}] (#2 + 0.70*\R, #3 - 0.50*\R) circle (\R);
    \or
        \draw[fill={black!0!white}] (#2 + 0.65*\R, #3 + 0.65*\R) circle (\R);
        \draw[fill={black!0!white}] (#2 - 0.65*\R, #3 - 0.65*\R) circle (\R);
        \draw[fill={black!0!white}] (#2 + 0.65*\R, #3 - 0.65*\R) circle (\R);
        \draw[fill={black!0!white}] (#2 - 0.65*\R, #3 + 0.65*\R) circle (\R);
    \or
        \draw[fill={black!0!white}] (#2 + 0.85*\R, #3 + 0.85*\R) circle (\R);
        \draw[fill={black!0!white}] (#2 - 0.85*\R, #3 - 0.85*\R) circle (\R);
        \draw[fill={black!0!white}] (#2 + 0.85*\R, #3 - 0.85*\R) circle (\R);
        \draw[fill={black!0!white}] (#2 - 0.85*\R, #3 + 0.85*\R) circle (\R);
        \draw[fill={black!0!white}] (#2 +       0, #3 +       0) circle (\R);
    \fi
}

\newcommand{\bang}[1]{
    \node[shape=starburst, starburst points=#1, starburst point height=0.5cm, 
        draw, black, fill=white,
        minimum width=1.5cm, minimum height=1.5cm,
    ] {};
}

\newcommand{\illustrateCoagulation}[0]{
    \begin{tikzpicture}

        \def\R{0.2}
        \def\P{1.1}

        \def\Xa{-1*\P}
        \def\Ya{+3*\P}
        \def\Xb{+1*\P}
        \def\Yb{+3*\P}
        \def\Xc{+0*\P}
        \def\Yc{-3*\P}

        \begin{scope}[decoration={
            markings,
            mark=at position 0.5 with {\arrow{latex}}}
        ]
            \draw[postaction={decorate}] (\Xa, \Ya) -- (  0,   0);
            \draw[postaction={decorate}] (\Xb, \Yb) -- (  0,   0);
            \draw[postaction={decorate}] (  0,   0) -- (\Xc, \Yc);
        \end{scope}

        \particle{3}{\Xa}{\Ya}
        \particle{1}{\Xb}{\Yb}
        \particle{4}{\Xc}{\Yc}

        \bang{10}

        \draw[draw=white,fill=white] (0, -4*\P) circle (\R);

    \end{tikzpicture}
}

\newcommand{\illustrateFragmentation}[0]{
    \begin{tikzpicture}

        \def\R{0.2}
        \def\P{1.1}

        \def\Xa{-1.0*\P}
        \def\Ya{+3.0*\P}
        \def\Xb{+1.0*\P}
        \def\Yb{+3.0*\P}
        \def\Xc{-2.0*\P}
        \def\Yc{-2.8*\P}
        \def\Xd{-1.0*\P}
        \def\Yd{-3.1*\P}
        \def\Xe{+0.0*\P}
        \def\Ye{-3.3*\P}
        \def\Xf{+1.0*\P}
        \def\Yf{-3.1*\P}
        \def\Xg{+2.0*\P}
        \def\Yg{-2.8*\P}

        \begin{scope}[decoration={
            markings,
            mark=at position 0.5 with {\arrow{latex}}}
        ]
            \draw[postaction={decorate}] (\Xa, \Ya) -- (  0,   0);
            \draw[postaction={decorate}] (\Xb, \Yb) -- (  0,   0);
            \draw[postaction={decorate}] (  0,   0) -- (\Xc, \Yc);
            \draw[postaction={decorate}] (  0,   0) -- (\Xd, \Yd);
            \draw[postaction={decorate}] (  0,   0) -- (\Xe, \Ye);
            \draw[postaction={decorate}] (  0,   0) -- (\Xf, \Yf);
            \draw[postaction={decorate}] (  0,   0) -- (\Xg, \Yg);
        \end{scope}

        \particle{5}{\Xa}{\Ya}
        \particle{3}{\Xb}{\Yb}
        \particle{1}{\Xc}{\Yc}
        \particle{2}{\Xd}{\Yd}
        \particle{3}{\Xe}{\Ye}
        \particle{1}{\Xf}{\Yf}
        \particle{1}{\Xg}{\Yg}

        \bang{9}

        \draw[draw=white,fill=white] (0, -4*\P) circle (\R);

    \end{tikzpicture}
}

\newcommand{\illustrateBouncing}[0]{
    \begin{tikzpicture}

        \def\R{0.2}
        \def\P{1.1}

        \def\Xa{-1*\P}
        \def\Ya{+3*\P}
        \def\Xb{+1*\P}
        \def\Yb{+3*\P}
        \def\Xc{-1*\P}
        \def\Yc{-3*\P}
        \def\Xd{+1*\P}
        \def\Yd{-3*\P}

        \begin{scope}[decoration={
            markings,
            mark=at position 0.5 with {\arrow{latex}}}
        ]
            \draw[postaction={decorate}] (\Xa, \Ya) -- (  0,   0);
            \draw[postaction={decorate}] (\Xb, \Yb) -- (  0,   0);
            \draw[postaction={decorate}] (  0,   0) -- (\Xc, \Yc);
            \draw[postaction={decorate}] (  0,   0) -- (\Xd, \Yd);
        \end{scope}

        \particle{3}{\Xa}{\Ya}
        \particle{1}{\Xb}{\Yb}
        \particle{3}{\Xc}{\Yc}
        \particle{1}{\Xd}{\Yd}

        \bang{13}

        \draw[draw=white,fill=white] (0, -4*\P) circle (\R);

    \end{tikzpicture}
}

\begin{figure}[h!]
    \centering
    \makebox[\textwidth]{
        \begin{minipage}{.33\paperwidth}
            \centering
          	\subfloat[Coagulation]{  % TODO Rename?
                \label{:a}
                \illustrateCoagulation
          	}
        \end{minipage}%
        \begin{minipage}{.33\paperwidth}
            \centering
          	\subfloat[Fragmentation]{
                \label{:b}
                \illustrateFragmentation
          	}
        \end{minipage}
        \begin{minipage}{.33\paperwidth}
            \centering
          	\subfloat[Bouncing]{
                \label{:b}
                \illustrateBouncing
          	}
        \end{minipage}%
    }
    \caption{Illustration of Dust Particle Collision Outcomes}
\end{figure} 

        \clearpage

        Assuming a hard decision boundary, the fragmentation probability $P^\text{frag}_{ij}$ 
        for a collision involving two particles of masses $m_i$ and $m_j$ could be expressed as
        \begin{equation}
            P^\text{frag}_{ij}
            =
            \begin{cases}
                1 & \text{if}\ \Delta v_{ij} >= v_\text{frag}\\
                0 & \text{else}
            \end{cases}
        \end{equation}

        A more elegant way though is given by in the 2022 paper by Stammler and Birnstiel 
        \cite{stammler_birnstiel_2022}. Here, a correction to the fragmentation probability is 
        made using the Maxwell-Boltzmann distribution. \\

        Adopting this method, we define
        \begin{equation}
            x:=\frac{3}{2}\cdot\bigg(
                \frac{v_\text{frag}}{\Delta v_{ij}}
            \bigg)^2
        \end{equation}
        and, with this, the fragmentation probability 
        \begin{equation}
            P^\text{frag}_{ij}=(1+x)\cdot e^{-x}
        \end{equation}

        The resulting probability distributions for the first method and the second, more sensible 
        one, are displayed in \cref{fig:fragmentation_probability}.

        % \newpage
\vfill

\begin{figure}[h!]
    \centering
    % \begin{minipage}{.5\linewidth}
    %   \centering
    %   \subfloat[]{
    %     \label{:a}
    %     \includegraphics[width=\linewidth]{24/coagulation_probability_from_cutoff_velocity.pdf}
    %   }
    % \end{minipage}%
    \begin{minipage}{.5\linewidth}
      \centering
      \subfloat[]{
        \label{:b}
        \includegraphics[width=\linewidth]{24/fragmentation_probability_from_cutoff_velocity.pdf}
      }
    \end{minipage}%
    % \begin{minipage}{.5\linewidth}
    %   \centering
    %   \subfloat[]{
    %     \label{:a}
    %     \includegraphics[width=\linewidth]{24/coagulation_probability_from_Maxwell-Boltzmann.pdf}
    %   }
    % \end{minipage}%
    \begin{minipage}{.5\linewidth}
      \centering
      \subfloat[]{
        \label{:b}
        \includegraphics[width=\linewidth]{24/fragmentation_probability_from_Maxwell-Boltzmann.pdf}
      }
    \end{minipage}
    \caption{
        Dust Particle Collision Outcome Probabilities.\\
        \todo{Why not a perfect square in 1st case?}
    }
    \label{}
\end{figure}

\newpage


        % \todo{...}
        % \clearpage

        % Since the kinematics of circumstellar dust particles are influenced by both systematic and 
        % stochastic contributions (e.g. Keplerian and Brownian motion, respectively), occasional
        % particle collisions are to be expected. At a given time $t$ and position $\vec r$ in the
        % disk, let us write $$R^\text{coll}(m,m')$$ to denote the rate of collisions between two
        % particles carrying the masses $m$ and $m'$, respectively. \\

        % \todo{This rate will be defined further in... [ref. chapter]} \\
        
        % The outcome of a collision between such a particle pair depends (\todo{among other things})
        % on the relative velocity between the two particles, and can be classified into 3 distinct 
        % categories, namely (pure hit-and-stick) coagulation, fragmentation, and bouncing.\\

        % \todo{They will be defined in... [ref. chapter]} \\ 
        
        % to denote the probability that such a collision will lead to a 
        % merging of the two collision partners into a single new particle. Analogously, 
        % labels that rate of collisions leading to fragmentation of the two 
        % particles into a range of newly created, smaller particles.
        % TODO "newly created" or "newly-created" ?
        % $R^\text{coag}(m,m',\vec r,t)$ label the rate of 
        % collisions leading to the merging of two particles carrying the masses $m$ and $m'$, respectively. 
        % Analogously, let $R^\text{frag}(m,m',\vec r,t)$ be the rate of collisions leading to particle 
        % fragmentation.\\
        
        % \todo{Line plot: outcome probability vs. particle radius}
        
    % }}}

% }}}
