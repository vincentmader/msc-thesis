% Dust Particle Kinematics {{{ 
\clearpage\section{Dust Particle Kinematics}

    The kinematics of dust particles in the disk are relevant to our model
    because the construction of the coagulation kernel requires the 
    definition of the collision rates between the dust particle populations 
    (each encoding a certain particle mass and radius).
    The collision rates in turn depend on the definition of the
    relative velocities between the two particles involved in a given collision. \\

    We include four processes in the calculation of the relative dust particle velocities, namely 
    the two systematic contributions of radial and azimuthal drift, and the two stochastic 
    contributions of Brownian and turbulent motion. \\

    We will start with the definition of the systematic contributions, then move on to define 
    the stochastic contributions, and finally combine the four different effects into the 
    total relative velocity value $\Delta v_{ij}$ for a given pair of particles from 
    bin $i$ and $j$, respectively. \\

    For the definition of the radial and azimuthal velocities, as well as for Brownian motion, 
    we closely follow the work done by \cite{weidenschilling_1977}, \cite{nakagawa_1986}, 
    \cite{birnstiel_dullemond_brauer_2010}, and \cite{dullemond_dominik_2004}.

    % Radial Drift {{{ 
    \subsection{Radial Drift}

        Consider a particle suspended in a fluid flow. The ratio between the particle's 
        characteristic time scale and the characteristic time scale of the fluid flow
        is often referred to as \textit{Stokes' number}, named after George Gabriel Stokes. 
        In our model, it is given by
        \begin{equation}
            \text{St}_i=\frac{\rho_s\cdot a_i}{\Sigma_g\cdot\frac{\pi}{2}}
        \end{equation}
        % \todo{Talk about Stokes' numbers.}
        % \todo{Epstein regime} \\
        % TODO "Stokes number" vs. "Stokes' number"
        % Small particles have small Stokes numbers.

        Following the work done by \cite{birnstiel_dullemond_brauer_2010}, we will model the
        radial velocity $u_{r,i}$ of particles in the dust population corresponding to 
        a given mass bin $i$ as
        \begin{equation}
            \label{eq:radial_dust_velocity}
            u_{r,i}
            =\frac{u_{g}}{1+\text{St}_i^2}-\frac{2u_{n}}{\text{St}_i+\text{St}_i^{-1}}
        \end{equation}

        Here, the first term in the above equation is the \textit{drag term}. 
        The gas moves along 
        the radial axis with a radial velocity $u_g$
        which we defined in \cref{eq:radial_gas_velocity}.
        The value of the Stokes number affects 
        the extent to which the dust particles are coupled to the gas. In the case of a low 
        Stokes number, the dust is dragged along with the gas. \\

        The second term is related to the fact that due to the disk's own gas pressure 
        gradient, the gas in the disk orbits the disk center on a sub-keplerian 
        trajectory. Heavier particles do not feel this pressure as much and 
        thus move keplerian. As a result, there is a difference in the radial velocity
        of the dust particle and the gas through which it moves. Slowed down by the gas, 
        the particle loses momentum and starts drifting inwards. 
        The term $u_n$ labels the maximum drift velocity of a particle, which was derived 
        by \cite{weidenschilling_1977}, and is given by
        \begin{equation}
            u_{n}
            =-\pderiv{P_g}{r}\cdot\frac{E_d}{2\rho_g\Omega_K}
        \end{equation}
        Here, the drift efficiency parameter $E_d$ is set to 1 for simplicity. \\

        % \todo{particles of different size are coupled differently to the gas}
        The relative velocity is then given by
        \begin{equation}
            \Delta v_{ij}^\text{RD}
            =|u_{r,i}-u_{r,j}|
        \end{equation}
    
        See \hyperref[fig:relative_dust_particle_velocities]
        {\cref*{fig:relative_dust_particle_velocities}a}.

    % }}}
    % Azimuthal Motion {{{ 
    \subsection{Azimuthal Motion}

        Following \cite{birnstiel_dullemond_brauer_2010}, where 
        According to \cite{weidenschilling_1977} and \cite{nakagawa_1986}, the relative azimuthal 
        velocity $\Delta v^\text{AZ}_{ij}$ for gas-dominated drag between a pair of particles 
        $(i, j)$ can be written as
        \begin{equation}
            \Delta v^\text{AZ}_{ij}=\bigg|
                u_n\cdot\bigg(
                    \frac{1}{1+\text{St}_i^2}+
                    \frac{1}{1+\text{St}_j^2}
                \bigg)
            \bigg|
        \end{equation}

    
        For a visualization, see \hyperref[fig:relative_dust_particle_velocities]
        {\cref*{fig:relative_dust_particle_velocities}b}.

    % }}}
    % Differential Settling {{{ 
    % \subsection{Differential Settling}
    % }}}
    % Brownian Motion {{{ 
    \subsection{Brownian Motion}

        Here, we follow the work of \cite{dullemond_dominik_2004}, where the contribution to 
        relative velocity due to Brownian motion is given by
        \begin{equation}
            \Delta v_{ij}^\text{BR}
            =\sqrt{\frac{8k_BT}{\pi}\cdot\frac{m_i+m_j}{m_i\cdot m_j}}
        \end{equation}
        Note that this term will be the largest when both collision partners carry a very low mass.
        Brownian motion thus favors collisions in which at least one collision partner is a 
        low-mass particles. This can be seen reflected in 
        \hyperref[fig:relative_dust_particle_velocities]{
        \cref*{fig:relative_dust_particle_velocities}c}.

    % }}}
    % Turbulent Motion {{{ 
    \subsection{Turbulent Motion}

        For the calculation of the relative velocity contributions due to turbulent motion, we 
        adapt the work of \cite{ormel_cuzzi_2007}.
        There, relative velocity $\Delta v^\text{TU}_{ij}$ due to turbulence was defined as
        \begin{equation}
            \Delta v^\text{TU}_{ij} =
            \begin{cases}
                \Delta v_{ij}^\text{A} & \text{if } t_{sm} \leq t_s \\
                \Delta v_{ij}^\text{B} & \text{if } t_s < t_{sm} \leq t_n \\
                \Delta v_{ij}^\text{C} & \text{if } t_{sm} > t_n \\
            \end{cases}
        \end{equation}

        For this definition, three different cases have to be handled separately.

        \begin{enumerate}

            \item Limit of tightly coupled particles (below the Kolmogorov scale) \\
                See Section 3.4.1 of OC2007, their Eq. 26
                % \begin{equation}
                %     \Delta v_{ij}^1 
                %     = v_g \cdot \sqrt{\frac{t_1 - t_2}{t_1 + t_2} \cdot (\Delta u_1 - \Delta u_2)}
                % \end{equation}
                % with the two "velocities" $u_1$ and $u_2$ given by
                % \begin{equation}
                %     \Delta u_{1,2} 
                %     = \left(\frac{t_{1,2}}{t_n}\right)^2 \cdot 
                %         \left(\frac{t_{1,2}}{t_n} + \frac{1}{\sqrt{\text{Re}} }\right)^{-1} \\
                % \end{equation}
                \begin{equation}
                    \Delta v_{ij}^\text{A}
                    = v_g \cdot \sqrt{
                        \frac{\text{St}_i - \text{St}_j}{\text{St}_i + \text{St}_j} \cdot 
                        \left(
                            \frac{\text{St}_i^2}{\text{St}_i + \text{Re}^{-1/2}} -
                            \frac{\text{St}_j^2}{\text{St}_j + \text{Re}^{-1/2}}
                        \right)
                    }
                \end{equation} 

            \item At least one of the particles is not in the tightly coupled regime. \\
                Both particles have $\text{St} < 1$. \\
                See Section 3.4.2 of OC2007, their Eq. 28
                \begin{equation}
                    \Delta v_{ij}^\text{B} = v_g \cdot \sqrt{
                        \text{St}_i \cdot \left[
                            2y_a - (1+\varepsilon) + \frac{2}{1+\varepsilon} \cdot \left(
                                \frac{1}{1+y_a} + \frac{\varepsilon^3}{y_a+\varepsilon}
                            \right)
                        \right] 
                    }
                \end{equation}
                where $\varepsilon < 1$ is the ratio of the stopping times 
                and $y_a = 1.6$ (\todo{is what?}).
                % \begin{equation}
                %     \varepsilon = \frac{\min(t_1, t_2)}{\max(t_1, t_2)} < 1
                % \end{equation}
                % Ratio of St_smallest/St_largest
                % St1 = St_largest
                % \begin{equation}
                %     \Delta v_{ij}^\text{B} = v_g \cdot \sqrt{\Delta u_2 \cdot \text{St}_i}
                % \end{equation}
                % with 
                % \begin{align}
                %     \Delta u_1 
                %         &= \frac{1}{1 + 1.6} + \frac{\varepsilon^3}{1.6 + \varepsilon} \\
                %     \Delta u_2 
                %         &= 3.2 - (1 + \varepsilon) + \frac{2}{1 + \varepsilon} \cdot \Delta u_1
                % \end{align}
                % with 
                % and
                % \begin{equation}
                %     \text{St}_1 = \frac{\max(t_1, t_2)}{t_n}
                % \end{equation}

            \item At least one of the particles is not in the tightly coupled regime. \\ 
                At least one of the particles has $\text{St} \geq 1$. \\
                See Section 3.4.3 of OC2007, their Eq. 29
                \begin{equation}
                    \Delta v_{ij}^\text{C} = v_g \cdot \sqrt{
                        \frac{1}{1+\text{St}_i} + \frac{1}{1+\text{St}_j}
                    }
                \end{equation}
                % \begin{equation}
                %         \frac{t_n}{t_n + t_1} + \frac{t_n}{t_n + t_2}
                % \end{equation}

        \end{enumerate}

        \todo{Define $v_g$}. \\
        \todo{Define $\text{Re}$}. \\

        For a visualization of the results, see \hyperref[fig:relative_dust_particle_velocities]
        {\cref*{fig:relative_dust_particle_velocities}d}.

    % }}}
    % Total Relative Velocity {{{ 
    \subsection{Total Relative Velocity}

        The total relative dust particle velocity can be expressed as the root mean square (RMS) of 
        the individual relative velocity contributions:
        \begin{equation}
            \Delta v_{ij}
                = \sqrt{
                    \big(\Delta v^\text{RD}_{ij}\big)^2
                    + \big(\Delta v^\text{AZ}_{ij}\big)^2
                    + \big(\Delta v^\text{BR}_{ij}\big)^2
                    + \big(\Delta v^\text{TU}_{ij}\big)^2
                }
        \end{equation}
        A visualization of the total relative velocity's dependence on the involved masses of the
        two colliding particles can be seen in \cref{fig:total_relative_dust_particle_velocity}.

        \vfill

% \begin{figure}[h!]
%     \begin{center}
%         \includegraphics[width=.5\linewidth]{21/dv_tot.pdf}
%     \end{center}
%     \caption{Total Relative Dust Particle Velocity}
%     \label{fig:total_relative_dust_particle_velocity}
% \end{figure}

\begin{figure}[h!]
    \makebox[\textwidth]{
        \includegraphics[width=\paperwidth]{21/dv_tot_lin+log.pdf}
    }
    \caption{Total Relative Dust Particle Velocity $\Delta v_{ij}^\text{tot}$}
    \label{fig:total_relative_dust_particle_velocity}
\end{figure}

        \vfill

\begin{figure}[h!]
    \begin{center}
        \includegraphics[width=.5\linewidth]{21/dv_tot.pdf}
    \end{center}
    \caption{Total Relative Dust Particle Velocity}
    \label{fig:total_relative_dust_particle_velocity}
\end{figure}

\newpage
\vfill

\begin{figure}[h!]
    \centering
    \begin{minipage}{.5\linewidth}
      \centering
      \subfloat[Relative velocity due to radial Drift]{
        \label{:a}
        \includegraphics[width=\linewidth]{21/dv_RD.pdf}
      }
    \end{minipage}%
    \begin{minipage}{.5\linewidth}
      \centering
      \subfloat[Relative velocity due to azimuthal Motion]{
        \label{:b}
        \includegraphics[width=\linewidth]{21/dv_AZ.pdf}
      }
    \end{minipage}
    \begin{minipage}{.5\linewidth}
      \centering
      \subfloat[Relative velocity due to Brownian Motion]{
        \label{:c}
        \includegraphics[width=\linewidth]{21/dv_BR.pdf}
      }
    \end{minipage}%
    \begin{minipage}{.5\linewidth}
      \centering
      \subfloat[Relative velocity due to turbulence]{
        \label{:d}
        \includegraphics[width=\linewidth]{21/dv_TU.pdf}
      }
    \end{minipage}
    \caption{Relative Dust Particle Velocities due to radial drift (a), azimuthal motion (b), 
        Brownian motion (c), and turbulence (d).}
    \label{fig:relative_dust_particle_velocities}
    % TODO Assure consistent capitalization in captions.
\end{figure} \ \\


    % }}}

% }}}
% Dust Particle Collisions {{{ 
\clearpage\section{Dust Particle Collisions}

    % Collision Cross Section {{{ 
    \subsection{Collision Cross Section}
        
        As was mentioned in \cref{sec:dust_particle_mass_distribution}, in the context of 
        this thesis we will assume the dust particles to possess a perfectly spherical shape. 
        Under this assumption, the cross section $\sigma_{ij}$ for a collision of two dust 
        particles with radii $a_i$ and $a_j$ can be approximated by the area of a circle the 
        radius of which equals the sum of the two particles' radii. \\

        Therefore, the collision cross section can be written as:
        \begin{equation}
            \sigma_{ij} = \pi \cdot (a_i+a_j)^2
        \end{equation}

    % }}}
    % Collision Rate {{{ 
    \subsection{Collision Rate}

        The rate $R^\text{coll}_{ij}$ of collisions between pairs of particles from bins $i$ 
        and $j$ is given by the product of the corresponding collision cross section, the
        particles' relative velocity, and the number density of particles at the considered 
        location in the disk. 
        \begin{equation}                                            % ^ TODO: "considered?"
            R^\text{coll}_{ij}
                \approx \sigma_{ij} \cdot \Delta v_{ij}
        \end{equation}

        Note: The units of this collision rate are not $\text{s}^{-1}$! Instead, this rate is 
        to be interpreted as a collision rate "per particle density", with units of 
        $\text{m}^{3} \text{s}^{-1}$.

        \vfill

\begin{figure}[h!]
    \centering
    \begin{minipage}{.5\linewidth}
      \centering
      % \subfloat[Dust Particle Collision Cross Section]{
      \subfloat[Dust Particle Collision Cross Section]{
        \label{:a}
        \includegraphics[width=\linewidth]{22/collision_cross_section.pdf}
      }
    \end{minipage}%
    \begin{minipage}{.5\linewidth}
      \centering
      % \subfloat[Dust Particle Collision Rate]{
      \subfloat[Dust Particle Collision Rate]{
        \label{:b}
        \includegraphics[width=\linewidth]{23/collision_rate.pdf}
      }
    \end{minipage}
    \caption{
        (a) Dust Particle Collision Cross Section $\sigma_{ij}$ and 
        (b) Dust Particle Collision Rate $R_{ij}^\text{coll}$
    }
\end{figure}

        % TODO Define `minipage` plot with both coll. cross section & coll. rate.

    % }}}
    % Collision Outcomes {{{ 
    \subsection{Collision Outcomes}

        Depending on the relative velocity of the two particles involved in a collision,
        % TODO Note somewhere that it's ALWAYS a two-particle interaction.
        we will differentiate between three different collision outcome scenarios:
        \begin{enumerate}
            \item If the relative velocity is low, the two particles will merge into a single one.
                \\ (\todo{pure "hit-and-stick" coagulation})
            \item \todo{Bouncing?}
            \item If the relative velocity is large enough, the collision will lead to a shattering 
                (fragmenting) of the two particles.
        \end{enumerate}

        \todo{For each of the three possible outcomes, we will define an associated probability.} \\
        \todo{Let $P^\text{coag}$, $P^\text{frag}$, and $P^\text{bounce}$
        label the probabilities that a collision will lead to the respective outcome.} \\

        \todo{The rates at which such events occur are then given by}
        \begin{align}
            R^\text{coag}(m,m')   &= R^\text{coll}(m,m') \cdot P^\text{coag}(m,m') \\
            R^\text{frag}(m,m')   &= R^\text{coll}(m,m') \cdot P^\text{frag}(m,m') \\
            R^\text{bounce}(m,m') &= R^\text{coll}(m,m') \cdot P^\text{bounce}(m,m')
        \end{align}

        \todo{In our case: Assume $P^\text{bounce}=0$.}

        \vfill

\newcommand{\particle}[3]{%
    \ifcase#1
        % Zero
    \or
        \draw[fill={black!0!white}] (#2          , #3         ) circle (\R);
    \or
        \draw[fill={black!0!white}] (#2 - 0.65*\R, #3 + 0.30*\R) circle (\R);
        \draw[fill={black!0!white}] (#2 + 0.65*\R, #3 - 0.30*\R) circle (\R);
    \or
        \draw[fill={black!0!white}] (#2 - 0.70*\R, #3 - 0.50*\R) circle (\R);
        \draw[fill={black!0!white}] (#2 + 0.00*\R, #3 + 0.50*\R) circle (\R);
        \draw[fill={black!0!white}] (#2 + 0.70*\R, #3 - 0.50*\R) circle (\R);
    \or
        \draw[fill={black!0!white}] (#2 + 0.65*\R, #3 + 0.65*\R) circle (\R);
        \draw[fill={black!0!white}] (#2 - 0.65*\R, #3 - 0.65*\R) circle (\R);
        \draw[fill={black!0!white}] (#2 + 0.65*\R, #3 - 0.65*\R) circle (\R);
        \draw[fill={black!0!white}] (#2 - 0.65*\R, #3 + 0.65*\R) circle (\R);
    \or
        \draw[fill={black!0!white}] (#2 + 0.85*\R, #3 + 0.85*\R) circle (\R);
        \draw[fill={black!0!white}] (#2 - 0.85*\R, #3 - 0.85*\R) circle (\R);
        \draw[fill={black!0!white}] (#2 + 0.85*\R, #3 - 0.85*\R) circle (\R);
        \draw[fill={black!0!white}] (#2 - 0.85*\R, #3 + 0.85*\R) circle (\R);
        \draw[fill={black!0!white}] (#2 +       0, #3 +       0) circle (\R);
    \fi
}

\newcommand{\bang}[1]{
    \node[shape=starburst, starburst points=#1, starburst point height=0.5cm, 
        draw, black, fill=white,
        minimum width=1.5cm, minimum height=1.5cm,
    ] {};
}

\newcommand{\illustrateCoagulation}[0]{
    \begin{tikzpicture}

        \def\R{0.2}
        \def\P{1.1}

        \def\Xa{-1*\P}
        \def\Ya{+3*\P}
        \def\Xb{+1*\P}
        \def\Yb{+3*\P}
        \def\Xc{+0*\P}
        \def\Yc{-3*\P}

        \begin{scope}[decoration={
            markings,
            mark=at position 0.5 with {\arrow{latex}}}
        ]
            \draw[postaction={decorate}] (\Xa, \Ya) -- (  0,   0);
            \draw[postaction={decorate}] (\Xb, \Yb) -- (  0,   0);
            \draw[postaction={decorate}] (  0,   0) -- (\Xc, \Yc);
        \end{scope}

        \particle{3}{\Xa}{\Ya}
        \particle{1}{\Xb}{\Yb}
        \particle{4}{\Xc}{\Yc}

        \bang{10}

        \draw[draw=white,fill=white] (0, -4*\P) circle (\R);

    \end{tikzpicture}
}

\newcommand{\illustrateFragmentation}[0]{
    \begin{tikzpicture}

        \def\R{0.2}
        \def\P{1.1}

        \def\Xa{-1.0*\P}
        \def\Ya{+3.0*\P}
        \def\Xb{+1.0*\P}
        \def\Yb{+3.0*\P}
        \def\Xc{-2.0*\P}
        \def\Yc{-2.8*\P}
        \def\Xd{-1.0*\P}
        \def\Yd{-3.1*\P}
        \def\Xe{+0.0*\P}
        \def\Ye{-3.3*\P}
        \def\Xf{+1.0*\P}
        \def\Yf{-3.1*\P}
        \def\Xg{+2.0*\P}
        \def\Yg{-2.8*\P}

        \begin{scope}[decoration={
            markings,
            mark=at position 0.5 with {\arrow{latex}}}
        ]
            \draw[postaction={decorate}] (\Xa, \Ya) -- (  0,   0);
            \draw[postaction={decorate}] (\Xb, \Yb) -- (  0,   0);
            \draw[postaction={decorate}] (  0,   0) -- (\Xc, \Yc);
            \draw[postaction={decorate}] (  0,   0) -- (\Xd, \Yd);
            \draw[postaction={decorate}] (  0,   0) -- (\Xe, \Ye);
            \draw[postaction={decorate}] (  0,   0) -- (\Xf, \Yf);
            \draw[postaction={decorate}] (  0,   0) -- (\Xg, \Yg);
        \end{scope}

        \particle{5}{\Xa}{\Ya}
        \particle{3}{\Xb}{\Yb}
        \particle{1}{\Xc}{\Yc}
        \particle{2}{\Xd}{\Yd}
        \particle{3}{\Xe}{\Ye}
        \particle{1}{\Xf}{\Yf}
        \particle{1}{\Xg}{\Yg}

        \bang{9}

        \draw[draw=white,fill=white] (0, -4*\P) circle (\R);

    \end{tikzpicture}
}

\newcommand{\illustrateBouncing}[0]{
    \begin{tikzpicture}

        \def\R{0.2}
        \def\P{1.1}

        \def\Xa{-1*\P}
        \def\Ya{+3*\P}
        \def\Xb{+1*\P}
        \def\Yb{+3*\P}
        \def\Xc{-1*\P}
        \def\Yc{-3*\P}
        \def\Xd{+1*\P}
        \def\Yd{-3*\P}

        \begin{scope}[decoration={
            markings,
            mark=at position 0.5 with {\arrow{latex}}}
        ]
            \draw[postaction={decorate}] (\Xa, \Ya) -- (  0,   0);
            \draw[postaction={decorate}] (\Xb, \Yb) -- (  0,   0);
            \draw[postaction={decorate}] (  0,   0) -- (\Xc, \Yc);
            \draw[postaction={decorate}] (  0,   0) -- (\Xd, \Yd);
        \end{scope}

        \particle{3}{\Xa}{\Ya}
        \particle{1}{\Xb}{\Yb}
        \particle{3}{\Xc}{\Yc}
        \particle{1}{\Xd}{\Yd}

        \bang{13}

        \draw[draw=white,fill=white] (0, -4*\P) circle (\R);

    \end{tikzpicture}
}

\begin{figure}[h!]
    \centering
    \makebox[\textwidth]{
        \begin{minipage}{.33\paperwidth}
            \centering
          	\subfloat[Coagulation]{  % TODO Rename?
                \label{:a}
                \illustrateCoagulation
          	}
        \end{minipage}%
        \begin{minipage}{.33\paperwidth}
            \centering
          	\subfloat[Fragmentation]{
                \label{:b}
                \illustrateFragmentation
          	}
        \end{minipage}
        \begin{minipage}{.33\paperwidth}
            \centering
          	\subfloat[Bouncing]{
                \label{:b}
                \illustrateBouncing
          	}
        \end{minipage}%
    }
    \caption{Illustration of Dust Particle Collision Outcomes}
\end{figure} 


        \clearpage

        \todo{Threshold value for relative velocity:}
        \begin{equation}
            v_\text{threshold}:=\SI{1}{\meter\second^{-1}}
        \end{equation}

        \begin{equation}
            P^\text{frag}_{ij}
            =
            \begin{cases}
                1 & \text{if}\ \Delta v_{ij} >= v_\text{threshold}\\
                0 & \text{else}
            \end{cases}
        \end{equation}
        
        Correction using Maxwell-Boltzmann distribution: 
        (\cite{stammler_birnstiel_2022})
        \begin{equation}
            x:=\frac{3}{2}\cdot\bigg(
                \frac{v_\text{frag}}{\Delta v_{ij}}
            \bigg)^2
        \end{equation}
        
        \begin{equation}
            P^\text{frag}_{ij}=(1+x)\cdot e^{-x}
        \end{equation}

        \vfill

\begin{figure}[h!]
    \centering
    % \begin{minipage}{.5\linewidth}
    %   \centering
    %   \subfloat[]{
    %     \label{:a}
    %     \includegraphics[width=\linewidth]{24/coagulation_probability_from_cutoff_velocity.pdf}
    %   }
    % \end{minipage}%
    \begin{minipage}{.5\linewidth}
      \centering
      \subfloat[]{
        \label{:b}
        \includegraphics[width=\linewidth]{24/fragmentation_probability_from_cutoff_velocity.pdf}
      }
    \end{minipage}%
    % \begin{minipage}{.5\linewidth}
    %   \centering
    %   \subfloat[]{
    %     \label{:a}
    %     \includegraphics[width=\linewidth]{24/coagulation_probability_from_Maxwell-Boltzmann.pdf}
    %   }
    % \end{minipage}%
    \begin{minipage}{.5\linewidth}
      \centering
      \subfloat[]{
        \label{:b}
        \includegraphics[width=\linewidth]{24/fragmentation_probability_from_Maxwell-Boltzmann.pdf}
      }
    \end{minipage}
    \caption{
        Dust Particle Fragmentation Probability $P_{ij}^\text{frag}$.
        % \todo{Why not a perfect square in 1st case?}
    }
\end{figure}

\clearpage


        % \todo{...}
        % \clearpage

        % Since the kinematics of circumstellar dust particles are influenced by both systematic and 
        % stochastic contributions (e.g. Keplerian and Brownian motion, respectively), occasional
        % particle collisions are to be expected. At a given time $t$ and position $\vec r$ in the
        % disk, let us write $$R^\text{coll}(m,m')$$ to denote the rate of collisions between two
        % particles carrying the masses $m$ and $m'$, respectively. \\

        % \todo{This rate will be defined further in... [ref. chapter]} \\
        
        % The outcome of a collision between such a particle pair depends (\todo{among other things})
        % on the relative velocity between the two particles, and can be classified into 3 distinct 
        % categories, namely (pure hit-and-stick) coagulation, fragmentation, and bouncing.\\

        % \todo{They will be defined in... [ref. chapter]} \\ 
        
        % to denote the probability that such a collision will lead to a 
        % merging of the two collision partners into a single new particle. Analogously, 
        % labels that rate of collisions leading to fragmentation of the two 
        % particles into a range of newly created, smaller particles.
        % TODO "newly created" or "newly-created" ?
        % $R^\text{coag}(m,m',\vec r,t)$ label the rate of 
        % collisions leading to the merging of two particles carrying the masses $m$ and $m'$, respectively. 
        % Analogously, let $R^\text{frag}(m,m',\vec r,t)$ be the rate of collisions leading to particle 
        % fragmentation.\\
        
        % \todo{Line plot: outcome probability vs. particle radius}
        
    % }}}

% }}}



        % TODO I already talked about this above.
        % TODO: Write "time scale" or "time-scale" or "timescale"?
        % TODO Write "Stokes" or "Stokes'"?
