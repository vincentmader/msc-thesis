\section{Continuous Formulation}

    \todo{Smol. eq. looks very similar to coag., but $f$ instead of $\delta_d$, can't "reduce".} \\
    \todo{That's where the computational cost comes from!} \\

    \todo{Is the below correct?}
    \begin{align}
        L_\text{coag}(m)
            &= \int\limits_0^\infty R_\text{coll}(m, m') \cdot n(m) \cdot n(m') \ \dd m \ \dd m' \\
        G_\text{coag}(m)
            &= \int\limits_0^\infty \int\limits_0^\infty R_\text{coll}(m', m'') \cdot
                n(m') \cdot n(m'') \cdot f(m,m',m'') \ \dd m' \ \dd m'' 
    \end{align}

    \todo{For the loss:}
    \begin{itemize}
        \item Same as in coagulation.
        \item If two particles with masses $m$ and $m'$ collide, these two particles "disappear"
              from the distribution.
        \item Instead, new particles have to be added.
        \item Once again, there is the criterion of mass conservation.
    \end{itemize}

    \todo{For the gain:}
    \begin{itemize}
        \item Looks very similar to coagulation.
        \item But: Here we have the term $f(m,m',m'')$ in the integral.
        \item In coagulation, this term is equal to $\delta_D(m-m'-m'')$.
        \item If two masses fuse, there is only a single possible mass value that can result from 
              that, without breaking mass conservation.
        \item In fragmentation, this is different.
        \item If two particles collide, they can in principle fragment into a whole range of 
              differently sized particles, as long as mass is conserved.
    \end{itemize}

\section{Discretized Formulation}
