\todo{Plot top-down view of the disk.} \\

\assume{Gas is in hydrostatic equilibrium.} \\
\assume{Gas behaves isothermically.}

\section{A Single Gas Particle}

    \assume{Gas mean particle mass.}
    \begin{equation}
        m_g
            := 2.3 \cdot m_p
    \end{equation}
    Here, $m_p$ labels the mass of a single proton.

\newpage\section{Radial Gas Density Profile}

    Our next task will be to formulate a simple model for the spatial distribution of the disk's 
    gaseous contents. In this context, we will adopt cylindrical coordinates $(r, \varphi, z)$ and
    define both the surface and volume density profiles along the radial axis $r$.

    % Surface Density Profile {{{ 
    \subsection{Surface Density Profile}

        The gas surface density $\Sigma_g(r,\varphi)$ is directly related to the gas volume density
        $\rho_g(r,\varphi,z)$, and can be derived from it by integrating the volume density from 
        $-\infty$ to $+\infty$ along the $z$-axis, i.e. along the axis perpendicular to the disk:
        \begin{equation}
            \label{eq:relationship_between_gas_surface_density_and_gas_volume_density}
            \Sigma_g(r,\varphi)
                = \int\limits_{-\infty}^{\infty} \rho_g(r,\varphi,z)\ \dd z
        \end{equation}

        \assume{As noted before in (cite), we will assume the disk to be perfectly radially 
        symmetric.} \\
        For the density distributions, this means that
        \begin{align}
            \Sigma_g(r, \varphi) 
                &= \Sigma_g(r) 
            \ \ \ \text{and}\ \ \
            \\
            \rho_g(r, \varphi) 
                &= \rho_g(r)
        \end{align}
        Thus, we can simplify 
        \cref{eq:relationship_between_gas_surface_density_and_gas_volume_density} and write:
        \begin{equation}
            \Sigma_g(r)
                = \int\limits_{-\infty}^{\infty} \rho_g(r,z)\ \dd z
        \end{equation}
        
        \todo{So far, we do not yet have an expression for the gas volume density $\rho_g$.}
        Therefore, we can not derive the surface density $\Sigma_g$ by evaluating the above
        integral. Instead, we will go the other way round by making an ansatz for the surface gas
        density and then derive the volume density from that. \\

        In order to do this, we will 
        follow the work of Brauer, Dullemond, \& Henning \cite{brauer_dullemond_henning_2007},
        % Following the works of \cite{brauer_dullemond_henning_2007}
        % , \cite{andrews_2007}, and \cite{kitamura_2002}, 
        and make the assumption that the radial profile of the gas surface density can be 
        approximated by using an inverse power law of the form
        \begin{equation}
            \label{eq:gas_surface_density_profile_inverse_power_law_ansatz}
            \Sigma_g(r)=\Sigma_0\cdot\frac{1}{r^\sigma}
            \ \ \ \text{with}\ \ \
            \sigma:=0.8
        \end{equation}
        % TODO Why 0.8 ?

        \todo{Before we can use this, we need to define $\Sigma_0$}. \\

        The total gas mass present in the disk is given by $M_\text{gas}$, which we defined in 
        (\todo{cite}). \\
        \todo{Since the integration of the surface density over the entire disk surface must yield 
        this value, we can formulate the following condition:}
        \begin{align}
            M_\text{gas}
                % &= \int_{A_\text{disk}} \Sigma_g(r) \ \dd A \\
                &= \int\limits_{r_\text{min}}^{r_\text{max}} \Sigma_g(r) \cdot 2\pi r \ \dd r
        \end{align}
        % TODO Define r_min and r_max before.

        \todo{Above (cite): Discretize $r$-axis into $\mathcal N_r$ "bins"}
        \begin{equation}
            M_\text{gas}
                = \sum_{i=1}^{\mathcal N_r} \Sigma_g(r_i) \cdot 2\pi r_i \cdot \Delta r_i
        \end{equation}
        % TODO Define N_r before.

        Now we can plug in \cref{eq:gas_surface_density_profile_inverse_power_law_ansatz} and 
        rearrange for $\Sigma_0$:
        \begin{equation}
            \Sigma_0
                = M_\text{gas} \cdot \left[
                    \sum_{i=1}^{\mathcal N_m} 2\pi r_i^{1-\sigma} \cdot \Delta r_i
                \right]^{-1}
        \end{equation}
        % TODO Do this with dA instead of 4pi r dr ?

    % }}}
    % Volume Density Profile {{{ 
    \newpage\subsection{Volume Density Profile}

        \assume{Isothermal condition}
        \begin{equation}
            P = \rho_g \cdot c_s^2
        \end{equation}

        \assume{Disk is in hydrostatic equilibrium (see assumption above)}  % TODO
        \begin{equation}
            \deriv{P}{z} = - \rho_g \cdot g_z  % TODO Why `g_z` and not `g` ?
        \end{equation}

        Let us adopt cylindrical coordinates and consider a point $\vec r=(r,\phi,z)^T$ in the 
        disk. \\  % TODO Write with transposed `T` ?

        \assume{The disk is "much more wide than tall": (it is a disk, afterall...)}
        \begin{equation*}
            r\gg z 
        \end{equation*}

        \todo{Sketch disk} \\
        \vfill

\begin{figure}[h!]
    \centering
    \begin{tikzpicture}
        % Draw triangle.
        \draw (0,0) -- (12,0) -- (12, 3) -- cycle;
        % Draw angle.
        \draw (0, 0) ++ (0:5) arc (0:14:5);
        % Draw star and massive object.
        \node at (-0.2,  0) {$\star$};
        \node at (12.1, 3)    {$\cdot$};
        % Draw text.
        \node at (12.5,  1.5) {$z$};
        \node at (6,    -0.5) {$r$};
        \node at (6,     2.1) {$|\vec r|$};
        \node at (3,     0.4) {$\theta$};
        \node at (12.5,  3.1)   {$\vec r$};
    \end{tikzpicture}
    \caption{Illustration of Trigonometric relations used for expressing the $z$-component of the
        acceleration due to the gravitational influence of the star at the center of the disk.
    }
    \label{fig:trigonometry_schematic_for_gas_volume_density_plot}
\end{figure}

\clearpage


        The acceleration due to Newtonian gravity is given by
        \begin{equation}
            g=\frac{GM_*}{r^2+z^2}
        \end{equation}

        Making use of the laws of trigonometry, we can write
        \begin{equation}
            g_z=g\cdot\sin\theta
        \end{equation}
        and
        \begin{equation}
            \sin\theta=\frac{z}{\sqrt{r^2+z^2}}
        \end{equation}

        Plugging all of this into [cite: above], we can write
        \begin{align}
            \deriv{P}{z}
                &= -\rho_g \cdot g_z \\
                &= -\rho_g \cdot g \cdot \sin\theta \\
                &= -\rho_g \cdot \frac{GM_*}{r^2+z^2} \cdot\frac{z}{\sqrt{r^2+z^2}} \\
                &= -\rho_g \cdot z \cdot\frac{GM_*}{ (r^2+z^2)^{\frac{3}{2}} } \\
            \Rightarrow
            \deriv{P}{z}
                &\approx -\rho_g \cdot z \cdot \Omega_K^2
        \end{align}

        For the last step, we made use of the fact that $r \gg z$, which allows us to make the 
        approximation
        \begin{equation}
            \frac{GM_*}{(r^2+z^2)^\frac{3}{2}}
                \approx \frac{GM_*}{r^3}
                = \Omega_K^2
        \end{equation}

        \todo{Differentiate isothermal condition with respect to $z$:}
        \begin{equation}
            \deriv{P}{z}
                = \deriv{\rho_g}{z}\cdot c_s^2
        \end{equation}

        \todo{We now have two expressions for $\pderiv{P}{z}$}, in [cite eq 1] and [cite eq 2]. \\
        \todo{Equating both of these and solving for $\pderiv{\rho_g}{z}$ leads to the following 
        differential equation:}
        % TODO Is it an ODE or a PDE ? Should I write `dy/dx` here, or `del y / del x` ?
        \begin{equation}
            \deriv{\rho_g}{z}
                = - \left(\frac{\Omega_K}{c_s}\right)^2 \cdot \rho_g \cdot z
        \end{equation}

        This differential equation can be solved by making the ansatz of a Gaussian distribution
        \begin{equation}
            \rho_g(z)
                = 
                    \rho_0\cdot\exp\left(-\frac{z^2}{2H_p^2}\right)
        \end{equation}
        where
        \begin{equation}
            H_p
                :=\frac{\Omega_K}{c_s}
        \end{equation}
        is the so-called \textit{pressure scale height} of the disk. \\
        % TODO socalled vs so-called

        \todo{Now that we know the behavior of the gas volume density, we can easily derive it from 
        the gas surface density like this:} 
        \begin{equation}
            \rho_g(r)
                =\frac{\Sigma_g(r)}{\sqrt{2\pi}\cdot H_p}
        \end{equation}
        \todo{Why can we do this?}

    % }}}
    % Number Density Profile {{{ 
    \subsection{Number Density Profile}

        \todo{Derive $N_g(r_i)$} \\

        \begin{equation}
            N_g
                =\frac{\rho_g}{m_g}
        \end{equation}

        \todo{Is this really even needed? (Don't I only need dust number density?)} \\

        \todo{Plot gas number density $N_g(r_i)$ vs $r_i$.} \\  % `N_g -> N`

    % }}}
    % Plots of Density Profiles {{{

        \newpage
        \todo{Plot gas surface density $\Sigma_g(r)$ vs $r$.} \\
        \todo{Plot gas volume density $\rho_g(r_i)$ vs $r_i$.} \\
        \todo{Plot $\rho_g(z)$.}
        \begin{figure}[h!]
    \makebox[\textwidth]{
        \includegraphics[width=\paperwidth]{12/gas_surface_density.pdf}
    }
    \caption{Radial Profile of the Gas Surface Density $\Sigma_g$}
    \label{}
\end{figure}

\begin{figure}[h!]
    \makebox[\textwidth]{
        \includegraphics[width=\paperwidth]{12/gas_volume_density.pdf}
    }
    \caption{Radial Profile of the Gas Volume Density $\rho_g$}
    \label{}
\end{figure}


    % }}}

\newpage\section{Gas Particle Kinematics}

    \todo{There are both stochastic and systematic contributions to the gas kinematics.}

    \subsection{Thermal Gas Velocity}

        \todo{Sound speed:}
        \begin{equation}
            c_s = \sqrt{\frac{k_BT}{m_g}}
        \end{equation}

        \todo{Thermal velocity:} % TODO thermal SPEED
        \begin{equation}
            u_\text{th} = \sqrt{\frac{8}{\pi}} \cdot c_s
        \end{equation}

        \todo{Mean free path:}
        \begin{equation}
            \lambda_\text{mfp} = \frac{1}{n \cdot \sigma_{\text{H}_2}}
        \end{equation}

        \vfill

\todo{Why so small?}

\begin{figure}[h!]
    \makebox[\textwidth]{
        \includegraphics[width=\paperwidth]{12/thermal_gas_velocity.pdf}
    }
    \caption{Thermal Gas Velocity $u_th(r)$}
    \label{}
\end{figure}


    \newpage\subsection{Radial Gas Velocity}

        \todo{Plot radial profile of "speed" (thermal + ..?)}

        The viscous evolution of the gas disk can be described by the continuity equation 
        \cite{birnstiel_dullemond_brauer_2010}:
        \begin{equation}
            \pderiv{\Sigma_g}{t}-\frac{1}{r}\pderiv{}{r}\bigg(\Sigma_g \cdot r \cdot u_g\bigg) = S_g
        \end{equation}
        Here, $u_g$ labels the \textit{radial gas velocity}.
        \\
        % TODO: What is $S_g$
        % TODO: Does $u_g$ label radial velocity or speed?
        % TODO: Choose either $uor and $v$ as variable names for velocities.
        %       Be consistent across the entire thesis.

        A solution to this equation is given by \cite{lynden-bell_pringle_1974}, where the 
        radial velocity of the gas is given by
        \begin{equation}
            u_{g}
            =-\frac{3}{\Sigma_g\cdot\sqrt{r}}
                \cdot\pderiv{}{r}\bigg(\Sigma_g \cdot \nu_g \cdot \sqrt{r}\bigg)
        \end{equation}
        % TODO: Have I already defined the viscosity somewhere? Else define here.

        \todo{Viscosity:}
        \begin{equation}
            \nu_g = \frac{1}{2} \cdot u_\text{th} \cdot \lambda_\text{mfp}
        \end{equation}

        \vfill

\begin{figure}[h!]
    \makebox[\textwidth]{
        \includegraphics[width=\paperwidth]{12/radial_gas_velocity.pdf}
    }
    \caption{Radial Gas Velocity $u_g(r)$}
    \label{}
\end{figure}

