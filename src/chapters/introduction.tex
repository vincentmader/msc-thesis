\section{Historical Context}

    \todo{Ever since the time of the earliest human civilizations, ...} \\
    \todo{Humans have been intrigued by the ... night sky.} \\
    \todo{Even before the invention of telescopes, with the naked eye alone...} \\
    \todo{Several bright objects could be distinguished from the rest of the night sky} \\
    \todo{Distinguish planets from stars, they exhibit relative motion ("planetes").} \\
    \todo{Ptolemai: Earth as center of the universe.} \\
    \todo{Copernicus: Sun as center of the Solar System.} \\
    \todo{Kepler: Kepler's laws, ellipsoid trajectories.} \\
    \todo{Newton: law of gravitation.} \\
    \todo{Discovery of planets in our solar system.} \\
    \todo{Discovery of exo-planets} \\
    \todo{Question: How do planets form?}
        
\section{Planet Formation}

    % The Interstellar Medium {{{
    \subsection{The Interstellar Medium}

        % What is the interstellar medium?
        The term \textit{interstellar medium} (ISM) is used to denote the matter and radiation that 
        can be found in the expanse of space between the stellar systems inside a galaxy. 
        As we will discuss further in the following sections, the ISM is widely considered to serve 
        as the birthplace of proto-planetary disks (PPDs), and subsequently, stars and planets. \\

        At first glance, it might not be easy to imagine how such giant objects like planets and 
        even stars can form in the ISM. After all, one might naively say that it consists of mostly
        nothing. From the point of view of a human living on Earth, the interstellar regions of
        outer space appear to be a near-perfect vacuum. \\

        Measurements of the the molcular particle number density in the ISM have been made in 
        previous studies by e.g. \cite{burton_2013}. Here, they showed that the number of particles 
        per unit volume can range from $\SI{1e16}{\meter^{-3}}$ all the way down to 
        $\SI{1e3}{\meter^{-3}}$. \\

        The exact value of the particle number density depends of course heavily on the precise 
        position in the ISM. Nonetheless, a comparison of the mere orders of magnitude involved 
        here demonstrates the huge difference in particle abundances in the ISM when compared to 
        Earth, where the number of gas molecules inside the atmosphere is approximately 
        $\SI{1e25}{\meter^{-3}}$.
        \footnote{Here we assume the validity of the ideal gas law, a sea-level pressure of 
                  $p=\SI{1}{\bar}$, and a temperature of $T=\SI{25}{\celsius}$.}
        \\

        The fact that giant astronomical objects like PPDs, stars, and planets can form from the 
        material found in the ISM, can be understood more easily if one brings to mind the
        absolutely enormous scales of both space and time that are involved here. \\

        Over time scales of (\todo{approx. how many years?}), the material in the ISM 
        will, under the effect of gravity, gather in giant interstellar clouds. Examples for such 
        clouds are given by e.g. \todo{(examples for clouds)}, measurements of which can be seen 
        visualized in \todo{(cite figures)}. \\

        \todo{Images of interstellar gas clouds, star-forming regions.} \\
        \todo{How big are they? How many light years across, approximately?} \\

        \vfill
\begin{figure}[h!]
    \centering
    \begin{minipage}{.5\linewidth}
        \centering
        \subfloat[Orion Nebula (Messier 42) \cite{orion_nebula_hubble_2006}]{
            \label{:a}
      	  	\includegraphics[width=0.85\linewidth]{Orion_Nebula_-_Hubble_2006_mosaic_1080px}
      	}
    \end{minipage}%
    \begin{minipage}{.5\linewidth}
        \centering
        \subfloat[Eagle Nebula (Messier 16) \cite{eagle_nebula_eso_2009}]{
            \label{:b}
            \includegraphics[width=0.85\linewidth]{eso0926a.jpg}
      	}
    \end{minipage}
    \caption{Examples of star-forming regions: (a) The Orion Nebula (b) The Eagle Nebula
        (Zoom in to see the famous \textit{Pillars of Creation} at the center of the image).
    }
    \label{fig:star_forming_regions}
\end{figure}


        The material making up these clouds can be classified into three main categories:
        \begin{enumerate}
            \item Molecular Gas: \\
                \todo{Mostly diatomic hydrogen $\text{H}_2$.} \\
                \todo{some helium, lithium, and trace amounts of heavier elements.}
            \item Ionized Gas: (Plasma) \\ 
                \todo{}
            \item Dust: \\
                \todo{Agglomerates of ...} \\
                \todo{held together (weakly) by ...} \\
                \todo{Shape: (fractal?)}
        \end{enumerate}

        Now that we've taken a brief look at the material make-up and particle density of the ISM, 
        let us turn our attention to the question of how PPDs, stars, and planets can be formed 
        from the aforementioned clouds of gas and dust.

    % }}}
    % The Nebular Hypothesis {{{ 
    \subsection{The Nebular Hypothesis}

        The \textit{nebular hypothesis} is a widely accepted model for explaining the formation and 
        evolution of not only our own Solar System, but the planetary systems around other stars as 
        well. 

        It was postulated independently by Immanuel Kant        % TODO cite
        and Pierre-Simon Laplace                                % TODO cite
        in 1755 and 1796, respectively. \\                      % TODO "independently" is that true?

        The hypothesis is built upon the idea that the formation of proto-planetary disks (and thus 
        solar systems) can be explained via the gravitational collapse of giant interstellar clouds 
        of gas and dust. (\todo: Examples for such star-forming regions, see above [cite]) \\
        \todo{(molecular clouds)}

        \todo{What is there at the beginning?}
        \begin{itemize}
            \item a giant cloud of (mostly) gas in the interstellar medium 
                  (as said before: mostly di-atomic hydrogen, and mono-atomic helium)
        \end{itemize}
        \todo{How big is such a cloud?}
        \begin{itemize}
            \item ...
        \end{itemize}
        \todo{What happens?}
        \begin{itemize}
            \item The cloud collapses under its own gravity. \todo{Cause?}
            \item The cloud posseses a total angular momentum. It is most likely $\neq 0$.
            \item Therefore, the cloud does not just simply collapse into a single point. 
                  (\todo{...})
            \item The cloud flattens into a disk.
            \item The majority of the mass present in the cloud gathers in the center.
            \item The rest of the mass flattens into a disk orbiting the star 
                  (later: solar system).
            \item In the center/core, density and temperature increase to a point that nuclear 
                  fusion becomes possible.
        \end{itemize}

        \todo{Result: PPD} \\
        \todo{Images of PPDs.}

    % }}}
    % The Road from Dust to Planet {{{ 
    \subsection{The Road from Dust to Planet}

        \todo{Evolution over many orders of magnitude.} \\
        \todo{Start with gas.}

        \todo{Then: dust particles}
        \begin{itemize}
            \item How do they form?
        \end{itemize}

        \todo{Then: dust agglomerates}
        \begin{itemize}
            \item They form via Coulomb and/or Van-der-Waals interactions
            \item Particles collide and, under the right circumstances, stick together.
            \item Larger and larger agglomerates form.
            \item They possess a fractal shape (dust bunnies).
        \end{itemize}
        \todo{Image of dust bunnies/fractals.}

        \todo{Then: pebbles} \\
        \todo{Then: planetesimals} \\
        \todo{Then: planetary cores (may start accreting gas)} \\
        \todo{Then: terrestrial planets} \\
        \todo{Then: gas giants (outer regions?)} \\

        \todo{Visualize particle coagulation.}

    % }}}

\section{Motivation for this Thesis}

    \todo{Why is planet formation relevant tous at all?} \\
    \todo{Why is dust coagulation relevant to planet formation?} \\
    \todo{How can one model dust coagulation?} \\
    \todo{What are the challenges involved in that model? (int. of Smol. eq.)} \\
    \todo{How does this thesis help with that?} 

\section{Structure \& Layout of this Thesis}

    \todo{What are the "parts" of this thesis?}
    \begin{enumerate}
        \item Disk model: disk, gas, \& dust.
        \item Coagulation \& Fragmentation: Smoluchowski equation, kernel, integration.
        \item Sampling: Probability distribution, results, stability \& accuracy.
    \end{enumerate}

\newpage
\section{Preliminary Notes on Axis Discretization}

    Throughout this thesis, we will deal with several physical quantities with values that range
    across many orders of magnitude. Most notably, these are time, distance, and mass. \\

    Due to the fact that analytical solutions to the relevant differential equation(s) exist only 
    for a very limited set of cases, it will be necessary to make use of numerical integration 
    techniques to obtain approximate solutions. (...)
    As such, the axes for the above-mentioned (?) physical quantities
    will have to be discretized in an appropriate manner. \\

    As an example, we will shortly demonstrate how one might do this for the mass axis $m$.
    The construction of discrete axes for both time $t$ and space (i.e. distance from the star $r$)
    can then be done entirely analogously to what we will do here: \\

    The continuous range of particle masses $m$ present in the disk is partitioned into a set of 
    $\mathcal N_m\in\mathbb N$ intervals, which we will refer to as "bins" from this point onwards. 
    Each of these bins can be uniquely characterized by an index 
    $i\in[1,\ \mathcal N_m]\cap\mathbb N$, and assigned a mass value $m_i^\text{c}$, 
    which is situated at the center of the bin (ergo the superfix "c").\\
    
    To derive an expression that can be used for calculating the mass values at the \textit{bin
    centers}, let us first define the mass values at the \textit{bin boundaries}. To do this,
    consider a collection of $\mathcal N_m+1$ appropriately spaced grid points $m_i^\text{b}$. 
    The first and last of these values, i.e. the lower and upper boundaries of the discrete mass 
    grid, shall be labeled $m_\text{min}$ and $m_\text{max}$, respectively. \\

    In [cite linear] and [cite logarithmic], we will go into more detail on how exactly the values 
    for $m_i^\text{b}$ and $m_i^\text{c}$ are to be defined. \\
    % TODO Assure consistent usage of \text{b} and \text{c}.

    A schematic represenation of the continuous mass axis and its discrete analogon can be seen 
    visualized in \cref{fig:continuous_and_discrete_mass_axis} below.
    % TODO Display `Figure` instead of `fig`.

    \vfill

\begin{figure}[h!]
    \begin{center}
        \begin{tikzpicture}
            \def\N{5}      % This is the number of cells drawn (to the left of "..." separator).
            \def\M{6}      % This is the number of boundary arrows drawn (left of "...").
            \def\W{1.5}    % This is a cell's width.
            \def\H{\W}     % This is a cell's height.
            \def\L{\W/4}   % This is an arrow's length.
            \def\P{\W/8}   % This is the padding between arrow & cell.
            \def\R{\W/32}  % This is the padding between arrow & text.

            % Draw continuous mass axis.
            \draw [|-to](0, 2.5*\H) -- (\N*\W+4*\W, 2.5*\H);
            \draw [-](\W, 2.45*\H) -- (\W, 2.55*\H);
            \draw [-](\N*\W+3*\W, 2.45*\H) -- (\N*\W+3*\W, 2.55*\H);
            \node[] at (-2*\P, 2.5*\H) {$0$};
            \node[] at (\N*\W+4*\W+2*\P, 2.5*\H) {$m$};

            % Draw arrow from continuous to discretized mass axis.
            % \node[] at (\N*\W/2+1.5*\W, 2.25*\H) {$\Downarrow$};

            % Draw cells...
            % ...to the left of "..." separator.
            \foreach \x in {1, ..., \N} {
                \draw[draw=black] (\x*\W, 0) rectangle ++(\W, \H);
            }
            % ...to the right of "..." separator.
            \draw[draw=black] (\N*\W+2*\W, 0) rectangle ++(\W, \H);

            % Draw crosses...
            % ...to the left of "..." separator.
            \foreach \x in {1, ..., \N} {
                \draw (\x*\W+\W/2, \H/2) node {\tiny x};
            }
            % ...to the right of "..." separator.
            \draw (\N*\W+2.5*\W, \H/2) node {\tiny x};

            % Draw "..." separator.
            \node[] at (\N*\W+1.5*\W, \H/2) {...};

            % Draw arrows labeling cell boundaries.
            \foreach \x in {1, ..., \M} {
                \draw [-to](\x*\W, \H+\L+\P) -- (\x*\W, \H+\P);
                \node[] at (\x*\W, \H+\L+\P+\L+\R) {$m_\x^\text{b}$};
            }
            \draw [-to](\N*\W+2*\W, \H+\L+\P) -- (\N*\W+2*\W, \H+\P);
            \draw [-to](\N*\W+3*\W, \H+\L+\P) -- (\N*\W+3*\W, \H+\P);
            \node[] at (\N*\W+2*\W, \H+\L+\P+\L+\R) {$m_{\mathcal N_m}^\text{b}$};
            \node[] at (\N*\W+3*\W, \H+\L+\P+\L+\R) {$m_{\mathcal N_m+1}^\text{b}$};

            % Draw arrows labeling cell centers.
            \foreach \x in {1, ..., \N} {
                \draw [-to](\x*\W+\W/2, 0-\L-\P) -- (\x*\W+\W/2, 0-\P);
                \node[] at (\x*\W+\W/2, 0-\L-\P-\L-\R) {$m_\x^\text{c}$};
            }
            \draw [-to](\N*\W+2.5*\W, 0-\L-\P) -- (\N*\W+2.5*\W, 0-\P);
            \node[] at (\N*\W+2.5*\W, 0-\L-\P-\L-\R) {$m_{\mathcal N_m}^\text{c}$};

            % Draw labels for `m_min` and `m_max`.
            \node[] at (\W, \H+\L+\P+\L+3*\R+\L+\L) {$m_\text{min}$};
            \node[] at (\N*\W+3*\W, \H+\L+\P+\L+3*\R+\L+\L) {$m_\text{max}$};
            \node[] at (\W, \H+\L+\P+\L+2*\R+\L) {$=$};
            \node[] at (\N*\W+3*\W, \H+\L+\P+\L+2*\R+\L) {$=$};
        \end{tikzpicture}
    \end{center}
    \caption{Schematic illustration of the discretized mass axis. After having defined a minimum 
        value $m_\text{min}$ and a maximum value $m_\text{max}$, the interval between these two 
        values is divided evenly into $\mathcal N_m$ bins. To do this, we first define the mass
        values at the bin boundaries and, from that, the mass values at the bin centers. 
        This can be done using either a linear or a logarithmic scaling (for details, see section
        [cite linear] and [cite logarithmic], respectively). The discretization of the 
        axes for both time $t$ and distance $r$ from the star is done completely analogously.}
    \label{fig:continuous_and_discrete_mass_axis}
\end{figure}


    % Axis Discretization with Linear Scaling {{{ 
    \subsection{Axis Discretization with Linear Scaling}

        The spacing of the grid points of course depends heavily on the utilized scaling, which for 
        simplicity shall be linear at the moment. Later on we will make use of a logarithmic scaling,
        to assure a better representation of all values along the wide range of dust particle masses 
        present in the proto-planetary disk.\\
    
        For now, the mass value $m_i^\text{b}$ (representing the lower boundary of a bin $i$) can be 
        expressed as
        \begin{align}
          m_i^\text{b}=m_\text{min}+(m_\text{max}-m_\text{min})\cdot\frac{i}{\mathcal N_m}
        \end{align}
    
        To calculate the mass values at the bin centers, all we need to do now is take the arithmetic 
        mean of its two boundaries values, i.e.:
        \begin{equation}
            m_i^\text{c}
                =\frac{m_i^\text{b}+m_{i+1}^\text{b}}{2}
        \end{equation}
        
        Thus, after having defined only the three numbers $\mathcal N_m$, $m_\text{min}$, and 
        $m_\text{max}$, it is possible to interpolate the values of all mass grid points sitting on the 
        boundaries and/or centers of the bins.\\
        
        The inverse transformation from mass to index can be determined easily by rearranging the above 
        relation. This leads to the following expression:
        \begin{equation}
            i(m)
                =\mathcal N_m\cdot\frac{m-m_\text{min}}{m_\text{max}-m_\text{min}}
        \end{equation}
        
        % Sidenote: In the linearily scaled mass grid, the bin "width" is constant, and independent of the 
        % bin index. This will change once the switch to a logarithmic grid is made! For now though, it is 
        % given by
        % \begin{align}
        %   \Delta m
        %   &=m_{i+1}^\text{b}-m_i^\text{b}\\
        %   &=\frac{m_\text{max}-m_\text{min}}{\mathcal N_m}
        % \end{align}
 
    % }}}
    % Axis Discretization with Logarithmic Scaling {{{ 
    \subsection{Axis Discretization with Logarithmic Scaling}

        As before, let $\mathcal N_m$ label the total number of bins, and let the upper and lower grid 
        boundaries be given by $m_\text{min}$ and $m_\text{max}$, respectively.\\
        
        Once again, we first define an expression for the grid points sitting directly on the bin 
        boundaries. Making use of an index $i\in\mathcal[1,N_m+1]$, they can be uniquely identified and
        calculated via
        \begin{equation}
            m_i^\text{b}
                =m_\text{min}\cdot\left(\frac{m_\text{max}}{m_\text{min}}\right)^{i/\mathcal N_m}
        \end{equation}
        To arrive at the mass values at the bin centers, we again take the mean. Contrary to the case of 
        a linear scaling, here we're not using the arithmetic mean though, but the geometric mean, i.e.
        \begin{align}
            m_i^\text{c}
                =\sqrt{m_i\cdot m_{i+1}}
        \end{align}
    
        As in the linear case, the inverse transformation can easily be arrived at by rearranging for 
        $i$. This leads to the following expression:
        \begin{align}
            i(m)
                =\mathcal N_m\cdot
                    \frac{\log(m)-\log(m_\text{min})}{\log(m_\text{max})-\log(m_\text{min})}
        \end{align}
        
        In contrast to the linear grid, where the bin "width" (i.e. the additive offset from one 
        bin to the next) is the same for all bins, in the logarithmic grid this is not the case.
        Instead, what stays the same for is the \textit{relative} mass increase from one bin to the
        next:
        \footnote{\todo{This is true only down to the machine precision of the utilized computer. 
        (?)}}
        \begin{equation}
            \frac{m_i^\text{c}}{m_i^\text{c}}
                =\frac{m_i^\text{b}}{m_i^\text{b}}
                =\text{const.}
                \forall i\in[1,\mathcal N_m]
        \end{equation}

    % }}}
