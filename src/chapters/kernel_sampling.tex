After having defined a simple disk model, and from that calculated the reaction rates for 
coagulation and fragmentation events, we will now focus on the main goal of this thesis: \\

We would like to find out whether it makes sense to utilize stochastic Monte Carlo sampling of the 
kernel matrix, in order to lower the cost of the Smoluchowski coagulation equation's numerical 
integration.

% Motivation {{{ 
\section{Motivation}

    Let us first think about why that would even be necessary.
    The computational cost of integrating the Smoluchowski equation within the context of our 
    model is affected by a number of the model's parameters: \\

    The first of these is given by the mass grid resolution $\mathcal N_m$, i.e. the number of 
    mass ``bins''. Remember that, in each time-step, we are interested in computing an updated 
    version of the discretized dust particle mass distribution function, which involves 
    updating each of its entries. \\

    For each of these entries $k$, we need to perform a loop over all collision pairs $(i, j)$ that 
    could potentially lead to a creation or destruction of a particle with mass $m_k$. 
    Having discretized the mass axis into $\mathcal N_m$ bins, the total number of collisions that
    we need to consider is given by $\mathcal N_m^2$. \\

    Since this needs to be done for each value of $k$, the total numerical cost $\mathcal C$ for 
    forwarding the mass distribution at a given location from one time-step to the next follows the 
    relationship
    \begin{equation}
        \mathcal C = \mathcal O\big( \mathcal N_m^3 \big)
    \end{equation}

    When considering only the simple case of pure hit-and-stick coagulation, this cubic 
    relationship turns into a quadratic one, since we can make use of the criterion of mass 
    conservation: For each particle mass $m_k$ that could emerge from a collision between 
    particles with masses $m_i$ and $m_j$, there is only a single value $m_j$ that satisfies the 
    mass conservation criterion $m_k = m_i + m_j$, and vice versa. In this case, the kernel matrix 
    does not even have to be cubic, but can be quadratic instead. \\

    The inclusion of fragmentation is what makes this more complicated. Here, a wide range of 
    differently-sized particles could emerge from any given collision, requiring the definition 
    of a cubic kernel matrix, and leading to the cubic dependence of the numerical cost on the mass 
    grid resolution.

    \clearpage

    Note that up until this point we have only considered the numerical cost of forwarding the mass 
    distribution a single time-step, at a single location, and assuming the individual dust 
    particles can be characterized entirely by their mass value alone. \\
    
    For the coagulation studies that are done in the context of this thesis, we only consider a 
    single position in the disk, located at $\vec r = \vec r_0$, defined in 
    \cref{eq:definition_of_position_of_interest}.
    A more sophisticated model would of course require the calculation of the mass distribution's 
    temporal evolution at several different positions in the disk. \\

    If both spatial and temporal discretization are included, the numerical complexity increases by 
    additional factors. 
    % Assuming a temporal resolution of $\mathcal N_t$ time-steps, we get
    % \begin{equation}
    %     \mathcal C = \mathcal O\big( 
    %         \mathcal N_m^3 \cdot 
    %         \mathcal N_t 
    %     \big)
    % \end{equation}
    Adopting cylindrical coordinates and assuming a 
    radial    resolution $\mathcal N_r$, 
    azimuthal resolution $\mathcal N_\phi$, 
    vertical  resolution $\mathcal N_z$, and 
    temporal  resolution $\mathcal N_t$, 
    the complexity becomes
    \begin{equation}
        \mathcal C = \mathcal O\big( 
            \mathcal N_m^3 \cdot 
            \mathcal N_t \cdot 
            \mathcal N_r \cdot \mathcal N_\phi \cdot \mathcal N_z
        \big)
    \end{equation}

    % In total, our discretized model requires the 
    % \\

    % Let us use $k \in [1, \mathcal N_m]$ to index these entries, and let $m_k$ label their
    % corresponding mass values. 
    % For each of these, we have to loop over all colliding particle pairs 
    % $(i, j)$ that could potentially lead to the 

    % In the simple case of pure hit-and-stick coagulation, this is a one-dimensional loop. Due 
    % to the conservation of mass, for each 

    In the context of this thesis, the dust particles are modeled as perfectly spherical bodies 
    with a shared solid mass density $\rho_s$. A possible next step for obtaining a more realistic
    model is given by the inclusion of a \textit{porosity parameter $p$}, which allows a 
    slightly more detailed description of the dust particles' internal structure, and that 
    structure's influence on the processes of coagulation, fragmentation, and bouncing. 
    Studies building upon this idea have already been made, for this see e.g. 
    \cite{ormel_2006} \cite{katakoa_2017}. \\

    The inclusion of a dust porosity parameter is probably the most significant argument speaking 
    for a Monte Carlo sampling of the kernel. While the cubic dependency of the numerical cost on 
    $\mathcal N_m$ is bad enough, the inclusion of porosity makes numerical integration even more 
    costly. \\

    The reason for this is the following: If porosity is included, the individual dust particles 
    can not be characterized by a single number alone anymore, instead two are needed. Thus, we 
    do not have a one-dimensional particle distribution $n(m)$, but a two-dimensional $n(m, p)$.
    Then, in each time-step and at each location, for each combination of mass and porosity, we 
    need to find out how the interaction between pairs of particles, each carrying an indiviual 
    combination of mass and porosity, influence the particle distribution. \\

    This requires the evaluation of a loop over $\mathcal N_m^3 \cdot \mathcal N_p^3$ cases.
    The total cost becomes 
    \begin{equation}
        \mathcal C = \mathcal O\big( 
            \mathcal N_m^3 \cdot 
            \mathcal N_p^3 \cdot 
            \mathcal N_t \cdot 
            \mathcal N_r \cdot \mathcal N_\phi \cdot \mathcal N_z
        \big)
    \end{equation}

    This last equation showcases a behavior that is quite expensive. Consider the following: 
    If we discretize our mass axis into 100 bins (which is not actually that much), and do the 
    same for the porosity axis, evaluating the complete sum on the right-hand side of the 
    Smoluchowski equation would already require the inclusion of
    \begin{equation}
        \mathcal N_m^3 \cdot \mathcal N_p^3 = 10^{12}
    \end{equation}
    terms. Note that this is for a single time-step, and at a single location in the disk! \\

    To allow studies of dust coagulation in a model including both a mass and porosity parameter,
    this is much too expensive to finish the simulations in a sensible time span. Thus, a method 
    for lowering the computational cost would be desirable. 
    
    \clearpage
    A first idea for how to do this could be to include only the most relevant collisions into the 
    sum over the kernel, such that fewer calculations have to be made. 
    What exactly we mean by ``relevant collision'' is of course yet to be defined, see
    \cref{sec:definition_of_the_sampling_probability} for this. \\

    In any case, a Monte Carlo sampling approach for lowering the computational cost seems 
    sensible, and will be implemented in the following sections.

    % \begin{itemize}
        % \item The numerical cost of integrating the Smoluchowski coagulation equation 
        %       within the context of the our model is affected by a number of parameters: 
        % \item The total cost is given by
        %     \begin{equation}
        %         \mathcal O\big(
        %             \mathcal N_m^3 \cdot % \mathcal N_p^3 \cdot 
        %             \mathcal N_t   \cdot 
        %             \mathcal N_r   \cdot \mathcal N_\phi \cdot \mathcal N_z
        %         \big)
        %     \end{equation}
    % \end{itemize}

    % The first of these we shall consider is the mass grid resolution $\mathcal N_m$, which 
    % labels the number of "bins" in our discretized mass axis. \\
    % Remember that in each time-step, we are interested in computing an updated version 
    % for each of the $\mathcal N_m$ entries of the discretized dust particle mass 
    % distribution function $n_k$.
    % To do this, we have to find out how the number of particles carrying said mass is influenced 
    % by collisions of particle pairs carrying every possible combination of mass values, of which 
    % there are $\mathcal N_m^2$.
    % % In the discretized Smoluchowski framework, this translates to a 
    % % summation over the kernel matrix $K_{kij}$. The sum goes over all collisions 
    % % between particles with masses $m_i$ and $m_j$ leading to a change to the distribution at 
    % % mass $m_k$, which requires us to sum over all $\mathcal N_m^2$ possible collision. 
    % As such, the numerical cost of the integration for one time-step is proportional to
    % $\mathcal N_m^3$. \\

    % In the context of this thesis, we chose to only model the disk along the radial 
    % axis, and only integrate the Smoluchowski equation at one specific radial location.
    % For a more complete model, it would of be necessary to resolve the disk spatially in a 
    % more sophisticated way. Assuming cylindrical coordinates, this lead would lead to an 
    % inclusion of $\mathcal N_r$, $\mathcal N_\phi$, $\mathcal N_z$ into the numerical cost. \\

    % Since this has to be done for each time-step, the computational cost of the entire 
    % integration is increased by an additional factor $\mathcal N_t$. \\

    % The inclusion of additional particle parameters, like e.g. dust particle porosity, 
    % is used [\todo{cite}]
    % to attain a more realistic model.

    % The inclusion of a porosity parameter $p$ leads to an increase of 
    % the computational cost of another factor $\mathcal N_p$.
    % the kernel dimension from $3$ to $6$

    % Due to the high dimensionality of the problem,

    % This would require the evaluation of a six-dimensional sum over the kernel matrix 
    % encoding coagulation and fragmentation events between particles modeled as 
    % mass and porosity.

    % each of the $\mathcal N_m$ mass grid bins. \\

    % If we then consider the total cost of the integration over $\mathcal N_t$ steps,
    % the cost is increased by that factor as well.

    % In the following we will consider the computational complexity of 
    % numerically integrating 
    % the Smoluchowski coagulation equation within 
    % the model we have built so far, as well as 
    % the complexity's dependence on some of the main model parameters: \\

    % In each time-step, and for each considered location in space,
    % we need to do a number of things: \\

    % For every mass value, we are interested in finding out how the number of particles 
    % carrying that mass is influenced by collisions of particles carrying every possible 
    % combination of mass values. \\
    % As such, the computational complexity is proportional to $\mathcal N_m^3$. \\

    % It gets even worse if we want to resolve the disk along all three spatial dimensions. \\
    % Then, there is an additional factor $\mathcal N_\phi \cdot \mathcal N_z$. \\

    % \todo{Therefore, the total computational complexity (assuming 3D spatial resolution) 
    % becomes proportional to}
    % \begin{equation}
    %     O\big(
    %         \mathcal N_m^3 \cdot \mathcal N_p^3 \cdot 
    %         \mathcal N_t   \cdot 
    %         \mathcal N_r   \cdot \mathcal N_\phi \cdot \mathcal N_z
    %     \big)
    % \end{equation}

    % Consider again the Smoluchowski equation in \cref{eq:discrete_smoluchowski_equation}. \\

    % When only the process of pure hit-and-stick coagulation is included into the model, the 
    % kernel matrix $K_{kij}^\text{coag}$ is actually quite sparse, since for given collision the 
    % dust particle mass distribution has to be modified at only very few mass grid points.
    % As we saw in [\todo{cite}], the two-dimensional integral could be collapsed into an 
    % integral over a single variable. \\

    % The inclusion of fragmentation is what really makes the integration of the Smoluchowski
    % equation numerically expensive. Here, integration will require the evaluation of the 
    % double sum over the kernel matrix. This stems from the fact that the collision of initially 
    % only two particles can lead to the creation of a wide range of particles of different sizes. \\
    
    % For higher mass grid resolutions $\mathcal N_m$, the integration of this differential equation 
    % can quickly become numerically expensive. \\

    % As the computational cost of its integration is proportional to the number of terms 
    % in the sum that have to be included, it seems feasible that a Monte Carlo sampling of 
    % the most relevant terms could be of interest. \\

    % \todo{It may be good enough to only resolve the disk along the radial axis, or even to focus 
    % on just a single point in the disk. Even then though, it gets more complicated if one wants 
    % to build a detailed model of dust particle coagulation.} \\

    % \todo{In the discussion up to this point, we made the assumption that all dust particles are 
    % perfectly spherically, and can thus be characterized entirely by the mass that they carry.} \\

    % \todo{This is of course quite simplistic. In reality, dust particles possess much more complex 
    % shapes.}xz    \todo{...}
    % \todo{An illustration of the so-called \textit{dust bunnies} resulting from collisions can be 
    % seen in [cite].} \\

    % \todo{A detailed description ... (impossible?)} \\
    % \todo{To at least somewhat include the dynamics of such particles, one can make the assumption 
    % that a dust particle can be characterized by not one, but two attributes: In addition to the 
    % particle mass, we define the \textit{particle porosity}, which gives a (Masz) for the
    % (Packungsdichte) of the particle.} \\

    % \todo{Studies including (...) have already been done by [cite].} \\

    % \todo{If porosity is included as a second attribute for the dust particles, it is no longer 
    % enough to formulate a three-dimensional matrix for the coagulation kernel. Instead, it 
    % becomes six-dimensional.} \\
    % \todo{The computational complexity increases by an additional factor $\mathcal N_p^3$ (?).}

    % \todo{The idea of this thesis:} \\
    % \todo{Speed up the integration via stochastic sampling of the coagulation kernel.} \\
    % \todo{In future studies (where porosity is included) this could lead to drastic improvements 
    % in the (integration times) by (excluding less relevant sum terms) (...)} \\

    % \todo{Goals?} \\
    % \todo{Metrics?} \\
    % \todo{Bare minimum: Mass conservation (detailed balance)} \\
    % \todo{Then: Accuracy}

% }}}
% Previous Works {{{
\section{Previous Works}

    Previous studies have already made use of Monte Carlo methods in an attempt to 
    bring down the numerical cost of the integration of the Smoluchowski equation. \\

    Notable examples for such prior work include e.g. the studies done for the 
    2006 paper by C. W. Ormel et. al. \cite{ormel_2006}, where a 
    direct Monte-Carlo simulation method (DSMC) is used, and 
    the more recent 2023 paper by M. Beutel \cite{beutel_2023},
    which makes use of a Representative Particle Monte Carlo method (RPMC). \\

    % The approach used for this thesis is slightly different, as we wish to 
    % sample \textit{particle collisions} $(i, j)$.

% }}}
% Basic Idea {{{ 
\section{Basic Idea}

    As noted above, the goal here is to reduce the cost of numerically integrating the 
    Smoluchowski coagulation equation. \\

    The basic idea is to not include all terms in the summation needed for the evaluation 
    of the Smoluchowski equation's right-hand side. This would lower the cost, as fewer 
    calculations have to be made. \\

    Instead of including \textit{all} collision pairs, we therefore need to identify the ``most 
    relevant'' ones.\footnote{
        It needs to be emphasized that we do \textit{not} sample $i$ and $j$ separately here! 
        Instead, we sample collision \textit{pairs} $(i, j)$.
    }
    Here, relevance is determined by a specific collision's impact on the overall 
    dust particle mass distribution. \\

    After having defined a sensible sampling probability distribution           
    $P_{ij}^\text{sample}$ we can, in each time-step, randomly select the most relevant 
    index pairs $(i, j)$ and construct a new kernel from that.
    This kernel is then used to forward the mass distribution to the next time-step. \\

    If it works, we should be able to decrease the number of terms over which the sum has
    to be evaluated, while still maintaining acceptable accuracy and stability. \\

    % Consider again the Smoluchowski equation in \cref{eq:discrete_smoluchowski_equation}.
    % As the computational cost of its integration is proportional to the number of terms 
    % in the sum that have to be included, it seems feasible that a Monte Carlo sampling of 
    % the most relevant terms could be of interest. \\

    % The idea here is to identify the most "relevant" collision pairs, i.e. index tuples $(i, j)$,
    % where relevance is determined by a specific collision's impact on the overall dust particle 
    % mass distribution. \\

    % After having defined a sensible sampling probability distribution $P_{ij}^\text{sample}$
    % we can, in each time-step, randomly select the most relevant index pairs $(i, j)$ and 
    % construct a new kernel from that. \\

    % The goal here is to decrease the number of terms over which the sum has to be evaluated,
    % while still maintaining acceptable accuracy and stability.

    % Note: We do \textit{not} sample $i$ and $j$ separately. Instead, we sample pairs $(i, j)$.

    % The idea is to try to lower the computational cost of integrating the Smoluchowski 
    % equation by including only the most "relevant" sum terms on the right hand side of 

    This basic idea is illustrated in \cref{fig:sampled_kernel_homogenous_probability}, where a 
    comparison between ``full'' and sampled kernel is displayed.
    Note that here, as a first example we are using a homogenous sampling probability.
    Since not all collision are relevant to the same extent for the evolution of the
    dust particle mass distribution, in the next section we will define a non-homogenous,
    more sensible sampling density. \\

    It should also be noted that we the displayed kernel matrix is symmetric around the 
    diagonal. This is only relevant for visualization though, and does not mean that the
    terms on the lower right of the matrix are actually defined as non-zero values.
    We can safely set them to zero, due to the symmetry $i\leftrightarrow j$ with regards
    to the particles partaking in a given collision. 

% }}}
% Sampled Kernel {{{ 

\begin{figure}[h!]
    \centering
    \begin{minipage}{.5\linewidth}
        \centering
      	\subfloat[Positive component of total kernel]{
            \label{:a}
      	  	\includegraphics[width=\linewidth]{93/K_gain.pdf}
      	}
    \end{minipage}%
    \begin{minipage}{.5\linewidth}
        \centering
      	\subfloat[Negative component of total kernel]{
            \label{:b}
      	  	\includegraphics[width=\linewidth]{93/K_loss.pdf}
      	}
    \end{minipage}
    \begin{minipage}{.5\linewidth}
        \centering
      	\subfloat[Positive component of sampled kernel]{
            \label{:a}
      	  	\includegraphics[width=\linewidth]{93/S_gain.pdf}
      	}
    \end{minipage}%
    \begin{minipage}{.5\linewidth}
        \centering
      	\subfloat[Negative component of sampled kernel]{
            \label{:b}
      	  	\includegraphics[width=\linewidth]{93/S_loss.pdf}
      	}
    \end{minipage}
    \caption{
        Illustration of the basic idea for kernel sampling: Only a fraction of the total
        number of possible particle collision is taken into account when defining the 
        kernel matrix. This number is determined by the sampling density $\rho_\text{sample}$,
        which is set to $1/2$ here. 
        The kernel is plotted for $k=40$ on a mass grid with resolution $\mathcal N_m=100$.
        Note that here, we are using a homogenous probability distribution for the collision 
        sampling. Since not all collisions are relevant for the temporal derivative of the 
        dust particle mass distribution to the same extent, we will define a non-homogenous
        probability distribution in the next section.
    }
    \label{fig:sampled_kernel_homogenous_probability}
\end{figure} 

% }}}
% Definition of the Sampling Probability {{{
\clearpage\section{Definition of the Sampling Probability}
\label{sec:definition_of_the_sampling_probability}

    As noted above, it is our goal to identify the most relevant collisions, i.e. those collisions 
    that lead to the biggest change in the dust particle mass distribution from one time-step to 
    the next.
    Having done that, we would like to encode this relevance of the collisions into a probability
    distribution. \\

    It should be chosen in such a way that the most influential collisions are also those that 
    possess highest probability of being selected at random using the probability distribution. 
    For this, we have to take two things into account: 
    \begin{enumerate}
        \item First, a certain collision might be deemed ``relevant'' because a large amount of mass
              is transferred between bins at each time that such a collision occurs.
        \item Secondly, a collision might also be relevant if only a small amount of mass is
              transferred at each collision event, but these events happens very frequently.
    \end{enumerate}

    In the following, we will try to encode these two conditions into a sensible probability 
    distribution. 
    For this, consider a collision $(i, j)$ between particles with masses $m_i$ and $m_j$. 
    Let $k$ be the index of a mass bin representing the population of particles with mass $m_k$. \\

    When such a collision $(i, j)$ takes place, the amount of mass that is transferred into
    a bin $k$ is given by the expression 
    \begin{equation}
      \Delta M_{kij} := m_k \cdot K_{kij}
    \end{equation}

    Also, the total mass transferred between all bins during such a collision is given by the sum 
    \begin{align}
        \label{eq:total_mass_being_moved_between_bins_on_collision_ij}
        \Delta M_{ij} 
        &:= \sum_{k=1}^{\mathcal N_m} \Delta M_{kij} \\
        &\ = \sum_{k=1}^{\mathcal N_m} m_k \cdot K_{kij}
    \end{align}
    Because no mass is created during collisions, this sum $\Delta M_{ij}$ is equal to 
    zero.\footnote{ 
        This is only really true in physical reality.
        The model discretization and usage of numerical methods introduces an error, 
        such that this sum will, in fact, \textit{not} be exactly equal to zero. 
        To assure the conservation of mass, and therefore the stability of the algorithm, it is
        important to keep this error as small as possible. Therefore, in our implementation we need
        to make sure that mass is conserved down to machine precision at each collision event.
        This must also be true when only an incomplete (sampled) version of the kernel is 
        used. As such, we must ensure that mass is conserved \textit{in detailed balance}.
    } \\

    To get an idea of the \textit{total net mass} $\Delta M^\text{net}_{ij}$ 
    that is being moved between bins, we can slightly alter the expression given in 
    \cref{eq:total_mass_being_moved_between_bins_on_collision_ij} and write 
    \begin{align}
        \Delta M^\text{net}_{ij} 
        &:= \sum_{k=1}^{\mathcal N_m} \big| \Delta M_{kij} \big| \\
        &\ = \sum_{k=1}^{\mathcal N_m} m_k \cdot \big| K_{kij} \big|
    \end{align}
    Defined like this, the back-and-forth movement of mass between bins does not lead to canceling, 
    and instead we get a measure for the total net transferred mass resulting from a collision
    $(i, j)$. \\

    Using this definition to quantify a collision's impact on the dust particle mass distribution,
    we can recognize the first proportionality relationship for our sampling probability
    $P^\text{sample}_{ij}$:
    \begin{equation}
        P_{ij}^\text{sample} \sim \Delta M^\text{net}_{ij}
    \end{equation}

    \clearpage

    Up until now, the sampling probability of a given collision does not depend on the rate at 
    which such collisions occur. 
    It seems plausible though that, all other factors being equal, more frequently occuring 
    collisions should also be sampled more frequently. \\

    To fix this, let us recognize the second proportionality
    relationship for the sampling probability, which depends directly on the dust particle 
    collision rate coefficient $R_{ij}^\text{coll}$:
    \begin{equation}
        P_{ij}^\text{sample} \sim R^\text{coll}_{ij}
    \end{equation}

    Remember that the coefficient $R^\text{coll}_{ij}$ does not label an actual 
    \textit{collision rate} with units of $\text{s}^{-1}$, but actually a collision rate normalized 
    by dust particle number density with units $\text{m}^{3}\ \text{s}^{-1}$. \\

    Since more collisions will occur at places where there are a lot of particles, the 
    sampling probability must depend on the number of particles per unit volume as well. 
    Therefore:
    \begin{align}
        P_{ij}^\text{sample} \sim N_i 
        \ \ \ \ \text{and} \ \ \ \
        P_{ij}^\text{sample} \sim N_j
    \end{align}

    Additionally, collisions involving high-mass particles will be weighted more significantly 
    than collisions involving low-mass particles:
    \begin{equation}
        P_{ij}^\text{sample} \sim m_i
        \ \ \ \ \text{and} \ \ \ \
        P_{ij}^\text{sample} \sim m_j
    \end{equation}

    % For a given collision $(i, j)$ between particles with masses $m_i$ and $m_j$, the amount of 
    % mass that is transferred into the bin 

    % Our first task will be to recognize proportionalities, before putting it all together.
    % The idea here is to define the sampling probability distribution in such a way that the most 
    % often sampled collisions are those that have the largest effect on the temporal evolution of 
    % the dust particle mass distribution. \\

    % Therefore, 
    % As such, we can recognize a proportionality relation between the dust particle collision 
    % rate coefficient $R^\text{coll}_{ij}$ and the sampling probability:

    % Also, it makes sense to define this probability distribution in such a way that the collisions 
    % that are most likely to be chosen are those that have the "largest effect" on the particle mass 
    % distribution function. With that, we mean those collisions that lead to the largest transfer of 
    % mass from one bin to another. \\
    
    % We define a \textit{weight matrix} $W_{ij}$ encoding this condition:
    % \begin{equation} 
    %     W_{ij}
    %         = \sum_{k=1}^{\mathcal N_m} m_k \cdot \big|K_{kij}\big|
    % \end{equation}

    The total sampling probability distribution is then directly proportional to the product of 
    the contributions that we discussed so far:
    \begin{equation}
        P_{ij}^\text{sample} \sim \Delta M_{ij}^\text{net} \cdot N_i \cdot N_j \cdot m_i \cdot m_j
    \end{equation}

    To turn this into an actual probability distribution though, we still have to perform a 
    normalization step, to assure that the sum over all probabilities is equal to one:
    \begin{equation}
        \sum_{i=1}^{\mathcal N_m}\sum_{j=1}^{\mathcal N_m}P_{ij} \overset{!}{=} 1
    \end{equation}

    To do this, we define the two preliminary expressions 
    \begin{align}
        F_{ij}^\text{sample} 
            &:= \Delta M_{ij}^\text{net} \cdot N_i \cdot N_j \cdot m_i \cdot m_j 
        \ \ \ \ \text{and} \\
        F^\text{sample} 
            &:= \sum_{i=1}^{\mathcal N_m} \sum_{j=1}^{\mathcal N_m} \mathcal P_{ij}
                % \sum_{i=1}^{\mathcal N_m} \sum_{j=1}^{\mathcal N_m}
               % W_{ij} \cdot N_i \cdot N_j \cdot m_i \cdot m_j
    \end{align}

    % To do this, we define the sum
    % \begin{equation}
    % \end{equation}
    and, with these, arrive at the normalized probability distribution: 
    \begin{align}
        \boxed{
        P_{ij}^\text{sample} 
            = \frac{F_{ij}^\text{sample}}{F^\text{sample}} 
            = \frac{
                \Delta M_{ij}^\text{net} \cdot N_i \cdot N_j \cdot m_i \cdot m_j
            }{
                % \sum_{\tilde i=1}^{\mathcal N_m} \sum_{\tilde j=1}^{\mathcal N_m} 
                % \Delta M_{\tilde i\tilde j}^\text{net} \cdot N_{\tilde i} \cdot N_{\tilde j} 
                % \cdot
                % m_{\tilde i} \cdot m_{\tilde j}
                \sum_{i'=1}^{\mathcal N_m} \sum_{j'=1}^{\mathcal N_m} 
                \Delta M_{i'j'}^\text{net} \cdot N_{i'} \cdot N_{j'} 
                \cdot m_{i'} \cdot m_{j'}
            }
        }
    \end{align}

    This probability distribution is what we will use in the following to construct a randomly
    sampled kernel. Its definition will always depend on the given dust particle mass 
    distribution: Depending on 
    what kind of particles are present at the moment (and the masses they carry), the most 
    relevant collisions can vary significantly. \\ 

    Note that this sampling will be implemented ``without returning'', i.e. each collision 
    will only be chosen a single time at most per time-step. \\

    Also, to filter out less relevant collisions, we will implement a threshold probability 
    value $P^\text{sample}_{min}$ that needs to be reached before a collision is deemed relevant 
    enough to be sampled. Assuming a mass grid resolution of $\mathcal N_m=100$, we will set this to
    \begin{equation}
        P^\text{sample}_{min} := 10^{-9}
    \end{equation}

% }}}
% First tests {{{ 
% \clearpage\section{First Tests}  % TODO Rename

    % \todo{Test (also for other sampling methods):}
    % $$K_\text{complete}=K(\mathcal{N}_s=\mathcal{N}_m^2)$$

% }}}
% Sampling "without returning" {{{ 
% \section{Sampling "without returning"}

    % The first case that we will consider will make use of "sampling without returning". Here, a 
    % particle pair $(i, j)$ is only allowed to be chosen once per time-step. \\
    % \todo{Plot probability distribution for various times.} \\
    % \todo{Plot $n(m,t)$ for various sampling densities.} \\
    % \todo{Plot $\deriv{n}{t}$ against sampling density, for constant mass grid resolution.} \\
    % \todo{Plot $\deriv{n}{t}$ against mass grid resolution, for constant sampling density.} \\

% }}}
% Sampling "with returning" {{{ 
% \section{Sampling "with returning"}

    % Now, we will consider the case of "sampling with returning", i.e. we will now allow particle 
    % pairs to be chosen more than once per time-step.

% }}}
% Introduction to Monte Carlo Sampling {{{
% \clearpage\section{Introduction to Monte Carlo Sampling}

% \todo{Before we start: Take a look at simple examples.} \\

    % Monte Carlo Integration in 1D {{{ 
    % \subsection{Monte Carlo Integration in 1D}

    %     \todo{Consider a function $f$ that is defined on an interval $D\subset\mathbb R$
    %     (from $x_\text{min}$ to $x_\text{max}$).} \\
    %     \todo{To demonstrate the basic idea of Monte Carlo integration, and without loss of 
    %     generality, let us for now assume that the image of $D$ under $f$ follows a Gaussian 
    %     distribution.}
    %     \begin{equation}
    %         f: 
    %         D \to \mathbb R, 
    %         x \mapsto \frac{1}{\sqrt{2\pi}\sigma}\exp\bigg(\frac{-(x-\mu)^2}{2\sigma^2}\bigg)
    %     \end{equation}

    %     We would now like to calculate an approximate value for the integral 
    %     \begin{equation}
    %         I = \int\limits_{x_\text{min}}^{x_\text{max}} f(x) \ \text dx
    %     \end{equation}

    %     Sample a total of $\mathcal N_\text{sample} \in \mathbb N$ points from the interval $D$. \\

    %     \todo{This will be done according to a probability distribution $P(x)$, the definition of 
    %     which will be the crucial thing to do. (?)} \\

    %     With $i \in [1, N_\text{sample}] \cap \mathbb N$, let $x_i$ label the chosen values. \\

    % }}}
    % Monte Carlo Integration in 2D {{{ 
    % \subsection{Monte Carlo Integration in 2D}

    % }}}

% }}}
% Visualizations {{{ 

    % Sampling of the Pure Coagulation Kernel {{{ 
    \cleardoublepage\section{Sampling of the Pure Coagulation Kernel}

        In \cref{fig:sampled_temporal_evolution_of_mass_distribution_for_only_coag}, the 
        temporal evolution of the dust particle mass distribution is plotted for 
        various different values of the sampling density $\rho_\text{sample}$. Here, 
        only the process of pure hit-and-stick coagulation is included into the kernel. \\
        % For an analogous plot with the full coagulation+fragmentation model, see 
        % \cref{fig:sampled_temporal_evolution_of_mass_distribution_for_full_model}. \\

        \textbf{Observation 1:} As can be seen when comparing the plots for different 
        values of $\rho_\text{sample}$, the accuracy of the integration suffers when 
        the sampling density is set to low values. This is of course not unexpected,
        since low sampling densities mean that a lot of possible collisions are neglected,
        leading to an incomplete picture of the particle collision dynamics. \\

        \textbf{Observation 2:} There are statistical fluctuations in the numerical solutions,
        such that the numerical solution is much less ``smooth'' as it would be if it were 
        calculated using the complete kernel matrix.
        This is to be expected as well, as not all collisions are taken into account 
        in each time-step, and there exists a certain under- or over-representation 
        of certain collisions due to the probabilistic nature of the method that is 
        used for selecting kernel the entries that will be included into the summation. \\

        % \textbf{Observation 3:} For the cases of $\rho_\text{sample} = 0.6$,
        % $\rho_\text{sample} = 0.7$, and $\rho_\text{sample} = 0.8$, the final state of the mass
        % distribution takes on a form that looks very much like what we see for the solution 
        % calculated using the complete kernel. In contrast to that, the first four plots show 
        % a significantly different final state. \todo{...} \\
        % One might wonder why that is, but there is a quite simple explanation:

        In \cref{fig:sampling_probability_for_only_coag}, the sampling probability 
        $P_{ij}^\text{sample}$ is plotted for various points in time throughout the 
        numerical integration. \cref{fig:nr_of_samples_for_only_coag} on the other 
        hand attempts to display the number of times $N^\text{sample}_{ij}$ that each 
        collisions $(i, j)$ was sampled. Since we are implementing a sampling 
        ``without returning'', this $N^\text{sample}_{ij}$ can only take on a value of 
        either 0 or 1. \\

        For the sampling density, we adopt a value of $\rho_\text{sample} = 0.5$ here. \\

        \textbf{Observation 3}: Initially, i.e. when forwarding the mass distribution from 
        the initial state to the next time-step for the first time, only a single entry of 
        the kernel matrix has to be considered for the summation. This changes as time 
        progresses, and the number of sampled collisions increases as more and more bins 
        in the mass distribution carry a non-zero value. \\

        \textbf{Observation 4:} Towards the end of the simulation, the number of sampled 
        collisions decreases again. In the simple case of hit-and-stick coagulation, on a 
        discretized mass axis with an upper limit $m_\text{max}$, at the end almost the 
        entire mass will be contained in the bin corresponding to the largest mass value. 
        As such, only collisions between particles from these high-mass bins need to be 
        considered.

        \clearpage
        \begin{figure}[h!]
            \makebox[\textwidth]{
                \includegraphics[width=\paperwidth]{
                104/3x2 rho_d vs. m, t, rho_sample, N_m=50, coag=True, frag=False.pdf}
            }
            \caption{
                Temporal evolution of the dust particle mass distribution under the influence
                of (only) pure hit-and-stick coagulation processes for various different 
                values of the sampling density $\rho_\text{sample}$.
            }
            \label{fig:sampled_temporal_evolution_of_mass_distribution_for_only_coag}
        \end{figure} 
        \clearpage
        \begin{figure}[h!]
            \makebox[\textwidth]{
                \includegraphics[width=\paperwidth]{105/coag=True frag=False rho_sample=0.5 P_ij.pdf}
            }
            \caption{
                % $P_{ij}^\text{sample}$ for
                % $\rho_\text{sample} = 0.5$,
                % $\mathcal N_m = 50$,
                % $t = $\textcolor{red}{?}
                Illustration of the collision sampling probability distribution 
                $P_{ij}^\text{sample}$ for $\rho_\text{sample} = 0.5$. Here, only the process 
                of pure stick-and-hit coagulation is included into the model.
            }
            \label{fig:sampling_probability_for_only_coag}
        \end{figure} 
        \clearpage
        \begin{figure}[h!]
            \makebox[\textwidth]{
                \includegraphics[width=\paperwidth]{105/coag=True frag=False rho_sample=0.5 S_ij.pdf}
            }
            \caption{
                % $\rho_\text{sample} = 0.5$,
                % $\mathcal N_m = 50$,
                % $t = $\textcolor{red}{?}
                Illustration of the amount of times $N^\text{sample}_{ij}$ that each collision 
                $(i,j)$ is sampled, for $\rho_\text{sample} = 0.5$. Here, only the process 
                of pure stick-and-hit coagulation is included into the model.
            }
            \label{fig:nr_of_samples_for_only_coag}
        \end{figure} 
    
    % }}}
    % Sampling of the Pure Fragmentation Kernel {{{ 
    % \cleardoublepage\section{Sampling of the Pure Fragmentation Kernel}
 
        % \clearpage
        % \begin{figure}[h!]
        %     \makebox[\textwidth]{
        %         \includegraphics[width=\paperwidth]{105/coag=False frag=True rho_sample=0.5 P_ij.pdf}
        %     }
        %     \caption{\todo{}}
        % \end{figure} 
        % \clearpage
        % \begin{figure}[h!]
        %     \makebox[\textwidth]{
        %         \includegraphics[width=\paperwidth]{105/coag=False frag=True rho_sample=0.5 S_ij.pdf}
        %     }
        %     \caption{\todo{what $\rho_\text{sample}$?}}
        % \end{figure} 
        % \clearpage
        % \begin{figure}[h!]
        %     \makebox[\textwidth]{
        %         \includegraphics[width=\paperwidth]{
        %         104/3x2 rho_d vs. m, t, rho_sample, N_m=50, coag=False, frag=True.pdf}
        %     }
        %     \caption{\todo{only fragmentation}}
        % \end{figure} 
          
        % \clearpage
        % \begin{figure}[h!]
        %     \centering
        %     \includegraphics[width=\textwidth]{
        %         3x2 rho_d vs. m, t, rho_sample, N_m=25, enable_fragmentation=False.pdf}
        %     \caption{}
        % \end{figure} 
        % \clearpage
        % \begin{figure}[h!]
        %     \centering
        %     \includegraphics[width=\textwidth]{
        %         3x2 rho_d vs. m, t, rho_sample, N_m=25, enable_fragmentation=True.pdf}
        %     \caption{}
        % \end{figure} 

    % }}}
    % Sampling of the Total Kernel {{{ 
    \cleardoublepage\section{Sampling of the Total Kernel}

        The plots in 
        \cref{fig:sampled_temporal_evolution_of_mass_distribution_for_full_model},
        \cref{fig:sampling_probability_for_full_model}, and 
        \cref{fig:nr_of_samples_for_full_model} display analogous results to 
        what we saw in 
        \cref{fig:sampled_temporal_evolution_of_mass_distribution_for_only_coag},
        \cref{fig:sampling_probability_for_only_coag}, and 
        \cref{fig:nr_of_samples_for_only_coag}. The difference is that here, 
        the full model is used, i.e. both coagulation and fragmentation 
        processes are taken into account when defining the kernel matrix. \\

        We will focus on the plots displaying the mass distribution's temporal 
        evolution first: \\

        \textbf{Observation 1:} Once again, there are visible fluctuations in the numerical 
        solution of the particle mass distribution. As one might expect, these decrease in 
        magnitude when the sampling density is increased. \\

        \textbf{Observation 2:} Even though the accuracy may suffer from low sampling densities,
        the results shown here can still be seen as a success, as the mass distribution
        \textit{does} evolve into a structure that visibily resembles the equilibrium that is
        reached in the ``complete'' solution. This is quite satisfactory, as it shows that the 
        sampling method could be implemented in a sensible manner. \\

        \textbf{Observation 3:} It might not be obvious at first sight, but the plots showcase
        another interesting behavior of the solutions resulting from a sampled kernel: 
        If you look closely, you will notice that the evolution of the mass distribution 
        progresses much slower towards the equilibrium state if the sampling density is low. 
        This of course is nothing unexpected: If not all collisions are taken into account,
        this effectively leads to a lower collision rate, and thus a slower evolution 
        of the distribution in general. \\
        % This can be seen most prominently in the first three plots. 
        % To visualize this more directly, we will take a closer look at
        % the magnitude of the temporal derivative of the mass distribution in [\todo{ref}]. \\

        % \textbf{Observation 1:} 
        % Equilibrium is reached 
        % more slowly
        % temporal derivative does not go to zero

        \textbf{Observation 4:} 
        It might not be that easy to recognize, since multiple distributions are plotted on 
        top of each other, another interesting feature one can notice is the existence 
        of peak-like structures at the upper end of the mass axis. 
        This could hint at an under- or over-representation of collisions involving particles 
        whose masses are close to the upper ``edge'' of the equilibrium state's mass distribution.
        This might be fixed by adjusting the definition of the sampling probability in a 
        sensible manner.
        % This is true 
        % Since we are 
        % plotting multiple distributions for different values of $t$ on top of each other, 
        % this might not be obvious, which why we present this feature in [\todo{ref}].
        \\
        
        Let us now take a look at the figures displaying the distribution of the 
        sampling probability and the distribution of sampled collisions.
        Like before, here we adopt a sampling density value of $\rho_\text{sample} = 0.5$. \\

        \textbf{Observation 5:}
        When comparing the distributions $P^\text{sample}_{ij}$ and $N^\text{sample}_{ij}$
        with the results from the previous section, we can see that the temporal 
        evolution behaves very similarly, at least during the early parts of the 
        simulation. \\

        \textbf{Observation 6:} Once big enough particles could form, allowing the 
        relative particle velocities to surpass the fragmentation threshold 
        velocity $v_\text{frag}$, the similarities come to an end. Because the low-mass 
        bins are ``filled up'' again by collison events leading to particle fragmentation,
        they will have to be taken into account here, in contrast to what we saw in the 
        case of pure coagulation. \\

        \textbf{Observation 7:} There exists a ``hole'' or ``gap'' in the sampling probability
        distribution $P_{ij}^\text{sample}$ for collisions where both particles carry 
        neither a very large nor a very small mass value. 
        Thus, towards the end of the simulation, when the equilibrium state begins to 
        settle in, the most relevant collisions appear to be those containing either 
        at least one large particke, or only small ones. This reflects the fact that the 
        particle populations with those two respective sizes are the most prevalent in 
        the distribution towards the end.
        % This can be explained by the 
        % fact that these collisions neither involve relative velocities
        \\

        % \todo{Plot zoomed in on peak at large masses.}

        % Observations:
        % \begin{itemize}
        %     \item Equilibrium is reached more slowly \\ 
        %         This makes sense: Some terms are ignored. \\
        %         See plot of temporal derivatives.
        %         % \todo{Quantify the increase in time! Plot!}
        %     \item There are fluctuations in the numerical solutions. \\ 
        %         This makes sense: Some terms are ignored.
        %     \item For constant sampling density, 
        %         the accuracy increases with the mass grid resolution \\
        %         % (\todo{Quantify gain! Plot!})
        %     \item For constant mass grid resolution, 
        %         the accuracy increases with the sampling density \\
        %         % (\todo{Quantify gain! Plot!})
        %     % \item ...
        % \end{itemize}
    
        \clearpage
        \begin{figure}[h!]
            \makebox[\textwidth]{
                \includegraphics[width=\paperwidth]{
                104/3x2 rho_d vs. m, t, rho_sample, N_m=50, coag=True, frag=True.pdf}
            }
            \caption{
                Temporal evolution of the dust particle mass distribution under the influence
                of both coagulation and fragmentation processes for various different 
                values of the sampling density $\rho_\text{sample}$.
            }
            \label{fig:sampled_temporal_evolution_of_mass_distribution_for_full_model}
        \end{figure} 
        \clearpage
        \begin{figure}[h!]
            \makebox[\textwidth]{
                \includegraphics[width=\paperwidth]{105/coag=True frag=True rho_sample=0.5 P_ij.pdf}
            }
            \caption{
                Illustration of the collision sampling probability distribution 
                $P_{ij}^\text{sample}$ for $\rho_\text{sample} = 0.5$. Here, both the process 
                of coagulation as well as fragmentation are included into the model.
                % $\mathcal N_m = 50$,
                % $t = $\textcolor{red}{?}
            }
            \label{fig:sampling_probability_for_full_model}
        \end{figure} 
        \clearpage
        \begin{figure}[h!]
            \makebox[\textwidth]{
                \includegraphics[width=\paperwidth]{105/coag=True frag=True rho_sample=0.5 S_ij.pdf}
            }
            \caption{
                % $\rho_\text{sample} = 0.5$,
                % $\mathcal N_m = 50$,
                % $t = $\textcolor{red}{?}
                Illustration of the amount of times $N^\text{sample}_{ij}$ that each collision 
                $(i,j)$ is sampled, for $\rho_\text{sample} = 0.5$. Here, both the process 
                of coagulation as well as fragmentation are included into the model.
            }
            \label{fig:nr_of_samples_for_full_model}
        \end{figure} 

    % }}}
% }}}
% Accuracy and Stability {{{ 

    \clearpage
    \section{Accuracy and Stability}

    Additionally to the stability error $\Delta_\text{stab}$ that we defined in
    \cref{eq:definition_of_integration_mass_error_stability}, and which is displayed 
    again below, the following definition of $\Delta_\text{acc}$ gives an 
    insight into the accuracy properties of the integration:
    % We define the stability and accuracy of the algorithm as
    \begin{align}
        \Delta_\text{stab}(t)
            &= \sum_i \frac{\rho^d_i(t) - \rho^d_i(t=0)}{\rho^d_i(t=0)} \\
        \Delta_\text{acc}(t)
            &= \sqrt{ \sum_i \left(
                \frac{\rho^d_i(t) - \rho^c_i(t)}{\rho^c_i(t)}
            \right)^2 }
    \end{align}
    Here, $\rho^c_i(t)$ denotes the fraction of the dust mass volume density 
    contained in a given bin $i$ when integrating the model 
    using the \textit{complete} kernel (hence the superfix ``c''). \\

    Instead of comparing the total mass density value summed over all bins with its 
    initial value at the start of the simulation, in the definition of the accuracy error 
    we calculate the root-mean-square of the relative deviation (at each mass bin)
    between the numerical solution arrived at using the sampled kernel, and the complete one. 
    The evolution of these two quantities with respect to time is displayed in 
    \cref{fig:stab_and_acc_error}. In green, we have the stability or mass error
    (where a dashed line signifies a negative value), and in blue we have the accuracy
    error that we just defined.\footnote{
        Note that there a missing values in the evolution of the accuracy error,
        which occur when the mass distribution in the ``complete'' solution contains 
        zeros. This is the case in the beginning, before coagulation processes can 
        transfer mass into each bin, and again at near the end, when the lower-mass 
        bins empty before fragmentation fills them up again. 
    }
    \\

    While the stability error could be kept near machine precision for the entirety of the 
    simulation, the accuracy error is much greater than 1. 
    This could be explained partly by statistical fluctuations induced by the 
    stochastic sampling of the kernel, and partly by the effectively lower reaction 
    rates, causing the mass distribution to ``lag'' behind the solution calculated with 
    the full kernel. \\

    Since fragmentation process only start to be relevant in the latter parts of the 
    integration, the statistical fluctuations can be assumed to 
    effect the accuracy error most significantly at the end as well, whereas the 
    lower reactions rates might well be the more significant contributor to 
    the error in the early stages of the simulation. \\

    This is due to the fact that, apart from a possibly slightly slower evolution, 
    for pure coagulation we can reproduce the results from a complete kernel quite 
    well even for lower sampling densities. This stems of course from the 
    sparsity of the kernel for pure hit-and-stick coagulation. \\

    In any case, while the results from the previous pages show a satisfactory 
    \textit{qualitative} behavior, the accuracy error supplies reasons for 
    doubting the utility of this method when low deviations from the complete 
    kernel solution is to be achieved. \\

    Still, it could be shown that the core idea of the method can be used to 
    reproduce the desired results in at least an approximate fashion. 
    
    \begin{figure}[h!]
        \makebox[\textwidth]{\includegraphics[width=\paperwidth]{
            106/accuracy_and_stability_vs_sampling_density}
        }
        \caption{
            Accuracy error $\Delta_\text{acc}$ and stability error $\Delta_\text{stab}$ for the 
            numerical integration of the Smoluchowski coagulation equation for various sampling 
            densities $\rho_\text{sample}$. 
            Stability is assured down to machine precision, accuracy on the other hand is 
            negatively impacted by statistical fluctuations arising from the stochastic 
            nature of the Monte Carlo method.
            For the stability error, the dashed line signifies a negative value. \\
        }
        \label{fig:stab_and_acc_error}
    \end{figure}
    % \todo{how to quantify accuracy?}
    % \begin{itemize}
    %     \item average over multiple runs
    % \end{itemize}
    
    % Let $\rho = \rho^\text{dust}$.
    % \begin{align}
    %     \rho^\text{dust}
    %         &= \sum_{i=1}^{\mathcal{N}_m} \rho_i%^\text{dust} 
    %     \\
    %     \rho_i%^\text{dust}
    %         &= n_i \cdot m_i \cdot \Delta m_i 
    % \end{align}
    % \begin{align}
    %     \Delta\rho_\text{A}
    %         &= \sqrt{\sum_{i=1}^{{\mathcal N}_m} \big(
    %             \Delta\rho_i%^\text{dust}
    %         \big)^2} 
    %         = \sqrt{\sum_{i=1}^{{\mathcal N}_m} \bigg(
    %             \rho_i^\text{sampled} - \rho_i^\text{complete}
    %         \bigg)^2} 
    % \end{align}
    % \begin{align}
    %         \rho &= \sum_{i=1}^{\mathcal{N}_m} \rho_i \\
    %     \Delta\rho_\text{B} &= \rho - \rho^\text{complete} \\
    % \end{align}

% }}}
    % Temporal Derivative {{{ 
    \clearpage
    
        \begin{figure}[h!]
            \makebox[\textwidth]{\includegraphics[width=\paperwidth]{
                107/dMdt vs t vs rho_sample, coag=True, frag=True.pdf}
            }
            \caption{
                Temporal evolution of the root-mean-square (RMS) of the temporal derivative 
                of the mass distribution.
                % $\sqrt{ \sum_{i=1}^{\mathcal N_m} \left( \frac{\Delta \rho_i}{\Delta t} \right)^2 }$
                Here, both the process of coagulation as well as
                fragmentation are included into the model. 
                Note that when sampling is enabled, the temporal derivative does not approach 
                zero as quickly as when it is disabled.
                This is most likely due to the stochastic nature of the Monte Carlo sampling
                method, leading to statistical fluctuations that prevent the mass distribution 
                from ultimately settling into a true equilibrium state.
                % Since not all possible particle 
                % collisions are taken into account when integrating with a sampled kernel,
                % the magnitude of this temporal derivative increases with the value of the 
                % sampling density.
            }
        \end{figure}
        \vfill
        \begin{figure}[h!]
            \makebox[\textwidth]{\includegraphics[width=\paperwidth]{
                107/dMdt vs t vs rho_sample, coag=True, frag=False.pdf}
            }
            \caption{
                Temporal evolution of the root-mean-square (RMS) of the temporal derivative 
                of the mass distribution.
                % $\sqrt{ \sum_{i=1}^{\mathcal N_m} \left( \frac{\Delta \rho_i}{\Delta t} \right)^2 }$
                Here, only the process of pure hit-and-stick coagulation
                is included into the model. Interestingly, here the temporal derivatives 
                that were arrived at using a sampled kernel initially approach zero much faster 
                than when using the complete kernel. At some point, also here the temporal 
                derivative of the ``complete'' solution overtakes the sampled soltions on its 
                way towards zero.
                \ \\
            }
        \end{figure}
        % \begin{figure}[h!]
        %     \makebox[\textwidth]{
        %         \includegraphics[width=\paperwidth]{107/dMdt vs t vs rho_sample, coag=False, frag=True.pdf}
        %     }
        %     \caption{only fragmentation}
        % \end{figure}

    % }}}
    % Percentage of Non-Zero Kernel Entries {{{ 
    
        \clearpage
        \begin{figure}[h!]
            \makebox[\textwidth]{\includegraphics[width=\paperwidth]{
                109/percentage non-zero kernel, coag=True, frag=True.pdf}
            }
            \caption{
                Percentage of non-zero entries in the sampled kernel matrix.
                Here, both coagulation and fragmentation are included in the model. 
                If $\rho_\text{sample}=1$, i.e. if all collisions 
                are taken into account when defining the kernel, this ratio is 
                approximately 30\%. Since a lot of these collisions are neglected when 
                a sampled kernel is used, the number 
                of non-zero kernel entries decreases with lower values of the sampling 
                density. This is especially true in the beginning, where fragmentation 
                processes do not yet happen much. As time progresses, more and more 
                entries of the kernel become relevant. As such, this ratio increases and, 
                in the end, converges towards the value of the sampling density.
            }
            \label{fig:percentage_non_zero_only_coag}
        \end{figure}
        \vfill
        \begin{figure}[h!]
            \makebox[\textwidth]{\includegraphics[width=\paperwidth]{
                109/percentage non-zero kernel, coag=True, frag=False.pdf}
            }
            \caption{
                Percentage non-zero entries in the sampled kernel matrix.
                Here, only the process of pure stick-and-hit coagulation 
                is included into the model. The number of non-zero entries 
                increases as time progresses, until there is a steep drop-off when basically 
                the entire mass has been transferred into the highest-mass 
                bin. Overall, the number of non-zero kernel entries is much lower 
                here than in the full model, reflecting the fact that 
                due to the Dirac $\delta$-function in \cref{eq:definition_of_f_coag}, 
                the definition of the
                pure coagulation kernel yields a much sparser matrix than one would have 
                if fragmentation processes are included, with only approximately 3\% of 
                the kernel being non-zero for $\rho_\text{sample}=1$.
            }
        \end{figure}
        % \clearpage
        % \begin{figure}[h!]
        %     \makebox[\textwidth]{
        %         \includegraphics[width=\paperwidth]{109/percentage non-zero kernel, coag=False, frag=True.pdf}
        %     }
        %     \caption{only fragmentation}
        % \end{figure}

    % }}} 
% Other {{{ 
    
    % \clearpage

    % Integrate Smoluchowski equation up to time $t=X?$.
    % Using the particle mass distribution function at that point in time, 
    % calculate the probability density $P_{ij}^\text{sample}$.
    
    % With this probability distribution, sample the kernel multiple times $\mathcal N_L$. 
    % For each time $l$, "perform the summation" to forward the mass distribution by one time-step.
    % Using that new mass distribution, calculate the numerical temporal derivative
    % \begin{equation}
    %     \frac{\Delta \rho_i^l}{\Delta t} = \frac{\Delta n_i^l(t)}{\Delta t} \cdot m_i \cdot \Delta m_i
    % \end{equation}
    
    % Calculate the average temporal derivative
    % \begin{equation}
    %     \frac{\Delta \rho^\text{avg}_i}{\Delta t} 
    %     = \sum_{l=1}^{\mathcal N_L} \frac{\Delta n_i^l(t)}{\Delta t} \cdot m_i \cdot \Delta m_i
    % \end{equation}
    
    % \todo{Plot $\frac{\rho_i^\text{avg}}{\Delta t}$ vs. $l$ or $\mathcal N_L$.} \\
    % \todo{Plot $\frac{\rho_i^\text{avg}}{\Delta t}$ $\rho_\text{sample}$.} \\
    % \todo{Plot $\frac{\rho_i^\text{avg}}{\Delta t}$ $\mathcal N_m$.}
    % \todo{Plot eff. sampling density vs. sampling density for various mass grid resolutions.} \\
    % \todo{Plot density error vs. mass grid resolution for various sampling densities.} \\
    % \todo{Plot density error vs. max. sampling density for various mass grid resolutions.} \\
    % \todo{? Plot eff. sampling density vs. mass grid resolution for various sampling densities.} \\

% \todo{sampling density} \\
% sample only half of kernel matrix $K_{kij} = K_{kji}$ (only "upper left" half) \\

% \clearpage

% \begin{table}[h!]
%     \begin{center}
%     \caption{\todo{error vs. sampling density}}
%     \begin{tabular}{r r r r r r r}
%               id 
%             & $\mathcal N_m$  
%             & $\mathcal N_\text{sample}$  
%             & $\rho_\text{sample}$  
%             & $\rho_\text{sample}^\text{total}$        
%             & ($\Delta \rho_\text{dust}$)
%             & $\Delta \rho_\text{dust}^\text{rel}$        
%             \\
%             \hline
%                 0 & & & 0.1 & & x & y \\
%                 1 & & & 0.2 & & x & y \\
%                 3 & & & 0.4 & & x & y \\
%                 3 & & & 0.6 & & x & y \\
%                 3 & & & 0.8 & & x & y \\
%                 3 & & & 1.0 & & x & y \\
%     \end{tabular}
%     \end{center}
% \end{table} 
% \begin{table}[h!]
%     \begin{center}
%     \caption{\todo{error vs. mass grid resolution}}
%     \begin{tabular}{r r r r r r r}
%               id 
%             & $\mathcal N_m$  
%             & $\mathcal N_\text{sample}$  
%             & $\rho_\text{sample}$  
%             & $\rho_\text{sample}^\text{total}$        
%             & ($\Delta \rho_\text{dust}$)
%             & $\Delta \rho_\text{dust}^\text{rel}$        
%             \\
%             \hline
%                 0 &  25 & & & & x & y \\
%                 1 &  50 & & & & x & y \\
%                 3 & 100 & & & & x & y \\
%                 3 & 200 & & & & x & y \\
%                 3 & 250 & & & & x & y \\
%                 3 & 500 & & & & x & y \\
%     \end{tabular}
%     \end{center}
% \end{table} \ \\ 

% \clearpage

% \clearpage


% \begin{table}[h!]
%     \begin{center}
%     \caption{}
%     \begin{tabular}{r r r r r r}
%               id 
%             & $\mathcal N_m$  
%             & $\mathcal N_\text{sample}$  
%             & $\rho_\text{sample}$  
%             & $\rho_\text{sample}^\text{total}$        
%             & $\Delta \rho$        
%             \\
%                 0 & (25) &      & & & x \\
%                 1 &  50  &      & & & x \\
%                 2 & 100  &      & & & x \\
%                 3 & 250  &      & & & x \\
%                 4 & 500  &      & & & x \\
%     \end{tabular}
%     \end{center}
% \end{table} \ \\ 


% % \begin{table}[h!]
% %     \begin{center}
% %     \caption{}
% %     \begin{tabular}{l & l}
% %         $\mathcal N_m$  
% %         & $\mathcal N_\text{sample}$  
% %     \end{tabular}
% %     \end{center}
% % \end{table} \ \\ 

% \begin{table}[h!]
%     \begin{center}
%     \caption{}
%     \begin{tabular}{r r}
%               $\mathcal N_m$  
%             & $\rho_\text{sample}$        
%             \\
%                 25 &  \\
%                 50 &  \\
%                100 &  \\
%                250 &  \\
%                500 &  \\
%     \end{tabular}
%     \end{center}
% \end{table} \ \\ 

% \begin{table}[h!]
%     \begin{center}
%     \caption{}
%     \begin{tabular}{r r r r r}
%               id 
%             & $\mathcal N_m$  
%             & $\mathcal N_\text{sample}$  
%             & $\rho_\text{sample}$  
%             & $\rho_\text{sample}^\text{total}$        
%             \\
%                 0 & (25) &      & & \\
%                 1 &  50  &      & & \\
%                 2 & 100  &      & & \\
%                 3 & 250  &      & & \\
%                 4 & 500  &      & & \\
%     \end{tabular}
%     \end{center}
% \end{table} \ \\ 

% \begin{table}[h!]
%     \begin{center}
%     \caption{}
%     \begin{tabular}{}
%               $\mathcal N_m$  
%             & $\mathcal N_\text{sample}$  
%             & $\rho_\text{sample}$  
%             & $\rho_\text{mc}^\text{tot}$        
%             \\
%                 (25)    & & & \\
%                  50     & & & \\
%                 100     & & & \\
%                 200     & & & \\
%     \end{tabular}
%     \end{center}
% \end{table} \ \\ 

% Sample kernel using probability from mass distribution.
% Calculate numerical temporal derivative
% \todo{Perform $\mathcal N_\text{r}$ runs with .}
% \todo{Plot }

% }}}
