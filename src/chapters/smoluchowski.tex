% The Smoluchowski Equation {{{ 
\section{The Smoluchowski Equation}

    Let us now turn our attention to a certain integro-differential equation, which serves as the
    centerpiece of the dust particle coagulation model that this thesis is built upon. \\

    The so-called \textit{Smoluchowski equation} provides a mathematical framework commonly used 
    to describe the process of particle aggregation, or \textit{coagulation}, into larger and 
    larger structures. \\

    This population balance equation is used in a wide range of physical and biological contexts. 
    (examples?) \\

    It is named after the Polish physicist Marian Smoluchowski, due to his treatment of the 
    equation in 1916 \cite{smoluchowski_1916}. \\

    % \todo{ODE or PDE?} \\
    \todo{For each choice of \textit{kernel function}, there exists a unique solution (cite).}
    (and initial conditions)

    % Continuous Formulation {{{
    \subsection{Continuous Formulation}

        \todo{For the evolution of the particle mass distribution function, (depending 
        \textit{only} on the particle mass), the Smoluchowski equation in its most general form 
        reads:}
    
        \begin{equation}
            \label{eq:continuous_smoluchowski_equation}
            \pderiv{n}{t}(m)
                =
                    \int\limits_0^\infty
                    \int\limits_0^\infty
                    K(m,m',m'')
                    \cdot n(m')
                    \cdot n(m'')
                    \ \dd m'
                    \ \dd m''
        \end{equation}

        \todo{Here, the term $K(m, m', m'')$ labels the \textit{coagulation kernel}. 
        It carries the information about how the number of particles with mass $m$ changes with 
        time under the influence of collisions between particles with masses $m'$ and $m''$.} \\

        \todo{We will discuss it in [cite section], and [coag-section] and [frag-section].}
 
    % }}}
    % Discretized Formulation {{{ 
    \newpage\subsection{Discretized Formulation}

        \todo{Now, discretize the Smoluchowski equation.} \\

        To do this, we perform the following two mappings:
        \begin{enumerate}
            \item Replace temporal derivative of particle mass distribution function with 
                finite difference.
                \begin{equation}
                    \pderiv{n}{t} \to \frac{\Delta n}{\Delta t}
                \end{equation}
            \item Replace integral over particle mass axis with discrete sum.
                \begin{equation}
                    \int_0^\infty \dd m' \to \sum_{i=1}^{\mathcal N_m} \Delta m_i
                    \ \ \ \text{and}\ \ \
                    \int_0^\infty \dd m'' \to \sum_{j=1}^{\mathcal N_m} \Delta m_j
                \end{equation} 
        \end{enumerate}

        \todo{This results in:}
        \begin{equation}
            \label{eq:discrete_smoluchowski_equation}
            \pderiv{n_k}{t}
                \approx \sum_{i=1}^{\mathcal N_m}\sum_{j=1}^{\mathcal N_m}
                    K(m_k,m_i,m_j)\cdot n_i\cdot n_j\cdot\Delta m_i\cdot\Delta m_j
        \end{equation}
        % TODO Later on I talk about `del n / del t` instead of `Delta n / Delta t`.
        %    -> Decide on which one of these to use.
    
        Let us define
        \begin{equation}
            K_{kij}
                := K(m_k,m_i,m_j)\cdot\Delta m_k
        \end{equation}
    
        If, in addition, we make use of the relationship $N_i=n_i\cdot\Delta m_i$ that was defined 
        in (\todo{cite}), then we can rewrite the Smoluchowski equation as
        \begin{equation}
            \pderiv{N_k}{t}
                \approx \sum_{i=1}^{\mathcal N_m}\sum_{j=1}^{\mathcal N_m} K_{kij}\cdot N_i\cdot N_j
        \end{equation}
    
    % }}}

% }}}
% The Kernel {{{ 
\newpage\section{The Kernel}

    In the following, we will factorize the kernel into two separate terms:
    \begin{equation}
        K(m,m',m'') = R(m',m'') \cdot X(m,m',m'') 
    \end{equation}

    The idea behind this distinction is the following: 
    \begin{enumerate}
        \item The first term $R(m',m'')$ is used to encode information about 
            \textit{how often collision events occur} per unit time between a given
            pair of particles carrying the masses $m'$ and $m''$. \\
            (\todo{It is defined as in [cite], separately for coagulation and fragmentation})
        \item In contrast to that, the second term $X(m,m',m'')$ helps us understand 
            \textit{how this affects the number of particles} carrying a mass value $m$ under any
            such collision, i.e. it gives information about how matter is redistributed onto the
            range of possible masses.
            (\todo{for each collision})
    \end{enumerate}

    % \todo{Talk about units! (not 1/second)} \\

    \todo{Kernel symmetry:} 
    \begin{itemize}
        \item \todo{(only use "upper left" half of the matrix)}
    \end{itemize}

    % \clearpage
    % \todo{...} \\

    % \todo{The second term, on the other hand, will be used to encode how the initial masses 
    % $m'$ and $m''$ are then distributed onto the range of particles resulting from such a 
    % collision, each carrying a mass of value $m$.} \\

    % Physical Units {{{ 
    % \subsection{Physical Units (\todo{Is this section needed?})}
    % }}}
    % Discretization {{{ 
    % \subsection{Discretization}

        % \begin{equation}
        %     K(m, m', m'')
        %         \to
        %             K_{kij}
        %         =
        %             K(m_k, m_i, m_j) \cdot\Delta m_k
        % \end{equation}

    % }}}
    % Decomposition into Gain and Loss Terms {{{ 
    \clearpage\subsection{Decomposition into Gain and Loss Terms}

        We can further decompose the "redistribution" kernel $K(m,m',m'')$ into a negative and a 
        positive contribution, which in the following we will address as the \textit{loss} and 
        \textit{gain} term $L$ and $G$, respectively:
        \begin{equation}
            K(m,m',m'') = L(m, m', m'') + G(m, m', m'') 
        \end{equation}
        % \begin{equation}
        %     K(m,m',m'') = R(m',m'') \cdot \bigg[G(m, m', m'') + L(m, m', m'')\bigg]
        % \end{equation}

        We will define these two terms as follows:
        \begin{align}
            L(m, m', m'') 
                &:= -R(m',m'') \cdot \frac{1}{2} \bigg[\delta_D(m-m')+\delta_D(m-m'')\bigg] \\
            G(m, m', m'') 
                &:= +R(m',m'') \cdot f(m, m', m'') 
        \end{align}

        \todo{The two separate Dirac distributions in the square brackets relate to the symmetry 
        of the kernel, which satisfies the following relationship.} \\
        \todo{Continuous case:}
        \begin{equation}
            K(m,m',m'') = K(m,m'',m')
        \end{equation}
        \todo{Discrete case:}
        \begin{equation}
            K_{kij} = K_{kji}
        \end{equation}
        \todo{I.e.: It is irrelevant whether a particle A collides with a particle B, or whether 
        a particle B collides with particle A. These are the same collision events}. \\
        \todo{They are both encoded into the kernel though, which is why we need the factor $1/2$ 
        in order to not count them twice.} \\
        \todo{For the numerical integration that we will do later, we will of course not sum over 
        the entire kernel and divide by 2 afterwards. That would lead to an unnecessary factor 2 
        in the necessary computations. Instead, we will map the lower right half of the kernel 
        matrix onto the top left (as we will see later).} \\

        When evaluating the sum in [cite Smol. eq.] during integration, this will have the following 
        effect:
        \begin{enumerate}
            \item For each collision between two particles $m'$ and $m''$, a single particle will 
                be removed from the distribution for each of these two masses, hence the term $-1$ 
                in the equation above. (\todo{fix this})
            \item At the same time, the total mass involved in the collision, which is given by the 
                sum $m_\text{tot} = m' + m''$, will have to be "redistributed" onto the "distribution" 
                in some sensible manner, necessarily respecting the conservation of mass. How exactly 
                this is done is determined by a function $f(m,m',m'')$, which will look differently 
                depending on whether one wishes to model pure (hit-and-stick) coagulation, or
                whether one is interested in collisions leading to particle fragmentation instead.
        \end{enumerate}

        \todo{Plugging this into [cite Smol. eq.]...}
        \begin{align}
            \deriv{n}{t} =
            &-\int\limits_0^\infty \int\limits_0^\infty R(m',m'') \cdot 
                \frac{1}{2}\bigg[\delta_D(m-m')+\delta_D(m-m'')\bigg]
                \cdot n(m') \cdot n(m'') \ \text dm' \ \text dm'' \\
            &+\int\limits_0^\infty \int\limits_0^\infty R(m',m'') \cdot f(m,m',m'')
                \cdot n(m') \cdot n(m'') \ \text dm' \ \text dm'' 
        \end{align}
        \todo{which, if we pull apart the two terms in the square brackets, can be 
        equivalently written as}
        \begin{align}
            \deriv{n}{t} =
            &-\int\limits_0^\infty \int\limits_0^\infty R(m',m'') \cdot \frac{1}{2}\delta_D(m-m')
                \cdot n(m') \cdot n(m'') \ \text dm' \ \text dm'' \\
            &-\int\limits_0^\infty \int\limits_0^\infty R(m',m'') \cdot \frac{1}{2}\delta_D(m-m'')
                \cdot n(m') \cdot n(m'') \ \text dm' \ \text dm'' \\
            &+\int\limits_0^\infty \int\limits_0^\infty R(m',m'') \cdot f(m,m',m'')
                \cdot n(m') \cdot n(m'') \ \text dm' \ \text dm''
        \end{align}

        \todo{Making use of the relationship}
        \begin{equation}
            f(x_0) = \int\limits_{-\infty}^{\infty} f(x) \cdot \delta_D(x-x_0) \ \dd x 
        \end{equation}
        \todo{for the Dirac delta distribution, ...} \\

        \todo{...we can reduce the 2D integral in each of the first two terms a 1D integral.}
        \begin{align}
            \deriv{n}{t} =
            &-\frac{1}{2} \int\limits_0^\infty R(m,m'') \cdot n(m) \cdot n(m'') \ \text dm'' \\
            &-\frac{1}{2} \int\limits_0^\infty R(m',m) \cdot n(m') \cdot n(m) \ \text dm' \\
            &+\int\limits_0^\infty \int\limits_0^\infty R(m',m'') \cdot f(m,m',m'')
                \cdot n(m') \cdot n(m'') \ \text dm' \ \text dm''
        \end{align}

        \todo{Since we can rename $m'' \to m'$ without changing ...} \\
        \todo{... and $R(m,m') = R(m',m)$ (...)} \\
        \todo{...}
        \begin{align}
            \deriv{n}{t} =
            &-\int\limits_0^\infty R(m,m') \cdot n(m) \cdot n(m') \ \text dm' \\
            &+\int\limits_0^\infty \int\limits_0^\infty R(m',m'') \cdot f(m,m',m'')
                \cdot n(m') \cdot n(m'') \ \text dm' \ \text dm''
        \end{align}

        (\todo{Explain ...})

        \todo{The discretized analogon to [cite last eq.] according to [cite sec. on discr.] reads}
        \begin{equation}
            \deriv{N}{t} =
            -\sum_{i=1}^{\mathcal N_m} 
                R_{ij} \cdot N_i \cdot N_j 
            +\sum_{i=1}^{\mathcal N_m} \sum_{j=1}^{\mathcal N_m} 
                R_{ij} \cdot f_{kij} \cdot N_i \cdot N_j
        \end{equation}
        (\todo{Define $f_{kij}$})

    % }}}
    % Decomposition into Coagulation and Fragmentation Kernel {{{ 
    \clearpage\subsection{Decomposition into Coagulation and Fragmentation Kernel}

        As mentioned above [where?], we will define two separate kernels to describe the influence 
        of both collisions leading to coagulation as well as fragmentation events, respectively:
        \begin{align}
            K^\text{coag}(m,m',m'') &= R^\text{coag}(m',m'') \cdot X^\text{coag}(m,m',m'') \\
            K^\text{frag}(m,m',m'') &= R^\text{frag}(m',m'') \cdot X^\text{frag}(m,m',m'')
        \end{align}

        After having done this, we can add these two sub-kernels to arrive at the total kernel:
        \begin{equation}
            X(m,m',m'') = X^\text{coag}(m,m',m'') + X^\text{frag}(m,m',m'')
        \end{equation}

        For completeness, it should be said that we could add another sub-kernel 
        $K^\text{bounce}$ to the sum, for collisions leading to particle bouncing events.
        We might as well neglect this though, since bouncing does not affect the dust particle mass 
        distribution in any way, and therefore we can set
        \begin{equation}
            K^\text{bounce}(m,m',m'') 
            = R^\text{bounce}(m',m'') \cdot \underbrace{X^\text{bounce}(m,m',m'')}_{=0}
        \end{equation}

    % }}}
    % Definition of the Kernel Mass Error {{{ 
    \subsection{Definition of the Kernel Mass Error}

        Here, it makes sense to distinguish between various definitions:

        \begin{enumerate}

            \item Kernel mass error $\Delta K_{ij}$ per second (and per density!)
                (given in \SI{}{\kilogram\ \meter^3\ \second^{-1}}): 
                \begin{align}
                    \Delta K_{ij} &= \sum_{k=1}^{\mathcal N_m} m_k \cdot K_{kij}
                \end{align}

            \item Kernel mass error $\Delta X_{ij}$ per collision 
                (given in \SI{}{\kilogram}): 
                \begin{align}
                    \Delta X_{ij}
                        = \frac{\Delta K_{ij}}{R^\text{coll}_{ij}}
                        = \sum_{k=1}^{\mathcal N_m} m_k \cdot X_{kij}
                \end{align}

            \item Relative kernel mass error $\Delta X_{ij}^\text{rel}$ per collision 
                (dimensionless):
                \begin{align}
                    \Delta X_{ij}^\text{rel} 
                        &= \sum_{k=1}^{\mathcal N_m} \frac{m_k}{m_i+m_j} \cdot X_{kij}
                \end{align}

            \item Total kernel error $\Delta X^\text{rel}$
                (dimensionless as wel):
                \begin{align}
                    \Delta X^\text{rel} &= \sqrt{
                        \sum_{i=1}^{\mathcal N_m}
                        \sum_{j=1}^{\mathcal N_m}
                        \big(\Delta X_{ij}^\text{rel}\big)^2
                    }
                \end{align}

        \end{enumerate}

        \todo{The last definition is what we will use to classify the "goodness" of our kernel.} 

    % }}}

% }}}
% Definition of the Coagulation Kernel {{{ 
\section{Definition of the Coagulation Kernel}

    \todo{As above, gain and loss:}
    \begin{align}
        L^\text{coag}(m, m', m'') 
            &:= -R(m',m'') \cdot \frac{1}{2} \bigg[\delta_D(m-m')+\delta_D(m-m'')\bigg] \\
        G^\text{coag}(m, m', m'') 
            &:= +R(m',m'') \cdot \frac{1}{2} \cdot \delta_D(m-m'-m'')
    \end{align}

    \todo{where (see above)}
    \begin{equation}
        f^\text{coag}(m,m',m'') = \frac{1}{2} \cdot \delta_D(m-m'-m'')
    \end{equation}

    \todo{The question is: How does the number of particles with mass $m$ change?} \\

    \todo{For the loss:}
    \begin{itemize}
        \item Particle with mass $m$ could collide with any other mass, labeled $m'$.
        \item This would then (here: pure coag.) lead to the merging of the particles.
        \item (A new particle with mass $m+m'$ is created and has to be added to the distribution.)
        \item A particle has to be removed from bin $m$ and $m'$.
        \item We sum (integrate) over all these masses $m'$.
    \end{itemize}

    \todo{For the gain}:
    \begin{itemize}
        \item How can a particle with mass $m$ be created via coagulation?
        \item For this to happen, two particles with masses $m'$ and $m''$ have to collide \& merge.
        \item Condition: Mass conservation $m=m'+m''$ ($\to\ \delta_D(m-m'-m'')$).
    \end{itemize}

    \todo{Total kernel for coagulation:}
    \begin{align}
        K^\text{coag}(m,m',m'') = R(m',m'') \cdot \frac{1}{2}
        \bigg[
            \delta_D(m-m'-m'') - \delta_D(m-m') - \delta_D(m-m'')
        \bigg]
    \end{align}

    \clearpage

    \todo{Total Smoluchowski equation for coagulation:}
    \begin{align}
        \deriv{n}{t} 
        =
        &-\int\limits_0^\infty R(m,m') \cdot n(m) \cdot n(m') \ \text dm' \\
        &+\int\limits_0^\infty \int\limits_0^\infty R(m',m'') \cdot 
            \frac{1}{2} \cdot \delta_D(m-m'-m'')
            \cdot n(m') \cdot n(m'') \ \text dm' \ \text dm'' \\
        =
        &-\int\limits_0^\infty R(m,m') \cdot n(m) \cdot n(m') \ \text dm'
        +\int\limits_0^\infty R(m',m-m') \cdot \frac{1}{2} \cdot n(m') \cdot n(m-m') \ \text dm' \\
        =
        &-\int\limits_0^\infty R(m,m') \cdot n(m) \cdot n(m') \ \text dm'
        +\int\limits_0^{m/2} R(m',m-m') \cdot n(m') \cdot n(m-m') \ \text dm'
    \end{align}
    % TODO Use multi-line equations here, no duplicate numbering! (both here and elsewhere)

    \todo{Explain: Why can I go from upper bound $\infty$ to upper bound $m$, then $m/2$?}

    \todo{Now define on discretized mass axis (needed for numerical integration)} \\

    Even in a highly simplified scenario, where only hit-and-stick coagulation is included, the 
    definition of the kernel $K_{kij}$ is not at all trivial. To assure both the consistency and 
    accuracy of the algorithm, one has to take care of two separate problems, namely:
    \begin{enumerate}
        \item The conservation of mass \textit{has} to be assured, otherwise the numerical solution can 
            not be assumed to remain stable for long. In the case of stick-and-hit coagulation, this 
            means that for every pair of colliding particles, a single new particle has to be created. 
            At the same time, the two initial particles have to be removed from the distribution. 
            During this process, the total mass should remain unaffected down to machine precision.
        \item When using a logarithmically spaced grid for the discretized mass axis, it can not be 
            assumed that after a collision of two dust particles with masses $m_i$ and $m_j$ the 
            resulting particle will carry a mass $m_k=m_i+m_j$ whose value can be mapped trivially onto 
            the grid. In general, the corresponding index will not be an integer, and instead lie 
            between somewhere between the two neighboring grid points with indices $k$ and $k+1$.
            Therefore, the result of the merging of $m_i$ and $m_j$ has to be divided in some sensible 
            way between these two neighboring bins.
    \end{enumerate}

    \todo{See next section [cite] for how to do this.}

    % The Kovetz-Olund Algorithm {{{ 
    \clearpage\subsection{The Kovetz-Olund Algorithm}
        
        An elegant way for solving the two problems listed above is given in the 1969 paper
        by Kovetz \& Olund \cite{kovetz_olund_1969}, where they used the following procedure:
        \begin{enumerate}
            \item The stick-and-hit coagulation kernel is split into two parts. 
                (\todo{as we did above})
                The first is the \textit{gain} of particles in bin $k$ due to the collision of 
                particles from the bins $i$ and $j$. The second is the \textit{loss} of particles 
                from bin $k$ due to collisions of particles in bin $k$ with particles from any 
                other bin $j$.
            
                Using this separation into gain \& loss, the dust particle mass distribution's
                temporal derivative can be expressed in the following form:
                \begin{equation}
                    \pderiv{N_k}{t}
                        = \sum_{i=0}^{\mathcal N_m} \sum_{j=0}^{\mathcal N_m}
                            K_{kij}^\text{gain}\cdot N_i\cdot N_j
                        - \sum_{j=0}^{\mathcal N_m} K_{kj}^\text{loss} \cdot N_k\cdot N_j
                \end{equation}
                In other words, the total kernel [from above, cite eq.] can be written as
                \begin{equation}
                    K_{kij} = K_{kij}^\text{gain} - K_{ij}^\text{loss}\cdot\delta_{ki}
                \end{equation}
                Splitting the kernel like this into a gain \& a loss term is a quite general
                approach, and can be used in more complex scenarios as well (including e.g.
                particle fragmentation processes).
            \item For the scenario of pure hit-and-stick coagulation, a unique discretization
                of the kernel can be defined such that both the number of particles and the
                conservation of total mass are handled correctly. To do this, consider a
                pair of colliding particles with indices $i$ and $j$. Then, let the index
                $\bar k$ be chosen in such a way that the condition
                \begin{equation}
                    m_{\bar k} \leq m_i + m_j < m_{\bar k+1}
                \end{equation}
                is satisfied.
            \item As stated before, in hit-and-stick coagulation, a single new particle emerges
                for each pair of colliding particles. Using the definitions from above, this
                condition can be expressed as follows:
                \begin{equation}
                    K_{\bar k,ij}^\text{gain} + K_{\bar k+1,ij}^\text{gain}
                    \overset{!}{=} K_{ij}^\text{loss}
                \end{equation}
            \item The second condition is that of mass conservation, which can be written as:
                \begin{equation}
                    m_{\bar k} \cdot K_{\bar k,ij}^\text{gain}
                    + m_{\bar k+1} \cdot K_{\bar k+1,ij}^\text{gain}
                    \overset{!}{=} (m_i+m_j) \cdot K_{ij}^\text{loss}
                \end{equation}
            \item Now, to map the resulting particle's mass onto two neighboring bins, let us    
                define a parameter $\varepsilon$ such that
                \begin{align}
                    K_{\bar k,ij}^\text{gain}
                        &=K_{ij}^\text{loss} \cdot (1-\varepsilon),\ \text{and}\\
                    K_{\bar k+1,ij}^\text{gain}
                        &=K_{ij}^\text{loss} \cdot \varepsilon
                \end{align}
                This assures that [equation from pt 3] is satisfied. If we now plug this
                definition into [equation from pt 4] and solve for $\varepsilon$, we
                arrive at
                \begin{equation}
                    \varepsilon
                        :=\frac{m_i+m_j-m_{\bar k}}{m_{\bar k+1}-m_{\bar k}}
                \end{equation}
        \end{enumerate}
        This is the Kovetz-Olund algorithm \cite{kovetz_olund_1969}, and it was also used in subsequent 
        coagulation studies by e.g. \cite{brauer_dullemond_henning_2007} and 
        \cite{birnstiel_dullemond_brauer_2010}.
    
    % }}}
    % Near-Zero Cancellation Handling {{{ 
    \clearpage\subsection{Near-Zero-Cancellation Handling}
    
        When using floating-point numbers following the representation defined
        by the IEEE-754 standard, it can occur that
        \begin{equation}
          a+b=a
          \ \ \ \ \ \text{for} \ \ \ \ \
          b\neq0
        \end{equation}
        Typically, this happens when
        \begin{equation}
            |b|<\varepsilon_m\cdot|a|
        \end{equation}
        Here, $\varepsilon_m$ labels the \textit{machine precision}.
        \todo{Add: How big is $\varepsilon_m$ for an f32, how big for an f64?}\\
        
        Let $i$ and $j$ once again be the indices used to label two colliding particles. Additionally, 
        assume now that particle $i$ is \textit{much smaller} than particle $j$.\\
        
        The detailed balance approach from above requires the removal of both the big and the small 
        particle from the mass distribution, followed by the re-insertion of a new particle carrying the 
        initial pair's combined mass. This new particle would then have a mass which is nearly identical 
        to that of the bigger one of the original two particles, it would be only a tiny bit heavier.\\
        
        In the approach defined above this would mean that $\bar k=j$, i.e. the resulting particle will 
        reside in the same bin as the larger original one. Also, it would follow that 
        $\varepsilon\ll1$.\\
        
        Let us now take a look at the particle mass distribution in the bin $\bar k$ and, more 
        specifically, by how much it changes from one timestep to the next. For this particular pair of 
        $i$ and $\bar k=j$, we can write:
        \begin{equation}
            \pderiv{n_{\bar k}}{t}
                =K_{\bar k,i\bar k}^{\text{gain}}\cdot n_i\cdot n_{\bar k}
                -K_{\bar ki}^{\text{loss}}\cdot n_i\cdot n_{\bar k}
        \end{equation}
        Plugging in [equation from above] leads to
        \begin{equation}
            \pderiv{n_{\bar k}}{t}
                =(1-\varepsilon)K_{\bar ki}^{\text{loss}}\cdot n_i\cdot n_{\bar k}
                -K_{\bar ki}^{\text{loss}}\cdot n_i\cdot n_{\bar k}
        \end{equation}
        Here, the two terms almost cancel each other out. What remains is a contribution which is 
        proportional to $\varepsilon$.\\
        
        If $\varepsilon$ is small enough, the double-precision accuracy of the floating point 
        representation will lead to breakdown of the method.\\
        \todo{rewrite this sentence, copied almost exactly from Kees}\\
        
        It is relatively easy to identify the particle pairs $(i,j)$ for which the scenario detailed 
        above will occur. Let $i$ (without loss of generality) be the index of the larger one of the two 
        colliding masses. Cancellation may then occur when the resulting $k$ is equal to $j$.\\
        
        In that case, we carry out the subtraction in [previous equation] analytically, and write:
        \begin{equation}
            \pderiv{n_{\bar k}}{t}
                =-\varepsilon K_{\bar ki}^{\text{loss}}\cdot n_i\cdot n_{\bar k}
        \end{equation}
        \todo{Elaborate on this, see "Dust Evolution with Binning Methods"}
    
        \newpage

% \begin{figure}[h!]
%     \centering
%     \begin{minipage}{.5\linewidth}
%         \centering
%       	\subfloat[]{
%             \label{:a}
%       	  	\includegraphics[width=\linewidth]{34/canc.pdf}
%       	}
%     \end{minipage}%
%     \begin{minipage}{.5\linewidth}
%         \centering
%       	\subfloat[]{
%             \label{:b}
%       	  	\includegraphics[width=\linewidth]{34/nocanc.pdf}
%       	}
%     \end{minipage}
%     \caption{}
% \end{figure}

% \newpage


\begin{figure}[h!]
    \centering
    \begin{minipage}{.5\linewidth}
        \centering
      	\subfloat[Coagulation Kernel without Cancellation Handling]{
            \label{:a}
      	  	% \includegraphics[width=\linewidth]{34/K_coag_canc.pdf}
      	  	\includegraphics[width=\linewidth]{34/error_K_coag_canc.pdf}
      	}
    \end{minipage}%
    \begin{minipage}{.5\linewidth}
        \centering
      	\subfloat[Coagulation Kernel with Cancellation Handling]{
            \label{:b}
      	  	% \includegraphics[width=\linewidth]{34/K_coag_nocanc.pdf}
      	  	\includegraphics[width=\linewidth]{34/error_K_coag_nocanc.pdf}
      	}
    \end{minipage}
    \caption{
        Influence of Near-Zero Cancellation on the Mass Error of the
        Coagulation Kernel
    }
    \label{}
\end{figure}


% \begin{figure}[h!]
%     \centering
%     \end{minipage}
%     \begin{minipage}{.5\linewidth}
%         \centering
%       	\subfloat[]{
%             \label{:a}
%       	}
%     \end{minipage}%
%     \begin{minipage}{.5\linewidth}
%         \centering
%       	\subfloat[]{
%             \label{:b}
%       	}
%     \end{minipage}
%     \caption{
%         Kernel Mass Error:
%         (a) Mass Error of Coagulation Kernel without Cancellation Handling
%         (b) Mass Error of Coagulation Kernel with Cancellation Handling
%         (c) Mass Error of Fragmentation Kernel
%         (d) Mass Error of Total Kernel
%     }
% \end{figure}

% \newpage


    % }}}

% }}}
% Definition of the Fragmentation Kernel {{{ 
\section{Definition of the Fragmentation Kernel}

    \todo{As above, gain and loss:}
    \begin{align}
        L^\text{coag}(m, m', m'') 
            &:= -R(m',m'') \cdot \frac{1}{2} \bigg[\delta_D(m-m')+\delta_D(m-m'')\bigg] \\
        G^\text{coag}(m, m', m'') 
            &:= +R(m',m'') \cdot \frac{1}{2} \cdot f(m,m',m'')
    \end{align}

    \todo{For the loss:}
    \begin{itemize}
        \item Same as in coagulation.
        \item If two particles with masses $m$ and $m'$ collide, these two particles "disappear"
              from the distribution.
        \item Instead, new particles have to be added.
        \item Once again, there is the criterion of mass conservation.
    \end{itemize}

    \todo{For the gain:}
    \begin{itemize}
        \item Looks very similar to coagulation.
        \item But: Here we have the term $f(m,m',m'')$ in the integral.
        \item In coagulation, this term is equal to $\delta_D(m-m'-m'')$.
        \item If two masses fuse, there is only a single possible mass value that can result from 
              that, without breaking mass conservation.
        \item In fragmentation, this is different.
        \item If two particles collide, they can in principle fragment into a whole range of 
              differently sized particles, as long as mass is conserved.
    \end{itemize}

    \todo{Smol. eq. looks very similar to coag., but $f$ instead of $\delta_D$, can't "reduce".} \\
    \todo{That's where the computational cost comes from!} \\

    \todo{How to model the distribution of particle masses resulting from a fragmentation event?} \\
    \todo{Many small ones, only a few bigger ones.} \\
    \todo{Different approaches [cite]}

    % Modeling the Mass Distribution Resulting from a Fragmentation Event {{{ 
    \subsection{Modeling the Mass Distribution Resulting from a Fragmentation Event
    (\todo{rename?})}

        \todo{Naive: Move all masses to the bin corresponding to lowest mass} \\
        \todo{i.e.: total "pulverization"} \\
        \todo{The entire mass is converted into tiny particles.} \\

        \todo{Set $k=0$.} \\
        Then
        \begin{equation}
            f^\text{frag}_{kij} = \frac{m_i + m_j}{m_k}
        \end{equation}
        \todo{Include $\theta_H$ ?}

    % }}}
    % The MRN distribution {{{ 
    \clearpage\subsection{The MRN Distribution}
        
        \todo{More sophisticated approach} \\
        \todo{given by [cite MRN]} \\
        \todo{inverse power law (?)}

        \begin{equation}
            f^\text{frag}_{kij} = (m_i + m_j) \cdot \xi_{kij}
        \end{equation}
        where $\xi_{kij}$ labels the fraction of $m_\text{tot} = m_i + m_j$ that "lands" in 
        a given bin $k$. It is defined as 
        \begin{equation}
            \xi_{kij} := \frac{m_k^q}{S}
        \end{equation}
        with $q=-\frac{11}{6}$ and
        \begin{equation}
            S := \sum_{k_\text{min}}^{k_\text{max}} m_k^{q+1} \cdot \Delta m_k
        \end{equation}

        \todo{Elaborate, explain, cite!}

    % }}}

% }}}
% Analytical Solutions {{{ 
\newpage\section{Analytical Solutions}

    Analytical solutions to the Smoluchowski coagulation equation exist if, and only if the kernel
    takes on one of the following three forms:
    \begin{enumerate}
        \item constant kernel
            $$K=1$$
        \item linear kernel
            $$K=m_1+m_2$$
        \item quadratic kernel
            $$K=m_1\cdot m_2$$
    \end{enumerate}
    \todo{Rewrite with $m$, $m'$, and $m''$.} \\
    \todo{Rewrite with $K$ and $X$. Take which?}

    % Constant Kernel {{{ 
    \subsection{Constant Kernel}

    \todo{Constant Kernel:} (general enough?)
    \begin{equation}
        K(m, m', m'')
            = 1
    \end{equation}

    \todo{Smoluchowski equation with constant kernel:}
    \begin{equation}
        \pderiv{n}{t}(m)
            = 
                \int\limits_0^\infty \int\limits_0^\infty
                n(m') \cdot n(m'')
                \ \dd m' \ \dd m''
    \end{equation}

    % }}}
    % Linear Kernel {{{ 
    \subsection{Linear Kernel}

    \todo{Linear Kernel:} (general enough?)
    \begin{equation}
        K(m, m', m'')
            = m' + m''
    \end{equation}

    \todo{Smoluchowski equation with linear kernel:}
    \begin{equation}
        \pderiv{n}{t}(m)
            = 
                \int\limits_0^\infty \int\limits_0^\infty
                (m' + m'') \cdot
                n(m') \cdot n(m'')
                \ \dd m' \ \dd m''
    \end{equation}

    % }}}
    % Quadratic Kernel {{{ 
    \subsection{Quadratic Kernel}

    \todo{Quadratic Kernel:} (general enough?)
    \begin{equation}
        K(m, m', m'')
            = m' \cdot m''
    \end{equation}

    \todo{Smoluchowski equation with quadratic kernel:}
    \begin{equation}
        \pderiv{n}{t}(m)
            = 
                \int\limits_0^\infty \int\limits_0^\infty
                (m' \cdot m'') \cdot
                n(m') \cdot n(m'')
                \ \dd m' \ \dd m''
    \end{equation}

    % }}}

% }}}
% Numerical Integration {{{
\newpage\section{Numerical Integration}

    % Definition of the Integration Mass Error {{{ 
    \subsection{Definition of the Integration Mass Error}

        \todo{Let us now take a look at how to use numerical integration of the discretized 
        Smoluchowski coagulation equation to arrive at an approximate solution for the temporal 
        evolution of the particle mass distribution function under the influence of dust particle 
        collisions.} \\

        \todo{To be able to compare different numerical schemes quantitatively, both with respect 
        to their accuracy as well as stability properties, it makes sense to first define a measure 
        of the total error of the integration.} \\

        \todo{As we already know from eq [...], the mass volume density can be calculcated from the 
        particle mass distribution.} \\ % TODO Rewrite this.

        \todo{Let $t_i$ and $t_f$ label the time at the start and end of the integration, respectively.
        The mass volume density's value at these two points in time can be written as:}

        \begin{align}
            % TODO Define `t_i` and `t_f`. Note that `t_i = t_0`
            % TODO Use subscript or superscript for `i` and `f` ?
            \rho_\text{i}
                &:=\sum_{k=0}^{\mathcal N_m} m_k\cdot n_k(t=t_\text{i})\cdot\Delta m_k
            \\
            \rho_\text{f}
                &:=\sum_{k=0}^{\mathcal N_m} m_k\cdot n_k(t=t_\text{f})\cdot\Delta m_k
        \end{align}

        \todo{After the integration has finished, the relative mass error (more specifically: density 
        error) then reads:}
        
        \begin{equation}
            \Delta\rho
                :=\frac{\rho_\text{f}-\rho_\text{i}}{\rho_\text{i}}
        \end{equation}

        In reality, this mass error should be exactly equal to zero. Otherwise, mass conservation would not be given, since the movement of mass from one point in space to another is not included in our model. \\
        
        The numerical integration introduces an error though, which ideally should be kept at machine precision. % TODO Rewrite this.

    % }}}
    % Explicit Euler Scheme {{{ 
    \subsection{Explicit Euler Scheme}

        A naive approach for solving the Smoluchowski equation numerically is given by the simple 
        explicit Euler integration scheme, which we will briefly address here. Assuming a constant 
        time-step size $\Delta t\in\mathbb R_+$, the elapsed time after $m\in\mathbb N_0$ integration 
        steps can be expressed as 
        \begin{equation}
            t_m
                =t_0+m\cdot\Delta t,
        \end{equation}
        where $t_0$ is the time value at the start of the integration.\\
        
        Now, the goal is to find the corresponding values $n_k^m$ for the particle mass 
        distribution:
        \begin{equation}
            n_k^m
                :=n(m=m_k,\ t=t_m)
        \end{equation}
        
        We define the differences
        \begin{align}
            \Delta n_k
                &=n_k^m-n_k^{m-1}
            \ \ \ \text{and}\ \ \
            \\
            \Delta t
                &=t_m-t_{m-1}
        \end{align}
        which, in turn, can be used to define the 1st order foward finite difference
        \begin{equation}
            \text{FFD}
                =\frac{\Delta n_k}{\Delta t}
        \end{equation}
        
        The finite difference can now be used as an approximation for the temporal derivative 
        of $n$:
        \begin{equation}
            \pderiv{n_k}{t}
                \approx\text{FFD}
        \end{equation}
        
        Plugging in \cref{eq:discrete_smoluchowski_equation} and rearranging for $\Delta n_k$ 
        leads to
        \begin{equation}
            \Delta n_k
            =\Delta t\cdot
                \sum_{i=1}^{\mathcal N_m}\sum_{j=1}^{\mathcal N_m}
                K_{kij}\cdot n_i\cdot n_j
        \end{equation}
        
        % TODO Be consistent: "time step" or "time-step" ?
        This can now be used an approximate value for the change in $n_k$ from one time-step to 
        the next, i.e.
        \begin{equation}
            \begin{split}
                n_k^m
                    &=n_k^{m-1}+\Delta n_k
                \\
                    &=n_k^{m-1}+\Delta t\cdot
                        \sum_{i=1}^{\mathcal N_m}\sum_{j=1}^{\mathcal N_m}
                        K_{kij}\cdot n_i\cdot n_j
            \end{split}
        \end{equation}
        
        % TODO Cite CFL ?
        % TODO Add CFL to abbreviations
        In order to ensure stability of the numerical integration, the step size needs to respect the 
        Courant-Friedrichs-Lewy (CFL) criterion, which provides an upper limit for the values that 
        $\Delta t$ can take on \cite{courant_friedrichs_lewy_1928}.\\
        
        \todo{Formulate CFL criterion for Smoluchowski equation.}

        \clearpage
        \todo{Plot $n(m,t)$ on a linear mass axis, with only coagulation.}
        \todo{Plot $n(m,t)$ on a linear mass axis, with only fragmentation. (?)}
        \todo{Plot $n(m,t)$ on a linear mass axis, with constant/linear kernel. (?)}

    % }}}
    % Implicit Radau Scheme {{{ 
    \subsection{Implicit Radau Scheme}

        \todo{Runge-Kutta scheme} \\
        \todo{Here, 4th order (correct?)} \\

        \todo{Built upon `scipy` Python library 
        (more specifically, `scipy.integrate.solve\_ivp` function)} \\
        \todo{Implemented by my supervisor Prof. Dr. Cornelis P. Dullemond [cite?]} \\

        \todo{Accuracy} \\
        \todo{Stability, Mass Conservation} \\

        \clearpage
        \todo{Plot $n(m,t)$ on log. mass/time axis, with only coagulation.}
        \todo{Plot $n(m,t)$ on log. mass/time axis, with only fragmentation.}
        \todo{Plot $n(m,t)$ on log. mass/time axis, with both coagulation \& "naive" fragmentation.}
        \todo{Plot $n(m,t)$ on log. mass/time axis, with both coagulation \& MRN fragmentation.}

    % }}}
% }}}
