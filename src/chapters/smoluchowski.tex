% Intro {{{ 

In the following, we will focus on the definition of a model for describing the 
processes involved in dust particle coagulation. \\

Adopting the formalism that was outlined in \cref{sec:dust_particle_mass_distribution},
we would like to study the behavior of the particle mass distribution function $n(m)$
under the influence of particle collisions. Both associative and dissociative process 
(coagulation and fragmentation) are to be taken into account.

% }}}
% The Smoluchowski Equation {{{ 
\section{The Smoluchowski Coagulation Equation}

    The centerpiece of the dust coagulation model that this thesis is built upon is provided 
    by the following integro-differential population balance equation:
    \begin{equation}
        \label{eq:continuous_smoluchowski_equation}
        \boxed{
        \pderiv{n}{t}(m)
            =
                \int\limits_0^\infty
                \int\limits_0^\infty
                K(m,m',m'')
                \cdot n(m')
                \cdot n(m'')
                \ \dd m'
                \ \dd m''
        }
    \end{equation}

    This it the so-called \textit{Smoluchowski coagulation equation}, which provides a 
    mathematical framework that can be used to model the process of particle aggregation, 
    or \textit{coagulation}, into larger and larger structures. \\

    The equation is named after the Polish physicist Marian Smoluchowski, due to his treatment 
    of the equation in his 1916 paper \cite{smoluchowski_1916}. 
    It finds applications in a wide range of physical and biological contexts. \\ 
    % todo (examples?) \\

    In the Smoluchowski coagulation equation, the term $K(m, m', m'')$ labels the so-called
    \textit{coagulation kernel}. 
    This is a function that we will use to encode the information about how the number of 
    particles with mass $m$ changes with time under the influence of collisions between 
    particles with masses $m'$ and $m''$. \\

    The effects of both coagulation and fragmentation processes can be defined in separate 
    kernel functions, that are subsequently added to acquire the total kernel. We will define 
    these functions separately in \cref{sec:coagulation_kernel} and \cref{sec:fragmentation_kernel},
    respectively.

    % \todo{ODE or PDE?} \\
    % \todo{For each choice of \textit{kernel function}, there exists a unique solution (cite).}
    % (and initial conditions)

    % Continuous Formulation {{{
    % \subsection{Continuous Formulation}

    %     \todo{For the evolution of the particle mass distribution function, (depending 
    %     \textit{only} on the particle mass), the Smoluchowski equation in its most general form 
    %     reads:}

    %     \todo{We will discuss it in [cite section], and [coag-section] and [frag-section].}
 
    % }}}

% }}}
    % Discretized Formulation of the Smoluchowski Coagulation Equation {{{ 
    \newpage\subsection{Discretized Formulation of the Smoluchowski Equation}

        To discretize the Smoluchowski coagulation equation, we need to define discretized 
        analogons for both the particle mass distribution function and its temporal derivative, 
        as well as write the integral on the right-hand side of the equation as a sum. \\

        Here we will assume that the particle mass axis from $m_\text{min}$ to $m_\text{max}$ 
        has been discretized into a grid $m_k$ consisting of $\mathcal N_m$ bins, and defined 
        according to what was outlined in \cref{sec:preliminary_notes_on_axis_discretization}, 
        where we layed out a method for axis discretization. \\

        For the mass distribution $n(m)$, we use the notation $n_k := n(m_k)$ as defined in
        \cref{eq:discrete_formulation_of_particle_mass_distribution}.
        To discretize its derivative, we can switch out the temporal derivative 
        of the particle mass distribution function with its finite difference analogon:
        \begin{equation}
            \pderiv{n}{t} \to \frac{\Delta n}{\Delta t}
        \end{equation}
        Note: Throughout this thesis, we will use the notation $\partial n/\partial t$ 
        interchangeably with the finite difference notation from above, since we will basically 
        always be talking about the numerical solution anyways. \\

        Additionally, we have to replace the integral over the particle mass axis 
        with a discrete sum:
        \begin{equation}
            \int_0^\infty \dd m' \to \sum_{i=1}^{\mathcal N_m} \Delta m_i
            \ \ \ \ \text{and}\ \ \ \
            \int_0^\infty \dd m'' \to \sum_{j=1}^{\mathcal N_m} \Delta m_j
        \end{equation} 

        Putting all this together, we arrive at the following expression:
        \begin{equation}
            \label{eq:discrete_smoluchowski_equation}
            \pderiv{n_k}{t}
                \approx \sum_{i=1}^{\mathcal N_m}\sum_{j=1}^{\mathcal N_m}
                    K(m_k,m_i,m_j)\cdot n_i\cdot n_j\cdot\Delta m_i\cdot\Delta m_j
        \end{equation}
        % TODO Later on I talk about `del n / del t` instead of `Delta n / Delta t`.
        %    -> Decide on which one of these to use.
    
        Let us now make use of the relationship $N_i=n_i \cdot \Delta m_i$ declared in 
        \cref{eq:relationship_between_dust_particle_mass_distribution_and_particle_number_density}
        and define the discrete \textit{kernel matrix}
        \begin{equation}
            \boxed{
                K_{kij} := K(m_k,m_i,m_j) \cdot \Delta m_k
            }
        \end{equation}

        With these two relationships, we are able to rewrite the equation and
        arrive at the \textit{discretized Smoluchowski coagulation equation}:
        % If, in addition, we make use of the relationship  that was defined 
        % in (\todo{cite}), then we can rewrite the Smoluchowski equation as
        \begin{equation}
            \boxed{
            \pderiv{N_k}{t}
                \approx \sum_{i=1}^{\mathcal N_m}\sum_{j=1}^{\mathcal N_m} K_{kij}\cdot N_i\cdot N_j
            }
        \end{equation}

        Here, it makes sense to bring attention to the fact that in the discretized Smoluchowski
        coagulation equation, we do not make use of the notation $n_k = n(m_k)$ for the dust
        particle mass distribution function. \\

        Instead, the equation is expressed in terms of the 
        number of particles $N_k = N(m_k)$, which labels the number of particles (per unit volume) 
        in a given bin characterized by an index $k$.

    % }}}
% The Kernel {{{ 
\clearpage\section{The Kernel}
    % Intro {{{ 
    
        In the following sections, we will focus on the definition of the kernel functions 
        for both pure coagulation, as well as pure fragmentation. These two functions can 
        then be added later to arrive at the total kernel function. \\

        For both of these cases, we will first consider the definition on the continuous 
        mass axis, and then implement the analogous definition using a mass axis 
        that was discretized according to \cref{sec:preliminary_notes_on_axis_discretization}. 
        Before we do that though, it will make sense to think about how the kernel function can be 
        decomposed and factorized in several different ways, the first of which 
        is detailed in the following section. 

        % \todo{(only use "upper left" half of the matrix)}

    % }}}
    % Decomposition into Collision Rate and Collision Details {{{ 
    \subsection{Decomposition into Collision Rate and Collision Details}
    
        We can factorize the kernel into two separate terms, given by
        \begin{equation}
            K(m, m', m'') = R(m', m'') \cdot X(m, m', m'') 
        \end{equation}
    
        The idea behind this distinction is the following: 
        \begin{enumerate}
    
            \item The first term $R(m',m'')$ is the reaction rate coefficient. It is used to 
                encode the information about \textit{how often collision events occur}
                per unit time, involving a given pair of particles carrying the masses $m'$ 
                and $m''$.
    
            \item In contrast to that, the second term $X(m,m',m'')$ helps us understand 
                \textit{how this affects the number of particles} carrying a mass value $m$ 
                under any such collisions, i.e. it gives information about how matter is 
                redistributed onto the range of possible masses as the result of a given 
                collision.
    
        \end{enumerate}
    
        % \todo{The second term, on the other hand, will be used to encode how the initial masses 
        % $m'$ and $m''$ are then distributed onto the range of particles resulting from such a 
        % collision, each carrying a mass of value $m$.} \\
    
        For the detailed definition of the first term $R(m', m'')$, 
        see \cref{sec:definition_of_reaction_rate_coefficient}. The second term $X(m, m', m'')$
        will be defined in the process of kernel construction in 
        \cref{sec:coagulation_kernel} and \cref{sec:fragmentation_kernel}.
        % For the detailed definition of the first term $R(m', m'')$, 
        % see \cref{sec:definition_of_reaction_rate_coefficient}. The second term $X(m, m', m'')$
        % will be defined in the process of the construction of the coagulation kernel in 
        % \cref{sec:coagulation_kernel} and the fragmentation kernel in 
        % \cref{sec:fragmentation_kernel}.
        All of these definitions will be made separately for the two cases of coagulation and
        fragmentation. 
    
    % }}}
    % Physical Units {{{ 
    % \subsection{Physical Units (\todo{Is this section needed?})}
    % \todo{Talk about units! (not 1/second)} \\
    % }}}
    % Discretization {{{ 
    % \subsection{Discretization}

        % \begin{equation}
        %     K(m, m', m'')
        %         \to
        %             K_{kij}
        %         =
        %             K(m_k, m_i, m_j) \cdot\Delta m_k
        % \end{equation}

    % }}}
    % Decomposition into Coagulation and Fragmentation Kernel {{{ 
    \subsection{Decomposition into Coagulation and Fragmentation Kernel}

        As was just mentioned, we will define two separate kernels to describe the influence 
        of both collisions leading to coagulation as well as fragmentation events, respectively:
        \begin{align}
            K^\text{coag}(m,m',m'') &= R^\text{coag}(m',m'') \cdot X^\text{coag}(m,m',m'') \\
            K^\text{frag}(m,m',m'') &= R^\text{frag}(m',m'') \cdot X^\text{frag}(m,m',m'')
        \end{align}

        After having done this, we can add these two sub-kernels to arrive at the total kernel:
        \begin{equation}
            X(m,m',m'') = X^\text{coag}(m,m',m'') + X^\text{frag}(m,m',m'')
        \end{equation}

        For completeness, it should be said that we could add another sub-kernel 
        $K^\text{bounce}$ to the sum, for collisions leading to particle bouncing events.
        We might as well neglect this though, since bouncing does not affect the dust particle mass 
        distribution in any way, and therefore we can set
        \begin{equation}
            K^\text{bounce}(m,m',m'') 
            = R^\text{bounce}(m',m'') \cdot \underbrace{X^\text{bounce}(m,m',m'')}_{=0}
        \end{equation}

    % }}}
    % Decomposition into Gain and Loss Term {{{ 
    \clearpage\subsection{Decomposition into Gain and Loss Term}

        We can also decompose the kernel $K(m,m',m'')$ into a negative and a positive 
        contribution, which in the following we will address as the \textit{loss} and 
        \textit{gain} components $L$ and $G$ of the kernel, respectively:
        \begin{equation}
            K(m,m',m'') = L(m, m', m'') + G(m, m', m'') 
        \end{equation}
        % \begin{equation}
        %     K(m,m',m'') = R(m',m'') \cdot \bigg[G(m, m', m'') + L(m, m', m'')\bigg]
        % \end{equation}

        For both sticking and fragmentation, it is possible to establish the following 
        general definition for these two components:
        \begin{align}
            \label{eq:definition_of_kernel_gain_and_loss}
            L(m, m', m'') 
                &:= -R(m',m'') \cdot \frac{1}{2} \bigg[\delta_D(m-m')+\delta_D(m-m'')\bigg] \\
            G(m, m', m'') 
                &:= +R(m',m'') \cdot f(m, m', m'') 
        \end{align}
        This definition (or one that is analogous to it) was also used in previous studies 
        of the dust coagulation process, see e.g. \cite{dullemond_dominik_2004},
        \cite{birnstiel_dullemond_brauer_2010}, \cite{okuzumi_2009}. 
        It will hopefully become clear in the following sections why it makes sense to adopt this 
        definition, and why it is presented in the form that is shown here.

        % \todo{The two separate Dirac distributions in the square brackets relate to the symmetry 
        % of the kernel, which satisfies the following relationship.} \\
        % \todo{Continuous case:}
        % \begin{equation}
        %     K(m,m',m'') = K(m,m'',m')
        % \end{equation}
        % \todo{Discrete case:}
        % \begin{equation}
        %     K_{kij} = K_{kji}
        % \end{equation}
        % \todo{I.e.: It is irrelevant whether a particle A collides with a particle B, or whether 
        % a particle B collides with particle A. These are the same collision events}. \\
        % \todo{They are both encoded into the kernel though, which is why we need the factor $1/2$ 
        % in order to not count them twice.} \\
        % \todo{For the numerical integration that we will do later, we will of course not sum over 
        % the entire kernel and divide by 2 afterwards. That would lead to an unnecessary factor 2 
        % in the necessary computations. Instead, we will map the lower right half of the kernel 
        % matrix onto the top left (as we will see later).} \\

        % When evaluating the sum in [cite Smol. eq.] during integration, this will have the following 
        % effect:
        % \begin{enumerate}
        %     \item For each collision between two particles $m'$ and $m''$, a single particle will 
        %         be removed from the distribution for each of these two masses, hence the term $-1$ 
        %         in the equation above. (\todo{fix this})
        %     \item At the same time, the total mass involved in the collision, which is given by the 
        %         sum $m_\text{tot} = m' + m''$, will have to be "redistributed" onto the "distribution" 
        %         in some sensible manner, necessarily respecting the conservation of mass. How exactly 
        %         this is done is determined by a function $f(m,m',m'')$, which will look differently 
        %         depending on whether one wishes to model pure (hit-and-stick) coagulation, or
        %         whether one is interested in collisions leading to particle fragmentation instead.
        % \end{enumerate}

        % \todo{Plugging this into [cite Smol. eq.]...}
        % \begin{align}
        %     \deriv{n}{t} =
        %     &-\int\limits_0^\infty \int\limits_0^\infty R(m',m'') \cdot 
        %         \frac{1}{2}\bigg[\delta_D(m-m')+\delta_D(m-m'')\bigg]
        %         \cdot n(m') \cdot n(m'') \ \text dm' \ \text dm'' \\
        %     &+\int\limits_0^\infty \int\limits_0^\infty R(m',m'') \cdot f(m,m',m'')
        %         \cdot n(m') \cdot n(m'') \ \text dm' \ \text dm'' 
        % \end{align}
        % \todo{which, if we pull apart the two terms in the square brackets, can be 
        % equivalently written as}
        % \begin{align}
        %     \deriv{n}{t} =
        %     &-\int\limits_0^\infty \int\limits_0^\infty R(m',m'') \cdot \frac{1}{2}\delta_D(m-m')
        %         \cdot n(m') \cdot n(m'') \ \text dm' \ \text dm'' \\
        %     &-\int\limits_0^\infty \int\limits_0^\infty R(m',m'') \cdot \frac{1}{2}\delta_D(m-m'')
        %         \cdot n(m') \cdot n(m'') \ \text dm' \ \text dm'' \\
        %     &+\int\limits_0^\infty \int\limits_0^\infty R(m',m'') \cdot f(m,m',m'')
        %         \cdot n(m') \cdot n(m'') \ \text dm' \ \text dm''
        % \end{align}

        % \todo{Making use of the relationship}
        % \begin{equation}
        %     f(x_0) = \int\limits_{-\infty}^{\infty} f(x) \cdot \delta_D(x-x_0) \ \dd x 
        % \end{equation}
        % \todo{for the Dirac delta distribution, ...} \\

        % \todo{...we can reduce the 2D integral in each of the first two terms a 1D integral.}
        % \begin{align}
        %     \deriv{n}{t} =
        %     &-\frac{1}{2} \int\limits_0^\infty R(m,m'') \cdot n(m) \cdot n(m'') \ \text dm'' \\
        %     &-\frac{1}{2} \int\limits_0^\infty R(m',m) \cdot n(m') \cdot n(m) \ \text dm' \\
        %     &+\int\limits_0^\infty \int\limits_0^\infty R(m',m'') \cdot f(m,m',m'')
        %         \cdot n(m') \cdot n(m'') \ \text dm' \ \text dm''
        % \end{align}

        % \todo{Since we can rename $m'' \to m'$ without changing ...} \\
        % \todo{... and $R(m,m') = R(m',m)$ (...)} \\
        % \todo{...}
        % \begin{align}
        %     \deriv{n}{t} =
        %     &-\int\limits_0^\infty R(m,m') \cdot n(m) \cdot n(m') \ \text dm' \\
        %     &+\int\limits_0^\infty \int\limits_0^\infty R(m',m'') \cdot f(m,m',m'')
        %         \cdot n(m') \cdot n(m'') \ \text dm' \ \text dm''
        % \end{align}

        % (\todo{Explain ...})

        % \todo{The discretized analogon to [cite last eq.] according to [cite sec. on discr.] reads}
        % \begin{equation}
        %     \deriv{N}{t} =
        %     -\sum_{i=1}^{\mathcal N_m} 
        %         R_{ij} \cdot N_i \cdot N_j 
        %     +\sum_{i=1}^{\mathcal N_m} \sum_{j=1}^{\mathcal N_m} 
        %         R_{ij} \cdot f_{kij} \cdot N_i \cdot N_j
        % \end{equation}
        % (\todo{Define $f_{kij}$})

    % }}}
    % Definition of the Kernel Mass Error {{{ 
    \subsection{Definition of the Kernel Mass Error}

        The coagulation framework that is given by the Smoluchowski equation and is 
        used throughout this work provides an easy method for expressing the mass 
        conservation error of the discretized kernel function. 
        It can be used to get a first look at the stability properties of the algorithm, 
        without actually running the integration scheme. It makes sense to distinguish 
        between a few different error definitions here:

        \begin{enumerate}

            \item Kernel mass error $\Delta K_{ij}$ per second (and per density!),
                given in \SI{}{\kilogram\ \meter^3\ \second^{-1}}: 
                \begin{align}
                    \Delta K_{ij} &= \sum_{k=1}^{\mathcal N_m} m_k \cdot K_{kij}
                \end{align}

            \item Kernel mass error $\Delta X_{ij}$ per collision, 
                given in \SI{}{\kilogram}: 
                \begin{align}
                    \Delta X_{ij}
                        = \frac{\Delta K_{ij}}{R^\text{coll}_{ij}}
                        = \sum_{k=1}^{\mathcal N_m} m_k \cdot X_{kij}
                \end{align}

            \item Relative kernel mass error $\Delta X_{ij}^\text{rel}$ per collision 
                (dimensionless):
                \begin{align}
                    \Delta X_{ij}^\text{rel} 
                        &= \sum_{k=1}^{\mathcal N_m} \frac{m_k}{m_i+m_j} \cdot X_{kij}
                \end{align}

            \item Total relative kernel mass error $\Delta X^\text{rel}$
                (also dimensionless):
                \begin{equation}
                    \boxed{
                        \Delta X^\text{rel} = \sqrt{
                            \sum_{i=1}^{\mathcal N_m}
                            \sum_{j=1}^{\mathcal N_m}
                            \big(\Delta X_{ij}^\text{rel}\big)^2
                        }
                    }
                \end{equation}

        \end{enumerate}

        The last definition is what we will use to classify the properties of our 
        kernel implementations regarding the conservation of mass during collisions.

    % }}}

% }}}
% Definition of the Coagulation Kernel {{{ 
\clearpage\section{Definition of the Coagulation Kernel}
\label{sec:coagulation_kernel}

    % Definition {{{ 

    Consider a collision between two particles with $m'$ and $m''$. 
    In the case of pure ``hit-and-stick'' coagulation, 
    % To understand the definition of the two terms $G$ and $L$, 
    we assume that these particles will merge as a result of the collison. \\

    If the coagulation kernel is defined in the right manner, then the integration of 
    the Smoluchowski equation should, for each collision, lead to the removal of one 
    particle from the mass distribution for each of the masses $m'$ and $m''$. At the same 
    time, a \textit{new} particle has to be inserted into the distribution at 
    $m=m_\text{tot}=m'+m''$. \\

    To implement this, we can define the positive and negative 
    components of the kernel in terms of the gain $G$ and loss $L$ as follows:
    \begin{align}
        \label{eq:definition_of_coagulation_kernel_gain_and_loss}
        L^\text{coag}(m, m', m'') 
            &:= -R(m',m'') \cdot \frac{1}{2} \bigg[\delta_D(m-m')+\delta_D(m-m'')\bigg] \\
        G^\text{coag}(m, m', m'') 
            &:= +R(m',m'') \cdot \frac{1}{2} \cdot \delta_D(m-m'-m'')
    \end{align}

    This formulation is equivalent to \cref{eq:definition_of_kernel_gain_and_loss},
    only that now we defined 
    \begin{equation}
        \label{eq:definition_of_f_coag}
        f^\text{coag}(m,m',m'') = \frac{1}{2} \cdot \delta_D(m-m'-m'')
    \end{equation}

    The reason for the presence of a factor $1/2$ will become clear at the end of this 
    section. \\

    The loss term $L$ assures that, for each collision, exactly one particle is removed 
    from the mass distribution for each of the collision partners' mass value.
    The gain term on the other hand makes sure that a single particle is inserted into the 
    mass distribution at $m = m' + m''$. \\

%     \todo{The question is: How does the number of particles with mass $m$ change?} \\

%     \todo{For the loss:}
%     \begin{itemize}
%         \item Particle with mass $m$ could collide with any other mass, labeled $m'$.
%         \item This would then (here: pure coag.) lead to the merging of the particles.
%         \item (A new particle with mass $m+m'$ is created and has to be added to the distribution.)
%         \item A particle has to be removed from bin $m$ and $m'$.
%         \item We sum (integrate) over all these masses $m'$.
%     \end{itemize}

%     \todo{For the gain}:
%     \begin{itemize}
%         \item How can a particle with mass $m$ be created via coagulation?
%         \item For this to happen, two particles with masses $m'$ and $m''$ have to collide \& merge.
%         \item Condition: Mass conservation $m=m'+m''$ ($\to\ \delta_D(m-m'-m'')$).
%     \end{itemize}

    The total coagulation kernel is given by the sum of the two components $G$ and $L$:
    \begin{align}
        K^\text{coag}(m,m',m'') 
        &= G(m,m',m'') + L(m,m',m'')
        \\
        &= R(m',m'') \cdot \frac{1}{2}
        \bigg[
            \delta_D(m-m'-m'') - \delta_D(m-m') - \delta_D(m-m'')
        \bigg]
    \end{align}

    If we plug this into \cref{eq:continuous_smoluchowski_equation}, we get the 
    continuous Smoluchowski equation for pure ``hit-and-stick'' coagulation:
    \begin{align}
        \pderiv{n}{t}(m)
            &=
                \int\limits_0^\infty
                \int\limits_0^\infty
                K_\text{coag}(m,m',m'')
                \cdot n(m')
                \cdot n(m'')
                \ \dd m'
                \ \dd m''
        \\
            &=
                \int\limits_0^\infty
                \int\limits_0^\infty
                R(m',m'') \cdot \frac{1}{2}
                \bigg[
                    \delta_D(m-m'-m'') - \delta_D(m-m') - \delta_D(m-m'')
                \bigg]
                \cdot n(m')
                \cdot n(m'')
                \ \dd m'
                \ \dd m''
    \end{align}

    To make this rather space-consuming equation a bit more manageable, we can pull apart the
    expression into a separate integral for each of the three sum terms in the kernel. \\

    Then we can make use of the identity for the Dirac $\delta$-distribution we already discussed
    in \cref{eq:dirac_delta_identity} and collapse one of the two integrals over the mass axis. \\

    \clearpage
    This leads to
    % \begin{align}
    %     \pderiv{n}{t}(m)
    %         =
    %             &\int\limits_0^\infty
    %             \int\limits_0^\infty
    %             R(m',m'') \cdot \frac{1}{2}
    %             \bigg[
    %                 \delta_D(m-m'-m'')
    %             \bigg]
    %             \cdot n(m')
    %             \cdot n(m'')
    %             \ \dd m'
    %             \ \dd m''
    %         \\ &
    %             -
    %             \int\limits_0^\infty
    %             \int\limits_0^\infty
    %             R(m',m'') \cdot \frac{1}{2}
    %             \bigg[
    %                 \delta_D(m-m')
    %             \bigg]
    %             \cdot n(m')
    %             \cdot n(m'')
    %             \ \dd m'
    %             \ \dd m''
    %         \\ &
    %             -
    %             \int\limits_0^\infty
    %             \int\limits_0^\infty
    %             R(m',m'') \cdot \frac{1}{2}
    %             \bigg[
    %                 \delta_D(m-m'')
    %             \bigg]
    %             \cdot n(m')
    %             \cdot n(m'')
    %             \ \dd m'
    %             \ \dd m''
    % \end{align}
    \begin{align}
        \pderiv{n}{t}(m)
            = &
                +\frac{1}{2} \int\limits_0^\infty
                R(m', m-m') 
                \cdot n(m')
                \cdot n(m-m')
                \ \dd m'
            \\ &
                -\frac{1}{2} \int\limits_0^\infty
                R(m, m') 
                \cdot n(m)
                \cdot n(m')
                \ \dd m'
            \\ &
                -\frac{1}{2} \int\limits_0^\infty
                R(m, m') 
                \cdot n(m)
                \cdot n(m')
                \ \dd m'
    \end{align}

    Note that here we performed a renaming of $m'' \to m'$ and $\text dm'' \to \text dm'$, 
    and made use of the symmetry $R(m, m') = R(m', m)$: It is irrelevant whether particle $a$ 
    hits particle $b$ or particle $b$ hits particle $a$. We regard these two scenarios as the 
    same event. \\
 
    We can now further simplify the expression combining the last two terms into a single 
    integral, which allows us to get rid of one of the two factors $1/2$:
    \begin{align}
        \boxed{\pderiv{n}{t}(m)
            =   
                \frac{1}{2} \cdot
                \int\limits_0^\infty
                R(m',m-m')
                \cdot n(m')
                \cdot n(m-m')
                \ \dd m'
                -
                \int\limits_0^\infty
                R(m,m')
                \cdot n(m)
                \cdot n(m')
                \ \dd m'
            }
    \end{align}

    Due to the symmetry $i \leftrightarrow j$, we can get rid of the other factor $1/2$ as well,
    if we want to. To achieve this, we will rewrite the integral as
    \begin{equation}
        \frac{1}{2} \int\limits_0^\infty \text dm'
        \to 
        \int\limits_0^{m/2} \text dm'
    \end{equation}
    We will make use of this in the numerical implementation later on, in order to avoid
    performing any unneeded computations. 

    % \todo{Total Smoluchowski equation for coagulation:}
    % \begin{align}
    %     \deriv{n}{t} 
    %     =
    %     &-\int\limits_0^\infty R(m,m') \cdot n(m) \cdot n(m') \ \text dm' \\
    %     &+\int\limits_0^\infty \int\limits_0^\infty R(m',m'') \cdot 
    %         \frac{1}{2} \cdot \delta_D(m-m'-m'')
    %         \cdot n(m') \cdot n(m'') \ \text dm' \ \text dm'' \\
    %     =
    %     &-\int\limits_0^\infty R(m,m') \cdot n(m) \cdot n(m') \ \text dm'
    %     +\int\limits_0^\infty R(m',m-m') \cdot \frac{1}{2} \cdot n(m') \cdot n(m-m') \ \text dm' \\
    %     =
    %     &-\int\limits_0^\infty R(m,m') \cdot n(m) \cdot n(m') \ \text dm'
    %     +\int\limits_0^{m/2} R(m',m-m') \cdot n(m') \cdot n(m-m') \ \text dm'
    % \end{align}
    % TODO Use multi-line equations here, no duplicate numbering! (both here and elsewhere)

    % }}}
    % Discretization of the Coagulation Kernel {{{ 
    \subsection{Discretization of the Coagulation Kernel}

        Having defined the coagulation kernel on a continuous mass axis,
        we now wish to implement the numerical analogon of the described model to allow 
        numerical simulations of the dust mass distribution's temporal evolution. \\

        Even in a highly simplified scenario, where only hit-and-stick coagulation is included, 
        the definition of the kernel $K_{kij}$ is not at all trivial. To assure both the 
        stability and the accuracy of the algorithm, one has to take care of two separate 
        problems, namely:
        \begin{enumerate}
            \item The conservation of mass \textit{has} to be assured, otherwise the numerical      
                solution can not be assumed to remain stable for long. In the case of hit-and-stick 
                coagulation, this means that for every pair of colliding particles, a single new 
                particle has to be created. At the same time, the two initial particles have to be 
                removed from the distribution. During this process, the total mass should remain 
                unaffected down to machine precision.
            \item When using a logarithmically spaced grid for the discretized mass axis, it can 
                not be assumed that after a collision of two dust particles with masses $m_i$ and 
                $m_j$ the resulting particle will carry a mass $m_k=m_i+m_j$ whose value can be 
                mapped trivially onto the grid. In general, the corresponding index will not be an 
                integer, and instead lie between somewhere between the two neighboring grid points 
                with indices $k$ and $k+1$.
                Therefore, the result of the merging of $m_i$ and $m_j$ has to be divided in some 
                sensible way between these two neighboring bins.
        \end{enumerate}

        % \todo{What then is $f^{coag}_{kij}$? (Define here or elsewhere?) 
        % (with canc. handling \& Kovetz-Olund)} \\

        In the following section, we will outline a method for assuring these conditions.

    % }}}
    % The Kovetz-Olund Algorithm {{{ 
    \clearpage\subsection{The Kovetz-Olund Algorithm}
        
        An elegant way for solving the two problems listed above is given in the 1969 paper
        by Kovetz \& Olund \cite{kovetz_olund_1969}, where they used the following procedure:
        \begin{enumerate}
            \item The hit-and-stick coagulation kernel is split into two parts, as we did above.
                The first is the \textit{gain} of particles in bin $k$ due to the collision of 
                particles from the bins $i$ and $j$. The second is the \textit{loss} of particles 
                from bin $k$ due to collisions of particles in bin $k$ with particles from any 
                other bin $j$. 
                Using this separation into gain \& loss terms (as well as the discrete 
                formalism for the Smoluchowski equation), the dust particle mass 
                distribution's temporal derivative can be expressed in the following form:
                \begin{equation}
                    \pderiv{N_k}{t}
                        = \sum_{i=0}^{\mathcal N_m} \sum_{j=0}^{\mathcal N_m}
                            K_{kij}^\text{gain}\cdot N_i\cdot N_j
                        - \sum_{j=0}^{\mathcal N_m} K_{kj}^\text{loss} \cdot N_k\cdot N_j
                \end{equation}
                % \todo{what is $K^\text{gain/loss}$?} \todo{Why double sum here?} \\
                In other words, the total kernel can be written as
                \begin{equation}
                    K_{kij} = K_{kij}^\text{gain} - K_{ij}^\text{loss}\cdot\delta_{ki}
                \end{equation}
                Splitting the kernel like this into a gain \& a loss term is a quite general
                approach, and can be used in more complex scenarios as well (including e.g.
                particle fragmentation processes). In our case, here the loss component is given by 
                the coagulation rate coefficent $R_{ij}^\text{coag}$, and the gain component 
                will shortly be defined from it.
            \item For the scenario of pure hit-and-stick coagulation, a unique discretization
                of the kernel can be defined such that both the number of particles and the
                conservation of total mass are handled correctly. To do this, consider a
                pair of colliding particles with indices $i$ and $j$. Then, let the index
                $\bar k$ be chosen in such a way that the condition
                \begin{equation}
                    m_{\bar k} \leq m_i + m_j < m_{\bar k+1}
                \end{equation}
                is satisfied.
            \item As stated before, in hit-and-stick coagulation, a single new particle emerges
                for each pair of colliding particles. Using the definitions from above, this
                condition can be expressed as follows:
                \begin{equation}
                    \label{eq:kovetz_olund_particle_conservation}
                    K_{\bar k,ij}^\text{gain} + K_{\bar k+1,ij}^\text{gain}
                    \overset{!}{=} K_{ij}^\text{loss}
                \end{equation}
            \item The second condition is that of mass conservation, which can be written as:
                \begin{equation}
                    \label{eq:kovetz_olund_mass_conservation}
                    m_{\bar k} \cdot K_{\bar k,ij}^\text{gain}
                    + m_{\bar k+1} \cdot K_{\bar k+1,ij}^\text{gain}
                    \overset{!}{=} (m_i+m_j) \cdot K_{ij}^\text{loss}
                \end{equation}
            \item Now, in order to map the resulting particle's mass onto the two 
                neighboring bins, let us define a parameter $\varepsilon$ such that
                \begin{align}
                    \label{eq:definition_of_kovetz_olund_epsilon_parameter}
                    K_{\bar k,ij}^\text{gain}
                        &=K_{ij}^\text{loss} \cdot (1-\varepsilon),\ \text{and}\\
                    K_{\bar k+1,ij}^\text{gain}
                        &=K_{ij}^\text{loss} \cdot \varepsilon
                \end{align}
                This assures that \cref{eq:kovetz_olund_particle_conservation} is satisfied. 
                If we now plug these definitions into \cref{eq:kovetz_olund_mass_conservation} 
                and solve for $\varepsilon$, we arrive at
                \begin{equation}
                    \varepsilon
                        :=\frac{m_i+m_j-m_{\bar k}}{m_{\bar k+1}-m_{\bar k}}
                \end{equation}
        \end{enumerate}
        This is the Kovetz-Olund algorithm \cite{kovetz_olund_1969}, which is frequently used 
        in studies of dust coagulation in proto-planetary disks, see e.g.
        \cite{brauer_dullemond_henning_2007} and \cite{birnstiel_dullemond_brauer_2010}.
    
    % }}}
    % Near-Zero Cancellation Handling {{{ 
    \clearpage\subsection{Near-Zero-Cancellation Handling}
    \label{sec:near_zero_cancellation_handling}
    
        When using floating-point numbers following the representation defined
        by the IEEE-754 standard, it can occur that
        \begin{equation}
          a + b = a
          \ \ \ \ \ \text{for} \ \ \ \ \
          b\neq0
        \end{equation}
        Typically, this happens when
        \begin{equation}
            |b| < \varepsilon_m \cdot |a|
        \end{equation}
        Here, $\varepsilon_m$ labels the \textit{machine precision}, which can be used to give 
        an upper bound on the relative approximation error that may happen in floating point 
        arithmetic due to rounding. For 64bit floats, this value is approximately given by
        \begin{equation}
            \varepsilon_{f64} \approx 10^{-16}
        \end{equation}
        
        Let $i$ and $j$ once again be the indices used to label two colliding particles. 
        Without loss of generalization, let us assume that particle $i$ is 
        \textit{much smaller} than particle $j$. \\
        
        The detailed balance approach from above requires the removal of both the big and the small 
        particle from the mass distribution, followed by the re-insertion of a new particle 
        carrying the initial pair's combined mass. This new particle would then have a mass which 
        is nearly identical to that of the bigger one of the original two particles, it would be 
        only a tiny bit heavier.\\
        
        In the approach defined above this would mean that $\bar k=j$, i.e. the resulting particle 
        will reside in the same bin as the larger original one. Also, it would follow that 
        $\varepsilon\ll1$.\\
        
        Let us now take a look at the particle mass distribution in the bin $\bar k$ and, more 
        specifically, by how much it changes from one timestep to the next. For this particular 
        pair of $i$ and $\bar k=j$, we can write:
        \begin{equation}
            \pderiv{n_{\bar k}}{t}
                =K_{\bar k,i\bar k}^{\text{gain}}\cdot n_i\cdot n_{\bar k}
                -K_{\bar ki}^{\text{loss}}\cdot n_i\cdot n_{\bar k}
        \end{equation}
        Plugging in \cref{eq:definition_of_kovetz_olund_epsilon_parameter} leads to
        \begin{equation}
            \label{eq:cancellation_handling_eq_1}
            \pderiv{n_{\bar k}}{t}
                =(1-\varepsilon)K_{\bar ki}^{\text{loss}}\cdot n_i\cdot n_{\bar k}
                -K_{\bar ki}^{\text{loss}}\cdot n_i\cdot n_{\bar k}
        \end{equation}
        Here, the two terms almost cancel each other out. What remains is a contribution which is 
        proportional to $\varepsilon$. But if $\varepsilon$ is small enough, then the 
        double-precision accuracy of the 64-bit floating point representation will lead to a
        breakdown of the algorithm. \\
        
        It is relatively easy though to identify the particle pairs $(i,j)$ for which the scenario 
        detailed above will occur. Let $i$ (without loss of generality) be the index of the larger 
        one of the two colliding masses. Cancellation may then occur when the resulting $k$ is 
        equal to $j$.\\
        
        In that case, we analytically carry out the subtraction in 
        \cref{eq:cancellation_handling_eq_1}, and write:
        \begin{equation}
            \pderiv{n_{\bar k}}{t}
                = -\varepsilon K_{\bar ki}^{\text{loss}}\cdot n_i\cdot n_{\bar k}
        \end{equation}

        % \todo{Elaborate on this, see "Dust Evolution with Binning Methods"} \\

        The effect that this corrected has on the kernel mass error can be seen visualized 
        in \cref{fig:effect_of_cancellation_handling_on_kernel_mass_error}.

    % }}}

% }}}
% Definition of the Fragmentation Kernel {{{ 
\clearpage\section{Definition of the Fragmentation Kernel}
\label{sec:fragmentation_kernel}

    Let us now focus on the definition of the kernel matrix for dust particle fragmentation 
    processes. As before, where we considered the simpler case of pure hit-and-stick coagulation,
    in the case of fragmentation we also separate the kernel function into a positive gain 
    component $G$ and a negative loss component $L$. \\

    These two kernel components can be expressed as
    \begin{align}
        L^\text{coag}(m, m', m'') 
            &:= -R(m',m'') \cdot \frac{1}{2} \bigg[\delta_D(m-m')+\delta_D(m-m'')\bigg] \\
        G^\text{coag}(m, m', m'') 
            &:= +R(m',m'') \cdot \frac{1}{2} \cdot f^\text{frag}(m,m',m'')
    \end{align}

    Here, it should be noted that the loss term is exactly identical to what we defined for pure
    hit-and-stick coagulation in \cref{eq:definition_of_coagulation_kernel_gain_and_loss}.
    If two particles with masses $m$ and $m'$ collide, then these two particles ``disappear''
    from the distribution. \\ 

    Replacing them, new particles will then have to be added to be distribution, under the strict 
    criterion of mass conservation. How this is implemented can be controlled via the definition 
    of the gain term or, more specifically, via the definition of the function 
    $f^\text{frag}(m, m', m'')$.
    The big distinction between the kernels for coagulation and fragmentation lies 
    in the way that this function is defined. \\

    In the scenario of pure hit-and-stick coagulation, it is given by a Dirac $\delta$ function,
    which allows us to reduce the double integral in the Smoluchowski equation (or double sum, in 
    the discrete case) into an integral/sum over a single axis. That is be very different here,
    and a reduction of the double integral will not be possible. This is why the numerical 
    cost of modeling fragmentation processes is significantly higher than if one were to only 
    consider pure coagulation. \\

    The reasoning behind the different definitions of the functions $f^\text{coag}$ and
    $f^\text{frag}$ is the following: If two particles merge into one, then (without breaking mass
    conservation) there is only a single possible mass value that can result from that. 
    
    In the case of fragmentation though, this is not true. 
    If two particles collide, they could in principle fragment into a whole range of 
    differently-sized particles, as long as mass is conserved. \\

    % The function $f^\text{coag}(m, m', m'')$ allows us to define the manner in which a collision's 
    % total mass is distributed onto a possibly large number of newly created particles. Once again, 
    % there is the criterion of mass conservation. It will be important to make sure that mass is 
    % conserved exactly down to the machine precision of 64-bit floating point numbers, 
    % as even small mass errors will accumulate over the large time scales we plan on running 
    % the simulation for. The sum of the newly created particles' masses should be exactly equal to 
    % that of the original colliding particles. \\

    In the following, we will attempt to define the function $f^\text{frag}$ in a sensible fashion.
    As we did before in other contexts, here as well we will adopt a shorthand index notation,
    defined by 
    \begin{equation}
        f^\text{frag}_{kij} := f^\text{frag}(m_k, m_i, m_j)
    \end{equation}

    There are of course various different approaches for how one might do this.

    % \todo{That's where the computational cost comes from!} \\
    % \todo{How to model the distribution of particle masses resulting from a fragmentation event?} \\
    % \todo{Many small ones, only a few bigger ones.} \\
    % \todo{Different approaches [cite]}

    % \todo{Smol. eq. looks very similar to coag., but $f$ instead of $\delta_D$, can't "reduce".} \\

    % Modeling the Mass Distribution Resulting from a Fragmentation Event {{{ 
    \subsection{Modeling the Resulting Mass Distribution}
    \label{sec:modeling_the_mass_distribution_resulting_from_fragmentation}

        An extremely simplified first approach for modeling the distribution of dust particle
        masses resulting from a collision event with subsequent particle fragmentation could 
        be derived from the
        naive assumption that, as a result of a collision, the entire mass involved in the
        collision is transferred into a single bin with index $k$. \\

        This is of course not a very sophisticated definition, but was used as a first 
        approach in the process of building and testing the model for this thesis in 
        computer code.

        \clearpage

        An implementation of this simple approach could be achieved via the definition of
        \begin{equation}
            f^\text{frag}_{kij} = \frac{m_i + m_j}{m_k}
        \end{equation}

        If, for example, we assume that the entire mass involved in the collision is to be
        transferred to the lowest-mass bin, i.e. the entire mass is converted into tiny particles,
        we would simply have to set $k=1$ (the index of the first bin in the mass grid).
        This might be imagined as ``complete pulverization'' of the colliding particles.

        % \todo{rename? how about complete annhiltation lool} \\
        % \todo{Naive: Move all masses to the bin corresponding to lowest mass} \\
        % \todo{i.e.: total "pulverization"} \\

        % \todo{Include $\theta_H$ ?}

    % }}}
    % The MRN distribution {{{ 
    \subsection{The MRN Distribution}
    \label{sec:the_mrn_distribution}
        
        A different method, which presents a bit more sophisticated approach, was given by 
        Mathis, Rumpl, and Nordsieck in their 1977 paper \cite{mathis_rumpl_nordsieck_1977}. \\

        There, the distribution of masses resulting from a collision of two dust particles 
        is modeled making use of an inverse power law with exponent $q := - 11/6$. \\
        % \todo{$? \sim m^q$} ($N$ or $n$ ??) \\

        We will implement it as follows:
        \begin{equation}
            f^\text{frag}_{kij} = \frac{m_i + m_j}{m_k} \cdot \xi_{kij}
        \end{equation}
        Here, $\xi_{kij}$ labels the fraction of the total mass $m_\text{tot} = m_i + m_j$ that 
        is transferred into a given bin $k$ as a result of the collision. 
        This fraction shall be defined as 
        \begin{equation}
            \xi_{kij} := \frac{m_k^q}{S}
        \end{equation}
        with the normalization factor
        \begin{equation}
            S := \sum_{k_\text{min}}^{k_\text{max}} m_k^q % \cdot \Delta m_k
        \end{equation}

        % \todo{Elaborate, explain, cite!}

    % }}}

% }}}
% The Kernel (Plots) {{{
\clearpage\section{Visualization of Kernel Matrices}
    
    On the following pages, we display a few visualizations of the kernel matrices we just 
    defined, plus a few short explanations for why they look the way that they do.

    % Coagulation Kernel on Linear Mass Axis {{{ 
    \subsection{Coagulation Kernel on Linear Mass Axis}
    \label{sec:coagulation_kernel_on_linear_mass_axis}

        To get started, we will first consider the definition of the kernel matrix 
        for pure hit-and-stick coagulation, defined on a mass axis using linear scaling.
        This is probably the simplest case one might think of in this context. \\

        The kernel matrix is shown in \cref{fig:K_coag_ij_vs_k_on_linear_mass_axis} for 
        various values of $k$, the value of which we use to index the different particle 
        masses that could result from a given collision.
        The negative component $L^\text{coag}_{kij}$ and the positive component 
        $G^\text{coag}_{kij}$ are drawn in different colors, namely blue and red, respectively. \\

        Let us try to understand why the plot looks the way it does: \\

        \textbf{Observation 1:} The displayed matrix is symmetric along the diagonal, meaning 
        it stays invariant under transformations $i \to j$ and $j \to i$. This is due to the 
        fact that we do not care whether a particle $a$ collides with a particle $b$, or a 
        particle $b$ collides with a particle $a$. These two scenarios describe the same
        event. \\

        \textbf{Observation 2:} The kernel loss (shown in blue) is visualized as two 
        straight lines. For the moment, let us focus on the vertical one: If a collision involving 
        a particle with mass $m_k$ occurs, then we will have to remove one particle from the 
        distribution at that bin $k$. This explains the vertical line's horizontal position at 
        $j = k$. In addition to the particle characterized by index $k$, such a collision could 
        involve any other particle mass, which explains why the loss component looks like a line, 
        instead of a point. An analogous reasoning can be made for the horizontal line. \\

        \textbf{Observation 3:} Even though we just mentioned that a given particle could 
        collide with particles from \textit{any other} mass bin, this is only true in principle. 
        As can be seen in the plot, the blue lines do not end at the edges of the kernel matrix, 
        but much earlier instead. This is due to the fact that we have to ensure the conservation 
        of mass over time. Consider two particles with masses $m_i$ and $m_j$. If their masses 
        do not satisfy the conditions $m_i + m_j < m_\text{max}$, then the collision and subsequent 
        merging will require the creation of a particle that can not be represented in the mass 
        grid, which would lead to a loss of mass. 
        In reality, mass can of course move out of the considered range, but we will not allow it 
        to do that in order to be able to better assert the algorithms mass conservation properties.
        As such, we can expect a ``traffic jam'' of dust mass accumulating in the highest-mass bin.
        \\

        \textbf{Observation 4:} The positive component (shown in red) follows a diagonal line. 
        This represents the influence of the mass conservation criterion for newly created 
        particles, which must satisfy the condition $m_\text{tot} = m_i + m_j$. 
        Therefore, $m_i = m_\text{tot} - m_j$, which can be interpreted as a linear function 
        of the kind ``$y = c - x$'', and leads to the representation as a diagonal
        downward-sloped line. \\

    % }}}
    % Coagulation Kernel on Linear Mass Axis (Plot) {{{ 
    \clearpage

        \begin{figure}[h!]
            \makebox[\textwidth]{
                \includegraphics[width=\paperwidth]{101/Kkij vs k, coag=True, frag=False.pdf}
            }
            \caption{ 
                Matrix layers of the kernel for pure hit-and-stick coagulation on a linear mass 
                axis with 
                lower mass grid boundary $m_\text{min}=10^{-6}\ \text{kg}$, 
                upper mass grid boundary $m_\text{min}=10^{-2}\ \text{kg}$, 
                mass grid resolution $\mathcal N_m=100$, and 
                collision rate coefficent $R_{ij}^\text{coll} := 1\ \text{m}^3\ \text{s}^{-1}$.
            }
            \label{fig:K_coag_ij_vs_k_on_linear_mass_axis}
        \end{figure}

    % }}}
    % Coagulation Kernel on Logarithmic Mass Axis {{{ 
    \subsection{Coagulation Kernel on Logarithmic Mass Axis}

        Next, we will define the coagulation kernel again, but making use of a logarithmic 
        scaling for the discretized mass axis instead. \\

        The results can be seen visualized in \cref{fig:K_coag_gain_ij_vs_k_on_log_mass_axis} 
        and \cref{fig:K_coag_loss_ij_vs_k_on_log_mass_axis}. Note that due to the logarithmic
        representation, here we need to plot gain and loss components separately. \\

        \textbf{Observation 1:} The negative ``loss'' component (shown in blue) is quite similar to 
        what we already saw in \cref{fig:K_coag_ij_vs_k_on_linear_mass_axis} for the case of a 
        linearly discretized mass axis. A difference is given by the extension of the blue lines, 
        which reach much further towards the edge of the kernel matrix than they did before.
        This can be explained by the logarithmic scaling. \\

        \textbf{Observation 2:} In contrast to the linear case, here the positive ``gain''
        component (shown in red) is not represented as a linear diagonal, but an 
        inverse power relation. This can be explained by the logarithmic scaling as well. 

        \textbf{Observation 3:} The majority of the kernel's entries carry a value of zero,
        with only about $3 \%$ of the entries being non-zero. 
        (Here, a mass grid resolution of $\mathcal N_m$ was used.)
       
    % }}}
    % Fragmentation Kernel on Logarithmic Mass Axis {{{ 
    \subsection{Fragmentation Kernel on Logarithmic Mass Axis}

        The positive and negative components of the kernel matrix for modeling pure 
        dust particle fragmentation can be seen in 
        \cref{fig:K_frag_gain_ij_vs_k_on_log_mass_axis} and 
        \cref{fig:K_frag_loss_ij_vs_k_on_log_mass_axis}, respectively. \\

        % Let us first tackle the similarities, before talking about 
        % the differences to what we saw in the case of the coagulation kernel on a 
        % logarithmically discretized mass axis. \\

        \textbf{Observation 1:} The negative ``loss'' component of the fragmentation kernel 
        looks quite similar to what we saw in the case of the coagulation kernel on a 
        logarithmically discretized mass axis. A difference can be noticed at the edges 
        of the kernel though. While the collisions of high-mass particles had to be 
        ignored in the case of hit-and-stick coagulation, we can include these collisons
        here. This can be explained by the fact that the particles resulting from a 
        collision+fragmentation event are \textit{always} smaller (and thus less massive) than 
        the original colliding particles. As such, no mass loss can occur here due to a 
        newly created particle's mass not being represented in the mass grid. \\

        \textbf{Observation 2:} The number of non-zero kernel entries is much, much higher 
        here than it was in the case of stick-and-hit coagulation. Here, about $30 \%$ of the 
        kernel's entries carry non-zero values. This can be explained by 
        the fact that a merge event always leads to the creation of exactly one new particle 
        from the two original ones, whereas this is not the case in fragmentation. 
        There, possibly very many new particles could be created as a result of a collision,
        and the mass distribution of newly created particles is modeled via the inverse power 
        law given by the MRN distribution (for details see \cref{sec:the_mrn_distribution}). \\

        \textbf{Observation 3:} Another thing one might notice is the existence of a 
        diagonal feature, for the kernel entries with $i=j$. In the plot, it can be seen as a
        bright line going from top right to top left. This is the same feature we could already see 
        in the relative velocity plot in \cref{fig:total_relative_dust_particle_velocity}, 
        as well as the plot of the collision rate coefficent in 
        \hyperref[fig:collision_cross_section_and_rate_coefficient]
        {\cref*{fig:collision_cross_section_and_rate_coefficient}b}.
        The feature can be explained by the fact that a pair of particles $(i,j)$ has a lower 
        chance of colliding, if $i = j$. Since same-sized particles also experience similar 
        kinematics, here the relative velocity is close to zero, which leads to a low 
        collision rate, and thus smaller values of the kernel entries. 

        % \todo{Plot pure fragmentation kernel on linear mass axis for various values of $k$.} \\

        % \clearpage
        % \begin{figure}[h!]
        %     \makebox[\textwidth]{
        %         \includegraphics[width=\paperwidth]{101/Kkij vs k, coag=False, frag=True.pdf}
        %     }
        %     \caption{ \todo{Add description!} \\ \todo{Add colorbar!} \\ (with correct z-limits!)}
        % \end{figure}

    % }}}
    % Total Kernel on Logarithmic Mass Axis {{{ 
    \clearpage
    \subsection{Total Kernel on Logarithmic Mass Axis}
        
        Now, we add the two kernel matrices $K_{kij}^\text{frag}$ and $K_{kij}^\text{coag}$
        to construct the total kernel, for the full model including both associative and 
        dissociative processes. \\

        Its gain and loss components can be seen visualized in 
        \cref{fig:K_tot_gain_ij_vs_k_on_log_mass_axis} and 
        \cref{fig:K_tot_loss_ij_vs_k_on_log_mass_axis}, respectively. \\

        % \textbf{Observation 1}: \\

        % visualization of the total kernel \\
        % visualization of the total kernel \\
        % visualization of the total kernel \\
        % visualization of the total kernel \\
        % defined on a discrete mass axis using logarithmic scaling \\
        % shown for various values of $k$ \\

        % \todo{Plot number of non-zero kernel entries vs. $k$, and give total non-zero entries.} 

    % }}}
    % Kernel Mass Error {{{ 
    \subsection{Kernel Mass Error}
        In \cref{fig:effect_of_cancellation_handling_on_kernel_mass_error} and
        \cref{fig:kernel_error_for_frag_and_total} we display the dimensionless kernel 
        error, i.e. the relative mass deviation resulting from a single collision,
        for various different scenarios. \\

        First, \cref{fig:effect_of_cancellation_handling_on_kernel_mass_error} shows the
        influence of near-zero cancellation handling on the error of the kernel 
        for pure coagulation
        (for implementation see \cref{sec:near_zero_cancellation_handling}). \\ 

        As the visualization shows, the mass error per collision of the coagulation 
        kernel can actually be decreased significantly via the inclusion of a procedure 
        for handling near-zero cancellation in floating point arithmetics. \\ 

        An additional error is introduced though by the inclusion of fragmentation, 
        as can be seen in \hyperref[fig:kernel_error_for_frag_and_total]
        {\cref*{fig:kernel_error_for_frag_and_total}a}. Here, it is not possible 
        to use the same procedure to get rid of this error, as no obvious cases of
        cancellation occur. \\

        The error of the total kernel is then shown in 
        \hyperref[fig:kernel_error_for_frag_and_total]
        {\cref*{fig:kernel_error_for_frag_and_total}b}. \\

        All of these (dimensionless/relative) errors could be kept on the order of $10^{-12}$.

    % }}}
    % Coagulation Kernel on Logarithmic Mass Axis (Plot) {{{ 
        
        \cleardoublepage
        \begin{figure}[h!]
            \makebox[\textwidth]{
                \includegraphics[width=\paperwidth]{102/Kkij_gain vs k, coag=True, frag=False.pdf}
            }
            \caption{ 
                Matrix layers of the positive kernel component for pure hit-and-stick 
                coagulation on a logarithmic mass axis,
                with 
                lower mass grid boundary $m_\text{min}=10^{-15}\ \text{kg}$, 
                upper mass grid boundary $m_\text{min}=10^{-8}\ \text{kg}$, and
                mass grid resolution $\mathcal N_m=100$.
            }
            \label{fig:K_coag_gain_ij_vs_k_on_log_mass_axis}
        \end{figure}

        \clearpage
        \begin{figure}[h!]
            \makebox[\textwidth]{
                \includegraphics[width=\paperwidth]{102/Kkij_loss vs k, coag=True, frag=False.pdf}
            }
            \caption{ 
                Matrix layers of the negative kernel component for pure hit-and-stick 
                coagulation on a logarithmic mass axis,
                with 
                lower mass grid boundary $m_\text{min}=10^{-15}\ \text{kg}$, 
                upper mass grid boundary $m_\text{min}=10^{-8}\ \text{kg}$, and
                mass grid resolution $\mathcal N_m=100$.
            }
            \label{fig:K_coag_loss_ij_vs_k_on_log_mass_axis}
        \end{figure}

    % }}}
    % Fragmentation Kernel on Logarithmic Mass Axis {{{ 

        \cleardoublepage
        \begin{figure}[h!]
            \makebox[\textwidth]{
                \includegraphics[width=\paperwidth]{102/Kkij_gain vs k, coag=False, frag=True.pdf}
            }
            \caption{ 
                Matrix layers of the positive kernel component for pure fragmentation
                on a logarithmic mass axis,
                with 
                lower mass grid boundary $m_\text{min}=10^{-15}\ \text{kg}$, 
                upper mass grid boundary $m_\text{min}=10^{-8}\ \text{kg}$, and
                mass grid resolution $\mathcal N_m=100$.
            }
            \label{fig:K_frag_gain_ij_vs_k_on_log_mass_axis}
        \end{figure}

        \clearpage
        \begin{figure}[h!]
            \makebox[\textwidth]{
                \includegraphics[width=\paperwidth]{102/Kkij_loss vs k, coag=False, frag=True.pdf}
            }
            \caption{ 
                Matrix layers of the negative kernel component for pure fragmentation
                on a logarithmic mass axis,
                with 
                lower mass grid boundary $m_\text{min}=10^{-15}\ \text{kg}$, 
                upper mass grid boundary $m_\text{min}=10^{-8}\ \text{kg}$, and
                mass grid resolution $\mathcal N_m=100$.
            }
            \label{fig:K_frag_loss_ij_vs_k_on_log_mass_axis}
        \end{figure}

    % }}}
    % Total Kernel on Logarithmic Mass Axis (Plot) {{{ 

        \cleardoublepage
        \begin{figure}[h!]
            \makebox[\textwidth]{
                \includegraphics[width=\paperwidth]{102/Kkij_gain vs k, coag=True, frag=True.pdf}
            }
            \caption{ 
                Matrix layers of the positive kernel component for both coagulation and 
                fragmentation on a logarithmic mass axis,
                with 
                lower mass grid boundary $m_\text{min}=10^{-15}\ \text{kg}$, 
                upper mass grid boundary $m_\text{min}=10^{-8}\ \text{kg}$, and
                mass grid resolution $\mathcal N_m=100$.
            }
            \label{fig:K_tot_gain_ij_vs_k_on_log_mass_axis}
        \end{figure}
        
        \clearpage
        \begin{figure}[h!]
            \makebox[\textwidth]{
                \includegraphics[width=\paperwidth]{102/Kkij_loss vs k, coag=True, frag=True.pdf}
            }
            \caption{ 
                Matrix layers of the negative kernel component for both coagulation and fragmentation on a logarithmic mass axis,
                with 
                lower mass grid boundary $m_\text{min}=10^{-15}\ \text{kg}$, 
                upper mass grid boundary $m_\text{min}=10^{-8}\ \text{kg}$, and
                mass grid resolution $\mathcal N_m=100$.
            }
            \label{fig:K_tot_loss_ij_vs_k_on_log_mass_axis}
        \end{figure}

    % }}}
    % Old Plots {{{ 
        
        % \clearpage
        % \begin{figure}[h!]
        %     \makebox[\textwidth]{
        %         \includegraphics[width=\paperwidth]{101/Kkij vs k, coag=True, frag=True.pdf}
        %     }
        %     \caption{ \todo{Add description!} \\ \todo{Add colorbar!} \\ (with correct z-limits!)}
        % \end{figure}

% }}}
    % Kernel Mass Error (Plot) {{{ 

    \newpage

% \begin{figure}[h!]
%     \centering
%     \begin{minipage}{.5\linewidth}
%         \centering
%       	\subfloat[]{
%             \label{:a}
%       	  	\includegraphics[width=\linewidth]{34/canc.pdf}
%       	}
%     \end{minipage}%
%     \begin{minipage}{.5\linewidth}
%         \centering
%       	\subfloat[]{
%             \label{:b}
%       	  	\includegraphics[width=\linewidth]{34/nocanc.pdf}
%       	}
%     \end{minipage}
%     \caption{}
% \end{figure}

% \newpage


\begin{figure}[h!]
    \centering
    \begin{minipage}{.5\linewidth}
        \centering
      	\subfloat[Coagulation Kernel without Cancellation Handling]{
            \label{:a}
      	  	% \includegraphics[width=\linewidth]{34/K_coag_canc.pdf}
      	  	\includegraphics[width=\linewidth]{34/error_K_coag_canc.pdf}
      	}
    \end{minipage}%
    \begin{minipage}{.5\linewidth}
        \centering
      	\subfloat[Coagulation Kernel with Cancellation Handling]{
            \label{:b}
      	  	% \includegraphics[width=\linewidth]{34/K_coag_nocanc.pdf}
      	  	\includegraphics[width=\linewidth]{34/error_K_coag_nocanc.pdf}
      	}
    \end{minipage}
    \caption{
        Influence of Near-Zero Cancellation on the Mass Error of the
        Coagulation Kernel
    }
    \label{}
\end{figure}


% \begin{figure}[h!]
%     \centering
%     \end{minipage}
%     \begin{minipage}{.5\linewidth}
%         \centering
%       	\subfloat[]{
%             \label{:a}
%       	}
%     \end{minipage}%
%     \begin{minipage}{.5\linewidth}
%         \centering
%       	\subfloat[]{
%             \label{:b}
%       	}
%     \end{minipage}
%     \caption{
%         Kernel Mass Error:
%         (a) Mass Error of Coagulation Kernel without Cancellation Handling
%         (b) Mass Error of Coagulation Kernel with Cancellation Handling
%         (c) Mass Error of Fragmentation Kernel
%         (d) Mass Error of Total Kernel
%     }
% \end{figure}

% \newpage


    % }}}
    % Cancellation {{{ 

        \begin{figure}[h!]
            \centering
            \begin{minipage}{.5\linewidth}
                \centering
              	\subfloat[Fragmentation kernel]{
                    \label{:a}
              	  	\includegraphics[width=\linewidth]{34/error_K_frag.pdf}
              	}
            \end{minipage}%
            \begin{minipage}{.5\linewidth}
                \centering
              	\subfloat[``Full'' kernel: coagulation + fragmentation]{
                    \label{:b}
              	  	\includegraphics[width=\linewidth]{34/error_K_tot.pdf}
              	}
            \end{minipage}
            \caption{Dimensionless mass error $\Delta X_{ij}$ for
                (a) Fragmentation kernel, and
                (b) Total kernel.
            }
            \label{fig:kernel_error_for_frag_and_total}
        \end{figure}

        % \todo{Add table: Kernel error for coag. / frag. / both, and for various mass axis resolutions.}

    % }}}

% }}}
% Analytical Solutions {{{ 
\newpage\section{Analytical Solutions}

    Analytical solutions to the Smoluchowski coagulation equation exist if, and only if the kernel
    takes on one of the following three forms:
    \begin{enumerate}
        \item Constant kernel:
            \begin{equation}
                K(m, m', m'') = A_1
            \end{equation}
        \item Linear kernel:
            \begin{equation}
                K(m, m', m'') = A_2 \cdot (m' + m'')
            \end{equation}
        \item Quadratic kernel:
            \begin{equation}
                K(m, m', m'') = A_3 \cdot (m' \cdot m'')
            \end{equation}
    \end{enumerate}

    Here, the three numbers $A_1$, $A_2$, and $A_3$ are constants. 
    The analytical solutions in these cases are discussed in detail 
    in the 1990 paper written by K. Ohtsuki \cite{ohtsuki_1990}.

    % Constant Kernel {{{ 
    \subsection{Constant Kernel}

    % \todo{Constant Kernel:} (general enough?)
    % \begin{equation}
    %     K(m, m', m'')
    %         = 1
    % \end{equation}

    When making use of the constant kernel, the Smoluchowski reads
    \begin{equation}
        \pderiv{n}{t}(m)
            = 
                \int\limits_0^\infty \int\limits_0^\infty
                n(m') \cdot n(m'')
                \ \dd m' \ \dd m''
        \label{eq:smoluchowski_with_constant_kernel}
    \end{equation}
    
    With the initial condition given by 
    \begin{align}
        n_1(t=0) &= n_0 
        \ \ \ \ \ \text{and} \\
        n_k(t=0) &= 0 
        \ \ \ \ \ \text{for } k>1
    \end{align}
    the solution to \cref{eq:smoluchowski_with_constant_kernel} is 
    given by \cite{smoluchowski_1916} \cite{ohtsuki_2017}
    \begin{equation}
        n_k(\tau) = n_0 \cdot g^2 \cdot (1-g)^{k-1}
    \end{equation}
    with 
    \begin{equation}
        g = (1+\tau/2)^{-1}
    \end{equation}
    where the normalized time is given by 
    \begin{equation}
        \tau = \frac{A_1 \cdot \rho \cdot t}{m_\text{min}}
    \end{equation}

    % }}}
    % Linear Kernel {{{ 
    \subsection{Linear Kernel}

    % \todo{Linear Kernel:} (general enough?)
    % \begin{equation}
    %     K(m, m', m'')
    %         = m' + m''
    % \end{equation}

    The Smoluchowski equation with a linear kernel can be written as
    \begin{equation}
        \pderiv{n}{t}(m)
            = 
                \int\limits_0^\infty \int\limits_0^\infty
                (m' + m'') \cdot
                n(m') \cdot n(m'')
                \ \dd m' \ \dd m''
        \label{eq:smoluchowski_with_linear_kernel}
    \end{equation}

    Let $\overline m_0$ label the average dust particle mass at time $t=0$.
    With the initial condition 
    \begin{equation}
        n(m, t=0) = C \cdot e^{-\lambda m}
    \end{equation}
    where 
    \begin{align}
        C = \frac{N_0}{\overline m_0}
        \ \ \ \ \ \text{and} \ \ \ \ \
        \lambda = \frac{1}{\overline m_0}
    \end{align}
    the solution to \cref{eq:smoluchowski_with_linear_kernel} according to
    \cite{safronov_1963}, \cite{safronov_1972} is given by 
    \begin{equation}
        n(m, \tau) = \frac{N_0 \cdot g}{m \cdot \sqrt{1-g}} \cdot e^{-\lambda \cdot m(2-g)}
        \cdot I_1 \cdot 2\lambda m \sqrt{1-g}
    \end{equation}
    with
    \begin{equation}
        g = e^{-\tau}
    \end{equation}
    where $I_1$ is the modified Bessel function and $\tau = A_2 \rho_g t$ is the normalized time.

    % }}}
    % Quadratic Kernel {{{ 
    \subsection{Quadratic Kernel}

    % \todo{Quadratic Kernel:} (general enough?)
    % \begin{equation}
    %     K(m, m', m'')
    %         = m' \cdot m''
    % \end{equation}

    For the quadratic kernel, the Smoluchowski equation reads
    \begin{equation}
        \pderiv{n}{t}(m)
            = 
                \int\limits_0^\infty \int\limits_0^\infty
                (m' \cdot m'') \cdot
                n(m') \cdot n(m'')
                \ \dd m' \ \dd m''
        \label{eq:smoluchowski_with_quadratic_kernel}
    \end{equation}

    A solution to this equation is derived by in the work by \cite{wattis_2006}, 
    where they give it as 
    \begin{equation}
        n_k(t) \sim \begin{cases} 
            \frac{ 1 }{ \sqrt{2\pi} \cdot k^{5/2}\cdot t } \cdot \exp[-k(t-1-\log(t))]
            & \text{for } t < t_c
            \\
            \frac{ 1 }{ \sqrt{2\pi} \cdot k^{5/2}\cdot t }
            & \text{for } t \geq t_c
        \end{cases}
    \end{equation}

    Here, $t_c$ labels a critical time value, at which the behavior of the evolution changes.
    In \cite{wattis_2006}, it is be shown that this is actually the \textit{gelation time}.

    % }}}

% }}}
% Numerical Integration {{{
\section{Numerical Integration}

    Let us now take a look at how to use numerical integration of the discretized 
    Smoluchowski coagulation equation to arrive at an approximate solution for the temporal 
    evolution of the particle mass distribution function under the influence of dust particle 
    collisions. 

    % Definition of the Integration Mass Error {{{ 
    \subsection{Definition of the Integration Mass Error}

        To be able to evaluate a model implementation with regards to its stability properties,
        it makes sense to define a measure for the total mass error of the integration at 
        a certain point in time. \\

        Let $t_i$ and $t_f$ label the time at the start and end of the integration, respectively.
        The mass volume density's value at these two points in time can be written as:
        \begin{align}
            % TODO Define `t_i` and `t_f`. Note that `t_i = t_0`
            % TODO Use subscript or superscript for `i` and `f` ?
            \rho_\text{i}^d
                &:=\sum_{k=0}^{\mathcal N_m} m_k\cdot n_k(t=t_\text{i})\cdot\Delta m_k
            \\
            \rho_\text{f}^d
                &:=\sum_{k=0}^{\mathcal N_m} m_k\cdot n_k(t=t_\text{f})\cdot\Delta m_k
        \end{align}

        The total mass arror accumulated over the course of the integration is then given by
        \begin{equation}
            \label{eq:definition_of_integration_mass_error_stability}
            \Delta_\text{stab} =\frac{\rho_f^d - \rho_i^d}{\rho_i^d}
        \end{equation}

        % the relative mass error (more specifically: density 
        % error) then reads:}
        % To be able to compare different numerical schemes quantitatively, both with respect 
        % to their accuracy as well as stability properties, it makes sense to first define a measure 
        % of the total error of the integration. \\
        % As we already know from [\todo{ref}], the mass volume density can be calculcated from 
        % the particle mass distribution. \\ % TODO Rewrite this.

        % \begin{equation}
        %     \Delta\rho
        %         :=\frac{\rho_\text{f}-\rho_\text{i}}{\rho_\text{i}}
        % \end{equation}

        In reality, this mass error should be exactly equal to zero. Otherwise, mass conservation would not be given, since the movement of mass from one point in space to another is not included in our model. 
        The numerical integration introduces an error though, which ideally should be kept at machine precision. % TODO Rewrite this.

    % }}}
    % Explicit Euler Integration Scheme {{{ 
    \subsection{Explicit Euler Integration Scheme}
    \label{sec:explicit_euler}

        A naive approach for solving the Smoluchowski equation numerically is given by the simple 
        explicit Euler integration scheme, which we will briefly address here. Assuming a constant 
        time-step size $\Delta t\in\mathbb R_+$, the elapsed time after $m\in\mathbb N_0$ integration 
        steps can be expressed as 
        \begin{equation}
            t_m
                =t_0+m\cdot\Delta t,
        \end{equation}
        where $t_0$ is the time value at the start of the integration.\\
        
        Now, the goal is to find the corresponding values $n_k^m$ for the particle mass 
        distribution:
        \begin{equation}
            n_k^m
                :=n(m=m_k,\ t=t_m)
        \end{equation}
        
        We define the differences
        \begin{align}
            \Delta n_k
                &=n_k^m-n_k^{m-1}
            \ \ \ \text{and}\ \ \
            \\
            \Delta t
                &=t_m-t_{m-1}
        \end{align}
        which, in turn, can be used to define the 1st order forward finite difference
        \begin{equation}
            \text{FFD}
                =\frac{\Delta n_k}{\Delta t}
        \end{equation}
        
        The finite difference can now be used as an approximation for the temporal derivative 
        of $n$:
        \begin{equation}
            \pderiv{n_k}{t}
                \approx\text{FFD}
        \end{equation}
        
        Plugging in \cref{eq:discrete_smoluchowski_equation} and rearranging for $\Delta n_k$ 
        leads to
        \begin{equation}
            \Delta n_k
            =\Delta t\cdot
                \sum_{i=1}^{\mathcal N_m}\sum_{j=1}^{\mathcal N_m}
                K_{kij}\cdot n_i\cdot n_j
        \end{equation}
        
        This can now be used an approximate value for the change in $n_k$ from one time-step to 
        the next, i.e.
        \begin{equation}
            \begin{split}
                n_k^m
                    &=n_k^{m-1}+\Delta n_k
                \\
                    &=n_k^{m-1}+\Delta t\cdot
                        \sum_{i=1}^{\mathcal N_m}\sum_{j=1}^{\mathcal N_m}
                        K_{kij}\cdot n_i\cdot n_j
            \end{split}
        \end{equation}
        
        In order to ensure stability of the numerical integration, the step size needs to respect 
        the Courant-Friedrichs-Lewy (CFL) criterion, which provides an upper limit for the values 
        that $\Delta t$ can take on \cite{courant_friedrichs_lewy_1928}. \\

        A discussion regarding the formulation of the CFL criterion in the context of the 
        Smoluchowski coagulation equation is given by e.g. \cite{laibe_lombart_2021}.
        
        % \todo{Formulate CFL criterion for Smoluchowski equation.} \\
        % \todo{Plot $n(m,t)$ on a linear mais axis, with only coagulation.} \\
        % \todo{Plot $n(m,t)$ on a linear mass axis, with only fragmentation. (?)} \\
        % \todo{Plot $n(m,t)$ on a linear mass axis, with constant/linear kernel. (?)}

    % }}}
    % Implicit Radau Integration Scheme {{{ 
    \clearpage\subsection{Implicit Radau Integration Scheme}

        Since we would like to study the temporal evolution of the dust particle mass 
        distribution with particle masses ranging over multiple orders of magnitude, it makes 
        sense to use a logarithmic scaling for the mass axis discretization. \\

        Similarly, because we want to follow the mass distribution's temporal evolution
        over very long time-spans, the temporal axis will be discretized using a 
        logarithmic scaling as well. \\

        This means that the difference in time $\Delta t$ from a step to the next will not 
        stay constant over time, and instead satisfies the relationship 
        \begin{equation}
            \Delta t \sim t
        \end{equation}

        Due to this, we must be careful not to disrespect the CFL criterion, which places 
        an upper limit on the step size $\Delta t$. Choosing a larger value would lead to 
        a breakdown of the algorithm, since stability would no longer be assured. \\

        To deal with that problem, it makes sense to make use of an implicit integration scheme. \\

        For the studies made in the context of this thesis, here we chose to utilize 
        a 4th-order Radau integration scheme. This scheme belongs to a class of 
        implicit Runge-Kutta methods that can be used for solving initial value problems. \\

        An implementation of this algorithm is given in the `scipy` Python library. More 
        specifically, we make use of the `scipy.integrate.solve\_ivp` function.

        % Due to the wide range of orders of magnitude needed for describing 

        % We use an implicit 4th-order Radau integration scheme.

        % The Radau integration scheme is a numerical integration method 

        % \todo{Runge-Kutta scheme} \\
        % \todo{Here, 4th order (correct?)} \\

        % \todo{Built upon `scipy` Python library 
        % (more specifically, `scipy.integrate.solve\_ivp` function)} \\
        % \todo{Implemented by my supervisor Prof. Dr. Cornelis P. Dullemond [cite?]} \\

        % \todo{Accuracy} \\
        % \todo{Stability, Mass Conservation} \\

    % }}}

% }}}
% Numerical Integration (Plots) {{{ 
\section{Visualization of Integration Results}

    Having defined the disk model, the kernel matrix, and the integration scheme, now it 
    is the time to integrate the Smoluchowski coagulation equation. \\ 

    This is done for several different cases, namely: 
    \begin{enumerate}
        \item Pure coagulation (on a lin. mass axis)
        \item Pure coagulation (on a log. mass axis)
        \item Pure fragmentation (on a log. mass axis)
        \item Full model of coagulation \textit{and} fragmentation (on a log. mass axis)
    \end{enumerate}

    The results are displayed in the following sections.

    % Pure Coagulation on Linear Mass Axis {{{ 
    \subsection{Pure Coagulation on Linear Mass Axis}

        As an introduction, let us start with a very simple first example: The numerical 
        integration of the Smoluchowski coagulation equation is to be carried out making 
        use of the kernel matrix that we defined in 
        \cref{sec:coagulation_kernel_on_linear_mass_axis}, i.e. with only pure 
        hit-and-stick coagulation processes being included into the model.
        For the definition of the discretized mass axis, we make use of a linear scaling 
        here. For the integration, we use the simple forward Euler method that was 
        described in \cref{sec:explicit_euler}. \\

        For simplicity, the reaction rate coefficent is set to 1 here, as we're only 
        interesting in getting a qualitative first view on the evolution of the 
        mass distribution, which we initialize with all particles sharing an initial mass 
        value of $m_0 = 1\ \text{kg}$. The total mass volume density contained in the 
        distribution is set to $\rho^d=1\ \text{kg} \text{m}^{-3}$ as well. \\

        \textbf{Observation 1:} As we would expect in the case of pure coagulation, the
        distribution of particle masses shifts towards higher mass values as time progress. \\

        \textbf{Observation 2:} The height of the graph stays relatively constant over time. 
        This is due to the fact that here we are not plotting the number of particles per bin 
        against the mass values corresponding to each bin, but instead the fraction of the 
        total mass volume density in each bin. This representation thus gives an intuitive 
        first insight into the mass conservation properties of the algorithm. \\

        \textbf{Observation 3:} Though not shown here (since the plot would not look very
        interesting), the total mass contained in the distribution is conserved exactly here.
        The mass error is equal to zero, which stems from the more simple numerical 
        treatment that can be done when using a linearly scaled mass axis. This exact conservation 
        of mass will be lost once the switch to a logarithmic scaling is made. \\

        \textbf{Observation 4:} As described by the CFL criterion, the usage of an explicit 
        Euler algorithm will lead to a breakdown of the method when the time-steps become too 
        large. This can be observed when increasing the size of the time-step $\Delta t$,
        but is not yet the case in the displayed simulation run.

        \vfill

        \begin{figure}[h!]
            \makebox[\textwidth]{
                \includegraphics[width=\paperwidth]{112/euler_explicit_integration_on_linear_mass_axis.pdf}
            }
            \caption{ 
                Temporal evolution of the dust particle mass distribution under the 
                influence of only pure hit-and-stick coagulation. Here, the mass axis 
                was discretized making use of a linear scaling. \\ \
            }
        \end{figure}

        % \clearpage

    % }}}
    % Pure Coagulation on Logarithmic Mass Axis {{{ 
    \cleardoublepage\subsection{Pure Coagulation on Logarithmic Mass Axis}
    \label{sec:integration_results_pure_coagulation_on_log_mass_axis}

        As before, we start with all mass being concentrated in the lowest bin, i.e. all 
        particles possessing the same small mass value. As the simulation progresses, more 
        and more particle collsions occur. In this model, 100 \% of collisions lead to a 
        merging of the two particles. \\

        As such, the particles in the disk grow larger and larger, while the number of particles 
        becomes smaller and smaller. \\

        At some point, the upper boundary of the discretized mass axis is reached. This can be 
        seen on the right side of \cref{fig:temporal_evolution_of_mass_dist_for_pure_coagulation},
        where more and more particles gather in the highest-mass bin, and a peak is formed. \\

        The stability properties of numerical integration with only pure coagulation
        are quite good. As we can see in 
        \cref{fig:temporal_evolution_of_stability_mass_error_for_pure_coagulation},
        the unitless relative mass error stays consistently on the order of $10^{-15}$. \\
        
        As such, mass conservation is assured down to machine precision here.

        \vfill
        \begin{figure}[h!]
            \makebox[\textwidth]{
                \includegraphics[width=\paperwidth]{103/mass_error coag=True frag=False m0=0.pdf}
            }
            \caption{
                % For the definition of the error, see 
                % \cref{eq:definition_of_integration_mass_error_stability}.
                Relative mass error (stability error) 
                $\Delta_\text{stab}(t) = \frac{ \rho_d(t) - \rho_d(t=0) }{ \rho_d(t=0) }$ 
                of the numerical integration as a function of time. Here,
                only the process of pure coagulation is included in the model. \\
            }
            \label{fig:temporal_evolution_of_stability_mass_error_for_pure_coagulation}
        \end{figure} 
        \clearpage
        \begin{figure}[h!]
            \makebox[\textwidth]{
                \includegraphics[width=\paperwidth]{103/mass_distr coag=True frag=False m0=0.pdf}
            }
            \caption{
                Temporal evolution of the dust particle mass distribution under the 
                influence of only pure coagulation. 
            }
            \label{fig:temporal_evolution_of_mass_dist_for_pure_coagulation}
        \end{figure} 
        \vfill
        \begin{figure}[h!]
            \makebox[\textwidth]{
                \includegraphics[width=\paperwidth]{103/mass_deriv coag=True frag=False m0=0.pdf}
            }
            \caption{
                Root-mean-squared (RMS) temporal derivative of the dust particle mass distribution 
                as a function of time. Here, only the process of pure coagulation is 
                included into the model. \\
            }
            \label{fig:rms_temporal_derivative_of_mass_dist_for_pure_coagulation}
        \end{figure} 

    % }}}
    % Pure Fragmentation on Linear Mass Axis {{{ 
    % \clearpage\subsection{Pure Fragmentation}

    %     \vfill
    %     \begin{figure}[h!]
    %         \makebox[\textwidth]{
    %             \includegraphics[width=\paperwidth]{103/mass_distr coag=False frag=True m0=0.pdf}
    %         }
    %         \caption{
    %             Temporal evolution of the dust particle mass distribution under the 
    %             influence of pure fragmentation. \\
    %             \ \\
    %             \todo{$\mathcal N_m = XXX$} 
    %         }
    %     \end{figure} 
    %     \clearpage
    %     \begin{figure}[h!]
    %         \makebox[\textwidth]{
    %             \includegraphics[width=\paperwidth]{103/mass_error coag=False frag=True m0=0.pdf}
    %         }
    %         \caption{
    %             Temporal Evolution of the Numerical Integration's Mass Error. For the error 
    %             definition see [\todo{cite}]. In the visualized simulation the model 
    %             includes only particle fragmentation.
    %         }
    %     \end{figure} 
    %     \begin{figure}[h!]
    %         \makebox[\textwidth]{
    %             \includegraphics[width=\paperwidth]{103/mass_deriv coag=False frag=True m0=0.pdf}
    %         }
    %         \caption{
    %             Root Mean Squared Temporal Derivative as a Function of Time
    %             with only Fragmentation enabled.
    %         }
    %     \end{figure} 

    % }}}
    % Pure Fragmentation on Logarithmic Mass Axis {{{ 
    \cleardoublepage\subsection{Pure Fragmentation on Logarithmic Mass Axis}
    \label{sec:pure_fragmentation_on_log_mass_axis}

        Next, we will consider the scenario in which particle fragmentation is included into 
        the model, but coagulation is not. In this case, it makes sense to not initialize the 
        particle mass distribution as a Dirac $\delta$-function centered on the lowest-mass 
        bin, but choose a larger initial mass for the particles instead.
        Otherwise, we would not see anything happen at all in 
        \cref{fig:temporal_evolution_of_mass_dist_for_pure_coagulation}, where the particle 
        mass distribution is shown at various points in time. \\

        The behavior is what one might expect. Initially, we have larger particles, which 
        are shattered on collision with another particle, such that the mass is 
        redistributed into the lower-mass bins. \\

        As we can see in
        \cref{fig:temporal_evolution_of_stability_mass_error_for_pure_fragmentation},
        the inclusion of fragmentation processes into the model leads to the 
        introduction of a mass error, which increases with time. \\

        If we were to run the simulation for longer time-spans, this error would 
        continue to grow, without any boundary in sight. This can be explained by 
        the utilization of a logarithmically scaled mass axis. \\

        Even if the mass error per collision is just a tiny number, it can become 
        arbitrarily large, as it is amplified by the size of the time-step $\Delta t$,
        which itself increases with time.

        \vfill
        \begin{figure}[h!]
            \makebox[\textwidth]{
                \includegraphics[width=\paperwidth]{103/mass_error coag=False frag=True m0=25.pdf}
            }
            \caption{
                % For the definition of the error, see 
                % \cref{eq:definition_of_integration_mass_error_stability}. 
                Relative mass error (stability error) 
                $\Delta_\text{stab}(t) = \frac{ \rho_d(t) - \rho_d(t=0) }{ \rho_d(t=0) }$ 
                of the numerical integration as a function of time. Here,
                only the process of pure fragmentation is included in the model. \\
            }
            \label{fig:temporal_evolution_of_stability_mass_error_for_pure_fragmentation}
        \end{figure} 
        \clearpage
        \begin{figure}[h!]
            \makebox[\textwidth]{
                \includegraphics[width=\paperwidth]{103/mass_distr coag=False frag=True m0=25.pdf}
            }
            \caption{
                Temporal evolution of the dust particle mass distribution under the 
                influence of only pure fragmentation.
                Here, the initial mass distribution is centered 
                not on the first bin like in the other runs, but on one corresponding a 
                higher mass value.
            }
            \label{fig:temporal_evolution_of_mass_dist_for_pure_fragmentation}
        \end{figure} 
        \vfill
        \begin{figure}[h!]
            \makebox[\textwidth]{
                \includegraphics[width=\paperwidth]{103/mass_deriv coag=False frag=True m0=25.pdf}
            }
            \caption{
                Root-mean-squared (RMS) temporal derivative of the dust particle mass distribution 
                as a function of time. Here, only the process of fragmentation is included into 
                the model. \\
            }
            \label{fig:rms_temporal_derivative_of_mass_dist_for_pure_fragmentation}
        \end{figure} 

    % }}}
    % Full Model {{{ 
    \cleardoublepage\subsection{Full Model}

        Let us now take a look at the full model, where the processes of both coagulation and 
        fragmentation are included into the kernel matrix. The temporal evolution of the dust 
        particle mass distribution is displayed in
        \cref{fig:temporal_evolution_of_mass_dist_for_full_model}. \\

        At the beginning, this evolution behaves very similarly to what we saw in the case of 
        pure hit-and-stick coagulation in
        \cref{sec:integration_results_pure_coagulation_on_log_mass_axis}.

        As a consequence of collisions between the initially very low-mass particles, 
        larger bodies are formed via the merging of smaller ones. As such, the low-mass bins 
        are emptied, and the mass contained within them is transferred into bins encoding 
        more massive particles. \\

        As the coagulation process goes on, larger and larger particles form. Eventually,
        some of them may become large enough that the relative velocity between them and 
        some of the other particles surpasses the fragmentation threshold velocity 
        $v_\text{frag}$. \\

        Here, the observed behavior changes. After having emptied by the sticking process, 
        low-mass bins get filled up again. As associative and dissociative processes work 
        against each other, at some point the particle mass distribution function reaches 
        an equilibrium. \\

        When the equilibrium is reached, the temporal derivative of the mass distribution 
        function goes to zero. This can be seen visualized in 
        \cref{fig:rms_temporal_derivative_of_mass_dist_for_full_model}, where we plot 
        the root-mean-square of the mass distribution's temporal derivative vector as a 
        function of time. \\

        In addition, the relative mass error of the integration that we defined in 
        \cref{eq:definition_of_integration_mass_error_stability} is plotted in 
        \cref{fig:temporal_evolution_of_stability_mass_error_for_full_model}. \\

        The increase of the error towards the end of the time-span over which we 
        integrate is introduced because of the inclusion of dust particle 
        fragmentation processes into the kernel, as was discussed in
        \cref{sec:pure_fragmentation_on_log_mass_axis}.

        \vfill
        \begin{figure}[h!]
            \makebox[\textwidth]{
                \includegraphics[width=\paperwidth]{103/mass_error coag=True frag=True m0=0.pdf}
            }
            \caption{
                % Temporal Evolution of the Numerical Integration's Mass Error. For the error 
                % definition see [\todo{cite}]. In the visualized simulation the full model 
                % containing both coagulation and fragmentation processes is included.
                Relative mass error (stability error) 
                $\Delta_\text{stab}(t) = \frac{ \rho_d(t) - \rho_d(t=0) }{ \rho_d(t=0) }$ 
                of the numerical integration as a function of time. Here,
                both coagulation and fragmentation processes are 
                included in the model. \\
                % For the definition of the error, see 
                % \cref{eq:definition_of_integration_mass_error_stability}.
            }
            \label{fig:temporal_evolution_of_stability_mass_error_for_full_model}
        \end{figure} 
        \clearpage
        \begin{figure}[h!]
            \makebox[\textwidth]{
                \includegraphics[width=\paperwidth]{103/mass_distr coag=True frag=True m0=0.pdf}
            }
            \caption{
                Temporal evolution of the dust particle mass distribution under the 
                influence of both coagulation and fragmentation processes.
            }
            \label{fig:temporal_evolution_of_mass_dist_for_full_model}
        \end{figure} 
        \vfill
        \begin{figure}[h!]
            \makebox[\textwidth]{
                \includegraphics[width=\paperwidth]{103/mass_deriv coag=True frag=True m0=0.pdf}
            }
            \caption{
                Root-mean-squared (RMS) temporal derivative of the dust particle mass distribution 
                as a function of time. Here, both the process of coagulation and fragmentation is 
                included into the model.
            }
            \label{fig:rms_temporal_derivative_of_mass_dist_for_full_model}
        \end{figure} 

    % }}}
    % Old {{{ 

        % \clearpage
        % \todo{Plot $n(m,t)$ on log. mass/time axis, with only coagulation.} \\
        % \clearpage
        % \todo{Plot $n(m,t)$ on log. mass/time axis, with only fragmentation.} \\
        % \clearpage
        % \todo{Plot $n(m,t)$ on log. mass/time axis, with both coagulation \& "naive" fragmentation.}
        % \\
        % \clearpage
        % \todo{Plot $n(m,t)$ on log. mass/time axis, with both coagulation \& MRN fragmentation.} \\

        % TODO Add these plots to chapters about kernel definition?
        % \clearpage
        % \todo{Plot temporal evolution of mass distribution function for pure coagulation 
        % (on log. axis)}
        % \clearpage
        % \todo{Plot temporal evolution of mass distribution function for only fragmentation 
        % (on log. axis)}

    % }}}

% }}}
