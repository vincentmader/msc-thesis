% \clearpage

% \begin{figure}[h!]
%     \makebox[\textwidth]{
%         \includegraphics[width=\linewidth]{dust_fragmentation.png}
%     }
%     \caption{}
% \end{figure}

\vfill

\begin{figure}[h!]
    \centering
    % \begin{minipage}{.5\linewidth}
    %   \centering
    %   \subfloat[]{
    %     \label{:a}
    %     \includegraphics[width=\linewidth]{24/coagulation_probability_from_cutoff_velocity.pdf}
    %   }
    % \end{minipage}%
    \begin{minipage}{.5\linewidth}
      \centering
      \subfloat[Frag. Prob. Assuming A Hard Decision Boundary]{
        \label{:b}
        \includegraphics[width=\linewidth]{24/fragmentation_probability_from_cutoff_velocity.pdf}
      }
    \end{minipage}%
    % \begin{minipage}{.5\linewidth}
    %   \centering
    %   \subfloat[]{
    %     \label{:a}
    %     \includegraphics[width=\linewidth]{24/coagulation_probability_from_Maxwell-Boltzmann.pdf}
    %   }
    % \end{minipage}%
    \begin{minipage}{.5\linewidth}
      \centering
      \subfloat[Correction Using Maxwell-Boltzmann Distribution]{
        \label{:b}
        \includegraphics[width=\linewidth]{24/fragmentation_probability_from_Maxwell-Boltzmann.pdf}
      }
    \end{minipage}
    \caption{
        Dust particle fragmentation probability $P_{ij}^\text{frag}$.
        % \todo{Why not a perfect square in 1st case?}
        \\ % Note: These newlines are for assuring the same vertical position 
        \  %       for this plot and the other ones (dv_rel_tot, dv_rel multi + coll)
    }
    \label{fig:fragmentation_probability}
\end{figure}
