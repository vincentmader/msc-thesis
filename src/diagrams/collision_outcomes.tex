\vfill

\newcommand{\particle}[3]{%
    \ifcase#1
        % Zero
    \or
        \draw[fill={black!0!white}] (#2          , #3         ) circle (\R);
    \or
        \draw[fill={black!0!white}] (#2 - 0.65*\R, #3 + 0.30*\R) circle (\R);
        \draw[fill={black!0!white}] (#2 + 0.65*\R, #3 - 0.30*\R) circle (\R);
    \or
        \draw[fill={black!0!white}] (#2 - 0.70*\R, #3 - 0.50*\R) circle (\R);
        \draw[fill={black!0!white}] (#2 + 0.00*\R, #3 + 0.50*\R) circle (\R);
        \draw[fill={black!0!white}] (#2 + 0.70*\R, #3 - 0.50*\R) circle (\R);
    \or
        \draw[fill={black!0!white}] (#2 + 0.65*\R, #3 + 0.65*\R) circle (\R);
        \draw[fill={black!0!white}] (#2 - 0.65*\R, #3 - 0.65*\R) circle (\R);
        \draw[fill={black!0!white}] (#2 + 0.65*\R, #3 - 0.65*\R) circle (\R);
        \draw[fill={black!0!white}] (#2 - 0.65*\R, #3 + 0.65*\R) circle (\R);
    \or
        \draw[fill={black!0!white}] (#2 + 0.85*\R, #3 + 0.85*\R) circle (\R);
        \draw[fill={black!0!white}] (#2 - 0.85*\R, #3 - 0.85*\R) circle (\R);
        \draw[fill={black!0!white}] (#2 + 0.85*\R, #3 - 0.85*\R) circle (\R);
        \draw[fill={black!0!white}] (#2 - 0.85*\R, #3 + 0.85*\R) circle (\R);
        \draw[fill={black!0!white}] (#2 +       0, #3 +       0) circle (\R);
    \fi
}

\newcommand{\bang}[1]{
    \node[shape=starburst, starburst points=#1, starburst point height=0.5cm, 
        draw, black, fill=white,
        minimum width=1.5cm, minimum height=1.5cm,
    ] {};
}

\newcommand{\illustrateCoagulation}[0]{
    \begin{tikzpicture}

        \def\R{0.2}
        \def\P{1.1}

        \def\Xa{-1*\P}
        \def\Ya{+3*\P}
        \def\Xb{+1*\P}
        \def\Yb{+3*\P}
        \def\Xc{+0*\P}
        \def\Yc{-3*\P}

        \begin{scope}[decoration={
            markings,
            mark=at position 0.5 with {\arrow{latex}}}
        ]
            \draw[postaction={decorate}] (\Xa, \Ya) -- (  0,   0);
            \draw[postaction={decorate}] (\Xb, \Yb) -- (  0,   0);
            \draw[postaction={decorate}] (  0,   0) -- (\Xc, \Yc);
        \end{scope}

        \particle{3}{\Xa}{\Ya}
        \particle{1}{\Xb}{\Yb}
        \particle{4}{\Xc}{\Yc}

        \bang{10}

        \draw[draw=white,fill=white] (0, -4*\P) circle (\R);

    \end{tikzpicture}
}

\newcommand{\illustrateFragmentation}[0]{
    \begin{tikzpicture}

        \def\R{0.2}
        \def\P{1.1}

        \def\Xa{-1.0*\P}
        \def\Ya{+3.0*\P}
        \def\Xb{+1.0*\P}
        \def\Yb{+3.0*\P}
        \def\Xc{-2.0*\P}
        \def\Yc{-2.8*\P}
        \def\Xd{-1.0*\P}
        \def\Yd{-3.1*\P}
        \def\Xe{+0.0*\P}
        \def\Ye{-3.3*\P}
        \def\Xf{+1.0*\P}
        \def\Yf{-3.1*\P}
        \def\Xg{+2.0*\P}
        \def\Yg{-2.8*\P}

        \begin{scope}[decoration={
            markings,
            mark=at position 0.5 with {\arrow{latex}}}
        ]
            \draw[postaction={decorate}] (\Xa, \Ya) -- (  0,   0);
            \draw[postaction={decorate}] (\Xb, \Yb) -- (  0,   0);
            \draw[postaction={decorate}] (  0,   0) -- (\Xc, \Yc);
            \draw[postaction={decorate}] (  0,   0) -- (\Xd, \Yd);
            \draw[postaction={decorate}] (  0,   0) -- (\Xe, \Ye);
            \draw[postaction={decorate}] (  0,   0) -- (\Xf, \Yf);
            \draw[postaction={decorate}] (  0,   0) -- (\Xg, \Yg);
        \end{scope}

        \particle{5}{\Xa}{\Ya}
        \particle{3}{\Xb}{\Yb}
        \particle{1}{\Xc}{\Yc}
        \particle{2}{\Xd}{\Yd}
        \particle{3}{\Xe}{\Ye}
        \particle{1}{\Xf}{\Yf}
        \particle{1}{\Xg}{\Yg}

        \bang{9}

        \draw[draw=white,fill=white] (0, -4*\P) circle (\R);

    \end{tikzpicture}
}

\newcommand{\illustrateBouncing}[0]{
    \begin{tikzpicture}

        \def\R{0.2}
        \def\P{1.1}

        \def\Xa{-1*\P}
        \def\Ya{+3*\P}
        \def\Xb{+1*\P}
        \def\Yb{+3*\P}
        \def\Xc{-1*\P}
        \def\Yc{-3*\P}
        \def\Xd{+1*\P}
        \def\Yd{-3*\P}

        \begin{scope}[decoration={
            markings,
            mark=at position 0.5 with {\arrow{latex}}}
        ]
            \draw[postaction={decorate}] (\Xa, \Ya) -- (  0,   0);
            \draw[postaction={decorate}] (\Xb, \Yb) -- (  0,   0);
            \draw[postaction={decorate}] (  0,   0) -- (\Xc, \Yc);
            \draw[postaction={decorate}] (  0,   0) -- (\Xd, \Yd);
        \end{scope}

        \particle{3}{\Xa}{\Ya}
        \particle{1}{\Xb}{\Yb}
        \particle{3}{\Xc}{\Yc}
        \particle{1}{\Xd}{\Yd}

        \bang{13}

        \draw[draw=white,fill=white] (0, -4*\P) circle (\R);

    \end{tikzpicture}
}

\begin{figure}[h!]
    \centering
    \makebox[\textwidth]{
        \begin{minipage}{.33\paperwidth}
            \centering
          	\subfloat[Coagulation]{  % TODO Rename?
                \label{:a}
                \illustrateCoagulation
          	}
        \end{minipage}%
        \begin{minipage}{.33\paperwidth}
            \centering
          	\subfloat[Fragmentation]{
                \label{:b}
                \illustrateFragmentation
          	}
        \end{minipage}
        \begin{minipage}{.33\paperwidth}
            \centering
          	\subfloat[Bouncing]{
                \label{:b}
                \illustrateBouncing
          	}
        \end{minipage}%
    }
    \caption{
        Illustration of dust particle collision outcomes
        (a) Coagulation: Two particles collide and merge into a single new one that
            carries the combined mass of the initial particles.
        (b) Fragmentation: The colliding particles are destroyed during the collision.
            The total mass is subsequently distributed onto a wide range of differently-sized,
            newly-created particles.
        (c) Bouncing: Two particles collide, but no mass is exchanged between them.
    }
    \label{fig:collision_outcomes}
\end{figure} 
