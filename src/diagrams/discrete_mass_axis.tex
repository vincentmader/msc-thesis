\vfill

\begin{figure}[h!]
    \begin{center}
        \begin{tikzpicture}
            \def\N{5}      % This is the number of cells drawn (to the left of "..." separator).
            \def\M{6}      % This is the number of boundary arrows drawn (left of "...").
            \def\W{1.5}    % This is a cell's width.
            \def\H{\W}     % This is a cell's height.
            \def\L{\W/4}   % This is an arrow's length.
            \def\P{\W/8}   % This is the padding between arrow & cell.
            \def\R{\W/32}  % This is the padding between arrow & text.

            % Draw continuous mass axis.
            \draw [|-to](0, 2.5*\H) -- (\N*\W+4*\W, 2.5*\H);
            \draw [-](\W, 2.45*\H) -- (\W, 2.55*\H);
            \draw [-](\N*\W+3*\W, 2.45*\H) -- (\N*\W+3*\W, 2.55*\H);
            \node[] at (-2*\P, 2.5*\H) {$0$};
            \node[] at (\N*\W+4*\W+2*\P, 2.5*\H) {$m$};

            % Draw arrow from continuous to discretized mass axis.
            % \node[] at (\N*\W/2+1.5*\W, 2.25*\H) {$\Downarrow$};

            % Draw cells...
            % ...to the left of "..." separator.
            \foreach \x in {1, ..., \N} {
                \draw[draw=black] (\x*\W, 0) rectangle ++(\W, \H);
            }
            % ...to the right of "..." separator.
            \draw[draw=black] (\N*\W+2*\W, 0) rectangle ++(\W, \H);

            % Draw crosses...
            % ...to the left of "..." separator.
            \foreach \x in {1, ..., \N} {
                \draw (\x*\W+\W/2, \H/2) node {\tiny x};
            }
            % ...to the right of "..." separator.
            \draw (\N*\W+2.5*\W, \H/2) node {\tiny x};

            % Draw "..." separator.
            \node[] at (\N*\W+1.5*\W, \H/2) {...};

            % Draw arrows labeling cell boundaries.
            \foreach \x in {1, ..., \M} {
                \draw [-to](\x*\W, \H+\L+\P) -- (\x*\W, \H+\P);
                \node[] at (\x*\W, \H+\L+\P+\L+\R) {$m_\x^\text{b}$};
            }
            \draw [-to](\N*\W+2*\W, \H+\L+\P) -- (\N*\W+2*\W, \H+\P);
            \draw [-to](\N*\W+3*\W, \H+\L+\P) -- (\N*\W+3*\W, \H+\P);
            \node[] at (\N*\W+2*\W, \H+\L+\P+\L+\R) {$m_{\mathcal N_m}^\text{b}$};
            \node[] at (\N*\W+3*\W, \H+\L+\P+\L+\R) {$m_{\mathcal N_m+1}^\text{b}$};

            % Draw arrows labeling cell centers.
            \foreach \x in {1, ..., \N} {
                \draw [-to](\x*\W+\W/2, 0-\L-\P) -- (\x*\W+\W/2, 0-\P);
                \node[] at (\x*\W+\W/2, 0-\L-\P-\L-\R) {$m_\x^\text{c}$};
            }
            \draw [-to](\N*\W+2.5*\W, 0-\L-\P) -- (\N*\W+2.5*\W, 0-\P);
            \node[] at (\N*\W+2.5*\W, 0-\L-\P-\L-\R) {$m_{\mathcal N_m}^\text{c}$};

            % Draw labels for `m_min` and `m_max`.
            \node[] at (\W, \H+\L+\P+\L+3*\R+\L+\L) {$m_\text{min}$};
            \node[] at (\N*\W+3*\W, \H+\L+\P+\L+3*\R+\L+\L) {$m_\text{max}$};
            \node[] at (\W, \H+\L+\P+\L+2*\R+\L) {$=$};
            \node[] at (\N*\W+3*\W, \H+\L+\P+\L+2*\R+\L) {$=$};
        \end{tikzpicture}
    \end{center}
    \caption{Schematic illustration of the discretized mass axis. After having defined a minimum 
        value $m_\text{min}$ and a maximum value $m_\text{max}$, the interval between these two 
        values is divided evenly into $\mathcal N_m$ bins. To do this, we first define the mass
        values at the bin boundaries and, from that, the mass values at the bin centers. 
        This can be done using either a linear or a logarithmic scaling (for details, see section
        [cite linear] and [cite logarithmic], respectively). The discretization of the 
        axes for both time $t$ and distance $r$ from the star is done completely analogously.}
    \label{fig:continuous_and_discrete_mass_axis}
\end{figure}
