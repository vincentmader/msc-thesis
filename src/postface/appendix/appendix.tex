\section{"Code / Repository"}

Git repository:
\url{https://github.com/vincentmader/msc-thesis} % TODO

\section{Abbreviations}

    \begin{table}[h!]
        % \begin{center}
        % \caption{}
        \begin{tabular}{|l|l|}
            \hline
            \textbf{Abbreviation}   & \textbf{Meaning} \\
            \hline
            ISM                     & Interstellar Medium \\
            \hline
            MRN                     & Mathis-Rumpl-Nordsieck \\
            \hline
            PPD                     & Proto-Planetary Disk \\
            \hline
            RMS                     & Root Mean Square \\
            \hline
            SI                      & Système International (International System of Units) \\
            \hline                  % ^ TODO Do I use this? -> Specify unit system in introduction.
        \end{tabular}
        % \end{center}
    \end{table}

\section{Variables}

    \begin{table}[h!]
        % \begin{center}
        % \caption{}
        \begin{tabular}{|l|l|l|}
            \hline
            \textbf{Symbol}     & \textbf{SI Units}     & \textbf{Description}
            \\ \hline
            $K(m,m',m'')$      & m$^3$ s$^{-1}$        & Kernel Function
            \\ \hline
            $K_{kij}$           & kg m$^3$ s$^{-1}$     & Kernel Matrix, equivalent to 
                                                          $K(m_k,m_i,m_j) \cdot \Delta m_k$
            \\ \hline
            $L_\odot$           & W                     & Solar Luminosity 
            \\ \hline
            $L_*$               & W                     & Stellar Luminosity 
            \\ \hline
            $M_\odot$           & kg                    & Solar Mass 
            \\ \hline
            $M_*$               & kg                    & Stellar Mass 
            \\ \hline
            $P_\text{bounce}$   & 1                     & Bouncing Probability
            \\ \hline
            $P_\text{frag}$     & 1                     & Fragmentation Probability
            \\ \hline
            $P_\text{coag}$     & 1                     & Coagulation Probability
            \\ \hline
            $R^\text{coll}_{ij}$& m$^3$ s$^{-1}$        & Dust Particle Collision Rate,
                                                          normalized to dust density
            \\ \hline
            $\text{St}$         & 1                     & Stokes Number 
            \\ \hline
            $\rho_s$            & kg m$^{-3}$           & Solid Density of Dust Particles 
            \\ \hline
            $\rho_g$            & kg m$^{-3}$           & Mass Volume Density of Gas Particles 
            \\ \hline
            $\rho_d$            & kg m$^{-3}$           & Mass Volume Density of Dust Particles 
            \\ \hline
            $\rho_s$            & 1                     & Solid Density of Dust Particles 
            \\ \hline
            $\sigma_{ij}$       & m$^2$                 & Dust Particle Collision Cross Section,
            \\ \hline
            $\rho_\text{sample}$& kg m$^{-3}$           & Monte Carlo Sampling Density
            \\ \hline
            $\Delta v_{ij}$     & m s$^{-1}$            & Relative Dust Particle Velocity
            \\ \hline
        \end{tabular}
        % \end{center}
    \end{table}

    \begin{align}
        \rho_g &= \rho_\text{gas} \\
        \rho_d &= \rho_\text{dust} \\
        \rho_s &= \rho_\text{solid} \\
    \end{align}
    \begin{align}
        \rho_\text{sample} \\
        \rho_\text{sample}^\text{total} 
    \end{align}
    
\section{Assumptions}

    \begin{itemize}
        \item The nebular hypothesis is accurate. 
        \item The disk formed via gravitational collapse of a giant interstellar gas cloud.
        \item $q_\text{disk-to-star}=0.01$.
        \item $q_\text{dust-to-gas}=0.01$.
        \item Gas mean particle mass $m_g=\mu\cdot m_p$ with $\mu:=2.3$.
        \item The star behaves like our own Solar System's Sun.
        \item The star radiates perfectly "isotropic".
        \item The disk has perfect radial symmetry.
        \item The disk is "much more wide than tall".
        \item The gas in the disk behaves "isothermal".
        \item The gas in the disk is in hydrostatic equilibrium.
        \item Dust particles are perfectly spherical.
        \item Dust particles all have density $\rho_s$.
    \end{itemize}

% \section{\todo{Move}}

%     The Maxwell-Boltzmann distribution function is given by 
%     \begin{equation}
%         f(v)
%         =\left(\frac{m}{2\pi\cdot k_BT}\right)^{3/2}
%         \cdot4\pi v^2\cdot\exp\left(-\frac{mv^2}{2k_BT}\right)
%     \end{equation}

%     The most probable dust particle speed can easily be determined by 
%     setting the first derivative of $f$ with respect to $v$ equal to 0:
%     \begin{equation}
%         0\overset{!}{=}\deriv{f}{v}
%     \end{equation}
%     Rearanging for the velocity leads to the expression
%     \begin{equation}
%         \hat{v}=\sqrt{\frac{2k_BT}{m}}
%     \end{equation}

%     The RMS speed can be calculated via
%     \begin{align}
%         v_\text{RMS}
%         =\sqrt{\overline{v^2}}
%         &=\sqrt{\int\limits_0^\infty v^2\cdot f(v)\ \text{d}v}\\
%         &=\sqrt{\frac{3k_BT}{m}}
%     \end{align}

%     We're not actually interested in the most probable particle speed.
%     Instead, what we want to look at is the arithmetic mean particle speed.
%     It can be calculated from the integral 
%     \begin{equation}
%         \bar{v}
%         =\int\limits_0^\infty v\cdot f(v)\ \text{d}v
%     \end{equation}
%     and then be expressed as
%     \begin{equation}
%         \bar{v}
%         =\sqrt{\frac{8k_BT}{\pi\cdot m}}
%     \end{equation}
%     \todo{Now with reduced particle mass: See above}
