\newpage
\underline{Abbreviations}:
\begin{table}[h!]
    % \begin{center}
    % \caption{}
    \begin{tabular}{|l|l|}
        \hline
        \textbf{Abbreviation}   & \textbf{Meaning} \\
        \hline
        ISM                     & Interstellar Medium \\
        \hline
        MRN                     & Mathis-Rumpl-Nordsieck \\
        \hline
        PPD                     & Proto-Planetary Disk \\
        \hline
        RMS                     & Root Mean Square \\
        \hline
        SI                      & Système International (International System of Units) \\
        \hline                  % ^ TODO Do I use this? -> Specify unit system in introduction.
    \end{tabular}
    % \end{center}
\end{table} \ \\ 

\underline{Variables}:
\begin{table}[h!]
    % \begin{center}
    % \caption{}
    \begin{tabular}{|l|l|l|}
        \hline
        \textbf{Symbol}     & \textbf{SI Units}     & \textbf{Description}
        \\ \hline
        $L_\odot$           & W                     & Solar Luminosity 
        \\ \hline
        $L_*$               & W                     & Stellar Luminosity 
        \\ \hline
        $M_\odot$           & kg                    & Solar Mass 
        \\ \hline
        $M_*$               & kg                    & Stellar Mass 
        \\ \hline
        $P_\text{bounce}$   & -                     & Bouncing Probability
        \\ \hline
        $P_\text{frag}$     & -                     & Fragmentation Probability
        \\ \hline
        $P_\text{coag}$     & -                     & Coagulation Probability
        \\ \hline
        $\text{St}$         & -                     & Stokes Number 
        \\ \hline
        $\rho_d$            & kg m$^{-3}$           & Mass Volume Density of Dust Particles 
        \\ \hline
        $\rho_g$            & kg m$^{-3}$           & Mass Volume Density of Gas Particles 
        \\ \hline
    \end{tabular}
    % \end{center}
\end{table} \ \\ 

\underline{\smash{Assumptions}}:
\begin{itemize}
    \item The nebular hypothesis is accurate. 
    \item The disk formed via gravitational collapse of a giant interstellar gas cloud.
    \item $q_\text{disk-to-star}=0.01$.
    \item $q_\text{dust-to-gas}=0.01$.
    \item Gas mean particle mass $m_g=2.3\cdot m_p$.
    \item The star behaves like our own Solar System's Sun.
    \item The star radiates perfectly "isotropic".
    \item The disk has perfect radial symmetry.
    \item The disk is "much more wide than tall".
    \item The gas in the disk behaves "isothermal".
    \item The gas in the disk is in hydrostatic equilibrium.
    \item Dust particles are perfectly spherical.
    \item Dust particles all have density $\rho_d$.
\end{itemize}
