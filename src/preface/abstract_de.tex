\cleardoublepage\chapter*{Zusammenfassung}

\thispagestyle{NoHeader}

\begin{otherlanguage}{german}

\textit{Kontext:}
Die Smoluchowski-Koagulationsgleichung stellt ein Modell zur Verfügung, das zur Bestimmung der 
zeitlichen Entwicklung der Verteilung von Staubteilchenmassen in protoplanetaren Scheiben verwendet 
werden kann. Wenn sowohl Koagulations- als auch Fragmentierungsprozesse berücksichtigt werden, dann 
erfordert die numerische Integration dieser Gleichung die Auswertung eines Doppelintegrals über die 
Reaktionskernel-Matrix, wodurch dieser Ansatz aus numerischer Sicht recht aufwendig wird.
\\ \ \\
\textit{Ziele:}
Das Ziel dieser Arbeit ist es zu untersuchen, ob sich die Nutzung von stochastischem 
Monte Carlo Sampling anbietet, um die numerischen Kosten der Integration zu senken.
Genauer gesagt ist hier die Idee, nur die relevantesten Staubteilchen-Kollisionen 
bei der Aufsummierung mit zu berücksichtigen.
Relevanz wird hierbei durch den Einfluss einer gegebenen Kollision auf die Massenverteilung der 
Staubteilchen bestimmt.
\\ \ \\
\textit{Methoden:}
Nach der Definition eines vereinfachten Scheibenmodells bestimmen wir die Reaktionsrate sowie die 
Reaktionskernel-Matrix für sowohl Koagulations- als auch Fragmentierungsprozesse.
Die Integration der Smoluchowski-Gleichung wird mithilfe eines impliziten
Radau-Integrationsverfahrens vierter Ordnung durchgeführt. 
Mit einer geeigneten Wahrscheinlichkeitsverteilung wird dann das stochastische Sampling der 
Kernel-Matrix vollzogen. Parameterstudien der Sampling-Dichte werden sowohl für den 
einfacheren Fall der reinen Koagulation, als auch unter Einbezug von Fragmentierung gemacht.
\\ \ \\
\textit{Ergebnisse:}
Das allgemeine Verhalten der zeitlichen Entwicklung der Massenverteilung kann reproduziert werden, 
auch wenn weite Teile des Kernels vernachlässigt werden. Wenn sowohl Koagulation als auch 
Fragmentierung einbezogen sind, nähert sich die numerische Lösung nach einigen hundert Jahren dem
erwarteten Gleichgewichtszustand an. Es muss jedoch angemerkt werden, dass aufgrund statistischer
Schwankungen, die aus der Sampling-Methode resultieren, die zeitliche Ableitung der Verteilung weit 
über Null bleibt. Dies, sowie die effektiv niedrigeren Kollisionsraten, führt zu einem Verlust an 
Genauigkeit. Die Stabilität des Algorithmus in Bezug auf Massenerhaltung konnte bis auf $10^{-12}$ 
sichergestellt werden.
\\ \ \\
\textit{Schlüsse:}
Ob es sinnvoll ist, den in dieser Thesis erprobten Ansatz des Kernel Samplings zu verwenden, 
hängt vermutlich von den spezifischen Anforderungen an die Rechenressourcen sowie der gewünschten 
Genauigkeit ab, da ein Kompromiss zwischen den beiden gefunden werden muss. 
Wenn bei der Modellierung der Staubteilchen mehr als nur ein Attribut berücksichtigt wird 
(e.g. Masse und Porosität), dann könnte die Anwendung dieser Methode mehr Sinn machen.
Es kann davon ausgegangen werden, dass der Anteil an vernachlässigbaren Einträgen in der
Kernel-Matrix mit der Dimensionalität des Problems schneller anwächst als die Zahl der relevanten
Einträge. Für detailliertere Studien, die zusätzliche Partikeleigenschaften einschließen, 
könnte es sich deshalb durchaus lohnen, diesen Ansatz zu verfolgen.

\end{otherlanguage}
