\cleardoublepage\chapter*{Abstract}
\thispagestyle{NoHeader}

\textit{Context:}
The Smoluchowski coagulation equation provides a framework that can be used for modeling the 
temporal evolution of the mass distribution of dust particles in proto-planetary disks.
When both coagulation and fragmentation processes are taken into account,
% When both associative and dissociative processes are included into the model 
% (i.e. coagulation and fragmentation), 
numerical integration requires the evaluation of a double integral over the 
reaction kernel matrix, which makes this approach quite numerically costly.
% This effect is ... inclusion of more 
% Previous studies regarding the temporal evolution of a proto-planetary disk's 
% dust mass distribution based on the model given by the Smoluchowski coagulation
% equation have so far 
\\ \ \\
\textit{Aims:}
The goal of this work is to examine the feasibility of utilizing stochastic Monte Carlo sampling 
of the kernel matrix in order to lower the integration's computational cost.
More specifically, the idea is to sample the most relevant dust particle collisions,
where relevance is determined by a given collision's impact on the evolution of the 
dust particle mass distribution.
\\ \ \\
\textit{Methods:}
After the construction of a simplified disk model, we define the reaction rates and kernel matrices
for both coagulation and fragmentation processes. The Smoluchowski equation is then numerically 
integrated using a 4th order implicit Radau integration scheme.
% using either the complete kernel 
% or an incomplete, sampled one. 
An appropriate sampling probability distribution is defined, 
% with which the kernel is then sampled.
with which an incomplete kernel matrix is constructed.
Parameter studies of the sampling density 
% $\rho_\text{sample}$ 
% as well as the mass grid resolution 
% $\mathcal N_m$ 
are made for both the simple case of pure hit-and-stick coagulation, as well as with 
fragmentation included into the model.
% as well as for the case where fragmentation processes are included into the model.
% For this, \todo{two different definitions of the sampling probabilit}
% sampling, define probability density
% parameter studies for various sampling densities and ...
\\ \ \\
\textit{Results:}
The general behavior of the mass distribution's temporal evolution can be reproduced, even if 
parts of the kernel are neglected.
When both coagulation and fragmentation are included, the numerical solution after a few hundred 
years resembles the expected equilibrium state.
% Even when parts of the kernel are left out when evaluating the sum on the right-hand side of 
% the discretized Smoluchowski coagulation equation, the general behavior of the mass distribution 
% under time can approximately be reproduced. 
It must be noted though, that
due to statistical fluctuations arising from the stochastic nature of the sampling method, 
the distribution's temporal 
derivative stays well above zero. This, as well as the effectively lower collisions rates,
leads to a loss with regards to accuracy. The stability of the algorithm with regards to mass 
conservation could be ensured down to $10^{-12}$.
% As one might expect, the magnitude of these fluctuations depends on the 
% value of the sampling density. Since leaving out some of the collisions effectively leads to a 
% lower total collision rate, the evolution progress more slowly for lower sampling densities. 
% During the integration, stability is kept well below 1, \todo{at around $10^{-12}$}. 
% \todo{Accuracy when compared to the complete solution suffers greatly.}
% We use quotation marks 
% here, because it needs to be noted that fluctuations arising from the stochastic nature of the 
% Monte Carlo method prevent the temporal derivative from going to zero. 
% of the mass distribution 
% As one might expect, stochastic Monte Carlo sampling of the kernel matrix 
% When we do not include the entire kernel, 
% When not the entire 
% Several key observations could be made: Stochastic kernel sampling leads to 
% probabilistically
% The inclusion of 
% stability assured
% quite ok 
\\ \ \\
\textit{Conclusions:}
Whether it makes sense to use this collision sampling approach likely depends on the specific 
requirements that are imposed on the computational resources, as well as the desired accuracy, since 
there is a trade-off between the two. 
If the dust particles are modeled as possessing more than one attribute (e.g. mass and porosity),
% When the dust particles are described as entities whose only 
% property is their mass, \todo{might only make sense for very high grid resolutions,} 
% or when only an approximate, first look at the results is desired.
% If more attributes, like e.g. particle porosity, are included into the model, 
the utility value of this method could increase.
As the amount of ``noise'' in the kernel can be expected to grow faster with the dimensionality of 
the problem than the number of relevant kernel entries, for studies of more sophisticated models
including additional particle attributes, it may well be worthwhile to consider this approach.

% due to the higher dimensionality of the problem. 

% It may well be that this specific method finds it uses mainly in cases where a first look at the 
% results of a given simulation is desired. 
% For studies where detailed 

% If, on the other hand, other particle attributes, like e.g. porosity are included into the model,
% we expect this approach 

% There is a trade-off between accuracy and numerical cost.

% \todo{Stability is ensured down to machine precision.} \\
% \todo{Accuracy suffers from low sampling density values. ?} \\
% \todo{nr. of non-zero kernel entries} \\

% \todo{Schluesse/Folgerungen?}
% \todo{Was geht besser?}

    % \item 
    % \item Because a lower sampling density lead to a larger accuracy error,
    %     the question of whether the sampling approach used in this thesis should be 
    %     used likely cannot be answered in a definitive way.
    %     % it cannot be stated clearly whether or not the 
    % \item 
    % \item What can be said though, is that the effectiveness of this approach will likely increase
    %     with the dimensionality of the problem, due to the Monte Carlo nature of the method.
    % \item The inclusion of a dust particle porosity parameter, which in conventional methods would 
    %     lead to a large increase in the numerical cost of the algorithm, could be facilitated 
    %     by using this stochastic sampling.
    % \item An illustrative reasoning for this is given by the fact that the amount of \textit{noise}
    %     in the kernel, i.e. the number of kernel entries that are largely irrelevant to the 
    %     temporal evolution of the mass distribution, grows faster than the number of entries 

% The numerical integration results are 
% The integrations' properties with regard to stability and accuracy
% Parameter studies for the influence of 

% \begin{align}
%     \frac{\text d n}{\text dt} 
%     = \int\limits_{-\infty}^{\infty} \int\limits_{-\infty}^{\infty}
%         K(m, m', m'') \cdot n(m') \cdot n(m'') \ \text dm' \ \text dm''
% \end{align}

% As such, it "is often" used in

% In previous studies using 
% this equation the assumption has ("often/mostly/so far?") been made that
% an individual dust particle can be characterized entirely by the value of its mass. \\
% See e.g. \cite{birnstiel_dullemond_brauer_2010} \cite{dullemond_dominik_2004} \\

% Here, the assumption is made that 

% \begin{equation}
%     \mathcal O(\mathcal N_m^3)
%     \cdot
%     \mathcal O(\mathcal N_p^3)
% \end{equation}

% poses a challenge to 
% the ... numerical integration of the Smoluchowski equation, since 

% Previous studies using this ("equation/idea/framework") 
% have  assumed that 

% Previous studies regarding the temporal evolution of the mass distribution of dust particles 
% in ("young/early?") proto-planetary disks 
% using the framework provided by the Smoluchowski coagulation equation

% \ \\
% The model provided by the Smoluchowski coagulation equation 
% "suffers from" 
% large increase in computational cost when introducing additional parameters like e.g. 
% particle porosity. 

% It requires the evaluation of a "sum".

% The Smoluchowski coagulation equation provides a framework that can be used for modeling the 
% temporal evolution of the distribution of dust particle masses in proto-planetary disks.



% The agglomeration of dust particles via a combination of Van der Waals and electrostatic forces
% is believed to play an important role in early planet formation inside proto-planetary disks. 
% A commonly used framework for modeling the temporal evolution of the 
% distribution of dust particle masses is given by the Smoluchowski coagulation equation. \\

% In this work we examine the feasibility of using Monte Carlo sampling of the kernel matrix
% for lowering the computational cost of the Smoluchowski. 
% The integration of the equation requires that one, in each time-step, 
% carries out the evaluation of a ``high-dimensional'' sum. \\

% The idea is to ignore the less relevant collisions,
% where the relevance of a collision between particles $i$ and $j$ is given by
% % = \frac{X_{ij}}{ \sum_{i=1}^\mathcal{N_m} \sum_{j=1}^{\mathcal N_m} X_{ij }}
% \begin{align}
%     P_{ij} 
%     \sim W_{ij} \cdot \rho_i \cdot \rho_j
% \end{align}
% with
% \begin{align}
%     W_{ij} 
%     = \sum_{k=1}^{\mathcal N_m} m_k \cdot K_{kij}
% \end{align}

% In this work we attempt to demonstrate a possible 
% In the context of studies of early planet


% proto-planetary disk

% In this work we investigate a possibility of lowering the computational cost of 
% the Smoluchowski coagulation equation's numerical integration. 

% of using Monte Carlo kernel sampling 
% to lower the computational cost 

% of colliding dust particle pairs 

% The agglomeration of dust particles via a combination of Van der Waals and electrostatic forces
% is believed to play an important role in early planet formation inside proto-planetary disks. 
% Once large enough, the gravitational influence of these bodies can 
% become strong enough to enable accretion of the surrounding gas (?), and thus, planetary core
% formation. (?) \\

% Studies regarding the process of dust coagulation % into larger and larger structures 
% are often based on the Smoluchowski coagulation equation. \todo{[cite]}
% This integro-differential equation allows the construction of a simplified model for the influence 
% of both coagulation and fragmentation processes onto the distribution of dust particle masses in 
% the disk. \\

% The formulation of a model for the growth of these dust particles has so far been quite limited by
% computational constraints. 

% The integration requires the evaluation of a multi-dimensional sum.

% The inclusion of grain properties other than their mass, like e.g.
% particle porosity, quickly leads to a drastic increase in the computational resources required for
% the numerical integration of the afore-mentioned Smoluchowski equation. \\

% It is the goal of this thesis to examine the feasibility of utilizing stochastic Monte Carlo 
% sampling of colliding circumstellar dust particle pairs in order to lower the integration's 
% computational cost. \\

% To do this, we define a simple disk model...
% The probability distribution...

% The sampling is done by assigning each collision $(i,j)$ between particles from bins 
% $i$ and $j$ a weight factor given by
% \begin{equation}
%     W_{ij}=
%         \underbrace{
%             \sum_k m_k\cdot\big|K_{kij}\big|
%         }_\text{a)}
%         \cdot
%         \underbrace{
%             N_i \cdot m_i
%             \vphantom{\sum_k m_k\cdot\big|K_{kij}\big|}
%         }_\text{b)}
%         \cdot
%         \underbrace{
%             N_j \cdot m_j
%             \vphantom{\sum_k m_k\cdot\big|K_{kij}\big|}
%         }_\text{c)}
% \end{equation}
% Normalization then leads to the probability distribution
% \begin{equation}
%     P_{ij}=
%         \frac{
%             W_{ij}
%         }{
%             \sum_{i,j} W_{ij}
%         }
% \end{equation}

% \todo{Results} 
