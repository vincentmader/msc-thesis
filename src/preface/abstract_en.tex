\chapter*{Abstract}
\thispagestyle{NoHeader}

The Smoluchowski coagulation equation provides a framework for modeling the temporal evolution of 
a particle group's distribution of masses under the influence of particle collisions and 
possible merge or fragmentation events. 
As such, it can be used to study the coagulation of particles in young circumstellar disks,
and the role dust plays in the early phases of planetary formation. \\

Previous studies of dust particle coagulation built using the Smoluchowski equation have made the 
simplifying assumption that the individual dust particles can be characterized entirely by the 
value of their mass (see e.g. 
    \cite{birnstiel_dullemond_brauer_2010}, 
    \cite{dullemond_dominik_2004}). 
This is due to the fact that the numerical integration of the Smoluchowski equation from one 
time step to the next involves the evaluation of a summation over the coagulation kernel. 
The inclusion of a second attribute like, for instance \textit{particle porosity}, greatly 
increases the dimensionality of the kernel, which leads to a significant increase in the 
computational cost required for the integration. \\

In this thesis we explore the possibility of lowering this computational cost via a stochastic 
Monte Carlo sampling of the kernel matrix. The central idea is to include only the most 
relevant sum terms, where relevance is determined by both the rate of a given collision,  
as well as by the effect it will have on the particle mass distribution function. \\

\todo{Stability is ensured down to machine precision.} \\
\todo{Accuracy suffers from low sampling density values. ?} \\
\todo{nr. of non-zero kernel entries} \\

\todo{Schluesse/Folgerungen?}
\todo{Was geht besser?}

% The numerical integration results are 
% The integrations' properties with regard to stability and accuracy
% Parameter studies for the influence of 

% \begin{align}
%     \frac{\text d n}{\text dt} 
%     = \int\limits_{-\infty}^{\infty} \int\limits_{-\infty}^{\infty}
%         K(m, m', m'') \cdot n(m') \cdot n(m'') \ \text dm' \ \text dm''
% \end{align}

% As such, it "is often" used in

% In previous studies using 
% this equation the assumption has ("often/mostly/so far?") been made that
% an individual dust particle can be characterized entirely by the value of its mass. \\
% See e.g. \cite{birnstiel_dullemond_brauer_2010} \cite{dullemond_dominik_2004} \\

% Here, the assumption is made that 

% \begin{equation}
%     \mathcal O(\mathcal N_m^3)
%     \cdot
%     \mathcal O(\mathcal N_p^3)
% \end{equation}

% poses a challenge to 
% the ... numerical integration of the Smoluchowski equation, since 

% Previous studies using this ("equation/idea/framework") 
% have  assumed that 

% Previous studies regarding the temporal evolution of the mass distribution of dust particles 
% in ("young/early?") proto-planetary disks 
% using the framework provided by the Smoluchowski coagulation equation

% \ \\
% The model provided by the Smoluchowski coagulation equation 
% "suffers from" 
% large increase in computational cost when introducing additional parameters like e.g. 
% particle porosity. 

% It requires the evaluation of a "sum".

% The Smoluchowski coagulation equation provides a framework that can be used for modeling the 
% temporal evolution of the distribution of dust particle masses in proto-planetary disks.



% The agglomeration of dust particles via a combination of Van der Waals and electrostatic forces
% is believed to play an important role in early planet formation inside proto-planetary disks. 
% A commonly used framework for modeling the temporal evolution of the 
% distribution of dust particle masses is given by the Smoluchowski coagulation equation. \\

% In this work we examine the feasibility of using Monte Carlo sampling of the kernel matrix
% for lowering the computational cost of the Smoluchowski. 
% The integration of the equation requires that one, in each time step, 
% carries out the evaluation of a "high-dimensional" sum. \\

% The idea is to ignore the less relevant collisions,
% where the relevance of a collision between particles $i$ and $j$ is given by
% % = \frac{X_{ij}}{ \sum_{i=1}^\mathcal{N_m} \sum_{j=1}^{\mathcal N_m} X_{ij }}
% \begin{align}
%     P_{ij} 
%     \sim W_{ij} \cdot \rho_i \cdot \rho_j
% \end{align}
% with
% \begin{align}
%     W_{ij} 
%     = \sum_{k=1}^{\mathcal N_m} m_k \cdot K_{kij}
% \end{align}

% In this work we attempt to demonstrate a possible 
% In the context of studies of early planet


% proto-planetary disk

% In this work we investigate a possibility of lowering the computational cost of 
% the Smoluchowski coagulation equation's numerical integration. 

% of using Monte Carlo kernel sampling 
% to lower the computational cost 

% of colliding dust particle pairs 

% The agglomeration of dust particles via a combination of Van der Waals and electrostatic forces
% is believed to play an important role in early planet formation inside proto-planetary disks. 
% Once large enough, the gravitational influence of these bodies can 
% become strong enough to enable accretion of the surrounding gas (?), and thus, planetary core
% formation. (?) \\

% Studies regarding the process of dust coagulation % into larger and larger structures 
% are often based on the Smoluchowski coagulation equation. \todo{[cite]}
% This integro-differential equation allows the construction of a simplified model for the influence 
% of both coagulation and fragmentation processes onto the distribution of dust particle masses in 
% the disk. \\

% The formulation of a model for the growth of these dust particles has so far been quite limited by
% computational constraints. 

% The integration requires the evaluation of a multi-dimensional sum.

% The inclusion of grain properties other than their mass, like e.g.
% particle porosity, quickly leads to a drastic increase in the computational resources required for
% the numerical integration of the afore-mentioned Smoluchowski equation. \\

% It is the goal of this thesis to examine the feasibility of utilizing stochastic Monte Carlo 
% sampling of colliding circumstellar dust particle pairs in order to lower the integration's 
% computational cost. \\

% To do this, we define a simple disk model...
% The probability distribution...

% The sampling is done by assigning each collision $(i,j)$ between particles from bins 
% $i$ and $j$ a weight factor given by
% \begin{equation}
%     W_{ij}=
%         \underbrace{
%             \sum_k m_k\cdot\big|K_{kij}\big|
%         }_\text{a)}
%         \cdot
%         \underbrace{
%             N_i \cdot m_i
%             \vphantom{\sum_k m_k\cdot\big|K_{kij}\big|}
%         }_\text{b)}
%         \cdot
%         \underbrace{
%             N_j \cdot m_j
%             \vphantom{\sum_k m_k\cdot\big|K_{kij}\big|}
%         }_\text{c)}
% \end{equation}
% Normalization then leads to the probability distribution
% \begin{equation}
%     P_{ij}=
%         \frac{
%             W_{ij}
%         }{
%             \sum_{i,j} W_{ij}
%         }
% \end{equation}

% \todo{Results} 
