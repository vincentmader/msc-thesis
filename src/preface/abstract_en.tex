\chapter*{Abstract}

\thispagestyle{NoHeader}

The agglomeration of dust particles via a combination of Van der Waals and electrostatic forces
is believed to play an important role in the early stages of planetesimal formation inside 
proto-planetary disks. Once large enough, the gravitational influence of these bodies can 
become strong enough to enable accretion of the surrounding gas (?), and thus, planetary core
formation. (?) \\

Studies regarding the process of dust coagulation % into larger and larger structures 
are often based on the Smoluchowski coagulation equation. % TODO Cite
This integro-differential equation allows the construction of a simplified model for the influence 
of both coagulation and fragmentation processes onto the distribution of dust particle masses in 
the disk. \\

The formulation of a model for the growth of these dust particles has so far been quite limited by
computational constraints. The inclusion of grain properties other than their mass, like e.g.
particle porosity, quickly leads to a drastic increase in the computational resources required for
the numerical integration of the afore-mentioned Smoluchowski equation. \\

It is the goal of this thesis to examine the feasibility of utilizing stochastic Monte Carlo 
sampling of colliding circumstellar dust particle pairs in order to lower the integration's 
computational cost. \\

% To do this, we define a simple disk model...
% The probability distribution...

The sampling is done by assigning each collision $(i,j)$ between particles from bins 
$i$ and $j$ a weight factor given by
\begin{equation}
    W_{ij}=
        \underbrace{
            \sum_k m_k\cdot\big|K_{kij}\big|
        }_\text{a)}
        \cdot
        \underbrace{
            N_i \cdot m_i
            \vphantom{\sum_k m_k\cdot\big|K_{kij}\big|}
        }_\text{b)}
        \cdot
        \underbrace{
            N_j \cdot m_j
            \vphantom{\sum_k m_k\cdot\big|K_{kij}\big|}
        }_\text{c)}
\end{equation}
Normalization then leads to the probability distribution
\begin{equation}
    P_{ij}=
        \frac{
            W_{ij}
        }{
            \sum_{i,j} W_{ij}
        }
\end{equation}

\todo{Results} 
